
% Default to the notebook output style

    


% Inherit from the specified cell style.




    
\documentclass{article}

    
    
    \usepackage{graphicx} % Used to insert images
    \usepackage{adjustbox} % Used to constrain images to a maximum size 
    \usepackage{color} % Allow colors to be defined
    \usepackage{enumerate} % Needed for markdown enumerations to work
    \usepackage{geometry} % Used to adjust the document margins
    \usepackage{amsmath} % Equations
    \usepackage{amssymb} % Equations
    \usepackage{eurosym} % defines \euro
    \usepackage[mathletters]{ucs} % Extended unicode (utf-8) support
    \usepackage[utf8x]{inputenc} % Allow utf-8 characters in the tex document
    \usepackage{fancyvrb} % verbatim replacement that allows latex
    \usepackage{grffile} % extends the file name processing of package graphics 
                         % to support a larger range 
    % The hyperref package gives us a pdf with properly built
    % internal navigation ('pdf bookmarks' for the table of contents,
    % internal cross-reference links, web links for URLs, etc.)
    \usepackage{hyperref}
    \usepackage{longtable} % longtable support required by pandoc >1.10
    \usepackage{booktabs}  % table support for pandoc > 1.12.2
    

    
    
    \definecolor{orange}{cmyk}{0,0.4,0.8,0.2}
    \definecolor{darkorange}{rgb}{.71,0.21,0.01}
    \definecolor{darkgreen}{rgb}{.12,.54,.11}
    \definecolor{myteal}{rgb}{.26, .44, .56}
    \definecolor{gray}{gray}{0.45}
    \definecolor{lightgray}{gray}{.95}
    \definecolor{mediumgray}{gray}{.8}
    \definecolor{inputbackground}{rgb}{.95, .95, .85}
    \definecolor{outputbackground}{rgb}{.95, .95, .95}
    \definecolor{traceback}{rgb}{1, .95, .95}
    % ansi colors
    \definecolor{red}{rgb}{.6,0,0}
    \definecolor{green}{rgb}{0,.65,0}
    \definecolor{brown}{rgb}{0.6,0.6,0}
    \definecolor{blue}{rgb}{0,.145,.698}
    \definecolor{purple}{rgb}{.698,.145,.698}
    \definecolor{cyan}{rgb}{0,.698,.698}
    \definecolor{lightgray}{gray}{0.5}
    
    % bright ansi colors
    \definecolor{darkgray}{gray}{0.25}
    \definecolor{lightred}{rgb}{1.0,0.39,0.28}
    \definecolor{lightgreen}{rgb}{0.48,0.99,0.0}
    \definecolor{lightblue}{rgb}{0.53,0.81,0.92}
    \definecolor{lightpurple}{rgb}{0.87,0.63,0.87}
    \definecolor{lightcyan}{rgb}{0.5,1.0,0.83}
    
    % commands and environments needed by pandoc snippets
    % extracted from the output of `pandoc -s`
    \DefineVerbatimEnvironment{Highlighting}{Verbatim}{commandchars=\\\{\}}
    % Add ',fontsize=\small' for more characters per line
    \newenvironment{Shaded}{}{}
    \newcommand{\KeywordTok}[1]{\textcolor[rgb]{0.00,0.44,0.13}{\textbf{{#1}}}}
    \newcommand{\DataTypeTok}[1]{\textcolor[rgb]{0.56,0.13,0.00}{{#1}}}
    \newcommand{\DecValTok}[1]{\textcolor[rgb]{0.25,0.63,0.44}{{#1}}}
    \newcommand{\BaseNTok}[1]{\textcolor[rgb]{0.25,0.63,0.44}{{#1}}}
    \newcommand{\FloatTok}[1]{\textcolor[rgb]{0.25,0.63,0.44}{{#1}}}
    \newcommand{\CharTok}[1]{\textcolor[rgb]{0.25,0.44,0.63}{{#1}}}
    \newcommand{\StringTok}[1]{\textcolor[rgb]{0.25,0.44,0.63}{{#1}}}
    \newcommand{\CommentTok}[1]{\textcolor[rgb]{0.38,0.63,0.69}{\textit{{#1}}}}
    \newcommand{\OtherTok}[1]{\textcolor[rgb]{0.00,0.44,0.13}{{#1}}}
    \newcommand{\AlertTok}[1]{\textcolor[rgb]{1.00,0.00,0.00}{\textbf{{#1}}}}
    \newcommand{\FunctionTok}[1]{\textcolor[rgb]{0.02,0.16,0.49}{{#1}}}
    \newcommand{\RegionMarkerTok}[1]{{#1}}
    \newcommand{\ErrorTok}[1]{\textcolor[rgb]{1.00,0.00,0.00}{\textbf{{#1}}}}
    \newcommand{\NormalTok}[1]{{#1}}
    
    % Define a nice break command that doesn't care if a line doesn't already
    % exist.
    \def\br{\hspace*{\fill} \\* }
    % Math Jax compatability definitions
    \def\gt{>}
    \def\lt{<}
    % Document parameters
    \title{Acquiring and wrangling with data}
    
    
    

    % Pygments definitions
    
\makeatletter
\def\PY@reset{\let\PY@it=\relax \let\PY@bf=\relax%
    \let\PY@ul=\relax \let\PY@tc=\relax%
    \let\PY@bc=\relax \let\PY@ff=\relax}
\def\PY@tok#1{\csname PY@tok@#1\endcsname}
\def\PY@toks#1+{\ifx\relax#1\empty\else%
    \PY@tok{#1}\expandafter\PY@toks\fi}
\def\PY@do#1{\PY@bc{\PY@tc{\PY@ul{%
    \PY@it{\PY@bf{\PY@ff{#1}}}}}}}
\def\PY#1#2{\PY@reset\PY@toks#1+\relax+\PY@do{#2}}

\expandafter\def\csname PY@tok@gd\endcsname{\def\PY@tc##1{\textcolor[rgb]{0.63,0.00,0.00}{##1}}}
\expandafter\def\csname PY@tok@gu\endcsname{\let\PY@bf=\textbf\def\PY@tc##1{\textcolor[rgb]{0.50,0.00,0.50}{##1}}}
\expandafter\def\csname PY@tok@gt\endcsname{\def\PY@tc##1{\textcolor[rgb]{0.00,0.27,0.87}{##1}}}
\expandafter\def\csname PY@tok@gs\endcsname{\let\PY@bf=\textbf}
\expandafter\def\csname PY@tok@gr\endcsname{\def\PY@tc##1{\textcolor[rgb]{1.00,0.00,0.00}{##1}}}
\expandafter\def\csname PY@tok@cm\endcsname{\let\PY@it=\textit\def\PY@tc##1{\textcolor[rgb]{0.25,0.50,0.50}{##1}}}
\expandafter\def\csname PY@tok@vg\endcsname{\def\PY@tc##1{\textcolor[rgb]{0.10,0.09,0.49}{##1}}}
\expandafter\def\csname PY@tok@m\endcsname{\def\PY@tc##1{\textcolor[rgb]{0.40,0.40,0.40}{##1}}}
\expandafter\def\csname PY@tok@mh\endcsname{\def\PY@tc##1{\textcolor[rgb]{0.40,0.40,0.40}{##1}}}
\expandafter\def\csname PY@tok@go\endcsname{\def\PY@tc##1{\textcolor[rgb]{0.53,0.53,0.53}{##1}}}
\expandafter\def\csname PY@tok@ge\endcsname{\let\PY@it=\textit}
\expandafter\def\csname PY@tok@vc\endcsname{\def\PY@tc##1{\textcolor[rgb]{0.10,0.09,0.49}{##1}}}
\expandafter\def\csname PY@tok@il\endcsname{\def\PY@tc##1{\textcolor[rgb]{0.40,0.40,0.40}{##1}}}
\expandafter\def\csname PY@tok@cs\endcsname{\let\PY@it=\textit\def\PY@tc##1{\textcolor[rgb]{0.25,0.50,0.50}{##1}}}
\expandafter\def\csname PY@tok@cp\endcsname{\def\PY@tc##1{\textcolor[rgb]{0.74,0.48,0.00}{##1}}}
\expandafter\def\csname PY@tok@gi\endcsname{\def\PY@tc##1{\textcolor[rgb]{0.00,0.63,0.00}{##1}}}
\expandafter\def\csname PY@tok@gh\endcsname{\let\PY@bf=\textbf\def\PY@tc##1{\textcolor[rgb]{0.00,0.00,0.50}{##1}}}
\expandafter\def\csname PY@tok@ni\endcsname{\let\PY@bf=\textbf\def\PY@tc##1{\textcolor[rgb]{0.60,0.60,0.60}{##1}}}
\expandafter\def\csname PY@tok@nl\endcsname{\def\PY@tc##1{\textcolor[rgb]{0.63,0.63,0.00}{##1}}}
\expandafter\def\csname PY@tok@nn\endcsname{\let\PY@bf=\textbf\def\PY@tc##1{\textcolor[rgb]{0.00,0.00,1.00}{##1}}}
\expandafter\def\csname PY@tok@no\endcsname{\def\PY@tc##1{\textcolor[rgb]{0.53,0.00,0.00}{##1}}}
\expandafter\def\csname PY@tok@na\endcsname{\def\PY@tc##1{\textcolor[rgb]{0.49,0.56,0.16}{##1}}}
\expandafter\def\csname PY@tok@nb\endcsname{\def\PY@tc##1{\textcolor[rgb]{0.00,0.50,0.00}{##1}}}
\expandafter\def\csname PY@tok@nc\endcsname{\let\PY@bf=\textbf\def\PY@tc##1{\textcolor[rgb]{0.00,0.00,1.00}{##1}}}
\expandafter\def\csname PY@tok@nd\endcsname{\def\PY@tc##1{\textcolor[rgb]{0.67,0.13,1.00}{##1}}}
\expandafter\def\csname PY@tok@ne\endcsname{\let\PY@bf=\textbf\def\PY@tc##1{\textcolor[rgb]{0.82,0.25,0.23}{##1}}}
\expandafter\def\csname PY@tok@nf\endcsname{\def\PY@tc##1{\textcolor[rgb]{0.00,0.00,1.00}{##1}}}
\expandafter\def\csname PY@tok@si\endcsname{\let\PY@bf=\textbf\def\PY@tc##1{\textcolor[rgb]{0.73,0.40,0.53}{##1}}}
\expandafter\def\csname PY@tok@s2\endcsname{\def\PY@tc##1{\textcolor[rgb]{0.73,0.13,0.13}{##1}}}
\expandafter\def\csname PY@tok@vi\endcsname{\def\PY@tc##1{\textcolor[rgb]{0.10,0.09,0.49}{##1}}}
\expandafter\def\csname PY@tok@nt\endcsname{\let\PY@bf=\textbf\def\PY@tc##1{\textcolor[rgb]{0.00,0.50,0.00}{##1}}}
\expandafter\def\csname PY@tok@nv\endcsname{\def\PY@tc##1{\textcolor[rgb]{0.10,0.09,0.49}{##1}}}
\expandafter\def\csname PY@tok@s1\endcsname{\def\PY@tc##1{\textcolor[rgb]{0.73,0.13,0.13}{##1}}}
\expandafter\def\csname PY@tok@kd\endcsname{\let\PY@bf=\textbf\def\PY@tc##1{\textcolor[rgb]{0.00,0.50,0.00}{##1}}}
\expandafter\def\csname PY@tok@sh\endcsname{\def\PY@tc##1{\textcolor[rgb]{0.73,0.13,0.13}{##1}}}
\expandafter\def\csname PY@tok@sc\endcsname{\def\PY@tc##1{\textcolor[rgb]{0.73,0.13,0.13}{##1}}}
\expandafter\def\csname PY@tok@sx\endcsname{\def\PY@tc##1{\textcolor[rgb]{0.00,0.50,0.00}{##1}}}
\expandafter\def\csname PY@tok@bp\endcsname{\def\PY@tc##1{\textcolor[rgb]{0.00,0.50,0.00}{##1}}}
\expandafter\def\csname PY@tok@c1\endcsname{\let\PY@it=\textit\def\PY@tc##1{\textcolor[rgb]{0.25,0.50,0.50}{##1}}}
\expandafter\def\csname PY@tok@kc\endcsname{\let\PY@bf=\textbf\def\PY@tc##1{\textcolor[rgb]{0.00,0.50,0.00}{##1}}}
\expandafter\def\csname PY@tok@c\endcsname{\let\PY@it=\textit\def\PY@tc##1{\textcolor[rgb]{0.25,0.50,0.50}{##1}}}
\expandafter\def\csname PY@tok@mf\endcsname{\def\PY@tc##1{\textcolor[rgb]{0.40,0.40,0.40}{##1}}}
\expandafter\def\csname PY@tok@err\endcsname{\def\PY@bc##1{\setlength{\fboxsep}{0pt}\fcolorbox[rgb]{1.00,0.00,0.00}{1,1,1}{\strut ##1}}}
\expandafter\def\csname PY@tok@mb\endcsname{\def\PY@tc##1{\textcolor[rgb]{0.40,0.40,0.40}{##1}}}
\expandafter\def\csname PY@tok@ss\endcsname{\def\PY@tc##1{\textcolor[rgb]{0.10,0.09,0.49}{##1}}}
\expandafter\def\csname PY@tok@sr\endcsname{\def\PY@tc##1{\textcolor[rgb]{0.73,0.40,0.53}{##1}}}
\expandafter\def\csname PY@tok@mo\endcsname{\def\PY@tc##1{\textcolor[rgb]{0.40,0.40,0.40}{##1}}}
\expandafter\def\csname PY@tok@kn\endcsname{\let\PY@bf=\textbf\def\PY@tc##1{\textcolor[rgb]{0.00,0.50,0.00}{##1}}}
\expandafter\def\csname PY@tok@mi\endcsname{\def\PY@tc##1{\textcolor[rgb]{0.40,0.40,0.40}{##1}}}
\expandafter\def\csname PY@tok@gp\endcsname{\let\PY@bf=\textbf\def\PY@tc##1{\textcolor[rgb]{0.00,0.00,0.50}{##1}}}
\expandafter\def\csname PY@tok@o\endcsname{\def\PY@tc##1{\textcolor[rgb]{0.40,0.40,0.40}{##1}}}
\expandafter\def\csname PY@tok@kr\endcsname{\let\PY@bf=\textbf\def\PY@tc##1{\textcolor[rgb]{0.00,0.50,0.00}{##1}}}
\expandafter\def\csname PY@tok@s\endcsname{\def\PY@tc##1{\textcolor[rgb]{0.73,0.13,0.13}{##1}}}
\expandafter\def\csname PY@tok@kp\endcsname{\def\PY@tc##1{\textcolor[rgb]{0.00,0.50,0.00}{##1}}}
\expandafter\def\csname PY@tok@w\endcsname{\def\PY@tc##1{\textcolor[rgb]{0.73,0.73,0.73}{##1}}}
\expandafter\def\csname PY@tok@kt\endcsname{\def\PY@tc##1{\textcolor[rgb]{0.69,0.00,0.25}{##1}}}
\expandafter\def\csname PY@tok@ow\endcsname{\let\PY@bf=\textbf\def\PY@tc##1{\textcolor[rgb]{0.67,0.13,1.00}{##1}}}
\expandafter\def\csname PY@tok@sb\endcsname{\def\PY@tc##1{\textcolor[rgb]{0.73,0.13,0.13}{##1}}}
\expandafter\def\csname PY@tok@k\endcsname{\let\PY@bf=\textbf\def\PY@tc##1{\textcolor[rgb]{0.00,0.50,0.00}{##1}}}
\expandafter\def\csname PY@tok@se\endcsname{\let\PY@bf=\textbf\def\PY@tc##1{\textcolor[rgb]{0.73,0.40,0.13}{##1}}}
\expandafter\def\csname PY@tok@sd\endcsname{\let\PY@it=\textit\def\PY@tc##1{\textcolor[rgb]{0.73,0.13,0.13}{##1}}}

\def\PYZbs{\char`\\}
\def\PYZus{\char`\_}
\def\PYZob{\char`\{}
\def\PYZcb{\char`\}}
\def\PYZca{\char`\^}
\def\PYZam{\char`\&}
\def\PYZlt{\char`\<}
\def\PYZgt{\char`\>}
\def\PYZsh{\char`\#}
\def\PYZpc{\char`\%}
\def\PYZdl{\char`\$}
\def\PYZhy{\char`\-}
\def\PYZsq{\char`\'}
\def\PYZdq{\char`\"}
\def\PYZti{\char`\~}
% for compatibility with earlier versions
\def\PYZat{@}
\def\PYZlb{[}
\def\PYZrb{]}
\makeatother


    % Exact colors from NB
    \definecolor{incolor}{rgb}{0.0, 0.0, 0.5}
    \definecolor{outcolor}{rgb}{0.545, 0.0, 0.0}



    
    % Prevent overflowing lines due to hard-to-break entities
    \sloppy 
    % Setup hyperref package
    \hypersetup{
      breaklinks=true,  % so long urls are correctly broken across lines
      colorlinks=true,
      urlcolor=blue,
      linkcolor=darkorange,
      citecolor=darkgreen,
      }
    % Slightly bigger margins than the latex defaults
    
    \geometry{verbose,tmargin=1in,bmargin=1in,lmargin=1in,rmargin=1in}
    
    

    \begin{document}
    
    
    \maketitle
    
    

    
    In previous sessions, we've talked about the underlying data structures
in the \texttt{Pandas} library. We've seen how to manipulate
\texttt{DataFrame} and \texttt{Series} objects in order to answer
questions regarding the data. Last week, we also saw how to use
\texttt{matplotlib} to visualize our analysis.

This week we will be diving into an important topic in \texttt{Pandas}:
aggregating and performing operations on specified groups of data
without modifying the underlying structure. According to Wes McKinney,

\begin{quote}
Categorizing a data set and applying a function to each group, whether
an aggregation or transformation, is often a critical component of a
data analysis workflow. After loading, merging, and preparing a data
set, a familiar task is to compute group statistics or possibly
\emph{pivot tables} for reporting or visualization purposes.
\texttt{Pandas} provides a flexible and high-performance
\texttt{groupby} facility, enabling you to slice and dice, and summarize
data sets in a natural way.
\end{quote}

    \begin{Verbatim}[commandchars=\\\{\}]
{\color{incolor}In [{\color{incolor}1}]:} \PY{k+kn}{from} \PY{n+nn}{\PYZus{}\PYZus{}future\PYZus{}\PYZus{}} \PY{k+kn}{import} \PY{n}{division}
        \PY{k+kn}{from} \PY{n+nn}{numpy.random} \PY{k+kn}{import} \PY{n}{randn}
        \PY{k+kn}{import} \PY{n+nn}{numpy} \PY{k+kn}{as} \PY{n+nn}{np}
        \PY{k+kn}{import} \PY{n+nn}{os}
        \PY{k+kn}{import} \PY{n+nn}{matplotlib.pyplot} \PY{k+kn}{as} \PY{n+nn}{plt}
        \PY{n}{np}\PY{o}{.}\PY{n}{random}\PY{o}{.}\PY{n}{seed}\PY{p}{(}\PY{l+m+mi}{12345}\PY{p}{)}
        \PY{n}{plt}\PY{o}{.}\PY{n}{rc}\PY{p}{(}\PY{l+s}{\PYZsq{}}\PY{l+s}{figure}\PY{l+s}{\PYZsq{}}\PY{p}{,} \PY{n}{figsize}\PY{o}{=}\PY{p}{(}\PY{l+m+mi}{10}\PY{p}{,} \PY{l+m+mi}{6}\PY{p}{)}\PY{p}{)}
        \PY{k+kn}{from} \PY{n+nn}{pandas} \PY{k+kn}{import} \PY{n}{Series}\PY{p}{,} \PY{n}{DataFrame}
        \PY{k+kn}{import} \PY{n+nn}{pandas} \PY{k+kn}{as} \PY{n+nn}{pd}
        \PY{n}{np}\PY{o}{.}\PY{n}{set\PYZus{}printoptions}\PY{p}{(}\PY{n}{precision}\PY{o}{=}\PY{l+m+mi}{4}\PY{p}{)}
\end{Verbatim}

    \section{Data Aggregation and Group
Operations}\label{data-aggregation-and-group-operations}

    \begin{Verbatim}[commandchars=\\\{\}]
{\color{incolor}In [{\color{incolor}2}]:} \PY{n}{pd}\PY{o}{.}\PY{n}{options}\PY{o}{.}\PY{n}{display}\PY{o}{.}\PY{n}{notebook\PYZus{}repr\PYZus{}html} \PY{o}{=} \PY{n+nb+bp}{False}
        \PY{o}{\PYZpc{}}\PY{k}{matplotlib} inline
\end{Verbatim}

    \subsection{GroupBy mechanics}\label{groupby-mechanics}

    \texttt{Pandas} was designed with a considerable deference to the
progress made in data aggregation techniques by developers for the
\texttt{R} programming language. The main mechanism is the
\emph{split-apply-combine} paradigm:

\begin{enumerate}
\def\labelenumi{\arabic{enumi}.}
\itemsep1pt\parskip0pt\parsep0pt
\item
  Data is \emph{split} into groups based on one or more provided
  \emph{keys},
\item
  A function is \emph{applied} to each group,
\item
  The results of all the function applications are \emph{combined} into
  a result object.
\end{enumerate}

As we will see, grouping keys are very flexible in nature. Some possible
types of keys are

\begin{itemize}
\itemsep1pt\parskip0pt\parsep0pt
\item
  A list or array of values sharing the length of the grouped column
\item
  A value indicating a column name
\item
  A dict or \texttt{Series} giving a correspondence between the values
  on the axis being grouped and the group names
\item
  A function to be invoked on the axis index or the individual labels in
  the index
\end{itemize}

    \begin{Verbatim}[commandchars=\\\{\}]
{\color{incolor}In [{\color{incolor}3}]:} \PY{n}{df} \PY{o}{=} \PY{n}{DataFrame}\PY{p}{(}\PY{p}{\PYZob{}}\PY{l+s}{\PYZsq{}}\PY{l+s}{key1}\PY{l+s}{\PYZsq{}} \PY{p}{:} \PY{p}{[}\PY{l+s}{\PYZsq{}}\PY{l+s}{a}\PY{l+s}{\PYZsq{}}\PY{p}{,} \PY{l+s}{\PYZsq{}}\PY{l+s}{a}\PY{l+s}{\PYZsq{}}\PY{p}{,} \PY{l+s}{\PYZsq{}}\PY{l+s}{b}\PY{l+s}{\PYZsq{}}\PY{p}{,} \PY{l+s}{\PYZsq{}}\PY{l+s}{b}\PY{l+s}{\PYZsq{}}\PY{p}{,} \PY{l+s}{\PYZsq{}}\PY{l+s}{a}\PY{l+s}{\PYZsq{}}\PY{p}{]}\PY{p}{,}
                        \PY{l+s}{\PYZsq{}}\PY{l+s}{key2}\PY{l+s}{\PYZsq{}} \PY{p}{:} \PY{p}{[}\PY{l+s}{\PYZsq{}}\PY{l+s}{one}\PY{l+s}{\PYZsq{}}\PY{p}{,} \PY{l+s}{\PYZsq{}}\PY{l+s}{two}\PY{l+s}{\PYZsq{}}\PY{p}{,} \PY{l+s}{\PYZsq{}}\PY{l+s}{one}\PY{l+s}{\PYZsq{}}\PY{p}{,} \PY{l+s}{\PYZsq{}}\PY{l+s}{two}\PY{l+s}{\PYZsq{}}\PY{p}{,} \PY{l+s}{\PYZsq{}}\PY{l+s}{one}\PY{l+s}{\PYZsq{}}\PY{p}{]}\PY{p}{,}
                        \PY{l+s}{\PYZsq{}}\PY{l+s}{data1}\PY{l+s}{\PYZsq{}} \PY{p}{:} \PY{n}{np}\PY{o}{.}\PY{n}{random}\PY{o}{.}\PY{n}{randn}\PY{p}{(}\PY{l+m+mi}{5}\PY{p}{)}\PY{p}{,}
                        \PY{l+s}{\PYZsq{}}\PY{l+s}{data2}\PY{l+s}{\PYZsq{}} \PY{p}{:} \PY{n}{np}\PY{o}{.}\PY{n}{random}\PY{o}{.}\PY{n}{randn}\PY{p}{(}\PY{l+m+mi}{5}\PY{p}{)}\PY{p}{\PYZcb{}}\PY{p}{)}
        \PY{n}{df}
\end{Verbatim}

            \begin{Verbatim}[commandchars=\\\{\}]
{\color{outcolor}Out[{\color{outcolor}3}]:}       data1     data2 key1 key2
        0 -0.204708  1.393406    a  one
        1  0.478943  0.092908    a  two
        2 -0.519439  0.281746    b  one
        3 -0.555730  0.769023    b  two
        4  1.965781  1.246435    a  one
\end{Verbatim}
        
    We can compute the mean of the column corresponding to ``data1'' by
using the group labels from ``key1''. This can be done in a number of
ways, but here is a straightforward example:

    \begin{Verbatim}[commandchars=\\\{\}]
{\color{incolor}In [{\color{incolor}4}]:} \PY{n}{grouped} \PY{o}{=} \PY{n}{df}\PY{p}{[}\PY{l+s}{\PYZsq{}}\PY{l+s}{data1}\PY{l+s}{\PYZsq{}}\PY{p}{]}\PY{o}{.}\PY{n}{groupby}\PY{p}{(}\PY{n}{df}\PY{p}{[}\PY{l+s}{\PYZsq{}}\PY{l+s}{key1}\PY{l+s}{\PYZsq{}}\PY{p}{]}\PY{p}{)}
        \PY{n}{grouped}
\end{Verbatim}

            \begin{Verbatim}[commandchars=\\\{\}]
{\color{outcolor}Out[{\color{outcolor}4}]:} <pandas.core.groupby.SeriesGroupBy object at 0x7f37d9157050>
\end{Verbatim}
        
    Notice that \texttt{grouped} is its own \texttt{Pandas} object, just
like a \texttt{Series} or \texttt{DataFrame}. Now we can compute simple
statistics just like we would with other objects.

    \begin{Verbatim}[commandchars=\\\{\}]
{\color{incolor}In [{\color{incolor}5}]:} \PY{n}{grouped}\PY{o}{.}\PY{n}{mean}\PY{p}{(}\PY{p}{)}
\end{Verbatim}

            \begin{Verbatim}[commandchars=\\\{\}]
{\color{outcolor}Out[{\color{outcolor}5}]:} key1
        a       0.746672
        b      -0.537585
        Name: data1, dtype: float64
\end{Verbatim}
        
    We can also pass an array of keys:

    \begin{Verbatim}[commandchars=\\\{\}]
{\color{incolor}In [{\color{incolor}6}]:} \PY{n}{means} \PY{o}{=} \PY{n}{df}\PY{p}{[}\PY{l+s}{\PYZsq{}}\PY{l+s}{data1}\PY{l+s}{\PYZsq{}}\PY{p}{]}\PY{o}{.}\PY{n}{groupby}\PY{p}{(}\PY{p}{[}\PY{n}{df}\PY{p}{[}\PY{l+s}{\PYZsq{}}\PY{l+s}{key1}\PY{l+s}{\PYZsq{}}\PY{p}{]}\PY{p}{,} \PY{n}{df}\PY{p}{[}\PY{l+s}{\PYZsq{}}\PY{l+s}{key2}\PY{l+s}{\PYZsq{}}\PY{p}{]}\PY{p}{]}\PY{p}{)}\PY{o}{.}\PY{n}{mean}\PY{p}{(}\PY{p}{)}
        \PY{n}{means}
\end{Verbatim}

            \begin{Verbatim}[commandchars=\\\{\}]
{\color{outcolor}Out[{\color{outcolor}6}]:} key1  key2
        a     one     0.880536
              two     0.478943
        b     one    -0.519439
              two    -0.555730
        Name: data1, dtype: float64
\end{Verbatim}
        
    This forms a grouping using a heierarchical index, as we have seen
earlier. To flatten the hierarchical index, as we have seen, we can call
\texttt{unstack()}.

    \begin{Verbatim}[commandchars=\\\{\}]
{\color{incolor}In [{\color{incolor}7}]:} \PY{n}{means}\PY{o}{.}\PY{n}{unstack}\PY{p}{(}\PY{p}{)}
\end{Verbatim}

            \begin{Verbatim}[commandchars=\\\{\}]
{\color{outcolor}Out[{\color{outcolor}7}]:} key2       one       two
        key1                    
        a     0.880536  0.478943
        b    -0.519439 -0.555730
\end{Verbatim}
        
    In these examples, the keys refer to \texttt{Series}, though they could
really be anything so long as the lengths match up. For example,
consider the following key arrays.

    \begin{Verbatim}[commandchars=\\\{\}]
{\color{incolor}In [{\color{incolor}8}]:} \PY{n}{states} \PY{o}{=} \PY{n}{np}\PY{o}{.}\PY{n}{array}\PY{p}{(}\PY{p}{[}\PY{l+s}{\PYZsq{}}\PY{l+s}{Ohio}\PY{l+s}{\PYZsq{}}\PY{p}{,} \PY{l+s}{\PYZsq{}}\PY{l+s}{California}\PY{l+s}{\PYZsq{}}\PY{p}{,} \PY{l+s}{\PYZsq{}}\PY{l+s}{California}\PY{l+s}{\PYZsq{}}\PY{p}{,} \PY{l+s}{\PYZsq{}}\PY{l+s}{Ohio}\PY{l+s}{\PYZsq{}}\PY{p}{,} \PY{l+s}{\PYZsq{}}\PY{l+s}{Ohio}\PY{l+s}{\PYZsq{}}\PY{p}{]}\PY{p}{)}
        \PY{n}{years} \PY{o}{=} \PY{n}{np}\PY{o}{.}\PY{n}{array}\PY{p}{(}\PY{p}{[}\PY{l+m+mi}{2005}\PY{p}{,} \PY{l+m+mi}{2005}\PY{p}{,} \PY{l+m+mi}{2006}\PY{p}{,} \PY{l+m+mi}{2005}\PY{p}{,} \PY{l+m+mi}{2006}\PY{p}{]}\PY{p}{)}
        \PY{n}{df}\PY{p}{[}\PY{l+s}{\PYZsq{}}\PY{l+s}{data1}\PY{l+s}{\PYZsq{}}\PY{p}{]}\PY{o}{.}\PY{n}{groupby}\PY{p}{(}\PY{p}{[}\PY{n}{states}\PY{p}{,} \PY{n}{years}\PY{p}{]}\PY{p}{)}\PY{o}{.}\PY{n}{mean}\PY{p}{(}\PY{p}{)}
\end{Verbatim}

            \begin{Verbatim}[commandchars=\\\{\}]
{\color{outcolor}Out[{\color{outcolor}8}]:} California  2005    0.478943
                    2006   -0.519439
        Ohio        2005   -0.380219
                    2006    1.965781
        Name: data1, dtype: float64
\end{Verbatim}
        
    If you're just interested in the column names, you can simply pass the
identifying string, or list of strings, as in the following examples:

    \begin{Verbatim}[commandchars=\\\{\}]
{\color{incolor}In [{\color{incolor}9}]:} \PY{n}{df}\PY{o}{.}\PY{n}{groupby}\PY{p}{(}\PY{l+s}{\PYZsq{}}\PY{l+s}{key1}\PY{l+s}{\PYZsq{}}\PY{p}{)}\PY{o}{.}\PY{n}{mean}\PY{p}{(}\PY{p}{)}
\end{Verbatim}

            \begin{Verbatim}[commandchars=\\\{\}]
{\color{outcolor}Out[{\color{outcolor}9}]:}          data1     data2
        key1                    
        a     0.746672  0.910916
        b    -0.537585  0.525384
\end{Verbatim}
        
    \begin{Verbatim}[commandchars=\\\{\}]
{\color{incolor}In [{\color{incolor}10}]:} \PY{n}{df}\PY{o}{.}\PY{n}{groupby}\PY{p}{(}\PY{p}{[}\PY{l+s}{\PYZsq{}}\PY{l+s}{key1}\PY{l+s}{\PYZsq{}}\PY{p}{,} \PY{l+s}{\PYZsq{}}\PY{l+s}{key2}\PY{l+s}{\PYZsq{}}\PY{p}{]}\PY{p}{)}\PY{o}{.}\PY{n}{mean}\PY{p}{(}\PY{p}{)}
\end{Verbatim}

            \begin{Verbatim}[commandchars=\\\{\}]
{\color{outcolor}Out[{\color{outcolor}10}]:}               data1     data2
         key1 key2                    
         a    one   0.880536  1.319920
              two   0.478943  0.092908
         b    one  -0.519439  0.281746
              two  -0.555730  0.769023
\end{Verbatim}
        
    \begin{Verbatim}[commandchars=\\\{\}]
{\color{incolor}In [{\color{incolor}11}]:} \PY{n}{df}\PY{o}{.}\PY{n}{groupby}\PY{p}{(}\PY{p}{[}\PY{l+s}{\PYZsq{}}\PY{l+s}{key1}\PY{l+s}{\PYZsq{}}\PY{p}{,} \PY{l+s}{\PYZsq{}}\PY{l+s}{key2}\PY{l+s}{\PYZsq{}}\PY{p}{]}\PY{p}{)}\PY{o}{.}\PY{n}{size}\PY{p}{(}\PY{p}{)}
\end{Verbatim}

            \begin{Verbatim}[commandchars=\\\{\}]
{\color{outcolor}Out[{\color{outcolor}11}]:} key1  key2
         a     one     2
               two     1
         b     one     1
               two     1
         dtype: int64
\end{Verbatim}
        
    \subsubsection{Iterating over groups}\label{iterating-over-groups}

    \texttt{groupby()} supports iteration. Specifically, \texttt{groupby()}
produces tuples containing the group name along the relevant data. For
example:

    \begin{Verbatim}[commandchars=\\\{\}]
{\color{incolor}In [{\color{incolor}12}]:} \PY{k}{for} \PY{n}{name}\PY{p}{,} \PY{n}{group} \PY{o+ow}{in} \PY{n}{df}\PY{o}{.}\PY{n}{groupby}\PY{p}{(}\PY{l+s}{\PYZsq{}}\PY{l+s}{key1}\PY{l+s}{\PYZsq{}}\PY{p}{)}\PY{p}{:}
             \PY{k}{print}\PY{p}{(}\PY{n}{name}\PY{p}{)}
             \PY{k}{print}\PY{p}{(}\PY{n}{group}\PY{p}{)}
\end{Verbatim}

    \begin{Verbatim}[commandchars=\\\{\}]
a
      data1     data2 key1 key2
0 -0.204708  1.393406    a  one
1  0.478943  0.092908    a  two
4  1.965781  1.246435    a  one
b
      data1     data2 key1 key2
2 -0.519439  0.281746    b  one
3 -0.555730  0.769023    b  two
    \end{Verbatim}

    If multiple keys are being passed, you can overload the name by making a
$n$-tuple of keys. For example:

    \begin{Verbatim}[commandchars=\\\{\}]
{\color{incolor}In [{\color{incolor}13}]:} \PY{k}{for} \PY{p}{(}\PY{n}{k1}\PY{p}{,} \PY{n}{k2}\PY{p}{)}\PY{p}{,} \PY{n}{group} \PY{o+ow}{in} \PY{n}{df}\PY{o}{.}\PY{n}{groupby}\PY{p}{(}\PY{p}{[}\PY{l+s}{\PYZsq{}}\PY{l+s}{key1}\PY{l+s}{\PYZsq{}}\PY{p}{,} \PY{l+s}{\PYZsq{}}\PY{l+s}{key2}\PY{l+s}{\PYZsq{}}\PY{p}{]}\PY{p}{)}\PY{p}{:}
             \PY{k}{print}\PY{p}{(}\PY{p}{(}\PY{n}{k1}\PY{p}{,} \PY{n}{k2}\PY{p}{)}\PY{p}{)}
             \PY{k}{print}\PY{p}{(}\PY{n}{group}\PY{p}{)}
\end{Verbatim}

    \begin{Verbatim}[commandchars=\\\{\}]
('a', 'one')
      data1     data2 key1 key2
0 -0.204708  1.393406    a  one
4  1.965781  1.246435    a  one
('a', 'two')
      data1     data2 key1 key2
1  0.478943  0.092908    a  two
('b', 'one')
      data1     data2 key1 key2
2 -0.519439  0.281746    b  one
('b', 'two')
     data1     data2 key1 key2
3 -0.55573  0.769023    b  two
    \end{Verbatim}

    This process is in general quite flexible. For example, suppose you want
to store the relevant \texttt{DataFrame} groups as a
native-\texttt{Python} dict. In this case, we can just wrap the
\texttt{groupby()} call with a list and a dict, which will thus be
stored into the dict.

    \begin{Verbatim}[commandchars=\\\{\}]
{\color{incolor}In [{\color{incolor}14}]:} \PY{n}{pieces} \PY{o}{=} \PY{n+nb}{dict}\PY{p}{(}\PY{n+nb}{list}\PY{p}{(}\PY{n}{df}\PY{o}{.}\PY{n}{groupby}\PY{p}{(}\PY{l+s}{\PYZsq{}}\PY{l+s}{key1}\PY{l+s}{\PYZsq{}}\PY{p}{)}\PY{p}{)}\PY{p}{)}
         \PY{n}{pieces}\PY{p}{[}\PY{l+s}{\PYZsq{}}\PY{l+s}{b}\PY{l+s}{\PYZsq{}}\PY{p}{]}
\end{Verbatim}

            \begin{Verbatim}[commandchars=\\\{\}]
{\color{outcolor}Out[{\color{outcolor}14}]:}       data1     data2 key1 key2
         2 -0.519439  0.281746    b  one
         3 -0.555730  0.769023    b  two
\end{Verbatim}
        
    By default, \texttt{groupby} groups on \texttt{axis=0}, which
corresponds to treating columns of data. Of course, you can specify
which axis you desire.

    \begin{Verbatim}[commandchars=\\\{\}]
{\color{incolor}In [{\color{incolor}15}]:} \PY{n}{df}\PY{o}{.}\PY{n}{dtypes}
\end{Verbatim}

            \begin{Verbatim}[commandchars=\\\{\}]
{\color{outcolor}Out[{\color{outcolor}15}]:} data1    float64
         data2    float64
         key1      object
         key2      object
         dtype: object
\end{Verbatim}
        
    \begin{Verbatim}[commandchars=\\\{\}]
{\color{incolor}In [{\color{incolor}16}]:} \PY{n}{grouped} \PY{o}{=} \PY{n}{df}\PY{o}{.}\PY{n}{groupby}\PY{p}{(}\PY{n}{df}\PY{o}{.}\PY{n}{dtypes}\PY{p}{,} \PY{n}{axis}\PY{o}{=}\PY{l+m+mi}{1}\PY{p}{)}
         \PY{n+nb}{dict}\PY{p}{(}\PY{n+nb}{list}\PY{p}{(}\PY{n}{grouped}\PY{p}{)}\PY{p}{)}
\end{Verbatim}

            \begin{Verbatim}[commandchars=\\\{\}]
{\color{outcolor}Out[{\color{outcolor}16}]:} \{dtype('float64'):       data1     data2
          0 -0.204708  1.393406
          1  0.478943  0.092908
          2 -0.519439  0.281746
          3 -0.555730  0.769023
          4  1.965781  1.246435, dtype('O'):   key1 key2
          0    a  one
          1    a  two
          2    b  one
          3    b  two
          4    a  one\}
\end{Verbatim}
        
    \subsubsection{Selecting a column or subset of
columns}\label{selecting-a-column-or-subset-of-columns}

    Indexing a \texttt{GroupBy} object created from a \texttt{DataFrame}
with a column name or array of column names has the efect of
\emph{selecting those columns} for aggregation. This means that:

\begin{Shaded}
\begin{Highlighting}[]
\NormalTok{df.groupby(}\StringTok{'key1'}\NormalTok{)[}\StringTok{'data1'}\NormalTok{]}
\NormalTok{df.groupby(}\StringTok{'key1'}\NormalTok{)[[}\StringTok{'data2'}\NormalTok{]]}
\end{Highlighting}
\end{Shaded}

is effectively identical to

\begin{Shaded}
\begin{Highlighting}[]
\NormalTok{df[}\StringTok{'data1'}\NormalTok{].groupby(df[}\StringTok{'key1'}\NormalTok{])}
\NormalTok{df[[}\StringTok{'data2'}\NormalTok{]].groupby(df[}\StringTok{'key1'}\NormalTok{])}
\end{Highlighting}
\end{Shaded}

    Why is this useful? If you're working with a large dataset for which
aggregating the entire \texttt{DataFrame} is out of the question, you
can speed up the process by specifying the particular columns you are
interested in.

    Here's an example: we can group a \texttt{DataFrame} according to a set
of keys, specify a particular column (in this case, \texttt{data2}), and
take the resulting mean.

    \begin{Verbatim}[commandchars=\\\{\}]
{\color{incolor}In [{\color{incolor}17}]:} \PY{n}{df}\PY{o}{.}\PY{n}{groupby}\PY{p}{(}\PY{p}{[}\PY{l+s}{\PYZsq{}}\PY{l+s}{key1}\PY{l+s}{\PYZsq{}}\PY{p}{,} \PY{l+s}{\PYZsq{}}\PY{l+s}{key2}\PY{l+s}{\PYZsq{}}\PY{p}{]}\PY{p}{)}\PY{p}{[}\PY{p}{[}\PY{l+s}{\PYZsq{}}\PY{l+s}{data2}\PY{l+s}{\PYZsq{}}\PY{p}{]}\PY{p}{]}\PY{o}{.}\PY{n}{mean}\PY{p}{(}\PY{p}{)}
\end{Verbatim}

            \begin{Verbatim}[commandchars=\\\{\}]
{\color{outcolor}Out[{\color{outcolor}17}]:}               data2
         key1 key2          
         a    one   1.319920
              two   0.092908
         b    one   0.281746
              two   0.769023
\end{Verbatim}
        
    The object returned here is a grouped \texttt{DataFrame} if a list or
array is passed and a grouped \texttt{Series} if just a column name is
passed.

    \begin{Verbatim}[commandchars=\\\{\}]
{\color{incolor}In [{\color{incolor}18}]:} \PY{n}{s\PYZus{}grouped} \PY{o}{=} \PY{n}{df}\PY{o}{.}\PY{n}{groupby}\PY{p}{(}\PY{p}{[}\PY{l+s}{\PYZsq{}}\PY{l+s}{key1}\PY{l+s}{\PYZsq{}}\PY{p}{,} \PY{l+s}{\PYZsq{}}\PY{l+s}{key2}\PY{l+s}{\PYZsq{}}\PY{p}{]}\PY{p}{)}\PY{p}{[}\PY{l+s}{\PYZsq{}}\PY{l+s}{data2}\PY{l+s}{\PYZsq{}}\PY{p}{]}
         \PY{n}{s\PYZus{}grouped}
\end{Verbatim}

            \begin{Verbatim}[commandchars=\\\{\}]
{\color{outcolor}Out[{\color{outcolor}18}]:} <pandas.core.groupby.SeriesGroupBy object at 0x7f37d9119350>
\end{Verbatim}
        
    \begin{Verbatim}[commandchars=\\\{\}]
{\color{incolor}In [{\color{incolor}19}]:} \PY{n}{s\PYZus{}grouped}\PY{o}{.}\PY{n}{mean}\PY{p}{(}\PY{p}{)}
\end{Verbatim}

            \begin{Verbatim}[commandchars=\\\{\}]
{\color{outcolor}Out[{\color{outcolor}19}]:} key1  key2
         a     one     1.319920
               two     0.092908
         b     one     0.281746
               two     0.769023
         Name: data2, dtype: float64
\end{Verbatim}
        
    \subsubsection{Grouping with dicts and
Series}\label{grouping-with-dicts-and-series}

    Grouping information may exist in a form other than an array. Let's
consider another example \texttt{DataFrame}.

    \begin{Verbatim}[commandchars=\\\{\}]
{\color{incolor}In [{\color{incolor}20}]:} \PY{n}{people} \PY{o}{=} \PY{n}{DataFrame}\PY{p}{(}\PY{n}{np}\PY{o}{.}\PY{n}{random}\PY{o}{.}\PY{n}{randn}\PY{p}{(}\PY{l+m+mi}{5}\PY{p}{,} \PY{l+m+mi}{5}\PY{p}{)}\PY{p}{,}
                            \PY{n}{columns}\PY{o}{=}\PY{p}{[}\PY{l+s}{\PYZsq{}}\PY{l+s}{a}\PY{l+s}{\PYZsq{}}\PY{p}{,} \PY{l+s}{\PYZsq{}}\PY{l+s}{b}\PY{l+s}{\PYZsq{}}\PY{p}{,} \PY{l+s}{\PYZsq{}}\PY{l+s}{c}\PY{l+s}{\PYZsq{}}\PY{p}{,} \PY{l+s}{\PYZsq{}}\PY{l+s}{d}\PY{l+s}{\PYZsq{}}\PY{p}{,} \PY{l+s}{\PYZsq{}}\PY{l+s}{e}\PY{l+s}{\PYZsq{}}\PY{p}{]}\PY{p}{,}
                            \PY{n}{index}\PY{o}{=}\PY{p}{[}\PY{l+s}{\PYZsq{}}\PY{l+s}{Joe}\PY{l+s}{\PYZsq{}}\PY{p}{,} \PY{l+s}{\PYZsq{}}\PY{l+s}{Steve}\PY{l+s}{\PYZsq{}}\PY{p}{,} \PY{l+s}{\PYZsq{}}\PY{l+s}{Wes}\PY{l+s}{\PYZsq{}}\PY{p}{,} \PY{l+s}{\PYZsq{}}\PY{l+s}{Jim}\PY{l+s}{\PYZsq{}}\PY{p}{,} \PY{l+s}{\PYZsq{}}\PY{l+s}{Travis}\PY{l+s}{\PYZsq{}}\PY{p}{]}\PY{p}{)}
         \PY{n}{people}\PY{o}{.}\PY{n}{ix}\PY{p}{[}\PY{l+m+mi}{2}\PY{p}{:}\PY{l+m+mi}{3}\PY{p}{,} \PY{p}{[}\PY{l+s}{\PYZsq{}}\PY{l+s}{b}\PY{l+s}{\PYZsq{}}\PY{p}{,} \PY{l+s}{\PYZsq{}}\PY{l+s}{c}\PY{l+s}{\PYZsq{}}\PY{p}{]}\PY{p}{]} \PY{o}{=} \PY{n}{np}\PY{o}{.}\PY{n}{nan} \PY{c}{\PYZsh{} Add a few NA values}
         \PY{n}{people}
\end{Verbatim}

            \begin{Verbatim}[commandchars=\\\{\}]
{\color{outcolor}Out[{\color{outcolor}20}]:}                a         b         c         d         e
         Joe     1.007189 -1.296221  0.274992  0.228913  1.352917
         Steve   0.886429 -2.001637 -0.371843  1.669025 -0.438570
         Wes    -0.539741       NaN       NaN -1.021228 -0.577087
         Jim     0.124121  0.302614  0.523772  0.000940  1.343810
         Travis -0.713544 -0.831154 -2.370232 -1.860761 -0.860757
\end{Verbatim}
        
    Suppose we have a group correspondence for the columns and want to sum
together the columns by group.

    \begin{Verbatim}[commandchars=\\\{\}]
{\color{incolor}In [{\color{incolor}21}]:} \PY{n}{mapping} \PY{o}{=} \PY{p}{\PYZob{}}\PY{l+s}{\PYZsq{}}\PY{l+s}{a}\PY{l+s}{\PYZsq{}}\PY{p}{:} \PY{l+s}{\PYZsq{}}\PY{l+s}{red}\PY{l+s}{\PYZsq{}}\PY{p}{,} \PY{l+s}{\PYZsq{}}\PY{l+s}{b}\PY{l+s}{\PYZsq{}}\PY{p}{:} \PY{l+s}{\PYZsq{}}\PY{l+s}{red}\PY{l+s}{\PYZsq{}}\PY{p}{,} \PY{l+s}{\PYZsq{}}\PY{l+s}{c}\PY{l+s}{\PYZsq{}}\PY{p}{:} \PY{l+s}{\PYZsq{}}\PY{l+s}{blue}\PY{l+s}{\PYZsq{}}\PY{p}{,}
                    \PY{l+s}{\PYZsq{}}\PY{l+s}{d}\PY{l+s}{\PYZsq{}}\PY{p}{:} \PY{l+s}{\PYZsq{}}\PY{l+s}{blue}\PY{l+s}{\PYZsq{}}\PY{p}{,} \PY{l+s}{\PYZsq{}}\PY{l+s}{e}\PY{l+s}{\PYZsq{}}\PY{p}{:} \PY{l+s}{\PYZsq{}}\PY{l+s}{red}\PY{l+s}{\PYZsq{}}\PY{p}{,} \PY{l+s}{\PYZsq{}}\PY{l+s}{f}\PY{l+s}{\PYZsq{}} \PY{p}{:} \PY{l+s}{\PYZsq{}}\PY{l+s}{orange}\PY{l+s}{\PYZsq{}}\PY{p}{\PYZcb{}}
\end{Verbatim}

    The \texttt{groupby} method can use the dict natively, allowing you to
form \texttt{GroupBy} objects on the fly.

    \begin{Verbatim}[commandchars=\\\{\}]
{\color{incolor}In [{\color{incolor}22}]:} \PY{n}{by\PYZus{}column} \PY{o}{=} \PY{n}{people}\PY{o}{.}\PY{n}{groupby}\PY{p}{(}\PY{n}{mapping}\PY{p}{,} \PY{n}{axis}\PY{o}{=}\PY{l+m+mi}{1}\PY{p}{)}
         \PY{n}{by\PYZus{}column}\PY{o}{.}\PY{n}{sum}\PY{p}{(}\PY{p}{)}
\end{Verbatim}

            \begin{Verbatim}[commandchars=\\\{\}]
{\color{outcolor}Out[{\color{outcolor}22}]:}             blue       red
         Joe     0.503905  1.063885
         Steve   1.297183 -1.553778
         Wes    -1.021228 -1.116829
         Jim     0.524712  1.770545
         Travis -4.230992 -2.405455
\end{Verbatim}
        
    The same holds for \texttt{Series} objects, which are structurally
similar to dicts.

    \begin{Verbatim}[commandchars=\\\{\}]
{\color{incolor}In [{\color{incolor}23}]:} \PY{n}{map\PYZus{}series} \PY{o}{=} \PY{n}{Series}\PY{p}{(}\PY{n}{mapping}\PY{p}{)}
         \PY{n}{map\PYZus{}series}
\end{Verbatim}

            \begin{Verbatim}[commandchars=\\\{\}]
{\color{outcolor}Out[{\color{outcolor}23}]:} a       red
         b       red
         c      blue
         d      blue
         e       red
         f    orange
         dtype: object
\end{Verbatim}
        
    \begin{Verbatim}[commandchars=\\\{\}]
{\color{incolor}In [{\color{incolor}24}]:} \PY{n}{people}\PY{o}{.}\PY{n}{groupby}\PY{p}{(}\PY{n}{map\PYZus{}series}\PY{p}{,} \PY{n}{axis}\PY{o}{=}\PY{l+m+mi}{1}\PY{p}{)}\PY{o}{.}\PY{n}{count}\PY{p}{(}\PY{p}{)}
\end{Verbatim}

            \begin{Verbatim}[commandchars=\\\{\}]
{\color{outcolor}Out[{\color{outcolor}24}]:}         blue  red
         Joe        2    3
         Steve      2    3
         Wes        1    2
         Jim        2    3
         Travis     2    3
\end{Verbatim}
        
    \subsubsection{Grouping with functions}\label{grouping-with-functions}

    In \texttt{Python}, functions are simply another data type. You can
actually use \texttt{groupby} to isolate members of your data set
according to a rule, defined by a function. For example:

    \begin{Verbatim}[commandchars=\\\{\}]
{\color{incolor}In [{\color{incolor}25}]:} \PY{n}{people}\PY{o}{.}\PY{n}{groupby}\PY{p}{(}\PY{n+nb}{len}\PY{p}{)}\PY{o}{.}\PY{n}{sum}\PY{p}{(}\PY{p}{)}
\end{Verbatim}

            \begin{Verbatim}[commandchars=\\\{\}]
{\color{outcolor}Out[{\color{outcolor}25}]:}           a         b         c         d         e
         3  0.591569 -0.993608  0.798764 -0.791374  2.119639
         5  0.886429 -2.001637 -0.371843  1.669025 -0.438570
         6 -0.713544 -0.831154 -2.370232 -1.860761 -0.860757
\end{Verbatim}
        
    You can mix and match functions with other data types that we have seen
above.

    \begin{Verbatim}[commandchars=\\\{\}]
{\color{incolor}In [{\color{incolor}26}]:} \PY{n}{key\PYZus{}list} \PY{o}{=} \PY{p}{[}\PY{l+s}{\PYZsq{}}\PY{l+s}{one}\PY{l+s}{\PYZsq{}}\PY{p}{,} \PY{l+s}{\PYZsq{}}\PY{l+s}{one}\PY{l+s}{\PYZsq{}}\PY{p}{,} \PY{l+s}{\PYZsq{}}\PY{l+s}{one}\PY{l+s}{\PYZsq{}}\PY{p}{,} \PY{l+s}{\PYZsq{}}\PY{l+s}{two}\PY{l+s}{\PYZsq{}}\PY{p}{,} \PY{l+s}{\PYZsq{}}\PY{l+s}{two}\PY{l+s}{\PYZsq{}}\PY{p}{]}
         \PY{n}{people}\PY{o}{.}\PY{n}{groupby}\PY{p}{(}\PY{p}{[}\PY{n+nb}{len}\PY{p}{,} \PY{n}{key\PYZus{}list}\PY{p}{]}\PY{p}{)}\PY{o}{.}\PY{n}{min}\PY{p}{(}\PY{p}{)}
\end{Verbatim}

            \begin{Verbatim}[commandchars=\\\{\}]
{\color{outcolor}Out[{\color{outcolor}26}]:}               a         b         c         d         e
         3 one -0.539741 -1.296221  0.274992 -1.021228 -0.577087
           two  0.124121  0.302614  0.523772  0.000940  1.343810
         5 one  0.886429 -2.001637 -0.371843  1.669025 -0.438570
         6 two -0.713544 -0.831154 -2.370232 -1.860761 -0.860757
\end{Verbatim}
        
    \subsubsection{Grouping by index levels}\label{grouping-by-index-levels}

    Finally, you can use heierarchical indexing to perform \texttt{groupby}
operations. To do this, pass the level number or name using the
\texttt{level} keyword.

    \begin{Verbatim}[commandchars=\\\{\}]
{\color{incolor}In [{\color{incolor}27}]:} \PY{n}{columns} \PY{o}{=} \PY{n}{pd}\PY{o}{.}\PY{n}{MultiIndex}\PY{o}{.}\PY{n}{from\PYZus{}arrays}\PY{p}{(}\PY{p}{[}\PY{p}{[}\PY{l+s}{\PYZsq{}}\PY{l+s}{US}\PY{l+s}{\PYZsq{}}\PY{p}{,} \PY{l+s}{\PYZsq{}}\PY{l+s}{US}\PY{l+s}{\PYZsq{}}\PY{p}{,} \PY{l+s}{\PYZsq{}}\PY{l+s}{US}\PY{l+s}{\PYZsq{}}\PY{p}{,} \PY{l+s}{\PYZsq{}}\PY{l+s}{JP}\PY{l+s}{\PYZsq{}}\PY{p}{,} \PY{l+s}{\PYZsq{}}\PY{l+s}{JP}\PY{l+s}{\PYZsq{}}\PY{p}{]}\PY{p}{,}
                                             \PY{p}{[}\PY{l+m+mi}{1}\PY{p}{,} \PY{l+m+mi}{3}\PY{p}{,} \PY{l+m+mi}{5}\PY{p}{,} \PY{l+m+mi}{1}\PY{p}{,} \PY{l+m+mi}{3}\PY{p}{]}\PY{p}{]}\PY{p}{,} 
                                             \PY{n}{names}\PY{o}{=}\PY{p}{[}\PY{l+s}{\PYZsq{}}\PY{l+s}{cty}\PY{l+s}{\PYZsq{}}\PY{p}{,} \PY{l+s}{\PYZsq{}}\PY{l+s}{tenor}\PY{l+s}{\PYZsq{}}\PY{p}{]}\PY{p}{)}
         \PY{n}{hier\PYZus{}df} \PY{o}{=} \PY{n}{DataFrame}\PY{p}{(}\PY{n}{np}\PY{o}{.}\PY{n}{random}\PY{o}{.}\PY{n}{randn}\PY{p}{(}\PY{l+m+mi}{4}\PY{p}{,} \PY{l+m+mi}{5}\PY{p}{)}\PY{p}{,} \PY{n}{columns}\PY{o}{=}\PY{n}{columns}\PY{p}{)}
         \PY{n}{hier\PYZus{}df}
\end{Verbatim}

            \begin{Verbatim}[commandchars=\\\{\}]
{\color{outcolor}Out[{\color{outcolor}27}]:} cty          US                            JP          
         tenor         1         3         5         1         3
         0      0.560145 -1.265934  0.119827 -1.063512  0.332883
         1     -2.359419 -0.199543 -1.541996 -0.970736 -1.307030
         2      0.286350  0.377984 -0.753887  0.331286  1.349742
         3      0.069877  0.246674 -0.011862  1.004812  1.327195
\end{Verbatim}
        
    \begin{Verbatim}[commandchars=\\\{\}]
{\color{incolor}In [{\color{incolor}28}]:} \PY{n}{hier\PYZus{}df}\PY{o}{.}\PY{n}{groupby}\PY{p}{(}\PY{n}{level}\PY{o}{=}\PY{l+s}{\PYZsq{}}\PY{l+s}{cty}\PY{l+s}{\PYZsq{}}\PY{p}{,} \PY{n}{axis}\PY{o}{=}\PY{l+m+mi}{1}\PY{p}{)}\PY{o}{.}\PY{n}{count}\PY{p}{(}\PY{p}{)}
\end{Verbatim}

            \begin{Verbatim}[commandchars=\\\{\}]
{\color{outcolor}Out[{\color{outcolor}28}]:} cty  JP  US
         0     2   3
         1     2   3
         2     2   3
         3     2   3
\end{Verbatim}
        
    \subsection{Data aggregation}\label{data-aggregation}

    Wes McKinney defines \emph{data aggregation} as any transformation that
takes a dataset or other array and produces scalar values. For example,
the simple statistical functions such as

\begin{itemize}
\itemsep1pt\parskip0pt\parsep0pt
\item
  mean
\item
  max
\item
  min
\item
  sum
\end{itemize}

are examples of operations taking arrays to numbers. Many of the
aggregations that we have seen so far have been optimized for
performance, but \texttt{Pandas} gives you the functionality to
implement customized aggregators.

    \begin{Verbatim}[commandchars=\\\{\}]
{\color{incolor}In [{\color{incolor}29}]:} \PY{n}{df}
\end{Verbatim}

            \begin{Verbatim}[commandchars=\\\{\}]
{\color{outcolor}Out[{\color{outcolor}29}]:}       data1     data2 key1 key2
         0 -0.204708  1.393406    a  one
         1  0.478943  0.092908    a  two
         2 -0.519439  0.281746    b  one
         3 -0.555730  0.769023    b  two
         4  1.965781  1.246435    a  one
\end{Verbatim}
        
    For example, we can use \texttt{quantile(x)} (not explicitly implemented
for \texttt{GroupBy}), which determines the value of $x$th percentile
given a \texttt{Series} of data.

    \begin{Verbatim}[commandchars=\\\{\}]
{\color{incolor}In [{\color{incolor}30}]:} \PY{n}{grouped} \PY{o}{=} \PY{n}{df}\PY{o}{.}\PY{n}{groupby}\PY{p}{(}\PY{l+s}{\PYZsq{}}\PY{l+s}{key1}\PY{l+s}{\PYZsq{}}\PY{p}{)}
         \PY{n}{grouped}\PY{p}{[}\PY{l+s}{\PYZsq{}}\PY{l+s}{data1}\PY{l+s}{\PYZsq{}}\PY{p}{]}\PY{o}{.}\PY{n}{quantile}\PY{p}{(}\PY{l+m+mf}{0.9}\PY{p}{)}
\end{Verbatim}

            \begin{Verbatim}[commandchars=\\\{\}]
{\color{outcolor}Out[{\color{outcolor}30}]:} key1
         a       1.668413
         b      -0.523068
         Name: data1, dtype: float64
\end{Verbatim}
        
    More to the point, you can define your own functions and do
\texttt{groupby} operations with them. For example, if you are
interested in the range of your data sets, you can use:

    \begin{Verbatim}[commandchars=\\\{\}]
{\color{incolor}In [{\color{incolor}31}]:} \PY{k}{def} \PY{n+nf}{peak\PYZus{}to\PYZus{}peak}\PY{p}{(}\PY{n}{arr}\PY{p}{)}\PY{p}{:}
             \PY{k}{return} \PY{n}{arr}\PY{o}{.}\PY{n}{max}\PY{p}{(}\PY{p}{)} \PY{o}{\PYZhy{}} \PY{n}{arr}\PY{o}{.}\PY{n}{min}\PY{p}{(}\PY{p}{)}
         \PY{n}{grouped}\PY{o}{.}\PY{n}{agg}\PY{p}{(}\PY{n}{peak\PYZus{}to\PYZus{}peak}\PY{p}{)}
\end{Verbatim}

            \begin{Verbatim}[commandchars=\\\{\}]
{\color{outcolor}Out[{\color{outcolor}31}]:}          data1     data2
         key1                    
         a     2.170488  1.300498
         b     0.036292  0.487276
\end{Verbatim}
        
    Even method that aren't really aggregations, such as \texttt{describe},
still perform useful operations on \texttt{GroupBy} objects.

    \begin{Verbatim}[commandchars=\\\{\}]
{\color{incolor}In [{\color{incolor}32}]:} \PY{n}{grouped}\PY{o}{.}\PY{n}{describe}\PY{p}{(}\PY{p}{)}
\end{Verbatim}

            \begin{Verbatim}[commandchars=\\\{\}]
{\color{outcolor}Out[{\color{outcolor}32}]:}                data1     data2
         key1                          
         a    count  3.000000  3.000000
              mean   0.746672  0.910916
              std    1.109736  0.712217
              min   -0.204708  0.092908
              25\%    0.137118  0.669671
              50\%    0.478943  1.246435
              75\%    1.222362  1.319920
              max    1.965781  1.393406
         b    count  2.000000  2.000000
              mean  -0.537585  0.525384
              std    0.025662  0.344556
              min   -0.555730  0.281746
              25\%   -0.546657  0.403565
              50\%   -0.537585  0.525384
              75\%   -0.528512  0.647203
              max   -0.519439  0.769023
\end{Verbatim}
        
    We will continue with the \texttt{tips} dataset from previous weeks to
show off some of the more advanced features of aggregation. The data set
can be found on the course webpage, or
\href{http://www.math.grinnell.edu/~kraemerd17/courses/python/course-material/tips.csv}{here},
if you're lazy (like us).

    \begin{Verbatim}[commandchars=\\\{\}]
{\color{incolor}In [{\color{incolor}33}]:} \PY{n}{tips} \PY{o}{=} \PY{n}{pd}\PY{o}{.}\PY{n}{read\PYZus{}csv}\PY{p}{(}\PY{l+s}{\PYZsq{}}\PY{l+s}{tips.csv}\PY{l+s}{\PYZsq{}}\PY{p}{)}
         \PY{c}{\PYZsh{} Add tip percentage of total bill}
         \PY{n}{tips}\PY{p}{[}\PY{l+s}{\PYZsq{}}\PY{l+s}{tip\PYZus{}pct}\PY{l+s}{\PYZsq{}}\PY{p}{]} \PY{o}{=} \PY{n}{tips}\PY{p}{[}\PY{l+s}{\PYZsq{}}\PY{l+s}{tip}\PY{l+s}{\PYZsq{}}\PY{p}{]} \PY{o}{/} \PY{n}{tips}\PY{p}{[}\PY{l+s}{\PYZsq{}}\PY{l+s}{total\PYZus{}bill}\PY{l+s}{\PYZsq{}}\PY{p}{]}
         \PY{n}{tips}\PY{p}{[}\PY{p}{:}\PY{l+m+mi}{6}\PY{p}{]}
\end{Verbatim}

            \begin{Verbatim}[commandchars=\\\{\}]
{\color{outcolor}Out[{\color{outcolor}33}]:}    total\_bill   tip     sex smoker  day    time  size   tip\_pct
         0       16.99  1.01  Female     No  Sun  Dinner     2  0.059447
         1       10.34  1.66    Male     No  Sun  Dinner     3  0.160542
         2       21.01  3.50    Male     No  Sun  Dinner     3  0.166587
         3       23.68  3.31    Male     No  Sun  Dinner     2  0.139780
         4       24.59  3.61  Female     No  Sun  Dinner     4  0.146808
         5       25.29  4.71    Male     No  Sun  Dinner     4  0.186240
\end{Verbatim}
        
    \subsubsection{Column-wise and multiple function
application}\label{column-wise-and-multiple-function-application}

    As we've seen, aggregating a \texttt{Series} or all of the columns of a
\texttt{DataFrame} is a matter of using \texttt{aggregate} with the
desired function or calling a method like \texttt{mean} or \texttt{std}.
However, you may want to aggregate using a different function depending
on the column or multiple functions at once. Fotunately, this is
straightforward to do, which we will illustrate through a number of
examples. First, let's group the tips by \texttt{sex} and
\texttt{smoker}.

    \begin{Verbatim}[commandchars=\\\{\}]
{\color{incolor}In [{\color{incolor}34}]:} \PY{n}{grouped} \PY{o}{=} \PY{n}{tips}\PY{o}{.}\PY{n}{groupby}\PY{p}{(}\PY{p}{[}\PY{l+s}{\PYZsq{}}\PY{l+s}{sex}\PY{l+s}{\PYZsq{}}\PY{p}{,} \PY{l+s}{\PYZsq{}}\PY{l+s}{smoker}\PY{l+s}{\PYZsq{}}\PY{p}{]}\PY{p}{)}
\end{Verbatim}

    Descriptive statistics, such as \texttt{mean}, can be passed to the
aggregator as a string.

    \begin{Verbatim}[commandchars=\\\{\}]
{\color{incolor}In [{\color{incolor}35}]:} \PY{n}{grouped\PYZus{}pct} \PY{o}{=} \PY{n}{grouped}\PY{p}{[}\PY{l+s}{\PYZsq{}}\PY{l+s}{tip\PYZus{}pct}\PY{l+s}{\PYZsq{}}\PY{p}{]}
         \PY{n}{grouped\PYZus{}pct}\PY{o}{.}\PY{n}{agg}\PY{p}{(}\PY{l+s}{\PYZsq{}}\PY{l+s}{mean}\PY{l+s}{\PYZsq{}}\PY{p}{)}
\end{Verbatim}

            \begin{Verbatim}[commandchars=\\\{\}]
{\color{outcolor}Out[{\color{outcolor}35}]:} sex     smoker
         Female  No        0.156921
                 Yes       0.182150
         Male    No        0.160669
                 Yes       0.152771
         Name: tip\_pct, dtype: float64
\end{Verbatim}
        
    You can also pass a list of functions to do aggregation. If the function
is built-in, it passes as a string. Otherwise, one can simply pass the
function on its own.

    \begin{Verbatim}[commandchars=\\\{\}]
{\color{incolor}In [{\color{incolor}36}]:} \PY{n}{grouped\PYZus{}pct}\PY{o}{.}\PY{n}{agg}\PY{p}{(}\PY{p}{[}\PY{l+s}{\PYZsq{}}\PY{l+s}{mean}\PY{l+s}{\PYZsq{}}\PY{p}{,} \PY{l+s}{\PYZsq{}}\PY{l+s}{std}\PY{l+s}{\PYZsq{}}\PY{p}{,} \PY{n}{peak\PYZus{}to\PYZus{}peak}\PY{p}{]}\PY{p}{)}
\end{Verbatim}

            \begin{Verbatim}[commandchars=\\\{\}]
{\color{outcolor}Out[{\color{outcolor}36}]:}                    mean       std  peak\_to\_peak
         sex    smoker                                  
         Female No      0.156921  0.036421      0.195876
                Yes     0.182150  0.071595      0.360233
         Male   No      0.160669  0.041849      0.220186
                Yes     0.152771  0.090588      0.674707
\end{Verbatim}
        
    To label the columns assigned by \texttt{agg}, pass a tuple in for each
function, with the first element corresponding to the label of the
column.

    \begin{Verbatim}[commandchars=\\\{\}]
{\color{incolor}In [{\color{incolor}37}]:} \PY{n}{grouped\PYZus{}pct}\PY{o}{.}\PY{n}{agg}\PY{p}{(}\PY{p}{[}\PY{p}{(}\PY{l+s}{\PYZsq{}}\PY{l+s}{foo}\PY{l+s}{\PYZsq{}}\PY{p}{,} \PY{l+s}{\PYZsq{}}\PY{l+s}{mean}\PY{l+s}{\PYZsq{}}\PY{p}{)}\PY{p}{,} \PY{p}{(}\PY{l+s}{\PYZsq{}}\PY{l+s}{bar}\PY{l+s}{\PYZsq{}}\PY{p}{,} \PY{n}{np}\PY{o}{.}\PY{n}{std}\PY{p}{)}\PY{p}{]}\PY{p}{)}
\end{Verbatim}

            \begin{Verbatim}[commandchars=\\\{\}]
{\color{outcolor}Out[{\color{outcolor}37}]:}                     foo       bar
         sex    smoker                    
         Female No      0.156921  0.036421
                Yes     0.182150  0.071595
         Male   No      0.160669  0.041849
                Yes     0.152771  0.090588
\end{Verbatim}
        
    With a \texttt{DataFame} you have more options as you can specify as
list of functions to apply to all of the columns or different functions
per column.

    \begin{Verbatim}[commandchars=\\\{\}]
{\color{incolor}In [{\color{incolor}38}]:} \PY{n}{functions} \PY{o}{=} \PY{p}{[}\PY{l+s}{\PYZsq{}}\PY{l+s}{count}\PY{l+s}{\PYZsq{}}\PY{p}{,} \PY{l+s}{\PYZsq{}}\PY{l+s}{mean}\PY{l+s}{\PYZsq{}}\PY{p}{,} \PY{l+s}{\PYZsq{}}\PY{l+s}{max}\PY{l+s}{\PYZsq{}}\PY{p}{]}
         \PY{n}{result} \PY{o}{=} \PY{n}{grouped}\PY{p}{[}\PY{l+s}{\PYZsq{}}\PY{l+s}{tip\PYZus{}pct}\PY{l+s}{\PYZsq{}}\PY{p}{,} \PY{l+s}{\PYZsq{}}\PY{l+s}{total\PYZus{}bill}\PY{l+s}{\PYZsq{}}\PY{p}{]}\PY{o}{.}\PY{n}{agg}\PY{p}{(}\PY{n}{functions}\PY{p}{)}
         \PY{n}{result}
\end{Verbatim}

            \begin{Verbatim}[commandchars=\\\{\}]
{\color{outcolor}Out[{\color{outcolor}38}]:}               tip\_pct                     total\_bill                  
                         count      mean       max      count       mean    max
         sex    smoker                                                         
         Female No          54  0.156921  0.252672         54  18.105185  35.83
                Yes         33  0.182150  0.416667         33  17.977879  44.30
         Male   No          97  0.160669  0.291990         97  19.791237  48.33
                Yes         60  0.152771  0.710345         60  22.284500  50.81
\end{Verbatim}
        
    Here we are using what effectively amounts to a hierarchical index,
which we can then slice by choosing columns and subcolumns.

    \begin{Verbatim}[commandchars=\\\{\}]
{\color{incolor}In [{\color{incolor}39}]:} \PY{n}{result}\PY{p}{[}\PY{l+s}{\PYZsq{}}\PY{l+s}{tip\PYZus{}pct}\PY{l+s}{\PYZsq{}}\PY{p}{]}
\end{Verbatim}

            \begin{Verbatim}[commandchars=\\\{\}]
{\color{outcolor}Out[{\color{outcolor}39}]:}                count      mean       max
         sex    smoker                           
         Female No         54  0.156921  0.252672
                Yes        33  0.182150  0.416667
         Male   No         97  0.160669  0.291990
                Yes        60  0.152771  0.710345
\end{Verbatim}
        
    As above, a list of tuples with custom names can be passed:

    \begin{Verbatim}[commandchars=\\\{\}]
{\color{incolor}In [{\color{incolor}40}]:} \PY{n}{ftuples} \PY{o}{=} \PY{p}{[}\PY{p}{(}\PY{l+s}{\PYZsq{}}\PY{l+s}{Durchschnitt}\PY{l+s}{\PYZsq{}}\PY{p}{,} \PY{l+s}{\PYZsq{}}\PY{l+s}{mean}\PY{l+s}{\PYZsq{}}\PY{p}{)}\PY{p}{,} \PY{p}{(}\PY{l+s}{\PYZsq{}}\PY{l+s}{Abweichung}\PY{l+s}{\PYZsq{}}\PY{p}{,} \PY{n}{np}\PY{o}{.}\PY{n}{var}\PY{p}{)}\PY{p}{]}
         \PY{n}{grouped}\PY{p}{[}\PY{l+s}{\PYZsq{}}\PY{l+s}{tip\PYZus{}pct}\PY{l+s}{\PYZsq{}}\PY{p}{,} \PY{l+s}{\PYZsq{}}\PY{l+s}{total\PYZus{}bill}\PY{l+s}{\PYZsq{}}\PY{p}{]}\PY{o}{.}\PY{n}{agg}\PY{p}{(}\PY{n}{ftuples}\PY{p}{)}
\end{Verbatim}

            \begin{Verbatim}[commandchars=\\\{\}]
{\color{outcolor}Out[{\color{outcolor}40}]:}                    tip\_pct              total\_bill           
                       Durchschnitt Abweichung Durchschnitt Abweichung
         sex    smoker                                                
         Female No         0.156921   0.001327    18.105185  53.092422
                Yes        0.182150   0.005126    17.977879  84.451517
         Male   No         0.160669   0.001751    19.791237  76.152961
                Yes        0.152771   0.008206    22.284500  98.244673
\end{Verbatim}
        
    You can also specify the particular column of a \texttt{DataFrame} that
you want to do aggregation on by passing a dict of information. For
example,

    \begin{Verbatim}[commandchars=\\\{\}]
{\color{incolor}In [{\color{incolor}41}]:} \PY{n}{grouped}\PY{o}{.}\PY{n}{agg}\PY{p}{(}\PY{p}{\PYZob{}}\PY{l+s}{\PYZsq{}}\PY{l+s}{tip}\PY{l+s}{\PYZsq{}} \PY{p}{:} \PY{n}{np}\PY{o}{.}\PY{n}{max}\PY{p}{,} \PY{l+s}{\PYZsq{}}\PY{l+s}{size}\PY{l+s}{\PYZsq{}} \PY{p}{:} \PY{l+s}{\PYZsq{}}\PY{l+s}{sum}\PY{l+s}{\PYZsq{}}\PY{p}{\PYZcb{}}\PY{p}{)}
\end{Verbatim}

            \begin{Verbatim}[commandchars=\\\{\}]
{\color{outcolor}Out[{\color{outcolor}41}]:}                 tip  size
         sex    smoker            
         Female No       5.2   140
                Yes      6.5    74
         Male   No       9.0   263
                Yes     10.0   150
\end{Verbatim}
        
    You can overload a particular column's aggregator functions by making
the dict key corresponding to the column reference a list rather than
just one function.

    \begin{Verbatim}[commandchars=\\\{\}]
{\color{incolor}In [{\color{incolor}42}]:} \PY{n}{grouped}\PY{o}{.}\PY{n}{agg}\PY{p}{(}\PY{p}{\PYZob{}}\PY{l+s}{\PYZsq{}}\PY{l+s}{tip\PYZus{}pct}\PY{l+s}{\PYZsq{}} \PY{p}{:} \PY{p}{[}\PY{l+s}{\PYZsq{}}\PY{l+s}{min}\PY{l+s}{\PYZsq{}}\PY{p}{,} \PY{l+s}{\PYZsq{}}\PY{l+s}{max}\PY{l+s}{\PYZsq{}}\PY{p}{,} \PY{l+s}{\PYZsq{}}\PY{l+s}{mean}\PY{l+s}{\PYZsq{}}\PY{p}{,} \PY{l+s}{\PYZsq{}}\PY{l+s}{std}\PY{l+s}{\PYZsq{}}\PY{p}{]}\PY{p}{,}
                      \PY{l+s}{\PYZsq{}}\PY{l+s}{size}\PY{l+s}{\PYZsq{}} \PY{p}{:} \PY{l+s}{\PYZsq{}}\PY{l+s}{sum}\PY{l+s}{\PYZsq{}}\PY{p}{\PYZcb{}}\PY{p}{)}
\end{Verbatim}

            \begin{Verbatim}[commandchars=\\\{\}]
{\color{outcolor}Out[{\color{outcolor}42}]:}                 tip\_pct                               size
                             min       max      mean       std  sum
         sex    smoker                                             
         Female No      0.056797  0.252672  0.156921  0.036421  140
                Yes     0.056433  0.416667  0.182150  0.071595   74
         Male   No      0.071804  0.291990  0.160669  0.041849  263
                Yes     0.035638  0.710345  0.152771  0.090588  150
\end{Verbatim}
        
    \subsubsection{Returning aggregated data in ``unindexed''
form}\label{returning-aggregated-data-in-unindexed-form}

    Of course, you can unindex a hierarchical indexed \texttt{GroupBy}
object by Specifying the \texttt{as\_index} optional paramter.

    \begin{Verbatim}[commandchars=\\\{\}]
{\color{incolor}In [{\color{incolor}43}]:} \PY{n}{tips}\PY{o}{.}\PY{n}{groupby}\PY{p}{(}\PY{p}{[}\PY{l+s}{\PYZsq{}}\PY{l+s}{sex}\PY{l+s}{\PYZsq{}}\PY{p}{,} \PY{l+s}{\PYZsq{}}\PY{l+s}{smoker}\PY{l+s}{\PYZsq{}}\PY{p}{]}\PY{p}{,} \PY{n}{as\PYZus{}index}\PY{o}{=}\PY{n+nb+bp}{False}\PY{p}{)}\PY{o}{.}\PY{n}{mean}\PY{p}{(}\PY{p}{)}
\end{Verbatim}

            \begin{Verbatim}[commandchars=\\\{\}]
{\color{outcolor}Out[{\color{outcolor}43}]:}       sex smoker  total\_bill       tip      size   tip\_pct
         0  Female     No   18.105185  2.773519  2.592593  0.156921
         1  Female    Yes   17.977879  2.931515  2.242424  0.182150
         2    Male     No   19.791237  3.113402  2.711340  0.160669
         3    Male    Yes   22.284500  3.051167  2.500000  0.152771
\end{Verbatim}
        
    \subsection{Group-wise operations and
transformations}\label{group-wise-operations-and-transformations}

    Aggregation is but one kind of group operation. It is a special case of
more general data transformations, taking one-dimensional arrays and
reducing them to scalars. Here we will generalize this notion by showing
you how to use \texttt{apply} and \texttt{transform} methods on
\texttt{DataFrame} objects. Let's revisit our old \texttt{DataFrame} of
random data.

    \begin{Verbatim}[commandchars=\\\{\}]
{\color{incolor}In [{\color{incolor}44}]:} \PY{n}{df}
\end{Verbatim}

            \begin{Verbatim}[commandchars=\\\{\}]
{\color{outcolor}Out[{\color{outcolor}44}]:}       data1     data2 key1 key2
         0 -0.204708  1.393406    a  one
         1  0.478943  0.092908    a  two
         2 -0.519439  0.281746    b  one
         3 -0.555730  0.769023    b  two
         4  1.965781  1.246435    a  one
\end{Verbatim}
        
    Suppose we want to add a column containing group means for each index.
One way to do this is to aggregate, then merge:

    \begin{Verbatim}[commandchars=\\\{\}]
{\color{incolor}In [{\color{incolor}45}]:} \PY{n}{k1\PYZus{}means} \PY{o}{=} \PY{n}{df}\PY{o}{.}\PY{n}{groupby}\PY{p}{(}\PY{l+s}{\PYZsq{}}\PY{l+s}{key1}\PY{l+s}{\PYZsq{}}\PY{p}{)}\PY{o}{.}\PY{n}{mean}\PY{p}{(}\PY{p}{)}\PY{o}{.}\PY{n}{add\PYZus{}prefix}\PY{p}{(}\PY{l+s}{\PYZsq{}}\PY{l+s}{mean\PYZus{}}\PY{l+s}{\PYZsq{}}\PY{p}{)}
         \PY{n}{k1\PYZus{}means}
\end{Verbatim}

            \begin{Verbatim}[commandchars=\\\{\}]
{\color{outcolor}Out[{\color{outcolor}45}]:}       mean\_data1  mean\_data2
         key1                        
         a       0.746672    0.910916
         b      -0.537585    0.525384
\end{Verbatim}
        
    \begin{Verbatim}[commandchars=\\\{\}]
{\color{incolor}In [{\color{incolor}46}]:} \PY{n}{pd}\PY{o}{.}\PY{n}{merge}\PY{p}{(}\PY{n}{df}\PY{p}{,} \PY{n}{k1\PYZus{}means}\PY{p}{,} \PY{n}{left\PYZus{}on}\PY{o}{=}\PY{l+s}{\PYZsq{}}\PY{l+s}{key1}\PY{l+s}{\PYZsq{}}\PY{p}{,} \PY{n}{right\PYZus{}index}\PY{o}{=}\PY{n+nb+bp}{True}\PY{p}{)}
\end{Verbatim}

            \begin{Verbatim}[commandchars=\\\{\}]
{\color{outcolor}Out[{\color{outcolor}46}]:}       data1     data2 key1 key2  mean\_data1  mean\_data2
         0 -0.204708  1.393406    a  one    0.746672    0.910916
         1  0.478943  0.092908    a  two    0.746672    0.910916
         4  1.965781  1.246435    a  one    0.746672    0.910916
         2 -0.519439  0.281746    b  one   -0.537585    0.525384
         3 -0.555730  0.769023    b  two   -0.537585    0.525384
\end{Verbatim}
        
    This works but is somewhat inflexible. You can think of the operation as
transforming the two data columns using the \texttt{np.mean} function.
Returning to the \texttt{people DataFrame} from before, we can use the
\texttt{transform} method on \texttt{GroupBy}.

    \begin{Verbatim}[commandchars=\\\{\}]
{\color{incolor}In [{\color{incolor}47}]:} \PY{n}{key} \PY{o}{=} \PY{p}{[}\PY{l+s}{\PYZsq{}}\PY{l+s}{one}\PY{l+s}{\PYZsq{}}\PY{p}{,} \PY{l+s}{\PYZsq{}}\PY{l+s}{two}\PY{l+s}{\PYZsq{}}\PY{p}{,} \PY{l+s}{\PYZsq{}}\PY{l+s}{one}\PY{l+s}{\PYZsq{}}\PY{p}{,} \PY{l+s}{\PYZsq{}}\PY{l+s}{two}\PY{l+s}{\PYZsq{}}\PY{p}{,} \PY{l+s}{\PYZsq{}}\PY{l+s}{one}\PY{l+s}{\PYZsq{}}\PY{p}{]}
         \PY{n}{people}\PY{o}{.}\PY{n}{groupby}\PY{p}{(}\PY{n}{key}\PY{p}{)}\PY{o}{.}\PY{n}{mean}\PY{p}{(}\PY{p}{)}
\end{Verbatim}

            \begin{Verbatim}[commandchars=\\\{\}]
{\color{outcolor}Out[{\color{outcolor}47}]:}             a         b         c         d         e
         one -0.082032 -1.063687 -1.047620 -0.884358 -0.028309
         two  0.505275 -0.849512  0.075965  0.834983  0.452620
\end{Verbatim}
        
    \begin{Verbatim}[commandchars=\\\{\}]
{\color{incolor}In [{\color{incolor}48}]:} \PY{n}{people}\PY{o}{.}\PY{n}{groupby}\PY{p}{(}\PY{n}{key}\PY{p}{)}\PY{o}{.}\PY{n}{transform}\PY{p}{(}\PY{n}{np}\PY{o}{.}\PY{n}{mean}\PY{p}{)}
\end{Verbatim}

            \begin{Verbatim}[commandchars=\\\{\}]
{\color{outcolor}Out[{\color{outcolor}48}]:}                a         b         c         d         e
         Joe    -0.082032 -1.063687 -1.047620 -0.884358 -0.028309
         Steve   0.505275 -0.849512  0.075965  0.834983  0.452620
         Wes    -0.082032 -1.063687 -1.047620 -0.884358 -0.028309
         Jim     0.505275 -0.849512  0.075965  0.834983  0.452620
         Travis -0.082032 -1.063687 -1.047620 -0.884358 -0.028309
\end{Verbatim}
        
    What is going on here is that \texttt{transform} applies a function to
each group, then places the results in the appropriate locations. When
this reduces to the special case of scalar values, the answer is simply
broadcasted across all the relevant locations.

    Suppose instead you wanted to subtract the mean value from each group.
To do so, let's define a function \texttt{demean}, and proceed by

    \begin{Verbatim}[commandchars=\\\{\}]
{\color{incolor}In [{\color{incolor}49}]:} \PY{k}{def} \PY{n+nf}{demean}\PY{p}{(}\PY{n}{arr}\PY{p}{)}\PY{p}{:}
             \PY{k}{return} \PY{n}{arr} \PY{o}{\PYZhy{}} \PY{n}{arr}\PY{o}{.}\PY{n}{mean}\PY{p}{(}\PY{p}{)}
         \PY{n}{demeaned} \PY{o}{=} \PY{n}{people}\PY{o}{.}\PY{n}{groupby}\PY{p}{(}\PY{n}{key}\PY{p}{)}\PY{o}{.}\PY{n}{transform}\PY{p}{(}\PY{n}{demean}\PY{p}{)}
         \PY{n}{demeaned}
\end{Verbatim}

            \begin{Verbatim}[commandchars=\\\{\}]
{\color{outcolor}Out[{\color{outcolor}49}]:}                a         b         c         d         e
         Joe     1.089221 -0.232534  1.322612  1.113271  1.381226
         Steve   0.381154 -1.152125 -0.447807  0.834043 -0.891190
         Wes    -0.457709       NaN       NaN -0.136869 -0.548778
         Jim    -0.381154  1.152125  0.447807 -0.834043  0.891190
         Travis -0.631512  0.232534 -1.322612 -0.976402 -0.832448
\end{Verbatim}
        
    You can check that \texttt{demeanded} now has zero group means:

    \begin{Verbatim}[commandchars=\\\{\}]
{\color{incolor}In [{\color{incolor}50}]:} \PY{n}{demeaned}\PY{o}{.}\PY{n}{groupby}\PY{p}{(}\PY{n}{key}\PY{p}{)}\PY{o}{.}\PY{n}{mean}\PY{p}{(}\PY{p}{)}
\end{Verbatim}

            \begin{Verbatim}[commandchars=\\\{\}]
{\color{outcolor}Out[{\color{outcolor}50}]:}                 a             b  c             d  e
         one  0.000000e+00 -1.110223e-16  0  7.401487e-17  0
         two -2.775558e-17  0.000000e+00  0  0.000000e+00  0
\end{Verbatim}
        
    We will soon see that demeaning can be achieved using \texttt{apply} as
well.

    \subsubsection{Apply: General
split-apply-combine}\label{apply-general-split-apply-combine}

    There are three data transformation tools that we can use to build
analyses on our data. The first two, \texttt{aggregate} and
\texttt{transform}, are somewhat rigid in their capabilities. On the
flip side, this makes it easier on you the data analyst to perform data
transformations. The third tool is \texttt{apply}, which gives you
immense flexibility at the expense of intuitivity.

    Returning to the \texttt{tips.csv} data set, suppose we want to select
the top five \texttt{tip\_pct} values by group. We can write a function
to identify the top values of a \texttt{DataFrame} very easily:

    \begin{Verbatim}[commandchars=\\\{\}]
{\color{incolor}In [{\color{incolor}51}]:} \PY{k}{def} \PY{n+nf}{top}\PY{p}{(}\PY{n}{df}\PY{p}{,} \PY{n}{n}\PY{o}{=}\PY{l+m+mi}{5}\PY{p}{,} \PY{n}{column}\PY{o}{=}\PY{l+s}{\PYZsq{}}\PY{l+s}{tip\PYZus{}pct}\PY{l+s}{\PYZsq{}}\PY{p}{)}\PY{p}{:}
             \PY{k}{return} \PY{n}{df}\PY{o}{.}\PY{n}{sort\PYZus{}index}\PY{p}{(}\PY{n}{by}\PY{o}{=}\PY{n}{column}\PY{p}{)}\PY{p}{[}\PY{o}{\PYZhy{}}\PY{n}{n}\PY{p}{:}\PY{p}{]}
         \PY{n}{top}\PY{p}{(}\PY{n}{tips}\PY{p}{,} \PY{n}{n}\PY{o}{=}\PY{l+m+mi}{6}\PY{p}{)}
\end{Verbatim}

            \begin{Verbatim}[commandchars=\\\{\}]
{\color{outcolor}Out[{\color{outcolor}51}]:}      total\_bill   tip     sex smoker  day    time  size   tip\_pct
         109       14.31  4.00  Female    Yes  Sat  Dinner     2  0.279525
         183       23.17  6.50    Male    Yes  Sun  Dinner     4  0.280535
         232       11.61  3.39    Male     No  Sat  Dinner     2  0.291990
         67         3.07  1.00  Female    Yes  Sat  Dinner     1  0.325733
         178        9.60  4.00  Female    Yes  Sun  Dinner     2  0.416667
         172        7.25  5.15    Male    Yes  Sun  Dinner     2  0.710345
\end{Verbatim}
        
    Now if we group by \texttt{smoker}, say, and call \texttt{apply} with
this function, we get

    \begin{Verbatim}[commandchars=\\\{\}]
{\color{incolor}In [{\color{incolor}52}]:} \PY{n}{tips}\PY{o}{.}\PY{n}{groupby}\PY{p}{(}\PY{l+s}{\PYZsq{}}\PY{l+s}{smoker}\PY{l+s}{\PYZsq{}}\PY{p}{)}\PY{o}{.}\PY{n}{apply}\PY{p}{(}\PY{n}{top}\PY{p}{)}
\end{Verbatim}

            \begin{Verbatim}[commandchars=\\\{\}]
{\color{outcolor}Out[{\color{outcolor}52}]:}             total\_bill   tip     sex smoker   day    time  size   tip\_pct
         smoker                                                                   
         No     88        24.71  5.85    Male     No  Thur   Lunch     2  0.236746
                185       20.69  5.00    Male     No   Sun  Dinner     5  0.241663
                51        10.29  2.60  Female     No   Sun  Dinner     2  0.252672
                149        7.51  2.00    Male     No  Thur   Lunch     2  0.266312
                232       11.61  3.39    Male     No   Sat  Dinner     2  0.291990
         Yes    109       14.31  4.00  Female    Yes   Sat  Dinner     2  0.279525
                183       23.17  6.50    Male    Yes   Sun  Dinner     4  0.280535
                67         3.07  1.00  Female    Yes   Sat  Dinner     1  0.325733
                178        9.60  4.00  Female    Yes   Sun  Dinner     2  0.416667
                172        7.25  5.15    Male    Yes   Sun  Dinner     2  0.710345
\end{Verbatim}
        
    \texttt{top} is called on each piece of the \texttt{DataFrame}, then the
results are glued together using \texttt{pandas.concat}, labeling the
pieces with the group names.

If you pass a function to \texttt{apply} that takes other arguments or
keywords, you can pass these after the function:

    \begin{Verbatim}[commandchars=\\\{\}]
{\color{incolor}In [{\color{incolor}53}]:} \PY{n}{tips}\PY{o}{.}\PY{n}{groupby}\PY{p}{(}\PY{p}{[}\PY{l+s}{\PYZsq{}}\PY{l+s}{smoker}\PY{l+s}{\PYZsq{}}\PY{p}{,} \PY{l+s}{\PYZsq{}}\PY{l+s}{day}\PY{l+s}{\PYZsq{}}\PY{p}{]}\PY{p}{)}\PY{o}{.}\PY{n}{apply}\PY{p}{(}\PY{n}{top}\PY{p}{,} \PY{n}{n}\PY{o}{=}\PY{l+m+mi}{1}\PY{p}{,} \PY{n}{column}\PY{o}{=}\PY{l+s}{\PYZsq{}}\PY{l+s}{total\PYZus{}bill}\PY{l+s}{\PYZsq{}}\PY{p}{)}
\end{Verbatim}

            \begin{Verbatim}[commandchars=\\\{\}]
{\color{outcolor}Out[{\color{outcolor}53}]:}                  total\_bill    tip     sex smoker   day    time  size  \textbackslash{}
         smoker day                                                              
         No     Fri  94        22.75   3.25  Female     No   Fri  Dinner     2   
                Sat  212       48.33   9.00    Male     No   Sat  Dinner     4   
                Sun  156       48.17   5.00    Male     No   Sun  Dinner     6   
                Thur 142       41.19   5.00    Male     No  Thur   Lunch     5   
         Yes    Fri  95        40.17   4.73    Male    Yes   Fri  Dinner     4   
                Sat  170       50.81  10.00    Male    Yes   Sat  Dinner     3   
                Sun  182       45.35   3.50    Male    Yes   Sun  Dinner     3   
                Thur 197       43.11   5.00  Female    Yes  Thur   Lunch     4   
         
                           tip\_pct  
         smoker day                 
         No     Fri  94   0.142857  
                Sat  212  0.186220  
                Sun  156  0.103799  
                Thur 142  0.121389  
         Yes    Fri  95   0.117750  
                Sat  170  0.196812  
                Sun  182  0.077178  
                Thur 197  0.115982  
\end{Verbatim}
        
    Recall that \texttt{describe} seems to work okay on a \texttt{GroupBy}
object.

    \begin{Verbatim}[commandchars=\\\{\}]
{\color{incolor}In [{\color{incolor}54}]:} \PY{n}{result} \PY{o}{=} \PY{n}{tips}\PY{o}{.}\PY{n}{groupby}\PY{p}{(}\PY{l+s}{\PYZsq{}}\PY{l+s}{smoker}\PY{l+s}{\PYZsq{}}\PY{p}{)}\PY{p}{[}\PY{l+s}{\PYZsq{}}\PY{l+s}{tip\PYZus{}pct}\PY{l+s}{\PYZsq{}}\PY{p}{]}\PY{o}{.}\PY{n}{describe}\PY{p}{(}\PY{p}{)}
         \PY{n}{result}
\end{Verbatim}

            \begin{Verbatim}[commandchars=\\\{\}]
{\color{outcolor}Out[{\color{outcolor}54}]:} smoker       
         No      count    151.000000
                 mean       0.159328
                 std        0.039910
                 min        0.056797
                 25\%        0.136906
                 50\%        0.155625
                 75\%        0.185014
                 max        0.291990
         Yes     count     93.000000
                 mean       0.163196
                 std        0.085119
                 min        0.035638
                 25\%        0.106771
                 50\%        0.153846
                 75\%        0.195059
                 max        0.710345
         dtype: float64
\end{Verbatim}
        
    \begin{Verbatim}[commandchars=\\\{\}]
{\color{incolor}In [{\color{incolor}55}]:} \PY{n}{result}\PY{o}{.}\PY{n}{unstack}\PY{p}{(}\PY{l+s}{\PYZsq{}}\PY{l+s}{smoker}\PY{l+s}{\PYZsq{}}\PY{p}{)}
\end{Verbatim}

            \begin{Verbatim}[commandchars=\\\{\}]
{\color{outcolor}Out[{\color{outcolor}55}]:} smoker          No        Yes
         count   151.000000  93.000000
         mean      0.159328   0.163196
         std       0.039910   0.085119
         min       0.056797   0.035638
         25\%       0.136906   0.106771
         50\%       0.155625   0.153846
         75\%       0.185014   0.195059
         max       0.291990   0.710345
\end{Verbatim}
        
    What's really happening (for all you 151ers) is that when you invoke a
method like \texttt{describe}, it is actually just a shortcut for:

\begin{Shaded}
\begin{Highlighting}[]
\NormalTok{f = }\KeywordTok{lambda} \NormalTok{x: x.describe()}
\NormalTok{grouped.}\DataTypeTok{apply}\NormalTok{(f)}
\end{Highlighting}
\end{Shaded}

    \paragraph{Suppressing the group keys}\label{suppressing-the-group-keys}

    One point of style: if you prefer not working with hierarchical indices,
you can specify in the \texttt{groupby} call to treat the underlying
\texttt{DataFrame} as flat by choosing \texttt{group\_keys=False}.

    \begin{Verbatim}[commandchars=\\\{\}]
{\color{incolor}In [{\color{incolor}56}]:} \PY{n}{tips}\PY{o}{.}\PY{n}{groupby}\PY{p}{(}\PY{l+s}{\PYZsq{}}\PY{l+s}{smoker}\PY{l+s}{\PYZsq{}}\PY{p}{,} \PY{n}{group\PYZus{}keys}\PY{o}{=}\PY{n+nb+bp}{False}\PY{p}{)}\PY{o}{.}\PY{n}{apply}\PY{p}{(}\PY{n}{top}\PY{p}{)}
\end{Verbatim}

            \begin{Verbatim}[commandchars=\\\{\}]
{\color{outcolor}Out[{\color{outcolor}56}]:}      total\_bill   tip     sex smoker   day    time  size   tip\_pct
         88        24.71  5.85    Male     No  Thur   Lunch     2  0.236746
         185       20.69  5.00    Male     No   Sun  Dinner     5  0.241663
         51        10.29  2.60  Female     No   Sun  Dinner     2  0.252672
         149        7.51  2.00    Male     No  Thur   Lunch     2  0.266312
         232       11.61  3.39    Male     No   Sat  Dinner     2  0.291990
         109       14.31  4.00  Female    Yes   Sat  Dinner     2  0.279525
         183       23.17  6.50    Male    Yes   Sun  Dinner     4  0.280535
         67         3.07  1.00  Female    Yes   Sat  Dinner     1  0.325733
         178        9.60  4.00  Female    Yes   Sun  Dinner     2  0.416667
         172        7.25  5.15    Male    Yes   Sun  Dinner     2  0.710345
\end{Verbatim}
        
    \subsubsection{Example: Filling missing values with group-specific
values}\label{example-filling-missing-values-with-group-specific-values}

    When cleaning up missing data, in some cases you will filter out data
observations using \texttt{dropna}, but in others you may want to impute
(fill in) the NA values using a fixed value or some value derived from
the data. \texttt{fillna} is the right tool to use; for example, we can
fill in NA values with the mean:

    \begin{Verbatim}[commandchars=\\\{\}]
{\color{incolor}In [{\color{incolor}57}]:} \PY{n}{s} \PY{o}{=} \PY{n}{Series}\PY{p}{(}\PY{n}{np}\PY{o}{.}\PY{n}{random}\PY{o}{.}\PY{n}{randn}\PY{p}{(}\PY{l+m+mi}{6}\PY{p}{)}\PY{p}{)}
         \PY{n}{s}\PY{p}{[}\PY{p}{:}\PY{p}{:}\PY{l+m+mi}{2}\PY{p}{]} \PY{o}{=} \PY{n}{np}\PY{o}{.}\PY{n}{nan}
         \PY{n}{s}
\end{Verbatim}

            \begin{Verbatim}[commandchars=\\\{\}]
{\color{outcolor}Out[{\color{outcolor}57}]:} 0         NaN
         1   -1.549106
         2         NaN
         3    0.758363
         4         NaN
         5    0.862580
         dtype: float64
\end{Verbatim}
        
    \begin{Verbatim}[commandchars=\\\{\}]
{\color{incolor}In [{\color{incolor}58}]:} \PY{n}{s}\PY{o}{.}\PY{n}{fillna}\PY{p}{(}\PY{n}{s}\PY{o}{.}\PY{n}{mean}\PY{p}{(}\PY{p}{)}\PY{p}{)}
\end{Verbatim}

            \begin{Verbatim}[commandchars=\\\{\}]
{\color{outcolor}Out[{\color{outcolor}58}]:} 0    0.023946
         1   -1.549106
         2    0.023946
         3    0.758363
         4    0.023946
         5    0.862580
         dtype: float64
\end{Verbatim}
        
    Suppose you need the fill value to vary by group. As you may guess, you
need only group the data and use \texttt{apply} with a function that
calls \texttt{fillna} on each data chunk. Here is some sample data on
some US states divided into eastern and western states.

    \begin{Verbatim}[commandchars=\\\{\}]
{\color{incolor}In [{\color{incolor}59}]:} \PY{n}{states} \PY{o}{=} \PY{p}{[}\PY{l+s}{\PYZsq{}}\PY{l+s}{Ohio}\PY{l+s}{\PYZsq{}}\PY{p}{,} \PY{l+s}{\PYZsq{}}\PY{l+s}{New York}\PY{l+s}{\PYZsq{}}\PY{p}{,} \PY{l+s}{\PYZsq{}}\PY{l+s}{Vermont}\PY{l+s}{\PYZsq{}}\PY{p}{,} \PY{l+s}{\PYZsq{}}\PY{l+s}{Florida}\PY{l+s}{\PYZsq{}}\PY{p}{,}
                   \PY{l+s}{\PYZsq{}}\PY{l+s}{Oregon}\PY{l+s}{\PYZsq{}}\PY{p}{,} \PY{l+s}{\PYZsq{}}\PY{l+s}{Nevada}\PY{l+s}{\PYZsq{}}\PY{p}{,} \PY{l+s}{\PYZsq{}}\PY{l+s}{California}\PY{l+s}{\PYZsq{}}\PY{p}{,} \PY{l+s}{\PYZsq{}}\PY{l+s}{Idaho}\PY{l+s}{\PYZsq{}}\PY{p}{]}
         \PY{n}{group\PYZus{}key} \PY{o}{=} \PY{p}{[}\PY{l+s}{\PYZsq{}}\PY{l+s}{East}\PY{l+s}{\PYZsq{}}\PY{p}{]} \PY{o}{*} \PY{l+m+mi}{4} \PY{o}{+} \PY{p}{[}\PY{l+s}{\PYZsq{}}\PY{l+s}{West}\PY{l+s}{\PYZsq{}}\PY{p}{]} \PY{o}{*} \PY{l+m+mi}{4}
         \PY{n}{data} \PY{o}{=} \PY{n}{Series}\PY{p}{(}\PY{n}{np}\PY{o}{.}\PY{n}{random}\PY{o}{.}\PY{n}{randn}\PY{p}{(}\PY{l+m+mi}{8}\PY{p}{)}\PY{p}{,} \PY{n}{index}\PY{o}{=}\PY{n}{states}\PY{p}{)}
         \PY{n}{data}\PY{p}{[}\PY{p}{[}\PY{l+s}{\PYZsq{}}\PY{l+s}{Vermont}\PY{l+s}{\PYZsq{}}\PY{p}{,} \PY{l+s}{\PYZsq{}}\PY{l+s}{Nevada}\PY{l+s}{\PYZsq{}}\PY{p}{,} \PY{l+s}{\PYZsq{}}\PY{l+s}{Idaho}\PY{l+s}{\PYZsq{}}\PY{p}{]}\PY{p}{]} \PY{o}{=} \PY{n}{np}\PY{o}{.}\PY{n}{nan}
         \PY{n}{data}
\end{Verbatim}

            \begin{Verbatim}[commandchars=\\\{\}]
{\color{outcolor}Out[{\color{outcolor}59}]:} Ohio         -0.010032
         New York      0.050009
         Vermont            NaN
         Florida       0.852965
         Oregon       -0.955869
         Nevada             NaN
         California   -2.304234
         Idaho              NaN
         dtype: float64
\end{Verbatim}
        
    \begin{Verbatim}[commandchars=\\\{\}]
{\color{incolor}In [{\color{incolor}60}]:} \PY{n}{data}\PY{o}{.}\PY{n}{groupby}\PY{p}{(}\PY{n}{group\PYZus{}key}\PY{p}{)}\PY{o}{.}\PY{n}{mean}\PY{p}{(}\PY{p}{)}
\end{Verbatim}

            \begin{Verbatim}[commandchars=\\\{\}]
{\color{outcolor}Out[{\color{outcolor}60}]:} East    0.297647
         West   -1.630051
         dtype: float64
\end{Verbatim}
        
    We can fill the NA values using the group means:

    \begin{Verbatim}[commandchars=\\\{\}]
{\color{incolor}In [{\color{incolor}61}]:} \PY{n}{fill\PYZus{}mean} \PY{o}{=} \PY{k}{lambda} \PY{n}{g}\PY{p}{:} \PY{n}{g}\PY{o}{.}\PY{n}{fillna}\PY{p}{(}\PY{n}{g}\PY{o}{.}\PY{n}{mean}\PY{p}{(}\PY{p}{)}\PY{p}{)}
         \PY{n}{data}\PY{o}{.}\PY{n}{groupby}\PY{p}{(}\PY{n}{group\PYZus{}key}\PY{p}{)}\PY{o}{.}\PY{n}{apply}\PY{p}{(}\PY{n}{fill\PYZus{}mean}\PY{p}{)}
\end{Verbatim}

            \begin{Verbatim}[commandchars=\\\{\}]
{\color{outcolor}Out[{\color{outcolor}61}]:} Ohio         -0.010032
         New York      0.050009
         Vermont       0.297647
         Florida       0.852965
         Oregon       -0.955869
         Nevada       -1.630051
         California   -2.304234
         Idaho        -1.630051
         dtype: float64
\end{Verbatim}
        
    In another case, you might have pre-defined fill values in your code
that vary by group. Since the groups have a \texttt{name} attribute set,
internally, we can use that:

    \begin{Verbatim}[commandchars=\\\{\}]
{\color{incolor}In [{\color{incolor}62}]:} \PY{n}{fill\PYZus{}values} \PY{o}{=} \PY{p}{\PYZob{}}\PY{l+s}{\PYZsq{}}\PY{l+s}{East}\PY{l+s}{\PYZsq{}}\PY{p}{:} \PY{l+m+mf}{0.5}\PY{p}{,} \PY{l+s}{\PYZsq{}}\PY{l+s}{West}\PY{l+s}{\PYZsq{}}\PY{p}{:} \PY{o}{\PYZhy{}}\PY{l+m+mi}{1}\PY{p}{\PYZcb{}}
         \PY{n}{fill\PYZus{}func} \PY{o}{=} \PY{k}{lambda} \PY{n}{g}\PY{p}{:} \PY{n}{g}\PY{o}{.}\PY{n}{fillna}\PY{p}{(}\PY{n}{fill\PYZus{}values}\PY{p}{[}\PY{n}{g}\PY{o}{.}\PY{n}{name}\PY{p}{]}\PY{p}{)}
         
         \PY{n}{data}\PY{o}{.}\PY{n}{groupby}\PY{p}{(}\PY{n}{group\PYZus{}key}\PY{p}{)}\PY{o}{.}\PY{n}{apply}\PY{p}{(}\PY{n}{fill\PYZus{}func}\PY{p}{)}
\end{Verbatim}

            \begin{Verbatim}[commandchars=\\\{\}]
{\color{outcolor}Out[{\color{outcolor}62}]:} Ohio         -0.010032
         New York      0.050009
         Vermont       0.500000
         Florida       0.852965
         Oregon       -0.955869
         Nevada       -1.000000
         California   -2.304234
         Idaho        -1.000000
         dtype: float64
\end{Verbatim}
        
    \subsubsection{Example: Random sampling and
permutation}\label{example-random-sampling-and-permutation}

    Suppose you wanted to draw a random sample from a large dataset for
Monte Carlo simulation purposes or some other application. There are a
number of ways to perform the ``draws''; some are much more efficient
than others. One way is to select the first \texttt{K} elements of
\texttt{np.random.permutation(N)}, where \texttt{N} is the size of your
complete dataset and \texttt{K} the desired sample size. As a more fun
example, here's a way to construct a deck of English-style playing
cards:

    \begin{Verbatim}[commandchars=\\\{\}]
{\color{incolor}In [{\color{incolor}63}]:} \PY{c}{\PYZsh{} Hearts, Spades, Clubs, Diamonds}
         \PY{n}{suits} \PY{o}{=} \PY{p}{[}\PY{l+s}{\PYZsq{}}\PY{l+s}{H}\PY{l+s}{\PYZsq{}}\PY{p}{,} \PY{l+s}{\PYZsq{}}\PY{l+s}{S}\PY{l+s}{\PYZsq{}}\PY{p}{,} \PY{l+s}{\PYZsq{}}\PY{l+s}{C}\PY{l+s}{\PYZsq{}}\PY{p}{,} \PY{l+s}{\PYZsq{}}\PY{l+s}{D}\PY{l+s}{\PYZsq{}}\PY{p}{]}
         \PY{n}{card\PYZus{}val} \PY{o}{=} \PY{p}{(}\PY{n+nb}{range}\PY{p}{(}\PY{l+m+mi}{1}\PY{p}{,} \PY{l+m+mi}{11}\PY{p}{)} \PY{o}{+} \PY{p}{[}\PY{l+m+mi}{10}\PY{p}{]} \PY{o}{*} \PY{l+m+mi}{3}\PY{p}{)} \PY{o}{*} \PY{l+m+mi}{4}
         \PY{n}{base\PYZus{}names} \PY{o}{=} \PY{p}{[}\PY{l+s}{\PYZsq{}}\PY{l+s}{A}\PY{l+s}{\PYZsq{}}\PY{p}{]} \PY{o}{+} \PY{n+nb}{range}\PY{p}{(}\PY{l+m+mi}{2}\PY{p}{,} \PY{l+m+mi}{11}\PY{p}{)} \PY{o}{+} \PY{p}{[}\PY{l+s}{\PYZsq{}}\PY{l+s}{J}\PY{l+s}{\PYZsq{}}\PY{p}{,} \PY{l+s}{\PYZsq{}}\PY{l+s}{K}\PY{l+s}{\PYZsq{}}\PY{p}{,} \PY{l+s}{\PYZsq{}}\PY{l+s}{Q}\PY{l+s}{\PYZsq{}}\PY{p}{]}
         \PY{n}{cards} \PY{o}{=} \PY{p}{[}\PY{p}{]}
         \PY{k}{for} \PY{n}{suit} \PY{o+ow}{in} \PY{p}{[}\PY{l+s}{\PYZsq{}}\PY{l+s}{H}\PY{l+s}{\PYZsq{}}\PY{p}{,} \PY{l+s}{\PYZsq{}}\PY{l+s}{S}\PY{l+s}{\PYZsq{}}\PY{p}{,} \PY{l+s}{\PYZsq{}}\PY{l+s}{C}\PY{l+s}{\PYZsq{}}\PY{p}{,} \PY{l+s}{\PYZsq{}}\PY{l+s}{D}\PY{l+s}{\PYZsq{}}\PY{p}{]}\PY{p}{:}
             \PY{n}{cards}\PY{o}{.}\PY{n}{extend}\PY{p}{(}\PY{n+nb}{str}\PY{p}{(}\PY{n}{num}\PY{p}{)} \PY{o}{+} \PY{n}{suit} \PY{k}{for} \PY{n}{num} \PY{o+ow}{in} \PY{n}{base\PYZus{}names}\PY{p}{)}
         
         \PY{n}{deck} \PY{o}{=} \PY{n}{Series}\PY{p}{(}\PY{n}{card\PYZus{}val}\PY{p}{,} \PY{n}{index}\PY{o}{=}\PY{n}{cards}\PY{p}{)}
\end{Verbatim}

    So now we have a \texttt{Series} of length 52 whose index contains card
names and values are the ones used in blackjack and other games (to keep
things simple, I just let the ace be 1).

    \begin{Verbatim}[commandchars=\\\{\}]
{\color{incolor}In [{\color{incolor}64}]:} \PY{n}{deck}\PY{p}{[}\PY{p}{:}\PY{l+m+mi}{13}\PY{p}{]}
\end{Verbatim}

            \begin{Verbatim}[commandchars=\\\{\}]
{\color{outcolor}Out[{\color{outcolor}64}]:} AH      1
         2H      2
         3H      3
         4H      4
         5H      5
         6H      6
         7H      7
         8H      8
         9H      9
         10H    10
         JH     10
         KH     10
         QH     10
         dtype: int64
\end{Verbatim}
        
    Now, based on what we've just discussed, drawing a hand of five cards
from the desk could be written as:

    \begin{Verbatim}[commandchars=\\\{\}]
{\color{incolor}In [{\color{incolor}65}]:} \PY{k}{def} \PY{n+nf}{draw}\PY{p}{(}\PY{n}{deck}\PY{p}{,} \PY{n}{n}\PY{o}{=}\PY{l+m+mi}{5}\PY{p}{)}\PY{p}{:}
             \PY{k}{return} \PY{n}{deck}\PY{o}{.}\PY{n}{take}\PY{p}{(}\PY{n}{np}\PY{o}{.}\PY{n}{random}\PY{o}{.}\PY{n}{permutation}\PY{p}{(}\PY{n+nb}{len}\PY{p}{(}\PY{n}{deck}\PY{p}{)}\PY{p}{)}\PY{p}{[}\PY{p}{:}\PY{n}{n}\PY{p}{]}\PY{p}{)}
         \PY{n}{draw}\PY{p}{(}\PY{n}{deck}\PY{p}{)}
\end{Verbatim}

            \begin{Verbatim}[commandchars=\\\{\}]
{\color{outcolor}Out[{\color{outcolor}65}]:} QC    10
         4H     4
         KC    10
         5H     5
         8C     8
         dtype: int64
\end{Verbatim}
        
    Suppose you wnted two random cards from each suit. Because the suit is
the last character of each card name, we can group based on this and use
\texttt{apply}:

    \begin{Verbatim}[commandchars=\\\{\}]
{\color{incolor}In [{\color{incolor}66}]:} \PY{n}{get\PYZus{}suit} \PY{o}{=} \PY{k}{lambda} \PY{n}{card}\PY{p}{:} \PY{n}{card}\PY{p}{[}\PY{o}{\PYZhy{}}\PY{l+m+mi}{1}\PY{p}{]} \PY{c}{\PYZsh{} last letter is suit}
         \PY{n}{deck}\PY{o}{.}\PY{n}{groupby}\PY{p}{(}\PY{n}{get\PYZus{}suit}\PY{p}{)}\PY{o}{.}\PY{n}{apply}\PY{p}{(}\PY{n}{draw}\PY{p}{,} \PY{n}{n}\PY{o}{=}\PY{l+m+mi}{2}\PY{p}{)}
\end{Verbatim}

            \begin{Verbatim}[commandchars=\\\{\}]
{\color{outcolor}Out[{\color{outcolor}66}]:} C  8C      8
            9C      9
         D  KD     10
            6D      6
         H  10H    10
            7H      7
         S  3S      3
            4S      4
         dtype: int64
\end{Verbatim}
        
    \begin{Verbatim}[commandchars=\\\{\}]
{\color{incolor}In [{\color{incolor}67}]:} \PY{c}{\PYZsh{} alternatively}
         \PY{n}{deck}\PY{o}{.}\PY{n}{groupby}\PY{p}{(}\PY{n}{get\PYZus{}suit}\PY{p}{,} \PY{n}{group\PYZus{}keys}\PY{o}{=}\PY{n+nb+bp}{False}\PY{p}{)}\PY{o}{.}\PY{n}{apply}\PY{p}{(}\PY{n}{draw}\PY{p}{,} \PY{n}{n}\PY{o}{=}\PY{l+m+mi}{2}\PY{p}{)}
\end{Verbatim}

            \begin{Verbatim}[commandchars=\\\{\}]
{\color{outcolor}Out[{\color{outcolor}67}]:} 4C      4
         10C    10
         AD      1
         4D      4
         JH     10
         7H      7
         KS     10
         AS      1
         dtype: int64
\end{Verbatim}
        
    \subsubsection{Example: Group weighted average and
correlation}\label{example-group-weighted-average-and-correlation}

    Under the split-apply-combine paradigm of \texttt{groupby} operations
between columns in a \texttt{DataFrame} or two \texttt{Series}, such as
group weighted average, become a routine affair. As an example, take
this dataset containing group keys, values, and some weights:

    \begin{Verbatim}[commandchars=\\\{\}]
{\color{incolor}In [{\color{incolor}68}]:} \PY{n}{df} \PY{o}{=} \PY{n}{DataFrame}\PY{p}{(}\PY{p}{\PYZob{}}\PY{l+s}{\PYZsq{}}\PY{l+s}{category}\PY{l+s}{\PYZsq{}}\PY{p}{:} \PY{p}{[}\PY{l+s}{\PYZsq{}}\PY{l+s}{a}\PY{l+s}{\PYZsq{}}\PY{p}{,} \PY{l+s}{\PYZsq{}}\PY{l+s}{a}\PY{l+s}{\PYZsq{}}\PY{p}{,} \PY{l+s}{\PYZsq{}}\PY{l+s}{a}\PY{l+s}{\PYZsq{}}\PY{p}{,} \PY{l+s}{\PYZsq{}}\PY{l+s}{a}\PY{l+s}{\PYZsq{}}\PY{p}{,} \PY{l+s}{\PYZsq{}}\PY{l+s}{b}\PY{l+s}{\PYZsq{}}\PY{p}{,} \PY{l+s}{\PYZsq{}}\PY{l+s}{b}\PY{l+s}{\PYZsq{}}\PY{p}{,} \PY{l+s}{\PYZsq{}}\PY{l+s}{b}\PY{l+s}{\PYZsq{}}\PY{p}{,} \PY{l+s}{\PYZsq{}}\PY{l+s}{b}\PY{l+s}{\PYZsq{}}\PY{p}{]}\PY{p}{,}
                         \PY{l+s}{\PYZsq{}}\PY{l+s}{data}\PY{l+s}{\PYZsq{}}\PY{p}{:} \PY{n}{np}\PY{o}{.}\PY{n}{random}\PY{o}{.}\PY{n}{randn}\PY{p}{(}\PY{l+m+mi}{8}\PY{p}{)}\PY{p}{,}
                         \PY{l+s}{\PYZsq{}}\PY{l+s}{weights}\PY{l+s}{\PYZsq{}}\PY{p}{:} \PY{n}{np}\PY{o}{.}\PY{n}{random}\PY{o}{.}\PY{n}{rand}\PY{p}{(}\PY{l+m+mi}{8}\PY{p}{)}\PY{p}{\PYZcb{}}\PY{p}{)}
         \PY{n}{df}
\end{Verbatim}

            \begin{Verbatim}[commandchars=\\\{\}]
{\color{outcolor}Out[{\color{outcolor}68}]:}   category      data   weights
         0        a -1.218302  0.832619
         1        a -0.733233  0.371132
         2        a -0.851944  0.040553
         3        a -1.623093  0.554671
         4        b -0.279937  0.451246
         5        b  1.155034  0.725301
         6        b  0.227328  0.378451
         7        b -1.095511  0.840662
\end{Verbatim}
        
    The group weighted average by category would then be:

    \begin{Verbatim}[commandchars=\\\{\}]
{\color{incolor}In [{\color{incolor}69}]:} \PY{n}{grouped} \PY{o}{=} \PY{n}{df}\PY{o}{.}\PY{n}{groupby}\PY{p}{(}\PY{l+s}{\PYZsq{}}\PY{l+s}{category}\PY{l+s}{\PYZsq{}}\PY{p}{)}
         \PY{n}{get\PYZus{}wavg} \PY{o}{=} \PY{k}{lambda} \PY{n}{g}\PY{p}{:} \PY{n}{np}\PY{o}{.}\PY{n}{average}\PY{p}{(}\PY{n}{g}\PY{p}{[}\PY{l+s}{\PYZsq{}}\PY{l+s}{data}\PY{l+s}{\PYZsq{}}\PY{p}{]}\PY{p}{,} \PY{n}{weights}\PY{o}{=}\PY{n}{g}\PY{p}{[}\PY{l+s}{\PYZsq{}}\PY{l+s}{weights}\PY{l+s}{\PYZsq{}}\PY{p}{]}\PY{p}{)}
         \PY{n}{grouped}\PY{o}{.}\PY{n}{apply}\PY{p}{(}\PY{n}{get\PYZus{}wavg}\PY{p}{)}
\end{Verbatim}

            \begin{Verbatim}[commandchars=\\\{\}]
{\color{outcolor}Out[{\color{outcolor}69}]:} category
         a          -1.23478
         b          -0.05155
         dtype: float64
\end{Verbatim}
        
    As a less trivial example, consider a data set from Yahoo! Finance
containing end of day prices for a few stocks and the S\&P 500 index
(the \texttt{SPX} ticker):

    \begin{Verbatim}[commandchars=\\\{\}]
{\color{incolor}In [{\color{incolor}70}]:} \PY{n}{close\PYZus{}px} \PY{o}{=} \PY{n}{pd}\PY{o}{.}\PY{n}{read\PYZus{}csv}\PY{p}{(}\PY{l+s}{\PYZsq{}}\PY{l+s}{stock\PYZus{}px.csv}\PY{l+s}{\PYZsq{}}\PY{p}{,} \PY{n}{parse\PYZus{}dates}\PY{o}{=}\PY{n+nb+bp}{True}\PY{p}{,} \PY{n}{index\PYZus{}col}\PY{o}{=}\PY{l+m+mi}{0}\PY{p}{)}
         \PY{n}{close\PYZus{}px}\PY{o}{.}\PY{n}{info}\PY{p}{(}\PY{p}{)}
\end{Verbatim}

    \begin{Verbatim}[commandchars=\\\{\}]
<class 'pandas.core.frame.DataFrame'>
DatetimeIndex: 2214 entries, 2003-01-02 00:00:00 to 2011-10-14 00:00:00
Data columns (total 4 columns):
AAPL    2214 non-null float64
MSFT    2214 non-null float64
XOM     2214 non-null float64
SPX     2214 non-null float64
dtypes: float64(4)
memory usage: 86.5 KB
    \end{Verbatim}

    One task of interest might be to compute a \texttt{DataFrame} consisting
of the yearly correlations of daily returns (computed from percent
changes) with \texttt{SPX}. Here is one way to do it:

    \begin{Verbatim}[commandchars=\\\{\}]
{\color{incolor}In [{\color{incolor}71}]:} \PY{n}{close\PYZus{}px}\PY{p}{[}\PY{o}{\PYZhy{}}\PY{l+m+mi}{4}\PY{p}{:}\PY{p}{]}
\end{Verbatim}

            \begin{Verbatim}[commandchars=\\\{\}]
{\color{outcolor}Out[{\color{outcolor}71}]:}               AAPL   MSFT    XOM      SPX
         2011-10-11  400.29  27.00  76.27  1195.54
         2011-10-12  402.19  26.96  77.16  1207.25
         2011-10-13  408.43  27.18  76.37  1203.66
         2011-10-14  422.00  27.27  78.11  1224.58
\end{Verbatim}
        
    \begin{Verbatim}[commandchars=\\\{\}]
{\color{incolor}In [{\color{incolor}72}]:} \PY{n}{rets} \PY{o}{=} \PY{n}{close\PYZus{}px}\PY{o}{.}\PY{n}{pct\PYZus{}change}\PY{p}{(}\PY{p}{)}\PY{o}{.}\PY{n}{dropna}\PY{p}{(}\PY{p}{)}
         \PY{n}{spx\PYZus{}corr} \PY{o}{=} \PY{k}{lambda} \PY{n}{x}\PY{p}{:} \PY{n}{x}\PY{o}{.}\PY{n}{corrwith}\PY{p}{(}\PY{n}{x}\PY{p}{[}\PY{l+s}{\PYZsq{}}\PY{l+s}{SPX}\PY{l+s}{\PYZsq{}}\PY{p}{]}\PY{p}{)}
         \PY{n}{by\PYZus{}year} \PY{o}{=} \PY{n}{rets}\PY{o}{.}\PY{n}{groupby}\PY{p}{(}\PY{k}{lambda} \PY{n}{x}\PY{p}{:} \PY{n}{x}\PY{o}{.}\PY{n}{year}\PY{p}{)}
         \PY{n}{by\PYZus{}year}\PY{o}{.}\PY{n}{apply}\PY{p}{(}\PY{n}{spx\PYZus{}corr}\PY{p}{)}
\end{Verbatim}

            \begin{Verbatim}[commandchars=\\\{\}]
{\color{outcolor}Out[{\color{outcolor}72}]:}           AAPL      MSFT       XOM  SPX
         2003  0.541124  0.745174  0.661265    1
         2004  0.374283  0.588531  0.557742    1
         2005  0.467540  0.562374  0.631010    1
         2006  0.428267  0.406126  0.518514    1
         2007  0.508118  0.658770  0.786264    1
         2008  0.681434  0.804626  0.828303    1
         2009  0.707103  0.654902  0.797921    1
         2010  0.710105  0.730118  0.839057    1
         2011  0.691931  0.800996  0.859975    1
\end{Verbatim}
        
    There is of course nothing to stop you from computing inter-column
correlation:

    \begin{Verbatim}[commandchars=\\\{\}]
{\color{incolor}In [{\color{incolor}73}]:} \PY{c}{\PYZsh{} Annual correlation of Apple with Microsoft}
         \PY{n}{by\PYZus{}year}\PY{o}{.}\PY{n}{apply}\PY{p}{(}\PY{k}{lambda} \PY{n}{g}\PY{p}{:} \PY{n}{g}\PY{p}{[}\PY{l+s}{\PYZsq{}}\PY{l+s}{AAPL}\PY{l+s}{\PYZsq{}}\PY{p}{]}\PY{o}{.}\PY{n}{corr}\PY{p}{(}\PY{n}{g}\PY{p}{[}\PY{l+s}{\PYZsq{}}\PY{l+s}{MSFT}\PY{l+s}{\PYZsq{}}\PY{p}{]}\PY{p}{)}\PY{p}{)}
\end{Verbatim}

            \begin{Verbatim}[commandchars=\\\{\}]
{\color{outcolor}Out[{\color{outcolor}73}]:} 2003    0.480868
         2004    0.259024
         2005    0.300093
         2006    0.161735
         2007    0.417738
         2008    0.611901
         2009    0.432738
         2010    0.571946
         2011    0.581987
         dtype: float64
\end{Verbatim}
        
    \subsection{Pivot tables and
Cross-tabulation}\label{pivot-tables-and-cross-tabulation}

    A \emph{pivot table} is a data summerization tool. Pivot tables work by
aggregating a table of data by keys, where the data is organized
rectangularly with the group keys along the rows and columns. We can use
pivot tables in Python by using the \texttt{groupby} methodology. Using
DataFrames allows us to apply the \texttt{pivot\_table} method and we
can use the \texttt{pandas.pivot\_table} function. Besides acting as a
great way to access \texttt{groupby}, \texttt{pivot\_table} can add
partial totals or margins.

We can use the \texttt{pivot\_table} to calculate the group means of
people by \texttt{sex} and \texttt{smoker}.

    \begin{Verbatim}[commandchars=\\\{\}]
{\color{incolor}In [{\color{incolor}74}]:} \PY{n}{tips}\PY{o}{=}\PY{n}{pd}\PY{o}{.}\PY{n}{read\PYZus{}csv}\PY{p}{(}\PY{l+s}{\PYZsq{}}\PY{l+s}{tips.csv}\PY{l+s}{\PYZsq{}}\PY{p}{)}
         \PY{n}{tips}\PY{p}{[}\PY{l+s}{\PYZsq{}}\PY{l+s}{tip\PYZus{}pct}\PY{l+s}{\PYZsq{}}\PY{p}{]} \PY{o}{=} \PY{n}{tips}\PY{p}{[}\PY{l+s}{\PYZsq{}}\PY{l+s}{tip}\PY{l+s}{\PYZsq{}}\PY{p}{]}\PY{o}{/}\PY{n}{tips}\PY{p}{[}\PY{l+s}{\PYZsq{}}\PY{l+s}{total\PYZus{}bill}\PY{l+s}{\PYZsq{}}\PY{p}{]}
         
         \PY{n}{tips}\PY{o}{.}\PY{n}{pivot\PYZus{}table}\PY{p}{(}\PY{n}{index}\PY{o}{=}\PY{p}{[}\PY{l+s}{\PYZsq{}}\PY{l+s}{sex}\PY{l+s}{\PYZsq{}}\PY{p}{,} \PY{l+s}{\PYZsq{}}\PY{l+s}{smoker}\PY{l+s}{\PYZsq{}}\PY{p}{]}\PY{p}{)}
\end{Verbatim}

            \begin{Verbatim}[commandchars=\\\{\}]
{\color{outcolor}Out[{\color{outcolor}74}]:}                    size       tip   tip\_pct  total\_bill
         sex    smoker                                          
         Female No      2.592593  2.773519  0.156921   18.105185
                Yes     2.242424  2.931515  0.182150   17.977879
         Male   No      2.711340  3.113402  0.160669   19.791237
                Yes     2.500000  3.051167  0.152771   22.284500
\end{Verbatim}
        
    Suppose that now we only care about tip percentage, size of the group,
and day of the week. We can put \texttt{smoker} in the columns and
\texttt{day} in the rows, so that we yield a tables showing the group
averages of \texttt{tip\_pct} and \texttt{size} based on \texttt{smoker}
and \texttt{day}.

    \begin{Verbatim}[commandchars=\\\{\}]
{\color{incolor}In [{\color{incolor}75}]:} \PY{n}{tips}\PY{o}{.}\PY{n}{pivot\PYZus{}table}\PY{p}{(}\PY{p}{[}\PY{l+s}{\PYZsq{}}\PY{l+s}{tip\PYZus{}pct}\PY{l+s}{\PYZsq{}}\PY{p}{,}\PY{l+s}{\PYZsq{}}\PY{l+s}{size}\PY{l+s}{\PYZsq{}}\PY{p}{]}\PY{p}{,} \PY{n}{index}\PY{o}{=}\PY{p}{[}\PY{l+s}{\PYZsq{}}\PY{l+s}{sex}\PY{l+s}{\PYZsq{}}\PY{p}{,} \PY{l+s}{\PYZsq{}}\PY{l+s}{day}\PY{l+s}{\PYZsq{}}\PY{p}{]}\PY{p}{,}
                          \PY{n}{columns}\PY{o}{=}\PY{l+s}{\PYZsq{}}\PY{l+s}{smoker}\PY{l+s}{\PYZsq{}}\PY{p}{)}
\end{Verbatim}

            \begin{Verbatim}[commandchars=\\\{\}]
{\color{outcolor}Out[{\color{outcolor}75}]:}               tip\_pct                size          
         smoker             No       Yes        No       Yes
         sex    day                                         
         Female Fri   0.165296  0.209129  2.500000  2.000000
                Sat   0.147993  0.163817  2.307692  2.200000
                Sun   0.165710  0.237075  3.071429  2.500000
                Thur  0.155971  0.163073  2.480000  2.428571
         Male   Fri   0.138005  0.144730  2.000000  2.125000
                Sat   0.162132  0.139067  2.656250  2.629630
                Sun   0.158291  0.173964  2.883721  2.600000
                Thur  0.165706  0.164417  2.500000  2.300000
\end{Verbatim}
        
    Furthermore, if we also want data about general tipping percentages and
size of parties without regard to people smoking, we can use the
\texttt{margins} argument to calculate to corresponding group statistic.
Using \texttt{margins=True} calculates the partial totals of each
column.

    \begin{Verbatim}[commandchars=\\\{\}]
{\color{incolor}In [{\color{incolor}76}]:} \PY{n}{tips}\PY{o}{.}\PY{n}{pivot\PYZus{}table}\PY{p}{(}\PY{p}{[}\PY{l+s}{\PYZsq{}}\PY{l+s}{tip\PYZus{}pct}\PY{l+s}{\PYZsq{}}\PY{p}{,} \PY{l+s}{\PYZsq{}}\PY{l+s}{size}\PY{l+s}{\PYZsq{}}\PY{p}{]}\PY{p}{,} \PY{n}{index}\PY{o}{=}\PY{p}{[}\PY{l+s}{\PYZsq{}}\PY{l+s}{sex}\PY{l+s}{\PYZsq{}}\PY{p}{,} \PY{l+s}{\PYZsq{}}\PY{l+s}{day}\PY{l+s}{\PYZsq{}}\PY{p}{]}\PY{p}{,}
                          \PY{n}{columns}\PY{o}{=}\PY{l+s}{\PYZsq{}}\PY{l+s}{smoker}\PY{l+s}{\PYZsq{}}\PY{p}{,} \PY{n}{margins}\PY{o}{=}\PY{n+nb+bp}{True}\PY{p}{)}
\end{Verbatim}

            \begin{Verbatim}[commandchars=\\\{\}]
{\color{outcolor}Out[{\color{outcolor}76}]:}               tip\_pct                          size                    
         smoker             No       Yes       All        No       Yes       All
         sex    day                                                             
         Female Fri   0.165296  0.209129  0.199388  2.500000  2.000000  2.111111
                Sat   0.147993  0.163817  0.156470  2.307692  2.200000  2.250000
                Sun   0.165710  0.237075  0.181569  3.071429  2.500000  2.944444
                Thur  0.155971  0.163073  0.157525  2.480000  2.428571  2.468750
         Male   Fri   0.138005  0.144730  0.143385  2.000000  2.125000  2.100000
                Sat   0.162132  0.139067  0.151577  2.656250  2.629630  2.644068
                Sun   0.158291  0.173964  0.162344  2.883721  2.600000  2.810345
                Thur  0.165706  0.164417  0.165276  2.500000  2.300000  2.433333
         All          0.159328  0.163196  0.160803  2.668874  2.408602  2.569672
\end{Verbatim}
        
    The \texttt{All} columns show the average tipping percentage and size of
parties without regard to smoking. To use a different aggregate
function, we may use the \texttt{aggfunc} argument. For example we may
use the \texttt{len} function to calculate the frequency of group sizes.

    \begin{Verbatim}[commandchars=\\\{\}]
{\color{incolor}In [{\color{incolor}77}]:} \PY{n}{tips}\PY{o}{.}\PY{n}{pivot\PYZus{}table}\PY{p}{(}\PY{l+s}{\PYZsq{}}\PY{l+s}{tip\PYZus{}pct}\PY{l+s}{\PYZsq{}}\PY{p}{,} \PY{n}{index}\PY{o}{=}\PY{p}{[}\PY{l+s}{\PYZsq{}}\PY{l+s}{sex}\PY{l+s}{\PYZsq{}}\PY{p}{,} \PY{l+s}{\PYZsq{}}\PY{l+s}{smoker}\PY{l+s}{\PYZsq{}}\PY{p}{]}\PY{p}{,} \PY{n}{columns}\PY{o}{=}\PY{l+s}{\PYZsq{}}\PY{l+s}{day}\PY{l+s}{\PYZsq{}}\PY{p}{,}
                          \PY{n}{aggfunc}\PY{o}{=}\PY{n+nb}{len}\PY{p}{,} \PY{n}{margins}\PY{o}{=}\PY{n+nb+bp}{True}\PY{p}{)}
\end{Verbatim}

            \begin{Verbatim}[commandchars=\\\{\}]
{\color{outcolor}Out[{\color{outcolor}77}]:} day            Fri  Sat  Sun  Thur  All
         sex    smoker                          
         Female No        2   13   14    25   54
                Yes       7   15    4     7   33
         Male   No        2   32   43    20   97
                Yes       8   27   15    10   60
         All             19   87   76    62  244
\end{Verbatim}
        
    To replace empty values with zero, we can use the \texttt{fill\_value}
argument.

    \begin{Verbatim}[commandchars=\\\{\}]
{\color{incolor}In [{\color{incolor}78}]:} \PY{n}{tips}\PY{o}{.}\PY{n}{pivot\PYZus{}table}\PY{p}{(}\PY{l+s}{\PYZsq{}}\PY{l+s}{size}\PY{l+s}{\PYZsq{}}\PY{p}{,} \PY{n}{index}\PY{o}{=}\PY{p}{[}\PY{l+s}{\PYZsq{}}\PY{l+s}{time}\PY{l+s}{\PYZsq{}}\PY{p}{,} \PY{l+s}{\PYZsq{}}\PY{l+s}{sex}\PY{l+s}{\PYZsq{}}\PY{p}{,} \PY{l+s}{\PYZsq{}}\PY{l+s}{smoker}\PY{l+s}{\PYZsq{}}\PY{p}{]}\PY{p}{,}
                          \PY{n}{columns}\PY{o}{=}\PY{l+s}{\PYZsq{}}\PY{l+s}{day}\PY{l+s}{\PYZsq{}}\PY{p}{,} \PY{n}{aggfunc}\PY{o}{=}\PY{l+s}{\PYZsq{}}\PY{l+s}{sum}\PY{l+s}{\PYZsq{}}\PY{p}{,}\PY{n}{fill\PYZus{}value}\PY{o}{=}\PY{l+m+mi}{0}\PY{p}{)}
\end{Verbatim}

            \begin{Verbatim}[commandchars=\\\{\}]
{\color{outcolor}Out[{\color{outcolor}78}]:} day                   Fri  Sat  Sun  Thur
         time   sex    smoker                     
         Dinner Female No        2   30   43     2
                       Yes       8   33   10     0
                Male   No        4   85  124     0
                       Yes      12   71   39     0
         Lunch  Female No        3    0    0    60
                       Yes       6    0    0    17
                Male   No        0    0    0    50
                       Yes       5    0    0    23
\end{Verbatim}
        
    \subsubsection{Cross-tabulations:
crosstab}\label{cross-tabulations-crosstab}

    A \emph{cross-tabulation} is a special case of a pivot table that
computes group frequencies.

    \begin{Verbatim}[commandchars=\\\{\}]
{\color{incolor}In [{\color{incolor}79}]:} \PY{k+kn}{from} \PY{n+nn}{StringIO} \PY{k+kn}{import} \PY{n}{StringIO}
         \PY{n}{data} \PY{o}{=} \PY{l+s}{\PYZdq{}\PYZdq{}\PYZdq{}}\PY{l+s+se}{\PYZbs{}}
         \PY{l+s}{Sample    Gender    Handedness}
         \PY{l+s}{1    Female    Right\PYZhy{}handed}
         \PY{l+s}{2    Male    Left\PYZhy{}handed}
         \PY{l+s}{3    Female    Right\PYZhy{}handed}
         \PY{l+s}{4    Male    Right\PYZhy{}handed}
         \PY{l+s}{5    Male    Left\PYZhy{}handed}
         \PY{l+s}{6    Male    Right\PYZhy{}handed}
         \PY{l+s}{7    Female    Right\PYZhy{}handed}
         \PY{l+s}{8    Female    Left\PYZhy{}handed}
         \PY{l+s}{9    Male    Right\PYZhy{}handed}
         \PY{l+s}{10    Female    Right\PYZhy{}handed}\PY{l+s}{\PYZdq{}\PYZdq{}\PYZdq{}}
         \PY{n}{data} \PY{o}{=} \PY{n}{pd}\PY{o}{.}\PY{n}{read\PYZus{}table}\PY{p}{(}\PY{n}{StringIO}\PY{p}{(}\PY{n}{data}\PY{p}{)}\PY{p}{,} \PY{n}{sep}\PY{o}{=}\PY{l+s}{\PYZsq{}}\PY{l+s}{\PYZbs{}}\PY{l+s}{s+}\PY{l+s}{\PYZsq{}}\PY{p}{)}
         
         \PY{n}{data}
\end{Verbatim}

            \begin{Verbatim}[commandchars=\\\{\}]
{\color{outcolor}Out[{\color{outcolor}79}]:}    Sample  Gender    Handedness
         0       1  Female  Right-handed
         1       2    Male   Left-handed
         2       3  Female  Right-handed
         3       4    Male  Right-handed
         4       5    Male   Left-handed
         5       6    Male  Right-handed
         6       7  Female  Right-handed
         7       8  Female   Left-handed
         8       9    Male  Right-handed
         9      10  Female  Right-handed
\end{Verbatim}
        
    We could use \texttt{pivot\_table} to do this calculation, but
\texttt{pandas.crosstab} is a convenient.

    \begin{Verbatim}[commandchars=\\\{\}]
{\color{incolor}In [{\color{incolor}80}]:} \PY{n}{pd}\PY{o}{.}\PY{n}{crosstab}\PY{p}{(}\PY{n}{data}\PY{o}{.}\PY{n}{Gender}\PY{p}{,} \PY{n}{data}\PY{o}{.}\PY{n}{Handedness}\PY{p}{,} \PY{n}{margins}\PY{o}{=}\PY{n+nb+bp}{True}\PY{p}{)}
\end{Verbatim}

            \begin{Verbatim}[commandchars=\\\{\}]
{\color{outcolor}Out[{\color{outcolor}80}]:} Handedness  Left-handed  Right-handed  All
         Gender                                    
         Female                1             4    5
         Male                  2             3    5
         All                   3             7   10
\end{Verbatim}
        
    When using \texttt{crosstab}, we may use either an array or Series or a
list of arrays.

    \begin{Verbatim}[commandchars=\\\{\}]
{\color{incolor}In [{\color{incolor}81}]:} \PY{n}{pd}\PY{o}{.}\PY{n}{crosstab}\PY{p}{(}\PY{p}{[}\PY{n}{tips}\PY{o}{.}\PY{n}{time}\PY{p}{,} \PY{n}{tips}\PY{o}{.}\PY{n}{day}\PY{p}{]}\PY{p}{,} \PY{n}{tips}\PY{o}{.}\PY{n}{smoker}\PY{p}{,} \PY{n}{margins}\PY{o}{=}\PY{n+nb+bp}{True}\PY{p}{)}
\end{Verbatim}

            \begin{Verbatim}[commandchars=\\\{\}]
{\color{outcolor}Out[{\color{outcolor}81}]:} smoker        No  Yes  All
         time   day                
         Dinner Fri     3    9   12
                Sat    45   42   87
                Sun    57   19   76
                Thur    1    0    1
         Lunch  Fri     1    6    7
                Thur   44   17   61
         All          151   93  244
\end{Verbatim}
        
    \subsection{Example: 2012 Federal Election Commission
Database}\label{example-2012-federal-election-commission-database}

    We will be working with data from the 2012 US Presidential Election.
This dataset focuses on campaign contributions for presidential
candidates. The data can be loaded from:

    \begin{Verbatim}[commandchars=\\\{\}]
{\color{incolor}In [{\color{incolor}82}]:} \PY{n}{fec} \PY{o}{=} \PY{n}{pd}\PY{o}{.}\PY{n}{read\PYZus{}csv}\PY{p}{(}\PY{l+s}{\PYZsq{}}\PY{l+s}{P00000001\PYZhy{}ALL.csv}\PY{l+s}{\PYZsq{}}\PY{p}{)}
         
         \PY{n}{fec}\PY{o}{.}\PY{n}{info}\PY{p}{(}\PY{p}{)}
\end{Verbatim}

    \begin{Verbatim}[commandchars=\\\{\}]
<class 'pandas.core.frame.DataFrame'>
Int64Index: 1001731 entries, 0 to 1001730
Data columns (total 16 columns):
cmte\_id              1001731 non-null object
cand\_id              1001731 non-null object
cand\_nm              1001731 non-null object
contbr\_nm            1001731 non-null object
contbr\_city          1001712 non-null object
contbr\_st            1001727 non-null object
contbr\_zip           1001620 non-null object
contbr\_employer      988002 non-null object
contbr\_occupation    993301 non-null object
contb\_receipt\_amt    1001731 non-null float64
contb\_receipt\_dt     1001731 non-null object
receipt\_desc         14166 non-null object
memo\_cd              92482 non-null object
memo\_text            97770 non-null object
form\_tp              1001731 non-null object
file\_num             1001731 non-null int64
dtypes: float64(1), int64(1), object(14)
memory usage: 129.9+ MB
    \end{Verbatim}

    \begin{Verbatim}[commandchars=\\\{\}]
/home/alethiometryst/anaconda/lib/python2.7/site-packages/pandas/io/parsers.py:1159: DtypeWarning: Columns (6) have mixed types. Specify dtype option on import or set low\_memory=False.
  data = self.\_reader.read(nrows)
    \end{Verbatim}

    A sample data frame looks like this:

    \begin{Verbatim}[commandchars=\\\{\}]
{\color{incolor}In [{\color{incolor}83}]:} \PY{n}{fec}\PY{o}{.}\PY{n}{ix}\PY{p}{[}\PY{l+m+mi}{123456}\PY{p}{]}
\end{Verbatim}

            \begin{Verbatim}[commandchars=\\\{\}]
{\color{outcolor}Out[{\color{outcolor}83}]:} cmte\_id                             C00431445
         cand\_id                             P80003338
         cand\_nm                         Obama, Barack
         contbr\_nm                         ELLMAN, IRA
         contbr\_city                             TEMPE
         contbr\_st                                  AZ
         contbr\_zip                          852816719
         contbr\_employer      ARIZONA STATE UNIVERSITY
         contbr\_occupation                   PROFESSOR
         contb\_receipt\_amt                          50
         contb\_receipt\_dt                    01-DEC-11
         receipt\_desc                              NaN
         memo\_cd                                   NaN
         memo\_text                                 NaN
         form\_tp                                 SA17A
         file\_num                               772372
         Name: 123456, dtype: object
\end{Verbatim}
        
    One interesting aspect of this data set is the lack of partisanship as a
way to classify candidates. We can add this information to the dataset.
The way we are going to solve this problem is to create a dictionary
indicating the politcal party of each candidate. First, we need to find
out who all of the candidates are.

    \begin{Verbatim}[commandchars=\\\{\}]
{\color{incolor}In [{\color{incolor}84}]:} \PY{n}{unique\PYZus{}cands} \PY{o}{=} \PY{n}{fec}\PY{o}{.}\PY{n}{cand\PYZus{}nm}\PY{o}{.}\PY{n}{unique}\PY{p}{(}\PY{p}{)}
         \PY{n}{unique\PYZus{}cands}
\end{Verbatim}

            \begin{Verbatim}[commandchars=\\\{\}]
{\color{outcolor}Out[{\color{outcolor}84}]:} array(['Bachmann, Michelle', 'Romney, Mitt', 'Obama, Barack',
                "Roemer, Charles E. 'Buddy' III", 'Pawlenty, Timothy',
                'Johnson, Gary Earl', 'Paul, Ron', 'Santorum, Rick', 'Cain, Herman',
                'Gingrich, Newt', 'McCotter, Thaddeus G', 'Huntsman, Jon',
                'Perry, Rick'], dtype=object)
\end{Verbatim}
        
    \begin{Verbatim}[commandchars=\\\{\}]
{\color{incolor}In [{\color{incolor}85}]:} \PY{n}{unique\PYZus{}cands}\PY{p}{[}\PY{l+m+mi}{2}\PY{p}{]}
\end{Verbatim}

            \begin{Verbatim}[commandchars=\\\{\}]
{\color{outcolor}Out[{\color{outcolor}85}]:} 'Obama, Barack'
\end{Verbatim}
        
    We use \texttt{parties} to specify a dictionary over all of the
candidates.

    \begin{Verbatim}[commandchars=\\\{\}]
{\color{incolor}In [{\color{incolor}86}]:} \PY{n}{parties} \PY{o}{=} \PY{p}{\PYZob{}}\PY{l+s}{\PYZsq{}}\PY{l+s}{Bachmann, Michelle}\PY{l+s}{\PYZsq{}}\PY{p}{:} \PY{l+s}{\PYZsq{}}\PY{l+s}{Republican}\PY{l+s}{\PYZsq{}}\PY{p}{,}
                    \PY{l+s}{\PYZsq{}}\PY{l+s}{Cain, Herman}\PY{l+s}{\PYZsq{}}\PY{p}{:} \PY{l+s}{\PYZsq{}}\PY{l+s}{Republican}\PY{l+s}{\PYZsq{}}\PY{p}{,}
                    \PY{l+s}{\PYZsq{}}\PY{l+s}{Gingrich, Newt}\PY{l+s}{\PYZsq{}}\PY{p}{:} \PY{l+s}{\PYZsq{}}\PY{l+s}{Republican}\PY{l+s}{\PYZsq{}}\PY{p}{,}
                    \PY{l+s}{\PYZsq{}}\PY{l+s}{Huntsman, Jon}\PY{l+s}{\PYZsq{}}\PY{p}{:} \PY{l+s}{\PYZsq{}}\PY{l+s}{Republican}\PY{l+s}{\PYZsq{}}\PY{p}{,}
                    \PY{l+s}{\PYZsq{}}\PY{l+s}{Johnson, Gary Earl}\PY{l+s}{\PYZsq{}}\PY{p}{:} \PY{l+s}{\PYZsq{}}\PY{l+s}{Republican}\PY{l+s}{\PYZsq{}}\PY{p}{,}
                    \PY{l+s}{\PYZsq{}}\PY{l+s}{McCotter, Thaddeus G}\PY{l+s}{\PYZsq{}}\PY{p}{:} \PY{l+s}{\PYZsq{}}\PY{l+s}{Republican}\PY{l+s}{\PYZsq{}}\PY{p}{,}
                    \PY{l+s}{\PYZsq{}}\PY{l+s}{Obama, Barack}\PY{l+s}{\PYZsq{}}\PY{p}{:} \PY{l+s}{\PYZsq{}}\PY{l+s}{Democrat}\PY{l+s}{\PYZsq{}}\PY{p}{,}
                    \PY{l+s}{\PYZsq{}}\PY{l+s}{Paul, Ron}\PY{l+s}{\PYZsq{}}\PY{p}{:} \PY{l+s}{\PYZsq{}}\PY{l+s}{Republican}\PY{l+s}{\PYZsq{}}\PY{p}{,}
                    \PY{l+s}{\PYZsq{}}\PY{l+s}{Pawlenty, Timothy}\PY{l+s}{\PYZsq{}}\PY{p}{:} \PY{l+s}{\PYZsq{}}\PY{l+s}{Republican}\PY{l+s}{\PYZsq{}}\PY{p}{,}
                    \PY{l+s}{\PYZsq{}}\PY{l+s}{Perry, Rick}\PY{l+s}{\PYZsq{}}\PY{p}{:} \PY{l+s}{\PYZsq{}}\PY{l+s}{Republican}\PY{l+s}{\PYZsq{}}\PY{p}{,}
                    \PY{l+s}{\PYZdq{}}\PY{l+s}{Roemer, Charles E. }\PY{l+s}{\PYZsq{}}\PY{l+s}{Buddy}\PY{l+s}{\PYZsq{}}\PY{l+s}{ III}\PY{l+s}{\PYZdq{}}\PY{p}{:} \PY{l+s}{\PYZsq{}}\PY{l+s}{Republican}\PY{l+s}{\PYZsq{}}\PY{p}{,}
                    \PY{l+s}{\PYZsq{}}\PY{l+s}{Romney, Mitt}\PY{l+s}{\PYZsq{}}\PY{p}{:} \PY{l+s}{\PYZsq{}}\PY{l+s}{Republican}\PY{l+s}{\PYZsq{}}\PY{p}{,}
                    \PY{l+s}{\PYZsq{}}\PY{l+s}{Santorum, Rick}\PY{l+s}{\PYZsq{}}\PY{p}{:} \PY{l+s}{\PYZsq{}}\PY{l+s}{Republican}\PY{l+s}{\PYZsq{}}\PY{p}{\PYZcb{}}
\end{Verbatim}

    We can test our dictionary by viewing a section of the dataset to first
view the candidate and then view their political affiliation.

    \begin{Verbatim}[commandchars=\\\{\}]
{\color{incolor}In [{\color{incolor}87}]:} \PY{n}{fec}\PY{o}{.}\PY{n}{cand\PYZus{}nm}\PY{p}{[}\PY{l+m+mi}{123456}\PY{p}{:}\PY{l+m+mi}{123461}\PY{p}{]}
\end{Verbatim}

            \begin{Verbatim}[commandchars=\\\{\}]
{\color{outcolor}Out[{\color{outcolor}87}]:} 123456    Obama, Barack
         123457    Obama, Barack
         123458    Obama, Barack
         123459    Obama, Barack
         123460    Obama, Barack
         Name: cand\_nm, dtype: object
\end{Verbatim}
        
    \begin{Verbatim}[commandchars=\\\{\}]
{\color{incolor}In [{\color{incolor}88}]:} \PY{n}{fec}\PY{o}{.}\PY{n}{cand\PYZus{}nm}\PY{p}{[}\PY{l+m+mi}{123456}\PY{p}{:}\PY{l+m+mi}{123461}\PY{p}{]}\PY{o}{.}\PY{n}{map}\PY{p}{(}\PY{n}{parties}\PY{p}{)}
\end{Verbatim}

            \begin{Verbatim}[commandchars=\\\{\}]
{\color{outcolor}Out[{\color{outcolor}88}]:} 123456    Democrat
         123457    Democrat
         123458    Democrat
         123459    Democrat
         123460    Democrat
         Name: cand\_nm, dtype: object
\end{Verbatim}
        
    To calculate the number of contributions to each party, we use the
\texttt{value\_counts} function to sum the number of contributions of
each party.

    \begin{Verbatim}[commandchars=\\\{\}]
{\color{incolor}In [{\color{incolor}89}]:} \PY{c}{\PYZsh{} Add it as a column}
         
         \PY{n}{fec}\PY{p}{[}\PY{l+s}{\PYZsq{}}\PY{l+s}{party}\PY{l+s}{\PYZsq{}}\PY{p}{]} \PY{o}{=} \PY{n}{fec}\PY{o}{.}\PY{n}{cand\PYZus{}nm}\PY{o}{.}\PY{n}{map}\PY{p}{(}\PY{n}{parties}\PY{p}{)}
         
         \PY{n}{fec}\PY{p}{[}\PY{l+s}{\PYZsq{}}\PY{l+s}{party}\PY{l+s}{\PYZsq{}}\PY{p}{]}\PY{o}{.}\PY{n}{value\PYZus{}counts}\PY{p}{(}\PY{p}{)}
\end{Verbatim}

            \begin{Verbatim}[commandchars=\\\{\}]
{\color{outcolor}Out[{\color{outcolor}89}]:} Democrat      593746
         Republican    407985
         dtype: int64
\end{Verbatim}
        
    Unfortunately, this counts both the positive and negative contributions
to candidate's campaigns (negative values indcate refunds). Thus, to see
the total number of donations to candidates in the 2012 US election we
subset the receipt values.

    \begin{Verbatim}[commandchars=\\\{\}]
{\color{incolor}In [{\color{incolor}90}]:} \PY{p}{(}\PY{n}{fec}\PY{o}{.}\PY{n}{contb\PYZus{}receipt\PYZus{}amt} \PY{o}{\PYZgt{}} \PY{l+m+mi}{0}\PY{p}{)}\PY{o}{.}\PY{n}{value\PYZus{}counts}\PY{p}{(}\PY{p}{)}
\end{Verbatim}

            \begin{Verbatim}[commandchars=\\\{\}]
{\color{outcolor}Out[{\color{outcolor}90}]:} True     991475
         False     10256
         dtype: int64
\end{Verbatim}
        
    To just use positive contibutions we use the following code.

    \begin{Verbatim}[commandchars=\\\{\}]
{\color{incolor}In [{\color{incolor}91}]:} \PY{n}{fec} \PY{o}{=} \PY{n}{fec}\PY{p}{[}\PY{n}{fec}\PY{o}{.}\PY{n}{contb\PYZus{}receipt\PYZus{}amt} \PY{o}{\PYZgt{}} \PY{l+m+mi}{0}\PY{p}{]}
         
         \PY{n}{fec\PYZus{}mrbo} \PY{o}{=} \PY{n}{fec}\PY{p}{[}\PY{n}{fec}\PY{o}{.}\PY{n}{cand\PYZus{}nm}\PY{o}{.}\PY{n}{isin}\PY{p}{(}\PY{p}{[}\PY{l+s}{\PYZsq{}}\PY{l+s}{Obama, Barack}\PY{l+s}{\PYZsq{}}\PY{p}{,} \PY{l+s}{\PYZsq{}}\PY{l+s}{Romney, Mitt}\PY{l+s}{\PYZsq{}}\PY{p}{]}\PY{p}{)}\PY{p}{]}
\end{Verbatim}

    \subsubsection{Donation statistics by occupation and
employer}\label{donation-statistics-by-occupation-and-employer}

    One interesting question is the occupation of donors for each party. For
example, do lawyers donate more to Democrats or Republicans? To which
party do business executives donate more money?

    \begin{Verbatim}[commandchars=\\\{\}]
{\color{incolor}In [{\color{incolor}92}]:} \PY{n}{fec}\PY{o}{.}\PY{n}{contbr\PYZus{}occupation}\PY{o}{.}\PY{n}{value\PYZus{}counts}\PY{p}{(}\PY{p}{)}\PY{p}{[}\PY{p}{:}\PY{l+m+mi}{10}\PY{p}{]}
\end{Verbatim}

            \begin{Verbatim}[commandchars=\\\{\}]
{\color{outcolor}Out[{\color{outcolor}92}]:} RETIRED                                   233990
         INFORMATION REQUESTED                      35107
         ATTORNEY                                   34286
         HOMEMAKER                                  29931
         PHYSICIAN                                  23432
         INFORMATION REQUESTED PER BEST EFFORTS     21138
         ENGINEER                                   14334
         TEACHER                                    13990
         CONSULTANT                                 13273
         PROFESSOR                                  12555
         dtype: int64
\end{Verbatim}
        
    We can again use a dictionary to better define the occupation of the
donors, as well as the employers of the donors.

    \begin{Verbatim}[commandchars=\\\{\}]
{\color{incolor}In [{\color{incolor}93}]:} \PY{n}{occ\PYZus{}mapping} \PY{o}{=} \PY{p}{\PYZob{}}
            \PY{l+s}{\PYZsq{}}\PY{l+s}{INFORMATION REQUESTED PER BEST EFFORTS}\PY{l+s}{\PYZsq{}} \PY{p}{:} \PY{l+s}{\PYZsq{}}\PY{l+s}{NOT PROVIDED}\PY{l+s}{\PYZsq{}}\PY{p}{,}
            \PY{l+s}{\PYZsq{}}\PY{l+s}{INFORMATION REQUESTED}\PY{l+s}{\PYZsq{}} \PY{p}{:} \PY{l+s}{\PYZsq{}}\PY{l+s}{NOT PROVIDED}\PY{l+s}{\PYZsq{}}\PY{p}{,}
            \PY{l+s}{\PYZsq{}}\PY{l+s}{INFORMATION REQUESTED (BEST EFFORTS)}\PY{l+s}{\PYZsq{}} \PY{p}{:} \PY{l+s}{\PYZsq{}}\PY{l+s}{NOT PROVIDED}\PY{l+s}{\PYZsq{}}\PY{p}{,}
            \PY{l+s}{\PYZsq{}}\PY{l+s}{C.E.O.}\PY{l+s}{\PYZsq{}}\PY{p}{:} \PY{l+s}{\PYZsq{}}\PY{l+s}{CEO}\PY{l+s}{\PYZsq{}}
         \PY{p}{\PYZcb{}}
\end{Verbatim}

    \begin{Verbatim}[commandchars=\\\{\}]
{\color{incolor}In [{\color{incolor}94}]:} \PY{c}{\PYZsh{} If no mapping provided, return x}
         \PY{n}{f} \PY{o}{=} \PY{k}{lambda} \PY{n}{x}\PY{p}{:} \PY{n}{occ\PYZus{}mapping}\PY{o}{.}\PY{n}{get}\PY{p}{(}\PY{n}{x}\PY{p}{,} \PY{n}{x}\PY{p}{)}
         \PY{n}{fec}\PY{o}{.}\PY{n}{contbr\PYZus{}occupation} \PY{o}{=} \PY{n}{fec}\PY{o}{.}\PY{n}{contbr\PYZus{}occupation}\PY{o}{.}\PY{n}{map}\PY{p}{(}\PY{n}{f}\PY{p}{)}
         
         \PY{n}{emp\PYZus{}mapping} \PY{o}{=} \PY{p}{\PYZob{}}
            \PY{l+s}{\PYZsq{}}\PY{l+s}{INFORMATION REQUESTED PER BEST EFFORTS}\PY{l+s}{\PYZsq{}} \PY{p}{:} \PY{l+s}{\PYZsq{}}\PY{l+s}{NOT PROVIDED}\PY{l+s}{\PYZsq{}}\PY{p}{,}
            \PY{l+s}{\PYZsq{}}\PY{l+s}{INFORMATION REQUESTED}\PY{l+s}{\PYZsq{}} \PY{p}{:} \PY{l+s}{\PYZsq{}}\PY{l+s}{NOT PROVIDED}\PY{l+s}{\PYZsq{}}\PY{p}{,}
            \PY{l+s}{\PYZsq{}}\PY{l+s}{SELF}\PY{l+s}{\PYZsq{}} \PY{p}{:} \PY{l+s}{\PYZsq{}}\PY{l+s}{SELF\PYZhy{}EMPLOYED}\PY{l+s}{\PYZsq{}}\PY{p}{,}
            \PY{l+s}{\PYZsq{}}\PY{l+s}{SELF EMPLOYED}\PY{l+s}{\PYZsq{}} \PY{p}{:} \PY{l+s}{\PYZsq{}}\PY{l+s}{SELF\PYZhy{}EMPLOYED}\PY{l+s}{\PYZsq{}}\PY{p}{,}
         \PY{p}{\PYZcb{}}
\end{Verbatim}

    \begin{Verbatim}[commandchars=\\\{\}]
{\color{incolor}In [{\color{incolor}95}]:} \PY{c}{\PYZsh{} If no mapping provided, return x}
         \PY{n}{f} \PY{o}{=} \PY{k}{lambda} \PY{n}{x}\PY{p}{:} \PY{n}{emp\PYZus{}mapping}\PY{o}{.}\PY{n}{get}\PY{p}{(}\PY{n}{x}\PY{p}{,} \PY{n}{x}\PY{p}{)}
         \PY{n}{fec}\PY{o}{.}\PY{n}{contbr\PYZus{}employer} \PY{o}{=} \PY{n}{fec}\PY{o}{.}\PY{n}{contbr\PYZus{}employer}\PY{o}{.}\PY{n}{map}\PY{p}{(}\PY{n}{f}\PY{p}{)}
\end{Verbatim}

    Using a \texttt{pivot\_table} we can view data on people who donated at
least \$2 million.

    \begin{Verbatim}[commandchars=\\\{\}]
{\color{incolor}In [{\color{incolor}96}]:} \PY{n}{by\PYZus{}occupation} \PY{o}{=} \PY{n}{fec}\PY{o}{.}\PY{n}{pivot\PYZus{}table}\PY{p}{(}\PY{l+s}{\PYZsq{}}\PY{l+s}{contb\PYZus{}receipt\PYZus{}amt}\PY{l+s}{\PYZsq{}}\PY{p}{,}
                                         \PY{n}{index}\PY{o}{=}\PY{l+s}{\PYZsq{}}\PY{l+s}{contbr\PYZus{}occupation}\PY{l+s}{\PYZsq{}}\PY{p}{,}
                                         \PY{n}{columns}\PY{o}{=}\PY{l+s}{\PYZsq{}}\PY{l+s}{party}\PY{l+s}{\PYZsq{}}\PY{p}{,} \PY{n}{aggfunc}\PY{o}{=}\PY{l+s}{\PYZsq{}}\PY{l+s}{sum}\PY{l+s}{\PYZsq{}}\PY{p}{)}
         
         \PY{n}{over\PYZus{}2mm} \PY{o}{=} \PY{n}{by\PYZus{}occupation}\PY{p}{[}\PY{n}{by\PYZus{}occupation}\PY{o}{.}\PY{n}{sum}\PY{p}{(}\PY{l+m+mi}{1}\PY{p}{)} \PY{o}{\PYZgt{}} \PY{l+m+mi}{2000000}\PY{p}{]}
         \PY{n}{over\PYZus{}2mm}
\end{Verbatim}

            \begin{Verbatim}[commandchars=\\\{\}]
{\color{outcolor}Out[{\color{outcolor}96}]:} party                 Democrat       Republican
         contbr\_occupation                              
         ATTORNEY           11141982.97   7477194.430000
         CEO                 2074974.79   4211040.520000
         CONSULTANT          2459912.71   2544725.450000
         ENGINEER             951525.55   1818373.700000
         EXECUTIVE           1355161.05   4138850.090000
         HOMEMAKER           4248875.80  13634275.780000
         INVESTOR             884133.00   2431768.920000
         LAWYER              3160478.87    391224.320000
         MANAGER              762883.22   1444532.370000
         NOT PROVIDED        4866973.96  20565473.010000
         OWNER               1001567.36   2408286.920000
         PHYSICIAN           3735124.94   3594320.240000
         PRESIDENT           1878509.95   4720923.760000
         PROFESSOR           2165071.08    296702.730000
         REAL ESTATE          528902.09   1625902.250000
         RETIRED            25305116.38  23561244.489999
         SELF-EMPLOYED        672393.40   1640252.540000
\end{Verbatim}
        
    \begin{Verbatim}[commandchars=\\\{\}]
{\color{incolor}In [{\color{incolor}97}]:} \PY{n}{over\PYZus{}2mm}\PY{o}{.}\PY{n}{plot}\PY{p}{(}\PY{n}{kind}\PY{o}{=}\PY{l+s}{\PYZsq{}}\PY{l+s}{barh}\PY{l+s}{\PYZsq{}}\PY{p}{)}
\end{Verbatim}

            \begin{Verbatim}[commandchars=\\\{\}]
{\color{outcolor}Out[{\color{outcolor}97}]:} <matplotlib.axes.\_subplots.AxesSubplot at 0x7f37d8e6db90>
\end{Verbatim}
        
    \begin{center}
    \adjustimage{max size={0.9\linewidth}{0.9\paperheight}}{Acquiring and wrangling with data_files/Acquiring and wrangling with data_200_1.png}
    \end{center}
    { \hspace*{\fill} \\}
    
    Alternatively, we can view donors who gave to the campaigns of Barack
Obama or Mitt Romney. We do this by grouping by candidate name using the
\texttt{top} method that we learned earlier.

    \begin{Verbatim}[commandchars=\\\{\}]
{\color{incolor}In [{\color{incolor}98}]:} \PY{k}{def} \PY{n+nf}{get\PYZus{}top\PYZus{}amounts}\PY{p}{(}\PY{n}{group}\PY{p}{,} \PY{n}{key}\PY{p}{,} \PY{n}{n}\PY{o}{=}\PY{l+m+mi}{5}\PY{p}{)}\PY{p}{:}
             \PY{n}{totals} \PY{o}{=} \PY{n}{group}\PY{o}{.}\PY{n}{groupby}\PY{p}{(}\PY{n}{key}\PY{p}{)}\PY{p}{[}\PY{l+s}{\PYZsq{}}\PY{l+s}{contb\PYZus{}receipt\PYZus{}amt}\PY{l+s}{\PYZsq{}}\PY{p}{]}\PY{o}{.}\PY{n}{sum}\PY{p}{(}\PY{p}{)}
         
             \PY{c}{\PYZsh{} Order totals by key in descending order}
             \PY{k}{return} \PY{n}{totals}\PY{o}{.}\PY{n}{order}\PY{p}{(}\PY{n}{ascending}\PY{o}{=}\PY{n+nb+bp}{False}\PY{p}{)}\PY{p}{[}\PY{o}{\PYZhy{}}\PY{n}{n}\PY{p}{:}\PY{p}{]}
         
         \PY{n}{grouped} \PY{o}{=} \PY{n}{fec\PYZus{}mrbo}\PY{o}{.}\PY{n}{groupby}\PY{p}{(}\PY{l+s}{\PYZsq{}}\PY{l+s}{cand\PYZus{}nm}\PY{l+s}{\PYZsq{}}\PY{p}{)}
         \PY{n}{grouped}\PY{o}{.}\PY{n}{apply}\PY{p}{(}\PY{n}{get\PYZus{}top\PYZus{}amounts}\PY{p}{,} \PY{l+s}{\PYZsq{}}\PY{l+s}{contbr\PYZus{}occupation}\PY{l+s}{\PYZsq{}}\PY{p}{,} \PY{n}{n}\PY{o}{=}\PY{l+m+mi}{7}\PY{p}{)}
         
         \PY{n}{grouped}\PY{o}{.}\PY{n}{apply}\PY{p}{(}\PY{n}{get\PYZus{}top\PYZus{}amounts}\PY{p}{,} \PY{l+s}{\PYZsq{}}\PY{l+s}{contbr\PYZus{}employer}\PY{l+s}{\PYZsq{}}\PY{p}{,} \PY{n}{n}\PY{o}{=}\PY{l+m+mi}{10}\PY{p}{)}
\end{Verbatim}

            \begin{Verbatim}[commandchars=\\\{\}]
{\color{outcolor}Out[{\color{outcolor}98}]:} cand\_nm        contbr\_employer                   
         Obama, Barack  SOLIYA                                3.0
                        CARR ENTERPRISES                      3.0
                        PENN STATE DICKINSON SCHOOL OF LAW    3.0
                        CADUCEUS OCCUPATIONAL MEDICINE        3.0
                        N.A.                                  3.0
                        REAL ENERGY CONSULTING SERVICES       3.0
                        JPDSYSTEMS, LLC                       3.0
                        CASS REGIONAL MED. CENTER             2.5
                        ARCON CORP                            2.0
                        THE VICTORIA GROUP, INC.              2.0
         Romney, Mitt   EASTHAM CAPITAL                       5.0
                        GREGORY GALLIVAN                      5.0
                        DIRECT LENDERS LLC                    5.0
                        LOUGH INVESTMENT ADVISORY LLC         4.0
                        WATERWORKS INDUSRTIES                 3.0
                        WILL MERRIFIELD                       3.0
                        HONOLD COMMUNICTAIONS                 3.0
                        INDEPENDENT PROFESSIONAL              3.0
                        UPTOWN CHEAPSKATE                     3.0
                        UN                                    3.0
         Name: contb\_receipt\_amt, dtype: float64
\end{Verbatim}
        
    \subsubsection{Bucketing donation
amounts}\label{bucketing-donation-amounts}

    A useful way to analyze data is to use the \texttt{cut} function to
partition the data into comparable buckets.

    \begin{Verbatim}[commandchars=\\\{\}]
{\color{incolor}In [{\color{incolor}99}]:} \PY{n}{bins} \PY{o}{=} \PY{n}{np}\PY{o}{.}\PY{n}{array}\PY{p}{(}\PY{p}{[}\PY{l+m+mi}{0}\PY{p}{,} \PY{l+m+mi}{1}\PY{p}{,} \PY{l+m+mi}{10}\PY{p}{,} \PY{l+m+mi}{100}\PY{p}{,} \PY{l+m+mi}{1000}\PY{p}{,} \PY{l+m+mi}{10000}\PY{p}{,} \PY{l+m+mi}{100000}\PY{p}{,} \PY{l+m+mi}{1000000}\PY{p}{,} \PY{l+m+mi}{10000000}\PY{p}{]}\PY{p}{)}
         \PY{n}{labels} \PY{o}{=} \PY{n}{pd}\PY{o}{.}\PY{n}{cut}\PY{p}{(}\PY{n}{fec\PYZus{}mrbo}\PY{o}{.}\PY{n}{contb\PYZus{}receipt\PYZus{}amt}\PY{p}{,} \PY{n}{bins}\PY{p}{)}
         \PY{n}{labels}
\end{Verbatim}

            \begin{Verbatim}[commandchars=\\\{\}]
{\color{outcolor}Out[{\color{outcolor}99}]:} 411      (10, 100]
         412    (100, 1000]
         413    (100, 1000]
         414      (10, 100]
         415      (10, 100]
         416      (10, 100]
         417    (100, 1000]
         418      (10, 100]
         419    (100, 1000]
         420      (10, 100]
         421      (10, 100]
         422    (100, 1000]
         423    (100, 1000]
         424    (100, 1000]
         425    (100, 1000]
         {\ldots}
         701371        (10, 100]
         701372        (10, 100]
         701373        (10, 100]
         701374        (10, 100]
         701375        (10, 100]
         701376    (1000, 10000]
         701377        (10, 100]
         701378        (10, 100]
         701379      (100, 1000]
         701380    (1000, 10000]
         701381        (10, 100]
         701382      (100, 1000]
         701383          (1, 10]
         701384        (10, 100]
         701385      (100, 1000]
         Name: contb\_receipt\_amt, Length: 694282, dtype: category
         Categories (8, object): [(0, 1] < (1, 10] < (10, 100] < (100, 1000] < (1000, 10000] < (10000, 100000] < (100000, 1000000] < (1000000, 10000000]]
\end{Verbatim}
        
    Grouping the data by name and bin, we get a histogram by donation size.

    \begin{Verbatim}[commandchars=\\\{\}]
{\color{incolor}In [{\color{incolor}100}]:} \PY{n}{grouped} \PY{o}{=} \PY{n}{fec\PYZus{}mrbo}\PY{o}{.}\PY{n}{groupby}\PY{p}{(}\PY{p}{[}\PY{l+s}{\PYZsq{}}\PY{l+s}{cand\PYZus{}nm}\PY{l+s}{\PYZsq{}}\PY{p}{,} \PY{n}{labels}\PY{p}{]}\PY{p}{)}
          \PY{n}{grouped}\PY{o}{.}\PY{n}{size}\PY{p}{(}\PY{p}{)}\PY{o}{.}\PY{n}{unstack}\PY{p}{(}\PY{l+m+mi}{0}\PY{p}{)}
\end{Verbatim}

            \begin{Verbatim}[commandchars=\\\{\}]
{\color{outcolor}Out[{\color{outcolor}100}]:} cand\_nm              Obama, Barack  Romney, Mitt
          contb\_receipt\_amt                               
          (0, 1]                         493            77
          (1, 10]                      40070          3681
          (10, 100]                   372280         31853
          (100, 1000]                 153991         43357
          (1000, 10000]                22284         26186
          (10000, 100000]                  2             1
          (100000, 1000000]                3           NaN
          (1000000, 10000000]              4           NaN
\end{Verbatim}
        
    The data shows that Barack Obama received significantly more
contributions of smaller donation sizes. We can also sum the
contribution amounts and normalize the data to view a percentage of
total donations of each size by candidate:

    \begin{Verbatim}[commandchars=\\\{\}]
{\color{incolor}In [{\color{incolor}101}]:} \PY{n}{bucket\PYZus{}sums} \PY{o}{=} \PY{n}{grouped}\PY{o}{.}\PY{n}{contb\PYZus{}receipt\PYZus{}amt}\PY{o}{.}\PY{n}{sum}\PY{p}{(}\PY{p}{)}\PY{o}{.}\PY{n}{unstack}\PY{p}{(}\PY{l+m+mi}{0}\PY{p}{)}
          \PY{n}{bucket\PYZus{}sums}
\end{Verbatim}

            \begin{Verbatim}[commandchars=\\\{\}]
{\color{outcolor}Out[{\color{outcolor}101}]:} cand\_nm              Obama, Barack  Romney, Mitt
          contb\_receipt\_amt                               
          (0, 1]                      318.24         77.00
          (1, 10]                  337267.62      29819.66
          (10, 100]              20288981.41    1987783.76
          (100, 1000]            54798531.46   22363381.69
          (1000, 10000]          51753705.67   63942145.42
          (10000, 100000]           59100.00      12700.00
          (100000, 1000000]       1490683.08           NaN
          (1000000, 10000000]     7148839.76           NaN
\end{Verbatim}
        
    \begin{Verbatim}[commandchars=\\\{\}]
{\color{incolor}In [{\color{incolor}102}]:} \PY{n}{normed\PYZus{}sums} \PY{o}{=} \PY{n}{bucket\PYZus{}sums}\PY{o}{.}\PY{n}{div}\PY{p}{(}\PY{n}{bucket\PYZus{}sums}\PY{o}{.}\PY{n}{sum}\PY{p}{(}\PY{n}{axis}\PY{o}{=}\PY{l+m+mi}{1}\PY{p}{)}\PY{p}{,} \PY{n}{axis}\PY{o}{=}\PY{l+m+mi}{0}\PY{p}{)}
          \PY{n}{normed\PYZus{}sums}
\end{Verbatim}

            \begin{Verbatim}[commandchars=\\\{\}]
{\color{outcolor}Out[{\color{outcolor}102}]:} cand\_nm              Obama, Barack  Romney, Mitt
          contb\_receipt\_amt                               
          (0, 1]                    0.805182      0.194818
          (1, 10]                   0.918767      0.081233
          (10, 100]                 0.910769      0.089231
          (100, 1000]               0.710176      0.289824
          (1000, 10000]             0.447326      0.552674
          (10000, 100000]           0.823120      0.176880
          (100000, 1000000]         1.000000           NaN
          (1000000, 10000000]       1.000000           NaN
\end{Verbatim}
        
    \begin{Verbatim}[commandchars=\\\{\}]
{\color{incolor}In [{\color{incolor}103}]:} \PY{n}{normed\PYZus{}sums}\PY{p}{[}\PY{p}{:}\PY{o}{\PYZhy{}}\PY{l+m+mi}{2}\PY{p}{]}\PY{o}{.}\PY{n}{plot}\PY{p}{(}\PY{n}{kind}\PY{o}{=}\PY{l+s}{\PYZsq{}}\PY{l+s}{barh}\PY{l+s}{\PYZsq{}}\PY{p}{,} \PY{n}{stacked}\PY{o}{=}\PY{n+nb+bp}{True}\PY{p}{)}
\end{Verbatim}

            \begin{Verbatim}[commandchars=\\\{\}]
{\color{outcolor}Out[{\color{outcolor}103}]:} <matplotlib.axes.\_subplots.AxesSubplot at 0x7f38001d0b90>
\end{Verbatim}
        
    \begin{center}
    \adjustimage{max size={0.9\linewidth}{0.9\paperheight}}{Acquiring and wrangling with data_files/Acquiring and wrangling with data_211_1.png}
    \end{center}
    { \hspace*{\fill} \\}
    
    \subsubsection{Donation statistics by
state}\label{donation-statistics-by-state}

    We can also aggregate donations by candidate and state:

    \begin{Verbatim}[commandchars=\\\{\}]
{\color{incolor}In [{\color{incolor}104}]:} \PY{n}{grouped} \PY{o}{=} \PY{n}{fec\PYZus{}mrbo}\PY{o}{.}\PY{n}{groupby}\PY{p}{(}\PY{p}{[}\PY{l+s}{\PYZsq{}}\PY{l+s}{cand\PYZus{}nm}\PY{l+s}{\PYZsq{}}\PY{p}{,} \PY{l+s}{\PYZsq{}}\PY{l+s}{contbr\PYZus{}st}\PY{l+s}{\PYZsq{}}\PY{p}{]}\PY{p}{)}
          \PY{n}{totals} \PY{o}{=} \PY{n}{grouped}\PY{o}{.}\PY{n}{contb\PYZus{}receipt\PYZus{}amt}\PY{o}{.}\PY{n}{sum}\PY{p}{(}\PY{p}{)}\PY{o}{.}\PY{n}{unstack}\PY{p}{(}\PY{l+m+mi}{0}\PY{p}{)}\PY{o}{.}\PY{n}{fillna}\PY{p}{(}\PY{l+m+mi}{0}\PY{p}{)}
          \PY{n}{totals} \PY{o}{=} \PY{n}{totals}\PY{p}{[}\PY{n}{totals}\PY{o}{.}\PY{n}{sum}\PY{p}{(}\PY{l+m+mi}{1}\PY{p}{)} \PY{o}{\PYZgt{}} \PY{l+m+mi}{100000}\PY{p}{]}
          \PY{n}{totals}\PY{p}{[}\PY{p}{:}\PY{l+m+mi}{10}\PY{p}{]}
\end{Verbatim}

            \begin{Verbatim}[commandchars=\\\{\}]
{\color{outcolor}Out[{\color{outcolor}104}]:} cand\_nm    Obama, Barack  Romney, Mitt
          contbr\_st                             
          AK             281840.15      86204.24
          AL             543123.48     527303.51
          AR             359247.28     105556.00
          AZ            1506476.98    1888436.23
          CA           23824984.24   11237636.60
          CO            2132429.49    1506714.12
          CT            2068291.26    3499475.45
          DC            4373538.80    1025137.50
          DE             336669.14      82712.00
          FL            7318178.58    8338458.81
\end{Verbatim}
        
    Additionally, we may obtain the relative percentage of total donations
by state for each candidate.

    \begin{Verbatim}[commandchars=\\\{\}]
{\color{incolor}In [{\color{incolor}105}]:} \PY{n}{percent} \PY{o}{=} \PY{n}{totals}\PY{o}{.}\PY{n}{div}\PY{p}{(}\PY{n}{totals}\PY{o}{.}\PY{n}{sum}\PY{p}{(}\PY{l+m+mi}{1}\PY{p}{)}\PY{p}{,} \PY{n}{axis}\PY{o}{=}\PY{l+m+mi}{0}\PY{p}{)}
          \PY{n}{percent}\PY{p}{[}\PY{p}{:}\PY{l+m+mi}{10}\PY{p}{]}
\end{Verbatim}

            \begin{Verbatim}[commandchars=\\\{\}]
{\color{outcolor}Out[{\color{outcolor}105}]:} cand\_nm    Obama, Barack  Romney, Mitt
          contbr\_st                             
          AK              0.765778      0.234222
          AL              0.507390      0.492610
          AR              0.772902      0.227098
          AZ              0.443745      0.556255
          CA              0.679498      0.320502
          CO              0.585970      0.414030
          CT              0.371476      0.628524
          DC              0.810113      0.189887
          DE              0.802776      0.197224
          FL              0.467417      0.532583
\end{Verbatim}
        

    % Add a bibliography block to the postdoc
    
    
    
    \end{document}
