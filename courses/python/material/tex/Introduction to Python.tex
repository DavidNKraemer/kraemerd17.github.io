
% Default to the notebook output style

    


% Inherit from the specified cell style.




    
\documentclass{article}

    
    
    \usepackage{graphicx} % Used to insert images
    \usepackage{adjustbox} % Used to constrain images to a maximum size 
    \usepackage{color} % Allow colors to be defined
    \usepackage{enumerate} % Needed for markdown enumerations to work
    \usepackage{geometry} % Used to adjust the document margins
    \usepackage{amsmath} % Equations
    \usepackage{amssymb} % Equations
    \usepackage{eurosym} % defines \euro
    \usepackage[mathletters]{ucs} % Extended unicode (utf-8) support
    \usepackage[utf8x]{inputenc} % Allow utf-8 characters in the tex document
    \usepackage{fancyvrb} % verbatim replacement that allows latex
    \usepackage{grffile} % extends the file name processing of package graphics 
                         % to support a larger range 
    % The hyperref package gives us a pdf with properly built
    % internal navigation ('pdf bookmarks' for the table of contents,
    % internal cross-reference links, web links for URLs, etc.)
    \usepackage{hyperref}
    \usepackage{longtable} % longtable support required by pandoc >1.10
    \usepackage{booktabs}  % table support for pandoc > 1.12.2
    

    
    
    \definecolor{orange}{cmyk}{0,0.4,0.8,0.2}
    \definecolor{darkorange}{rgb}{.71,0.21,0.01}
    \definecolor{darkgreen}{rgb}{.12,.54,.11}
    \definecolor{myteal}{rgb}{.26, .44, .56}
    \definecolor{gray}{gray}{0.45}
    \definecolor{lightgray}{gray}{.95}
    \definecolor{mediumgray}{gray}{.8}
    \definecolor{inputbackground}{rgb}{.95, .95, .85}
    \definecolor{outputbackground}{rgb}{.95, .95, .95}
    \definecolor{traceback}{rgb}{1, .95, .95}
    % ansi colors
    \definecolor{red}{rgb}{.6,0,0}
    \definecolor{green}{rgb}{0,.65,0}
    \definecolor{brown}{rgb}{0.6,0.6,0}
    \definecolor{blue}{rgb}{0,.145,.698}
    \definecolor{purple}{rgb}{.698,.145,.698}
    \definecolor{cyan}{rgb}{0,.698,.698}
    \definecolor{lightgray}{gray}{0.5}
    
    % bright ansi colors
    \definecolor{darkgray}{gray}{0.25}
    \definecolor{lightred}{rgb}{1.0,0.39,0.28}
    \definecolor{lightgreen}{rgb}{0.48,0.99,0.0}
    \definecolor{lightblue}{rgb}{0.53,0.81,0.92}
    \definecolor{lightpurple}{rgb}{0.87,0.63,0.87}
    \definecolor{lightcyan}{rgb}{0.5,1.0,0.83}
    
    % commands and environments needed by pandoc snippets
    % extracted from the output of `pandoc -s`
    \DefineVerbatimEnvironment{Highlighting}{Verbatim}{commandchars=\\\{\}}
    % Add ',fontsize=\small' for more characters per line
    \newenvironment{Shaded}{}{}
    \newcommand{\KeywordTok}[1]{\textcolor[rgb]{0.00,0.44,0.13}{\textbf{{#1}}}}
    \newcommand{\DataTypeTok}[1]{\textcolor[rgb]{0.56,0.13,0.00}{{#1}}}
    \newcommand{\DecValTok}[1]{\textcolor[rgb]{0.25,0.63,0.44}{{#1}}}
    \newcommand{\BaseNTok}[1]{\textcolor[rgb]{0.25,0.63,0.44}{{#1}}}
    \newcommand{\FloatTok}[1]{\textcolor[rgb]{0.25,0.63,0.44}{{#1}}}
    \newcommand{\CharTok}[1]{\textcolor[rgb]{0.25,0.44,0.63}{{#1}}}
    \newcommand{\StringTok}[1]{\textcolor[rgb]{0.25,0.44,0.63}{{#1}}}
    \newcommand{\CommentTok}[1]{\textcolor[rgb]{0.38,0.63,0.69}{\textit{{#1}}}}
    \newcommand{\OtherTok}[1]{\textcolor[rgb]{0.00,0.44,0.13}{{#1}}}
    \newcommand{\AlertTok}[1]{\textcolor[rgb]{1.00,0.00,0.00}{\textbf{{#1}}}}
    \newcommand{\FunctionTok}[1]{\textcolor[rgb]{0.02,0.16,0.49}{{#1}}}
    \newcommand{\RegionMarkerTok}[1]{{#1}}
    \newcommand{\ErrorTok}[1]{\textcolor[rgb]{1.00,0.00,0.00}{\textbf{{#1}}}}
    \newcommand{\NormalTok}[1]{{#1}}
    
    % Define a nice break command that doesn't care if a line doesn't already
    % exist.
    \def\br{\hspace*{\fill} \\* }
    % Math Jax compatability definitions
    \def\gt{>}
    \def\lt{<}
    % Document parameters
    \title{Introduction to \texttt{Python}}
    
    
    

    % Pygments definitions
    
\makeatletter
\def\PY@reset{\let\PY@it=\relax \let\PY@bf=\relax%
    \let\PY@ul=\relax \let\PY@tc=\relax%
    \let\PY@bc=\relax \let\PY@ff=\relax}
\def\PY@tok#1{\csname PY@tok@#1\endcsname}
\def\PY@toks#1+{\ifx\relax#1\empty\else%
    \PY@tok{#1}\expandafter\PY@toks\fi}
\def\PY@do#1{\PY@bc{\PY@tc{\PY@ul{%
    \PY@it{\PY@bf{\PY@ff{#1}}}}}}}
\def\PY#1#2{\PY@reset\PY@toks#1+\relax+\PY@do{#2}}

\expandafter\def\csname PY@tok@gd\endcsname{\def\PY@tc##1{\textcolor[rgb]{0.63,0.00,0.00}{##1}}}
\expandafter\def\csname PY@tok@gu\endcsname{\let\PY@bf=\textbf\def\PY@tc##1{\textcolor[rgb]{0.50,0.00,0.50}{##1}}}
\expandafter\def\csname PY@tok@gt\endcsname{\def\PY@tc##1{\textcolor[rgb]{0.00,0.27,0.87}{##1}}}
\expandafter\def\csname PY@tok@gs\endcsname{\let\PY@bf=\textbf}
\expandafter\def\csname PY@tok@gr\endcsname{\def\PY@tc##1{\textcolor[rgb]{1.00,0.00,0.00}{##1}}}
\expandafter\def\csname PY@tok@cm\endcsname{\let\PY@it=\textit\def\PY@tc##1{\textcolor[rgb]{0.25,0.50,0.50}{##1}}}
\expandafter\def\csname PY@tok@vg\endcsname{\def\PY@tc##1{\textcolor[rgb]{0.10,0.09,0.49}{##1}}}
\expandafter\def\csname PY@tok@m\endcsname{\def\PY@tc##1{\textcolor[rgb]{0.40,0.40,0.40}{##1}}}
\expandafter\def\csname PY@tok@mh\endcsname{\def\PY@tc##1{\textcolor[rgb]{0.40,0.40,0.40}{##1}}}
\expandafter\def\csname PY@tok@go\endcsname{\def\PY@tc##1{\textcolor[rgb]{0.53,0.53,0.53}{##1}}}
\expandafter\def\csname PY@tok@ge\endcsname{\let\PY@it=\textit}
\expandafter\def\csname PY@tok@vc\endcsname{\def\PY@tc##1{\textcolor[rgb]{0.10,0.09,0.49}{##1}}}
\expandafter\def\csname PY@tok@il\endcsname{\def\PY@tc##1{\textcolor[rgb]{0.40,0.40,0.40}{##1}}}
\expandafter\def\csname PY@tok@cs\endcsname{\let\PY@it=\textit\def\PY@tc##1{\textcolor[rgb]{0.25,0.50,0.50}{##1}}}
\expandafter\def\csname PY@tok@cp\endcsname{\def\PY@tc##1{\textcolor[rgb]{0.74,0.48,0.00}{##1}}}
\expandafter\def\csname PY@tok@gi\endcsname{\def\PY@tc##1{\textcolor[rgb]{0.00,0.63,0.00}{##1}}}
\expandafter\def\csname PY@tok@gh\endcsname{\let\PY@bf=\textbf\def\PY@tc##1{\textcolor[rgb]{0.00,0.00,0.50}{##1}}}
\expandafter\def\csname PY@tok@ni\endcsname{\let\PY@bf=\textbf\def\PY@tc##1{\textcolor[rgb]{0.60,0.60,0.60}{##1}}}
\expandafter\def\csname PY@tok@nl\endcsname{\def\PY@tc##1{\textcolor[rgb]{0.63,0.63,0.00}{##1}}}
\expandafter\def\csname PY@tok@nn\endcsname{\let\PY@bf=\textbf\def\PY@tc##1{\textcolor[rgb]{0.00,0.00,1.00}{##1}}}
\expandafter\def\csname PY@tok@no\endcsname{\def\PY@tc##1{\textcolor[rgb]{0.53,0.00,0.00}{##1}}}
\expandafter\def\csname PY@tok@na\endcsname{\def\PY@tc##1{\textcolor[rgb]{0.49,0.56,0.16}{##1}}}
\expandafter\def\csname PY@tok@nb\endcsname{\def\PY@tc##1{\textcolor[rgb]{0.00,0.50,0.00}{##1}}}
\expandafter\def\csname PY@tok@nc\endcsname{\let\PY@bf=\textbf\def\PY@tc##1{\textcolor[rgb]{0.00,0.00,1.00}{##1}}}
\expandafter\def\csname PY@tok@nd\endcsname{\def\PY@tc##1{\textcolor[rgb]{0.67,0.13,1.00}{##1}}}
\expandafter\def\csname PY@tok@ne\endcsname{\let\PY@bf=\textbf\def\PY@tc##1{\textcolor[rgb]{0.82,0.25,0.23}{##1}}}
\expandafter\def\csname PY@tok@nf\endcsname{\def\PY@tc##1{\textcolor[rgb]{0.00,0.00,1.00}{##1}}}
\expandafter\def\csname PY@tok@si\endcsname{\let\PY@bf=\textbf\def\PY@tc##1{\textcolor[rgb]{0.73,0.40,0.53}{##1}}}
\expandafter\def\csname PY@tok@s2\endcsname{\def\PY@tc##1{\textcolor[rgb]{0.73,0.13,0.13}{##1}}}
\expandafter\def\csname PY@tok@vi\endcsname{\def\PY@tc##1{\textcolor[rgb]{0.10,0.09,0.49}{##1}}}
\expandafter\def\csname PY@tok@nt\endcsname{\let\PY@bf=\textbf\def\PY@tc##1{\textcolor[rgb]{0.00,0.50,0.00}{##1}}}
\expandafter\def\csname PY@tok@nv\endcsname{\def\PY@tc##1{\textcolor[rgb]{0.10,0.09,0.49}{##1}}}
\expandafter\def\csname PY@tok@s1\endcsname{\def\PY@tc##1{\textcolor[rgb]{0.73,0.13,0.13}{##1}}}
\expandafter\def\csname PY@tok@kd\endcsname{\let\PY@bf=\textbf\def\PY@tc##1{\textcolor[rgb]{0.00,0.50,0.00}{##1}}}
\expandafter\def\csname PY@tok@sh\endcsname{\def\PY@tc##1{\textcolor[rgb]{0.73,0.13,0.13}{##1}}}
\expandafter\def\csname PY@tok@sc\endcsname{\def\PY@tc##1{\textcolor[rgb]{0.73,0.13,0.13}{##1}}}
\expandafter\def\csname PY@tok@sx\endcsname{\def\PY@tc##1{\textcolor[rgb]{0.00,0.50,0.00}{##1}}}
\expandafter\def\csname PY@tok@bp\endcsname{\def\PY@tc##1{\textcolor[rgb]{0.00,0.50,0.00}{##1}}}
\expandafter\def\csname PY@tok@c1\endcsname{\let\PY@it=\textit\def\PY@tc##1{\textcolor[rgb]{0.25,0.50,0.50}{##1}}}
\expandafter\def\csname PY@tok@kc\endcsname{\let\PY@bf=\textbf\def\PY@tc##1{\textcolor[rgb]{0.00,0.50,0.00}{##1}}}
\expandafter\def\csname PY@tok@c\endcsname{\let\PY@it=\textit\def\PY@tc##1{\textcolor[rgb]{0.25,0.50,0.50}{##1}}}
\expandafter\def\csname PY@tok@mf\endcsname{\def\PY@tc##1{\textcolor[rgb]{0.40,0.40,0.40}{##1}}}
\expandafter\def\csname PY@tok@err\endcsname{\def\PY@bc##1{\setlength{\fboxsep}{0pt}\fcolorbox[rgb]{1.00,0.00,0.00}{1,1,1}{\strut ##1}}}
\expandafter\def\csname PY@tok@mb\endcsname{\def\PY@tc##1{\textcolor[rgb]{0.40,0.40,0.40}{##1}}}
\expandafter\def\csname PY@tok@ss\endcsname{\def\PY@tc##1{\textcolor[rgb]{0.10,0.09,0.49}{##1}}}
\expandafter\def\csname PY@tok@sr\endcsname{\def\PY@tc##1{\textcolor[rgb]{0.73,0.40,0.53}{##1}}}
\expandafter\def\csname PY@tok@mo\endcsname{\def\PY@tc##1{\textcolor[rgb]{0.40,0.40,0.40}{##1}}}
\expandafter\def\csname PY@tok@kn\endcsname{\let\PY@bf=\textbf\def\PY@tc##1{\textcolor[rgb]{0.00,0.50,0.00}{##1}}}
\expandafter\def\csname PY@tok@mi\endcsname{\def\PY@tc##1{\textcolor[rgb]{0.40,0.40,0.40}{##1}}}
\expandafter\def\csname PY@tok@gp\endcsname{\let\PY@bf=\textbf\def\PY@tc##1{\textcolor[rgb]{0.00,0.00,0.50}{##1}}}
\expandafter\def\csname PY@tok@o\endcsname{\def\PY@tc##1{\textcolor[rgb]{0.40,0.40,0.40}{##1}}}
\expandafter\def\csname PY@tok@kr\endcsname{\let\PY@bf=\textbf\def\PY@tc##1{\textcolor[rgb]{0.00,0.50,0.00}{##1}}}
\expandafter\def\csname PY@tok@s\endcsname{\def\PY@tc##1{\textcolor[rgb]{0.73,0.13,0.13}{##1}}}
\expandafter\def\csname PY@tok@kp\endcsname{\def\PY@tc##1{\textcolor[rgb]{0.00,0.50,0.00}{##1}}}
\expandafter\def\csname PY@tok@w\endcsname{\def\PY@tc##1{\textcolor[rgb]{0.73,0.73,0.73}{##1}}}
\expandafter\def\csname PY@tok@kt\endcsname{\def\PY@tc##1{\textcolor[rgb]{0.69,0.00,0.25}{##1}}}
\expandafter\def\csname PY@tok@ow\endcsname{\let\PY@bf=\textbf\def\PY@tc##1{\textcolor[rgb]{0.67,0.13,1.00}{##1}}}
\expandafter\def\csname PY@tok@sb\endcsname{\def\PY@tc##1{\textcolor[rgb]{0.73,0.13,0.13}{##1}}}
\expandafter\def\csname PY@tok@k\endcsname{\let\PY@bf=\textbf\def\PY@tc##1{\textcolor[rgb]{0.00,0.50,0.00}{##1}}}
\expandafter\def\csname PY@tok@se\endcsname{\let\PY@bf=\textbf\def\PY@tc##1{\textcolor[rgb]{0.73,0.40,0.13}{##1}}}
\expandafter\def\csname PY@tok@sd\endcsname{\let\PY@it=\textit\def\PY@tc##1{\textcolor[rgb]{0.73,0.13,0.13}{##1}}}

\def\PYZbs{\char`\\}
\def\PYZus{\char`\_}
\def\PYZob{\char`\{}
\def\PYZcb{\char`\}}
\def\PYZca{\char`\^}
\def\PYZam{\char`\&}
\def\PYZlt{\char`\<}
\def\PYZgt{\char`\>}
\def\PYZsh{\char`\#}
\def\PYZpc{\char`\%}
\def\PYZdl{\char`\$}
\def\PYZhy{\char`\-}
\def\PYZsq{\char`\'}
\def\PYZdq{\char`\"}
\def\PYZti{\char`\~}
% for compatibility with earlier versions
\def\PYZat{@}
\def\PYZlb{[}
\def\PYZrb{]}
\makeatother


    % Exact colors from NB
    \definecolor{incolor}{rgb}{0.0, 0.0, 0.5}
    \definecolor{outcolor}{rgb}{0.545, 0.0, 0.0}



    
    % Prevent overflowing lines due to hard-to-break entities
    \sloppy 
    % Setup hyperref package
    \hypersetup{
      breaklinks=true,  % so long urls are correctly broken across lines
      colorlinks=true,
      urlcolor=blue,
      linkcolor=darkorange,
      citecolor=darkgreen,
      }
    % Slightly bigger margins than the latex defaults
    
    \geometry{verbose,tmargin=1in,bmargin=1in,lmargin=1in,rmargin=1in}
    
    

    \begin{document}
    
    
    \maketitle
    
    

    
    \section{Outline}\label{outline}

    \begin{itemize}
\itemsep1pt\parskip0pt\parsep0pt
\item
  Introduction to \texttt{Python}
\item
  Installing \texttt{Python}
\item
  Installing \texttt{IPython}
\item
  Installing additional libraries

  \begin{itemize}
  \itemsep1pt\parskip0pt\parsep0pt
  \item
    \texttt{NumPy}
  \item
    \texttt{pandas}
  \item
    \texttt{Matplotlib}
  \end{itemize}
\item
  A brief overview of syntax in \texttt{Python}
\item
  Some Labs
\end{itemize}

    \section{Installing \texttt{Python}}\label{installing-python}

For this course, we will use \texttt{Python 2.7} over
\texttt{Python 3.4}. Why is this? No particular reason, although
\texttt{Python 2.7} is still the overall favorite of most developers. We
will only be covering tools that are agnostic to the \texttt{2.7-3.4}
war, so it shouldn't matter much.

If you plan to install \texttt{Python} to your computer locally, you
will have to install not only the \texttt{Python} libraries and
interpreters themselves but the additional software that we will use
throughout the course. This is a huge pain, and while this can be done
manually, we highly recommend you use a third-party package management
program.

We have experimented with each of the different package management
tools, and our favorite is Anaconda, provided by Continuum analytics.
However, you can make your set-up work with any package manager. We just
think it's the easiest with Anaconda.

    \subsection{Anaconda}\label{anaconda}

To install Anaconda, go the
\href{https://store.continuum.io/cshop/anaconda/}{Continuum Store} and
select ``Download Anaconda.'' This will require you to enter an email
address, though this does not have to be linked to promotional emails.
After doing this, you will be sent to a page to download the correct
version for your local operating system. Download the correct installer
for your OS, and run the corresponding installation programs once it is
complete.

For Unix users, this means moving to the Downloads directory and
entering

\begin{Shaded}
\begin{Highlighting}[]
\KeywordTok{bash} \NormalTok{Anaconda-2.1.0-[Linux or MacOSX]-x86_[32 or 64].sh}
\end{Highlighting}
\end{Shaded}

The instructions are on the download page. For Windows users, just
double-click the \texttt{.exe} file and follow the requisite
instructions.

Once Anaconda is installed, open a terminal and type

\begin{Shaded}
\begin{Highlighting}[]
\KeywordTok{conda} \NormalTok{update conda}
\KeywordTok{conda} \NormalTok{update ipython ipython-notebook ipython-qtconsole}
\end{Highlighting}
\end{Shaded}

This should install the IPython software we will use throughout the
course, while also updating all of the \texttt{Python} libraries.

    \subsubsection{Wakari Cloud}\label{wakari-cloud}

If you don't want to install \texttt{Python} packages locally, Continuum
Analytics has a cloud-based solution, Wakari. It's not designed for
extensive individual use, and it hasn't updated IPython to the latest
version, but it is permissible for our course. To get started, make an
account on \href{https://wakari.io/}{Wakari.io}, which will give you
your own directory for IPython projects.

    \subsection{Canopy}\label{canopy}

Canopy is an alternative to Anaconda, which includes its own integrated
development environment on top of a package manager. We don't recommend
using Canopy because we found it difficult to update to the latest
version of the IPython notebook, and there seems to be issues with the
\texttt{matplotlib} engine outside of the Canopy IDE.

However, if you desire, you can install Canopy by going to the
\href{https://store.enthought.com/downloads/}{Enthought Store} and
selecting ``DOWNLOAD Canopy Express''. This will require that you make
an account with Enthought. One benefit of making an account and then
identifying yourself as a student is that it can give you access to
tutorial videos on \texttt{Python} and its data analysis libraries (this
privilege can be finicky, however).

From there, simply install the Canopy program that you download from the
site, and open it. Canopy gives a graphical package manager, which you
might find appealing (we don't).

    \subsection{Manual installation}\label{manual-installation}

Manual installation is, simply put, a pain in the neck. We don't
recommend doing this, because it is quite complicated. To be fair, we
will call using \texttt{pip} as manual installation, even though this
simplifies true manual installation quite significantly. In fact, we
won't go into details (to discourage you from trying). It can be done,
by following (roughly) these steps:

\begin{itemize}
\item
  Downloading and installing \texttt{Python 2.7} (or \texttt{3.4}) from
  the \href{https://www.python.org/downloads/}{Official Website}
\item
  Hopefully, after the installation, you can call the program
  \texttt{python} from the terminal. This is not guaranteed. If you are
  lucky, download
  \href{https://bootstrap.pypa.io/get-pip.py}{get-pip.py}, and run
  \texttt{python get-pip.py} from the relevant directory.
\item
  This should install \texttt{pip}. From here, run

\begin{Shaded}
\begin{Highlighting}[]
\KeywordTok{pip} \NormalTok{install numpy}
\KeywordTok{pip} \NormalTok{install scipy}
\KeywordTok{pip} \NormalTok{install matplotlib}
\KeywordTok{pip} \NormalTok{install pandas}
\KeywordTok{pip} \NormalTok{install ipython[all]}
\end{Highlighting}
\end{Shaded}
\item
  If the gods are smiling on you, this will install everything you need
  for this class, and you should be good to go.
\end{itemize}

    \subsection{A simple \texttt{Python}
script}\label{a-simple-python-script}

Open a terminal (Windows users: your ``terminal'' is technically called
the Command Prompt, \texttt{cmd.exe}, but we will use ``terminal''
colloquially) and enter \texttt{ipython}. Check to see that the packages
we will be using in this course are properly installed by entering the
following code into the terminal:

\begin{Shaded}
\begin{Highlighting}[]
\KeywordTok{ipython}
\end{Highlighting}
\end{Shaded}

If everything is installed properly, this should launch the
\texttt{IPython} shell. This is a simple interpreter, which is handy to
use from time to time, but we will not use it in this course. However,
we can take this opportunity to write a simple line of code into the
interpreter:

\begin{Shaded}
\begin{Highlighting}[]
\DataTypeTok{print} \StringTok{"Hello, world!"}
\end{Highlighting}
\end{Shaded}

Congratulations, you have officially been inducted into the wonderful
world of computing in \texttt{Python}!

\section{Python basics}\label{python-basics}

\texttt{Python} works like other programming languages in that it has
some built-in types:

\begin{itemize}
\itemsep1pt\parskip0pt\parsep0pt
\item
  \texttt{int}, an integer type (default precision: 32-bit)
\item
  \texttt{float}, a floating-point type (default precision: 64-bit)
\item
  \texttt{long}, an integer type (unlimited precision)
\item
  \texttt{str}, a string literal type
\item
  \texttt{Boolean}, a boolean type
\end{itemize}

Variable initialization in \texttt{Python} is different than the
\texttt{C} family of languages because variable type is \textbf{not}
specified. In \texttt{C}, you have to declare the type immediately:

\begin{Shaded}
\begin{Highlighting}[]
\DataTypeTok{double} \NormalTok{var = }\FloatTok{1.61}\NormalTok{;}
\end{Highlighting}
\end{Shaded}

The equivalent \texttt{Python} code is

\begin{Shaded}
\begin{Highlighting}[]
\NormalTok{var = }\FloatTok{1.61}
\end{Highlighting}
\end{Shaded}

\subsubsection{Try it!}\label{try-it}

Open an \texttt{IPython Notebook} file.

\begin{Shaded}
\begin{Highlighting}[]
\KeywordTok{ipython} \NormalTok{notebook}
\end{Highlighting}
\end{Shaded}

Use the next few examples to get a feel for \texttt{Python}'s style of
variables.

    \begin{Verbatim}[commandchars=\\\{\}]
{\color{incolor}In [{\color{incolor}6}]:} \PY{n}{a} \PY{o}{=} \PY{l+m+mi}{10}
        \PY{n}{b} \PY{o}{=} \PY{l+m+mi}{20}
        \PY{k}{print} \PY{n}{a} \PY{o}{+} \PY{n}{b}
        \PY{k}{print} \PY{n}{a} \PY{o}{==} \PY{n}{b}
        \PY{k}{print} \PY{o+ow}{not} \PY{p}{(} \PY{n}{a} \PY{o}{\PYZgt{}} \PY{n}{b}\PY{p}{)}
\end{Verbatim}

    \begin{Verbatim}[commandchars=\\\{\}]
30
False
True
    \end{Verbatim}

    \subsection{Arithmetic}\label{arithmetic}

\texttt{Python} arithmetic is what you would expect. \texttt{+, -, *, /}
are all defined as normal, but the operation \texttt{x ** y} is
equivalent to $x^y$.

\subsection{Pointers in Python:}\label{pointers-in-python}

What do you expect will happen when you run the following code?

    \begin{Verbatim}[commandchars=\\\{\}]
{\color{incolor}In [{\color{incolor}7}]:} \PY{n}{x} \PY{o}{=} \PY{l+m+mf}{10.42}
        \PY{n}{y} \PY{o}{=} \PY{n}{x}
        \PY{n}{x} \PY{o}{+}\PY{o}{=} \PY{l+m+mf}{12.}
        \PY{k}{print} \PY{n}{y}
        
        \PY{n}{str1} \PY{o}{=} \PY{l+s}{\PYZdq{}}\PY{l+s}{Hello}\PY{l+s}{\PYZdq{}}
        \PY{n}{str2} \PY{o}{=} \PY{n}{str1}
        \PY{n}{str1} \PY{o}{+}\PY{o}{=} \PY{l+s}{\PYZdq{}}\PY{l+s}{ world!}\PY{l+s}{\PYZdq{}}
        \PY{k}{print} \PY{n}{str2}
\end{Verbatim}

    \begin{Verbatim}[commandchars=\\\{\}]
10.42
Hello
    \end{Verbatim}

    In \texttt{Python}, everything is a pointer, but \texttt{Python} will
try to behave like you intend it to (no horrible \texttt{C}-like
issues).

Boolean expressions in \texttt{Python} are designed with readability in
mind. What do you expect the following output will be?

    \begin{Verbatim}[commandchars=\\\{\}]
{\color{incolor}In [{\color{incolor}8}]:} \PY{n}{x} \PY{o}{=} \PY{l+m+mf}{1.23} 
        \PY{n}{y} \PY{o}{=} \PY{o}{\PYZhy{}}\PY{l+m+mf}{0.5}
        \PY{n}{bool\PYZus{}one} \PY{o}{=} \PY{p}{(}\PY{n}{x} \PY{o}{\PYZgt{}}\PY{o}{=} \PY{n}{y}\PY{p}{)}
        \PY{n}{bool\PYZus{}two} \PY{o}{=} \PY{p}{(}\PY{n}{y} \PY{o}{*}\PY{o}{*} \PY{l+m+mi}{2} \PY{o}{+} \PY{l+m+mi}{2} \PY{o}{\PYZlt{}} \PY{n}{x}\PY{p}{)}
        
        \PY{k}{print} \PY{p}{(}\PY{o+ow}{not} \PY{n}{bool\PYZus{}one}\PY{p}{)} \PY{o+ow}{or} \PY{n}{bool\PYZus{}two}
        \PY{k}{print} \PY{n}{bool\PYZus{}one} \PY{o+ow}{is} \PY{o+ow}{not} \PY{n}{bool\PYZus{}two}
\end{Verbatim}

    \begin{Verbatim}[commandchars=\\\{\}]
False
True
    \end{Verbatim}

    \subsection{Lists}\label{lists}

There is not a native array type in \texttt{Python} like there is in
\texttt{C} and \texttt{Java}. Instead, \texttt{Python} follows a similar
model to \texttt{Scheme} in that the core native ``iterable'' data type
is a (linked) list. Lists are initialized in \texttt{Python} by

\begin{Shaded}
\begin{Highlighting}[]
\NormalTok{my_list = [}\DecValTok{1}\NormalTok{, }\DecValTok{2}\NormalTok{, }\DecValTok{3}\NormalTok{, }\DecValTok{4}\NormalTok{, }\DecValTok{5}\NormalTok{]}
\end{Highlighting}
\end{Shaded}

Lists indices begin at 0 (Sorry, \texttt{MATLAB} and \texttt{R} users),
so \texttt{my\_list{[}0{]} = 1}. Lists are endowed with some very useful
functions:

\begin{itemize}
\itemsep1pt\parskip0pt\parsep0pt
\item
  \texttt{append}, which adds an element to the end of the list.
\item
  \texttt{index}, which searches a list for a particular value.
\item
  \texttt{insert}, which adds an element after a specified location.
\end{itemize}

One important concept in \texttt{Python} is list \emph{slicing}, which
allows you to quickly access sections of a list without much code
overhead. For example, if I wish to obtain the middle three elements of
\texttt{my\_list}, I can simply type

\begin{Shaded}
\begin{Highlighting}[]
\NormalTok{mid_three = my_list[}\DecValTok{1}\NormalTok{:}\DecValTok{4}\NormalTok{:}\DecValTok{1}\NormalTok{]}
\end{Highlighting}
\end{Shaded}

The braces \texttt{{[}1:4:1{]}} should be read as telling the list to
start at the 1th (that is, second) index and go until the 4th (that is,
fifth) index, incrementing by 1. So, the 1 is ``inclusive'', whereas the
4 is ``exclusive.'' One can reach the last element of a list easily
without knowing the length of the list by writing
\texttt{my\_list{[}-1{]}}. In general, the \texttt{-} indicates that you
want to traverse the list in the opposite direction.

\subsubsection{Try it!}\label{try-it}

What do you expect the following code will produce?

    \begin{Verbatim}[commandchars=\\\{\}]
{\color{incolor}In [{\color{incolor}9}]:} \PY{n}{example} \PY{o}{=} \PY{p}{[}\PY{l+m+mf}{3.}\PY{p}{,} \PY{l+m+mf}{5.}\PY{p}{,} \PY{l+m+mf}{7.}\PY{p}{,} \PY{l+m+mf}{9.}\PY{p}{,} \PY{l+m+mf}{11.}\PY{p}{,} \PY{l+m+mf}{14.}\PY{p}{]}
        
        \PY{n}{example}\PY{o}{.}\PY{n}{append}\PY{p}{(}\PY{l+m+mf}{82.}\PY{p}{)}
        
        \PY{n}{example}\PY{o}{.}\PY{n}{insert}\PY{p}{(}\PY{l+m+mi}{3}\PY{p}{,} \PY{l+m+mf}{1.}\PY{p}{)}
        
        \PY{k}{print} \PY{l+s}{\PYZdq{}}\PY{l+s}{Example: }\PY{l+s+se}{\PYZbs{}t}\PY{l+s}{\PYZdq{}}\PY{p}{,} \PY{n}{example}
        \PY{k}{print} \PY{l+s}{\PYZdq{}}\PY{l+s+se}{\PYZbs{}t}\PY{l+s+se}{\PYZbs{}t}\PY{l+s}{\PYZdq{}}\PY{p}{,} \PY{n}{example}\PY{p}{[}\PY{l+m+mi}{0}\PY{p}{:}\PY{l+m+mi}{2}\PY{p}{]} 
        \PY{k}{print} \PY{l+s}{\PYZdq{}}\PY{l+s+se}{\PYZbs{}t}\PY{l+s+se}{\PYZbs{}t}\PY{l+s}{\PYZdq{}}\PY{p}{,} \PY{n}{example}\PY{p}{[}\PY{l+m+mi}{3}\PY{p}{:}\PY{p}{]}
        \PY{k}{print} \PY{l+s}{\PYZdq{}}\PY{l+s+se}{\PYZbs{}t}\PY{l+s+se}{\PYZbs{}t}\PY{l+s}{\PYZdq{}}\PY{p}{,} \PY{n}{example}\PY{p}{[}\PY{p}{:}\PY{p}{:}\PY{o}{\PYZhy{}}\PY{l+m+mi}{1}\PY{p}{]}
\end{Verbatim}

    \begin{Verbatim}[commandchars=\\\{\}]
Example: 	[3.0, 5.0, 7.0, 1.0, 9.0, 11.0, 14.0, 82.0]
		[3.0, 5.0]
		[1.0, 9.0, 11.0, 14.0, 82.0]
		[82.0, 14.0, 11.0, 9.0, 1.0, 7.0, 5.0, 3.0]
    \end{Verbatim}

    \subsection{Conditionals}\label{conditionals}

\texttt{Python} uses \texttt{if-elif-else} branching very much like the
\texttt{C} family uses \texttt{if-else if-else} statements. There is no
\texttt{Python} equivalent to \texttt{switch}.

\begin{Shaded}
\begin{Highlighting}[]
\KeywordTok{if} \OtherTok{True}\NormalTok{:}
    \DataTypeTok{print} \StringTok{"this will happen"}
\KeywordTok{else}\NormalTok{: }
    \DataTypeTok{print} \StringTok{"this will not happen"}
\end{Highlighting}
\end{Shaded}

Notice that \texttt{Python} is sensitive to white-spaces. The
\texttt{IPython Notebook} is clever enough to handle proper spacing for
you.

\subsubsection{Try it!}\label{try-it}

What do you think will happen when you execute the following snippets of
code?

    \begin{Verbatim}[commandchars=\\\{\}]
{\color{incolor}In [{\color{incolor}10}]:} \PY{k}{if} \PY{l+m+mi}{1} \PY{o}{\PYZlt{}} \PY{l+m+mi}{0}\PY{p}{:}
             \PY{k}{print} \PY{l+s}{\PYZdq{}}\PY{l+s}{1 \PYZlt{} 0}\PY{l+s}{\PYZdq{}}
         \PY{k}{elif} \PY{l+m+mi}{1} \PY{o}{\PYZgt{}} \PY{l+m+mi}{2}\PY{p}{:}
             \PY{k}{print} \PY{l+s}{\PYZdq{}}\PY{l+s}{1 \PYZgt{} 2}\PY{l+s}{\PYZdq{}}
         \PY{k}{else}\PY{p}{:}
             \PY{k}{if} \PY{l+m+mi}{1} \PY{o}{\PYZlt{}}\PY{o}{=} \PY{l+m+mi}{1}\PY{p}{:}
             \PY{k}{print} \PY{l+s}{\PYZdq{}}\PY{l+s}{1 \PYZlt{}= 1}\PY{l+s}{\PYZdq{}}
\end{Verbatim}

    \begin{Verbatim}[commandchars=\\\{\}]

          File "<ipython-input-10-d824fa9aecd4>", line 7
        print "1 <= 1"
            \^{}
    IndentationError: expected an indented block


    \end{Verbatim}

    \begin{Verbatim}[commandchars=\\\{\}]
{\color{incolor}In [{\color{incolor}11}]:} \PY{k}{if} \PY{n+nb}{len}\PY{p}{(}\PY{l+s}{\PYZdq{}}\PY{l+s}{this}\PY{l+s}{\PYZdq{}}\PY{p}{)} \PY{o}{\PYZlt{}} \PY{l+m+mi}{4}\PY{p}{:}
             \PY{k}{print} \PY{l+s}{\PYZdq{}}\PY{l+s}{what do you think len() does?}\PY{l+s}{\PYZdq{}}
         \PY{k}{elif} \PY{p}{(}\PY{l+m+mi}{3} \PY{o}{+} \PY{l+m+mi}{4} \PY{o}{\PYZhy{}} \PY{l+m+mi}{6}\PY{o}{*}\PY{o}{*}\PY{l+m+mi}{2} \PY{o}{\PYZlt{}} \PY{l+m+mi}{0}\PY{p}{)}\PY{p}{:}
             \PY{k}{print} \PY{l+s}{\PYZdq{}}\PY{l+s}{7 \PYZhy{} 36 \PYZlt{} 0}\PY{l+s}{\PYZdq{}}
\end{Verbatim}

    \begin{Verbatim}[commandchars=\\\{\}]
7 - 36 < 0
    \end{Verbatim}

    The \texttt{if} condition can be used independently of \texttt{elif} and
\texttt{else}, much like in \texttt{C}. But if you use \texttt{elif},
you must use an \texttt{else}.

\subsubsection{Loops}\label{loops}

There are two main loop constructions in \texttt{Python}: \texttt{for}
and \texttt{while}. The \texttt{while} loop operates much like its
\texttt{C} counterpart does. It is given a boolean statement that, while
true, executes the subsequent code.

    \begin{Verbatim}[commandchars=\\\{\}]
{\color{incolor}In [{\color{incolor}12}]:} \PY{n}{a} \PY{o}{=} \PY{l+m+mi}{0}
         \PY{k}{while} \PY{n}{a} \PY{o}{\PYZlt{}} \PY{l+m+mi}{10}\PY{p}{:}
             \PY{k}{print} \PY{n}{a}\PY{o}{*}\PY{o}{*}\PY{l+m+mi}{2}
             \PY{n}{a} \PY{o}{+}\PY{o}{=} \PY{l+m+mi}{1}
\end{Verbatim}

    \begin{Verbatim}[commandchars=\\\{\}]
0
1
4
9
16
25
36
49
64
81
    \end{Verbatim}

    The \texttt{for} loop, by contrast, operates differently than the C
version. In C, there are three parts

\begin{Shaded}
\begin{Highlighting}[]
\KeywordTok{for} \NormalTok{(loop initialization ; condition ; update)}
\NormalTok{\{}
    \NormalTok{code;}
\NormalTok{\}}
\end{Highlighting}
\end{Shaded}

In this sense, \texttt{for} is structurally identical to \texttt{while};
it just is set up differently. In \texttt{Python}, \texttt{for} takes
some sort of \emph{iterable} type and iterates through it:

    \begin{Verbatim}[commandchars=\\\{\}]
{\color{incolor}In [{\color{incolor}13}]:} \PY{k}{for} \PY{n}{name} \PY{o+ow}{in} \PY{p}{[}\PY{l+s}{\PYZsq{}}\PY{l+s}{Adam}\PY{l+s}{\PYZsq{}}\PY{p}{,} \PY{l+s}{\PYZsq{}}\PY{l+s}{Amy}\PY{l+s}{\PYZsq{}}\PY{p}{,} \PY{l+s}{\PYZsq{}}\PY{l+s}{Alex}\PY{l+s}{\PYZsq{}}\PY{p}{,} \PY{l+s}{\PYZsq{}}\PY{l+s}{Alfonzo}\PY{l+s}{\PYZsq{}}\PY{p}{,} \PY{l+s}{\PYZsq{}}\PY{l+s}{Adrianne}\PY{l+s}{\PYZsq{}}\PY{p}{,} \PY{l+s}{\PYZsq{}}\PY{l+s}{Abbie}\PY{l+s}{\PYZsq{}}\PY{p}{,} \PY{l+s}{\PYZsq{}}\PY{l+s}{Al}\PY{l+s}{\PYZsq{}}\PY{p}{]}\PY{p}{:}
             \PY{k}{print} \PY{n}{name}\PY{p}{,} \PY{l+s}{\PYZdq{}}\PY{l+s}{is cool}\PY{l+s}{\PYZdq{}}
\end{Verbatim}

    \begin{Verbatim}[commandchars=\\\{\}]
Adam is cool
Amy is cool
Alex is cool
Alfonzo is cool
Adrianne is cool
Abbie is cool
Al is cool
    \end{Verbatim}

    The variable \texttt{name} cycles through the list of names, printing it
in each step of the loop. This is essentially the same as
\texttt{Java}'s enhanced \texttt{for} loop, and you will learn to love
it. If you don't, and you just want to have an index \texttt{i} to go
from 0 to 9, you can use

    \begin{Verbatim}[commandchars=\\\{\}]
{\color{incolor}In [{\color{incolor}14}]:} \PY{k}{for} \PY{n}{i} \PY{o+ow}{in} \PY{n+nb}{range}\PY{p}{(}\PY{l+m+mi}{10}\PY{p}{)}\PY{p}{:}
             \PY{k}{print} \PY{n}{i}
\end{Verbatim}

    \begin{Verbatim}[commandchars=\\\{\}]
0
1
2
3
4
5
6
7
8
9
    \end{Verbatim}

    This achieves the same effect as the C \texttt{for} loop.

\subsection{Functions}\label{functions}

In \texttt{Python}, functions are easy to program. The general syntax is
as follows:

\begin{Shaded}
\begin{Highlighting}[]
\KeywordTok{def} \NormalTok{foo(args):}
    \NormalTok{[do stuff]}
    \KeywordTok{return} \NormalTok{bar }\CommentTok{# optional}
\end{Highlighting}
\end{Shaded}

Simply pass the arguments you want to include into the function, do
whatever you want inside the function (don't forget the whitespace!),
and \texttt{return} as you would in \texttt{C}.

\subsubsection{Try it!}\label{try-it}

\begin{enumerate}
\def\labelenumi{\arabic{enumi}.}
\itemsep1pt\parskip0pt\parsep0pt
\item
  Write a function, \texttt{print\_args}, that prints each argument
  passed to it in one line.
\item
  Write a function, \texttt{list\_float\_to\_int}, that converts an
  integer list into a float list.
\item
  Write a function, \texttt{centered\_average}, which returns the
  arithmetic average of a list, excluding the largest and the smallest
  elements.
\end{enumerate}

\section{Lab Activity: The Bisection
Method}\label{lab-activity-the-bisection-method}

Those of you who have taken a few computer science courses may have
learned about something called the ``Binary Search'', which allows you
to quickly find an element of an unordered list. We will extend this
idea to allow us to find roots of continuous functions over a closed
interval.

The idea for this procedure comes from the Intermediate Value Theorem:
if you have a continuous function $f(x)$ on an interval $[a,b]$ for
which you know that $f(a)<0$ but $f(b)>0$, then there must be some
intermediary point $c$ satisfying $f(c)=0$. The Bisection Method
provides us with a procedure for finding such a $c$. Here's how it
works:

\begin{enumerate}
\def\labelenumi{\arabic{enumi}.}
\itemsep1pt\parskip0pt\parsep0pt
\item
  Start with a function $f(x)$ on an interval $[a,b]$ that you know it
  is continuous over.
\item
  Evaluate $f(a)$ and $f(b)$. Figure out which direction is positive and
  which one is negative. (Let's assume $f(a)<0$ and $f(b)>0$ for this
  description).
\item
  Choose the point $c'=\dfrac{a+b}{2}$, and evaluate $f(c')$.
\item
  If $f(c')<0$, then you know that the root of the function must be to
  the right of $c'$, so set $a=c'$ and repeat this search.
\item
  If $f(c')>0$, then you know that the root of the function must be to
  the left of $c'$, so set $b=c'$ and repeat this search.
\item
  Continue until you reach an arbitrary precision of your choice (or
  terminating after a finite number of steps).
\end{enumerate}

Your task is to implement this function in \texttt{Python}. Use this to
come up with an approximation of $\pi$. \emph{Bonus}: Store an array to
keep track of all of your $c'$ values.

\subsubsection{Our solution}\label{our-solution}

\begin{Shaded}
\begin{Highlighting}[]
\CharTok{from} \NormalTok{numpy }\CharTok{import} \NormalTok{sin}
\KeywordTok{def} \NormalTok{bisection_method(ufunc, a, b, N):}
    \NormalTok{err = }\FloatTok{1e-20}
    \KeywordTok{if} \NormalTok{ufunc(a) > }\DecValTok{0}\NormalTok{.: }\CommentTok{# forces f(a) to be a lower bound}
        \NormalTok{temp = a}
        \NormalTok{a = b}
        \NormalTok{b = temp}
    \NormalTok{c = }\FloatTok{0.5} \NormalTok{* (a + b)}
    \NormalTok{fc = ufunc(c)}
    \NormalTok{cvals = [c]}
    \KeywordTok{for} \NormalTok{i in }\DataTypeTok{range}\NormalTok{(N):}
        \KeywordTok{if} \NormalTok{-err < fc and fc < err: }\CommentTok{# if fc is within an error tolerance of 0}
            \KeywordTok{return} \NormalTok{c}
        \KeywordTok{elif} \NormalTok{fc < }\DecValTok{0}\NormalTok{.:}
            \NormalTok{a = c}
        \KeywordTok{else}\NormalTok{:}
            \NormalTok{b = c}
        \NormalTok{c = }\FloatTok{0.5} \NormalTok{* (a + b)}
        \NormalTok{cvals.append(c)}
        \NormalTok{fc = ufunc(c)}
    \KeywordTok{return} \NormalTok{c, cvals}
\end{Highlighting}
\end{Shaded}

Alternatively, you can implement this recursively.

\begin{Shaded}
\begin{Highlighting}[]
\KeywordTok{def} \NormalTok{bisection_method_rec(ufunc, a, b, N):}
    \NormalTok{err = }\FloatTok{1e-20}
    \KeywordTok{if} \NormalTok{ufunc(a) > }\DecValTok{0}\NormalTok{.:}
        \KeywordTok{return} \NormalTok{bisection_method_rec(ufunc, b, a, N)}
    \NormalTok{c = }\FloatTok{0.5} \NormalTok{* (a + b)}
    \NormalTok{fc = ufunc(c)}
    \KeywordTok{for} \NormalTok{i in }\DataTypeTok{range}\NormalTok{(N):}
        \KeywordTok{if} \DataTypeTok{abs}\NormalTok{(fc) < err or N == }\DecValTok{0}\NormalTok{:}
            \KeywordTok{return} \NormalTok{c}
        \KeywordTok{elif} \NormalTok{fc < }\DecValTok{0}\NormalTok{.:}
            \KeywordTok{return} \NormalTok{bisection_method_rec(ufunc, c, b, N}\DecValTok{-1}\NormalTok{)}
        \KeywordTok{else}\NormalTok{:}
            \KeywordTok{return} \NormalTok{bisection_method_rec(ufunc, a, c, N}\DecValTok{-1}\NormalTok{)}
    \KeywordTok{return} \NormalTok{c}
\end{Highlighting}
\end{Shaded}

    \begin{Verbatim}[commandchars=\\\{\}]
{\color{incolor}In [{\color{incolor}15}]:} \PY{k}{def} \PY{n+nf}{bisection\PYZus{}method\PYZus{}rec}\PY{p}{(}\PY{n}{ufunc}\PY{p}{,} \PY{n}{a}\PY{p}{,} \PY{n}{b}\PY{p}{,} \PY{n}{N}\PY{p}{)}\PY{p}{:}
             \PY{n}{err} \PY{o}{=} \PY{l+m+mf}{1e\PYZhy{}20}
             \PY{k}{if} \PY{n}{ufunc}\PY{p}{(}\PY{n}{a}\PY{p}{)} \PY{o}{\PYZgt{}} \PY{l+m+mf}{0.}\PY{p}{:}
                 \PY{k}{return} \PY{n}{bisection\PYZus{}method\PYZus{}rec}\PY{p}{(}\PY{n}{ufunc}\PY{p}{,} \PY{n}{b}\PY{p}{,} \PY{n}{a}\PY{p}{,} \PY{n}{N}\PY{p}{)}
             \PY{n}{c} \PY{o}{=} \PY{l+m+mf}{0.5} \PY{o}{*} \PY{p}{(}\PY{n}{a} \PY{o}{+} \PY{n}{b}\PY{p}{)}
             \PY{n}{fc} \PY{o}{=} \PY{n}{ufunc}\PY{p}{(}\PY{n}{c}\PY{p}{)}
             \PY{k}{for} \PY{n}{i} \PY{o+ow}{in} \PY{n+nb}{range}\PY{p}{(}\PY{n}{N}\PY{p}{)}\PY{p}{:}
                 \PY{k}{if} \PY{n+nb}{abs}\PY{p}{(}\PY{n}{fc}\PY{p}{)} \PY{o}{\PYZlt{}} \PY{n}{err} \PY{o+ow}{or} \PY{n}{N} \PY{o}{==} \PY{l+m+mi}{0}\PY{p}{:}
                     \PY{k}{return} \PY{n}{c}
                 \PY{k}{elif} \PY{n}{fc} \PY{o}{\PYZlt{}} \PY{l+m+mf}{0.}\PY{p}{:}
                     \PY{k}{return} \PY{n}{bisection\PYZus{}method\PYZus{}rec}\PY{p}{(}\PY{n}{ufunc}\PY{p}{,} \PY{n}{c}\PY{p}{,} \PY{n}{b}\PY{p}{,} \PY{n}{N}\PY{o}{\PYZhy{}}\PY{l+m+mi}{1}\PY{p}{)}
                 \PY{k}{else}\PY{p}{:}
                     \PY{k}{return} \PY{n}{bisection\PYZus{}method\PYZus{}rec}\PY{p}{(}\PY{n}{ufunc}\PY{p}{,} \PY{n}{a}\PY{p}{,} \PY{n}{c}\PY{p}{,} \PY{n}{N}\PY{o}{\PYZhy{}}\PY{l+m+mi}{1}\PY{p}{)}
             \PY{k}{return} \PY{n}{c}
\end{Verbatim}

    \begin{Verbatim}[commandchars=\\\{\}]
{\color{incolor}In [{\color{incolor}16}]:} \PY{k}{print} \PY{n}{bisection\PYZus{}method\PYZus{}rec}\PY{p}{(}\PY{k}{lambda} \PY{n}{x}\PY{p}{:} \PY{n}{x}\PY{o}{*}\PY{n}{x} \PY{o}{\PYZhy{}} \PY{l+m+mf}{3.}\PY{p}{,} \PY{l+m+mf}{0.}\PY{p}{,} \PY{l+m+mf}{2.}\PY{p}{,} \PY{l+m+mi}{100}\PY{p}{)}
\end{Verbatim}

    \begin{Verbatim}[commandchars=\\\{\}]
1.73205080757
    \end{Verbatim}

    \begin{Verbatim}[commandchars=\\\{\}]
{\color{incolor}In [{\color{incolor}17}]:} \PY{k+kn}{from} \PY{n+nn}{IPython.display} \PY{k+kn}{import} \PY{n}{Image}
         
         \PY{n}{Embed} \PY{o}{=} \PY{n}{Image}\PY{p}{(}\PY{l+s}{\PYZdq{}}\PY{l+s}{convergence.png}\PY{l+s}{\PYZdq{}}\PY{p}{)}
         
         \PY{n}{Embed}
\end{Verbatim}
\texttt{\color{outcolor}Out[{\color{outcolor}17}]:}
    
    \begin{center}
    \adjustimage{max size={0.6\linewidth}{0.9\paperheight}}{../Introduction to Python_27_0.png}
    \end{center}
    { \hspace*{\fill} \\}
    

    \section{Some advanced topics}\label{some-advanced-topics}

Here's just a brief overview of the more advanced things you can do in
\texttt{Python}. We won't use them in class, but it's good to know about
them.

\subsection{Classes}\label{classes}

\texttt{Python} can be object-oriented like \texttt{Java}. You declare
classes as follows:

\begin{Shaded}
\begin{Highlighting}[]
\KeywordTok{class} \NormalTok{My_Class:}
    \OtherTok{self}\NormalTok{.field_one}
    \OtherTok{self}\NormalTok{.field_two}
    
    \KeywordTok{def} \OtherTok{__init__}\NormalTok{(args): }\CommentTok{# Equivalent to constructors in Java}
        \CommentTok{# Initialization routine}
        \CommentTok{# Every class needs one}
       
    \KeywordTok{def} \NormalTok{method_one(args):}
        \NormalTok{[do stuff]}
\end{Highlighting}
\end{Shaded}

Object-oriented programming favors modularity, bringing wonderful
returns to scale. In \texttt{Python}, class fields and methods must
always be specified with the \texttt{self} keyword, similar to
\texttt{this} in \texttt{Java}. Unlike in \texttt{Java}, this is not
optional in \texttt{Python}. Method overriding is also not supported in
\texttt{Python}.

\subsection{Lambda functions}\label{lambda-functions}

\texttt{Python} can be functional like \texttt{Scheme}. A lambda
function uses the following syntax:

\begin{Shaded}
\begin{Highlighting}[]
   \KeywordTok{lambda} \NormalTok{x: x**}\DecValTok{2}
\end{Highlighting}
\end{Shaded}

For example, instead of including a predefined function in
\texttt{bisection\_method}, you could use a lambda function. Lambdas and
other household names for the functional programmer are built in, such
as: * \texttt{map} * \texttt{filter} * \texttt{reduce}

If you like Scheme, you can make \texttt{Python} as functional as you
want.

\subsection{Comprehensions}\label{comprehensions}

In \texttt{Python} one important and useful tool is called list (or
other Iterable) \emph{comprehensions}. They allow you to construct lists
from following rules, without the amount of code overhead that you might
see in \texttt{C} or \texttt{Java}.

For example. Suppose you want to construct a list of all nonnegative
even integers less than 20. The naive approach is to use some sort of
loop:

\begin{Shaded}
\begin{Highlighting}[]
\NormalTok{evens = []}
\KeywordTok{for} \NormalTok{i in }\DataTypeTok{range}\NormalTok{(}\DecValTok{20}\NormalTok{):}
    \KeywordTok{if} \NormalTok{i % }\DecValTok{2} \NormalTok{== }\DecValTok{0}\NormalTok{:}
        \NormalTok{evens.append(i)}
\end{Highlighting}
\end{Shaded}

Comprehensions allow you to take this routine and make it more concise.
Instead, you can write

\begin{Shaded}
\begin{Highlighting}[]
\NormalTok{evens = [i }\KeywordTok{for} \NormalTok{i in }\DataTypeTok{range}\NormalTok{(}\DecValTok{20}\NormalTok{) }\KeywordTok{if} \NormalTok{i % }\DecValTok{2} \NormalTok{== }\DecValTok{0}\NormalTok{]}
\end{Highlighting}
\end{Shaded}

For those of you who are comfortable with simple set theory, this is
equivalent to saying

\[ \textrm{Evens} = \{ i \in \mathbb{N} : i < 20, i\equiv 0 \bmod{2} \}\]

Comprehensions are a great way to make your code readable and concise.
They cost a bit more (in terms of microseconds), but often the
philosophy of \texttt{Python} is that readable code is preferable to
slightly-faster ugly code.


    % Add a bibliography block to the postdoc
    
    
    
    \end{document}
