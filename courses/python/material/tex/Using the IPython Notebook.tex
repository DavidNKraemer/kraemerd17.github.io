
% Default to the notebook output style

    


% Inherit from the specified cell style.




    
\documentclass{article}

    
    
    \usepackage{graphicx} % Used to insert images
    \usepackage{adjustbox} % Used to constrain images to a maximum size 
    \usepackage{color} % Allow colors to be defined
    \usepackage{enumerate} % Needed for markdown enumerations to work
    \usepackage{geometry} % Used to adjust the document margins
    \usepackage{amsmath} % Equations
    \usepackage{amssymb} % Equations
    \usepackage{eurosym} % defines \euro
    \usepackage[mathletters]{ucs} % Extended unicode (utf-8) support
    \usepackage[utf8x]{inputenc} % Allow utf-8 characters in the tex document
    \usepackage{fancyvrb} % verbatim replacement that allows latex
    \usepackage{grffile} % extends the file name processing of package graphics 
                         % to support a larger range 
    % The hyperref package gives us a pdf with properly built
    % internal navigation ('pdf bookmarks' for the table of contents,
    % internal cross-reference links, web links for URLs, etc.)
    \usepackage{hyperref}
    \usepackage{longtable} % longtable support required by pandoc >1.10
    \usepackage{booktabs}  % table support for pandoc > 1.12.2
    

    
    
    \definecolor{orange}{cmyk}{0,0.4,0.8,0.2}
    \definecolor{darkorange}{rgb}{.71,0.21,0.01}
    \definecolor{darkgreen}{rgb}{.12,.54,.11}
    \definecolor{myteal}{rgb}{.26, .44, .56}
    \definecolor{gray}{gray}{0.45}
    \definecolor{lightgray}{gray}{.95}
    \definecolor{mediumgray}{gray}{.8}
    \definecolor{inputbackground}{rgb}{.95, .95, .85}
    \definecolor{outputbackground}{rgb}{.95, .95, .95}
    \definecolor{traceback}{rgb}{1, .95, .95}
    % ansi colors
    \definecolor{red}{rgb}{.6,0,0}
    \definecolor{green}{rgb}{0,.65,0}
    \definecolor{brown}{rgb}{0.6,0.6,0}
    \definecolor{blue}{rgb}{0,.145,.698}
    \definecolor{purple}{rgb}{.698,.145,.698}
    \definecolor{cyan}{rgb}{0,.698,.698}
    \definecolor{lightgray}{gray}{0.5}
    
    % bright ansi colors
    \definecolor{darkgray}{gray}{0.25}
    \definecolor{lightred}{rgb}{1.0,0.39,0.28}
    \definecolor{lightgreen}{rgb}{0.48,0.99,0.0}
    \definecolor{lightblue}{rgb}{0.53,0.81,0.92}
    \definecolor{lightpurple}{rgb}{0.87,0.63,0.87}
    \definecolor{lightcyan}{rgb}{0.5,1.0,0.83}
    
    % commands and environments needed by pandoc snippets
    % extracted from the output of `pandoc -s`
    \DefineVerbatimEnvironment{Highlighting}{Verbatim}{commandchars=\\\{\}}
    % Add ',fontsize=\small' for more characters per line
    \newenvironment{Shaded}{}{}
    \newcommand{\KeywordTok}[1]{\textcolor[rgb]{0.00,0.44,0.13}{\textbf{{#1}}}}
    \newcommand{\DataTypeTok}[1]{\textcolor[rgb]{0.56,0.13,0.00}{{#1}}}
    \newcommand{\DecValTok}[1]{\textcolor[rgb]{0.25,0.63,0.44}{{#1}}}
    \newcommand{\BaseNTok}[1]{\textcolor[rgb]{0.25,0.63,0.44}{{#1}}}
    \newcommand{\FloatTok}[1]{\textcolor[rgb]{0.25,0.63,0.44}{{#1}}}
    \newcommand{\CharTok}[1]{\textcolor[rgb]{0.25,0.44,0.63}{{#1}}}
    \newcommand{\StringTok}[1]{\textcolor[rgb]{0.25,0.44,0.63}{{#1}}}
    \newcommand{\CommentTok}[1]{\textcolor[rgb]{0.38,0.63,0.69}{\textit{{#1}}}}
    \newcommand{\OtherTok}[1]{\textcolor[rgb]{0.00,0.44,0.13}{{#1}}}
    \newcommand{\AlertTok}[1]{\textcolor[rgb]{1.00,0.00,0.00}{\textbf{{#1}}}}
    \newcommand{\FunctionTok}[1]{\textcolor[rgb]{0.02,0.16,0.49}{{#1}}}
    \newcommand{\RegionMarkerTok}[1]{{#1}}
    \newcommand{\ErrorTok}[1]{\textcolor[rgb]{1.00,0.00,0.00}{\textbf{{#1}}}}
    \newcommand{\NormalTok}[1]{{#1}}
    
    % Define a nice break command that doesn't care if a line doesn't already
    % exist.
    \def\br{\hspace*{\fill} \\* }
    % Math Jax compatability definitions
    \def\gt{>}
    \def\lt{<}
    % Document parameters
    \title{Using the \texttt{IPython} Notebook}
    
    
    

    % Pygments definitions
    
\makeatletter
\def\PY@reset{\let\PY@it=\relax \let\PY@bf=\relax%
    \let\PY@ul=\relax \let\PY@tc=\relax%
    \let\PY@bc=\relax \let\PY@ff=\relax}
\def\PY@tok#1{\csname PY@tok@#1\endcsname}
\def\PY@toks#1+{\ifx\relax#1\empty\else%
    \PY@tok{#1}\expandafter\PY@toks\fi}
\def\PY@do#1{\PY@bc{\PY@tc{\PY@ul{%
    \PY@it{\PY@bf{\PY@ff{#1}}}}}}}
\def\PY#1#2{\PY@reset\PY@toks#1+\relax+\PY@do{#2}}

\expandafter\def\csname PY@tok@gd\endcsname{\def\PY@tc##1{\textcolor[rgb]{0.63,0.00,0.00}{##1}}}
\expandafter\def\csname PY@tok@gu\endcsname{\let\PY@bf=\textbf\def\PY@tc##1{\textcolor[rgb]{0.50,0.00,0.50}{##1}}}
\expandafter\def\csname PY@tok@gt\endcsname{\def\PY@tc##1{\textcolor[rgb]{0.00,0.27,0.87}{##1}}}
\expandafter\def\csname PY@tok@gs\endcsname{\let\PY@bf=\textbf}
\expandafter\def\csname PY@tok@gr\endcsname{\def\PY@tc##1{\textcolor[rgb]{1.00,0.00,0.00}{##1}}}
\expandafter\def\csname PY@tok@cm\endcsname{\let\PY@it=\textit\def\PY@tc##1{\textcolor[rgb]{0.25,0.50,0.50}{##1}}}
\expandafter\def\csname PY@tok@vg\endcsname{\def\PY@tc##1{\textcolor[rgb]{0.10,0.09,0.49}{##1}}}
\expandafter\def\csname PY@tok@m\endcsname{\def\PY@tc##1{\textcolor[rgb]{0.40,0.40,0.40}{##1}}}
\expandafter\def\csname PY@tok@mh\endcsname{\def\PY@tc##1{\textcolor[rgb]{0.40,0.40,0.40}{##1}}}
\expandafter\def\csname PY@tok@go\endcsname{\def\PY@tc##1{\textcolor[rgb]{0.53,0.53,0.53}{##1}}}
\expandafter\def\csname PY@tok@ge\endcsname{\let\PY@it=\textit}
\expandafter\def\csname PY@tok@vc\endcsname{\def\PY@tc##1{\textcolor[rgb]{0.10,0.09,0.49}{##1}}}
\expandafter\def\csname PY@tok@il\endcsname{\def\PY@tc##1{\textcolor[rgb]{0.40,0.40,0.40}{##1}}}
\expandafter\def\csname PY@tok@cs\endcsname{\let\PY@it=\textit\def\PY@tc##1{\textcolor[rgb]{0.25,0.50,0.50}{##1}}}
\expandafter\def\csname PY@tok@cp\endcsname{\def\PY@tc##1{\textcolor[rgb]{0.74,0.48,0.00}{##1}}}
\expandafter\def\csname PY@tok@gi\endcsname{\def\PY@tc##1{\textcolor[rgb]{0.00,0.63,0.00}{##1}}}
\expandafter\def\csname PY@tok@gh\endcsname{\let\PY@bf=\textbf\def\PY@tc##1{\textcolor[rgb]{0.00,0.00,0.50}{##1}}}
\expandafter\def\csname PY@tok@ni\endcsname{\let\PY@bf=\textbf\def\PY@tc##1{\textcolor[rgb]{0.60,0.60,0.60}{##1}}}
\expandafter\def\csname PY@tok@nl\endcsname{\def\PY@tc##1{\textcolor[rgb]{0.63,0.63,0.00}{##1}}}
\expandafter\def\csname PY@tok@nn\endcsname{\let\PY@bf=\textbf\def\PY@tc##1{\textcolor[rgb]{0.00,0.00,1.00}{##1}}}
\expandafter\def\csname PY@tok@no\endcsname{\def\PY@tc##1{\textcolor[rgb]{0.53,0.00,0.00}{##1}}}
\expandafter\def\csname PY@tok@na\endcsname{\def\PY@tc##1{\textcolor[rgb]{0.49,0.56,0.16}{##1}}}
\expandafter\def\csname PY@tok@nb\endcsname{\def\PY@tc##1{\textcolor[rgb]{0.00,0.50,0.00}{##1}}}
\expandafter\def\csname PY@tok@nc\endcsname{\let\PY@bf=\textbf\def\PY@tc##1{\textcolor[rgb]{0.00,0.00,1.00}{##1}}}
\expandafter\def\csname PY@tok@nd\endcsname{\def\PY@tc##1{\textcolor[rgb]{0.67,0.13,1.00}{##1}}}
\expandafter\def\csname PY@tok@ne\endcsname{\let\PY@bf=\textbf\def\PY@tc##1{\textcolor[rgb]{0.82,0.25,0.23}{##1}}}
\expandafter\def\csname PY@tok@nf\endcsname{\def\PY@tc##1{\textcolor[rgb]{0.00,0.00,1.00}{##1}}}
\expandafter\def\csname PY@tok@si\endcsname{\let\PY@bf=\textbf\def\PY@tc##1{\textcolor[rgb]{0.73,0.40,0.53}{##1}}}
\expandafter\def\csname PY@tok@s2\endcsname{\def\PY@tc##1{\textcolor[rgb]{0.73,0.13,0.13}{##1}}}
\expandafter\def\csname PY@tok@vi\endcsname{\def\PY@tc##1{\textcolor[rgb]{0.10,0.09,0.49}{##1}}}
\expandafter\def\csname PY@tok@nt\endcsname{\let\PY@bf=\textbf\def\PY@tc##1{\textcolor[rgb]{0.00,0.50,0.00}{##1}}}
\expandafter\def\csname PY@tok@nv\endcsname{\def\PY@tc##1{\textcolor[rgb]{0.10,0.09,0.49}{##1}}}
\expandafter\def\csname PY@tok@s1\endcsname{\def\PY@tc##1{\textcolor[rgb]{0.73,0.13,0.13}{##1}}}
\expandafter\def\csname PY@tok@kd\endcsname{\let\PY@bf=\textbf\def\PY@tc##1{\textcolor[rgb]{0.00,0.50,0.00}{##1}}}
\expandafter\def\csname PY@tok@sh\endcsname{\def\PY@tc##1{\textcolor[rgb]{0.73,0.13,0.13}{##1}}}
\expandafter\def\csname PY@tok@sc\endcsname{\def\PY@tc##1{\textcolor[rgb]{0.73,0.13,0.13}{##1}}}
\expandafter\def\csname PY@tok@sx\endcsname{\def\PY@tc##1{\textcolor[rgb]{0.00,0.50,0.00}{##1}}}
\expandafter\def\csname PY@tok@bp\endcsname{\def\PY@tc##1{\textcolor[rgb]{0.00,0.50,0.00}{##1}}}
\expandafter\def\csname PY@tok@c1\endcsname{\let\PY@it=\textit\def\PY@tc##1{\textcolor[rgb]{0.25,0.50,0.50}{##1}}}
\expandafter\def\csname PY@tok@kc\endcsname{\let\PY@bf=\textbf\def\PY@tc##1{\textcolor[rgb]{0.00,0.50,0.00}{##1}}}
\expandafter\def\csname PY@tok@c\endcsname{\let\PY@it=\textit\def\PY@tc##1{\textcolor[rgb]{0.25,0.50,0.50}{##1}}}
\expandafter\def\csname PY@tok@mf\endcsname{\def\PY@tc##1{\textcolor[rgb]{0.40,0.40,0.40}{##1}}}
\expandafter\def\csname PY@tok@err\endcsname{\def\PY@bc##1{\setlength{\fboxsep}{0pt}\fcolorbox[rgb]{1.00,0.00,0.00}{1,1,1}{\strut ##1}}}
\expandafter\def\csname PY@tok@mb\endcsname{\def\PY@tc##1{\textcolor[rgb]{0.40,0.40,0.40}{##1}}}
\expandafter\def\csname PY@tok@ss\endcsname{\def\PY@tc##1{\textcolor[rgb]{0.10,0.09,0.49}{##1}}}
\expandafter\def\csname PY@tok@sr\endcsname{\def\PY@tc##1{\textcolor[rgb]{0.73,0.40,0.53}{##1}}}
\expandafter\def\csname PY@tok@mo\endcsname{\def\PY@tc##1{\textcolor[rgb]{0.40,0.40,0.40}{##1}}}
\expandafter\def\csname PY@tok@kn\endcsname{\let\PY@bf=\textbf\def\PY@tc##1{\textcolor[rgb]{0.00,0.50,0.00}{##1}}}
\expandafter\def\csname PY@tok@mi\endcsname{\def\PY@tc##1{\textcolor[rgb]{0.40,0.40,0.40}{##1}}}
\expandafter\def\csname PY@tok@gp\endcsname{\let\PY@bf=\textbf\def\PY@tc##1{\textcolor[rgb]{0.00,0.00,0.50}{##1}}}
\expandafter\def\csname PY@tok@o\endcsname{\def\PY@tc##1{\textcolor[rgb]{0.40,0.40,0.40}{##1}}}
\expandafter\def\csname PY@tok@kr\endcsname{\let\PY@bf=\textbf\def\PY@tc##1{\textcolor[rgb]{0.00,0.50,0.00}{##1}}}
\expandafter\def\csname PY@tok@s\endcsname{\def\PY@tc##1{\textcolor[rgb]{0.73,0.13,0.13}{##1}}}
\expandafter\def\csname PY@tok@kp\endcsname{\def\PY@tc##1{\textcolor[rgb]{0.00,0.50,0.00}{##1}}}
\expandafter\def\csname PY@tok@w\endcsname{\def\PY@tc##1{\textcolor[rgb]{0.73,0.73,0.73}{##1}}}
\expandafter\def\csname PY@tok@kt\endcsname{\def\PY@tc##1{\textcolor[rgb]{0.69,0.00,0.25}{##1}}}
\expandafter\def\csname PY@tok@ow\endcsname{\let\PY@bf=\textbf\def\PY@tc##1{\textcolor[rgb]{0.67,0.13,1.00}{##1}}}
\expandafter\def\csname PY@tok@sb\endcsname{\def\PY@tc##1{\textcolor[rgb]{0.73,0.13,0.13}{##1}}}
\expandafter\def\csname PY@tok@k\endcsname{\let\PY@bf=\textbf\def\PY@tc##1{\textcolor[rgb]{0.00,0.50,0.00}{##1}}}
\expandafter\def\csname PY@tok@se\endcsname{\let\PY@bf=\textbf\def\PY@tc##1{\textcolor[rgb]{0.73,0.40,0.13}{##1}}}
\expandafter\def\csname PY@tok@sd\endcsname{\let\PY@it=\textit\def\PY@tc##1{\textcolor[rgb]{0.73,0.13,0.13}{##1}}}

\def\PYZbs{\char`\\}
\def\PYZus{\char`\_}
\def\PYZob{\char`\{}
\def\PYZcb{\char`\}}
\def\PYZca{\char`\^}
\def\PYZam{\char`\&}
\def\PYZlt{\char`\<}
\def\PYZgt{\char`\>}
\def\PYZsh{\char`\#}
\def\PYZpc{\char`\%}
\def\PYZdl{\char`\$}
\def\PYZhy{\char`\-}
\def\PYZsq{\char`\'}
\def\PYZdq{\char`\"}
\def\PYZti{\char`\~}
% for compatibility with earlier versions
\def\PYZat{@}
\def\PYZlb{[}
\def\PYZrb{]}
\makeatother


    % Exact colors from NB
    \definecolor{incolor}{rgb}{0.0, 0.0, 0.5}
    \definecolor{outcolor}{rgb}{0.545, 0.0, 0.0}



    
    % Prevent overflowing lines due to hard-to-break entities
    \sloppy 
    % Setup hyperref package
    \hypersetup{
      breaklinks=true,  % so long urls are correctly broken across lines
      colorlinks=true,
      urlcolor=blue,
      linkcolor=darkorange,
      citecolor=darkgreen,
      }
    % Slightly bigger margins than the latex defaults
    
    \geometry{verbose,tmargin=1in,bmargin=1in,lmargin=1in,rmargin=1in}
    
    

    \begin{document}
    
    
    \maketitle
    
    

    
    \section{Warm-up exercises}\label{warm-up-exercises}

\begin{enumerate}
\def\labelenumi{\arabic{enumi}.}
\item
  Write a function, \texttt{pos\_neg(a, b, is\_negative)}, which should
  return \texttt{True} if one of \texttt{a} or \texttt{b} is negative.
  But if \texttt{is\_negative} is \texttt{True}, return true only if
  both \texttt{a} and \texttt{b} are negative.
\item
  Write a function, \texttt{extra\_end(str)}, that takes a string and
  returns a new string formed by taking the last two characters of
  \texttt{str} and repeats it three times. \emph{Bonus}: handle cases in
  which \texttt{str} has length less than 2.
\item
  Write a function, \texttt{rotate\_left3(lst)}, that takes a list of 3
  integers and returns the elements ``rotated left'', so
  \texttt{{[}1, 2, 3{]}} yields \texttt{{[}2, 3, 1{]}}.
\end{enumerate}

    \begin{Verbatim}[commandchars=\\\{\}]
{\color{incolor}In [{\color{incolor}1}]:} \PY{k}{def} \PY{n+nf}{pos\PYZus{}neg}\PY{p}{(}\PY{n}{a}\PY{p}{,} \PY{n}{b}\PY{p}{,} \PY{n}{is\PYZus{}negative}\PY{p}{)}\PY{p}{:}
            \PY{k}{if} \PY{n}{is\PYZus{}negative}\PY{p}{:}
                \PY{k}{return} \PY{n}{a} \PY{o}{\PYZlt{}} \PY{l+m+mi}{0} \PY{o+ow}{and} \PY{n}{b} \PY{o}{\PYZlt{}} \PY{l+m+mi}{0}
            \PY{k}{else}\PY{p}{:}
                \PY{k}{return} \PY{n}{a} \PY{o}{\PYZlt{}} \PY{l+m+mi}{0} \PY{o+ow}{or} \PY{n}{b} \PY{o}{\PYZlt{}} \PY{l+m+mi}{0}
            
        \PY{k}{def} \PY{n+nf}{extra\PYZus{}end}\PY{p}{(}\PY{n+nb}{str}\PY{p}{)}\PY{p}{:}
            \PY{k}{return} \PY{n+nb}{str}\PY{p}{[}\PY{o}{\PYZhy{}}\PY{l+m+mi}{2}\PY{p}{:}\PY{p}{]} \PY{o}{*} \PY{l+m+mi}{3}
        
        \PY{k}{def} \PY{n+nf}{rotate\PYZus{}left3}\PY{p}{(}\PY{n}{lst}\PY{p}{)}\PY{p}{:}
            \PY{k}{return} \PY{n}{lst}\PY{p}{[}\PY{l+m+mi}{1}\PY{p}{:}\PY{p}{]} \PY{o}{+} \PY{p}{[}\PY{n}{lst}\PY{p}{[}\PY{l+m+mi}{0}\PY{p}{]}\PY{p}{]}
\end{Verbatim}

    \begin{Verbatim}[commandchars=\\\{\}]
{\color{incolor}In [{\color{incolor}2}]:} \PY{k}{print} \PY{n}{pos\PYZus{}neg}\PY{p}{(}\PY{l+m+mi}{1}\PY{p}{,} \PY{o}{\PYZhy{}}\PY{l+m+mi}{1}\PY{p}{,} \PY{n+nb+bp}{False}\PY{p}{)}
        \PY{k}{print} \PY{n}{extra\PYZus{}end}\PY{p}{(}\PY{l+s}{\PYZdq{}}\PY{l+s}{warm}\PY{l+s}{\PYZdq{}}\PY{p}{)}
        \PY{k}{print} \PY{n}{rotate\PYZus{}left3}\PY{p}{(}\PY{p}{[}\PY{l+m+mi}{5}\PY{p}{,} \PY{l+m+mi}{3}\PY{p}{,} \PY{l+m+mi}{2}\PY{p}{]}\PY{p}{)}
\end{Verbatim}

    \begin{Verbatim}[commandchars=\\\{\}]
True
rmrmrm
[3, 2, 5]
    \end{Verbatim}

    \section{About \texttt{IPython}}\label{about-ipython}

From \emph{Python for Data Analysis}:

\begin{quote}
The \texttt{IPython} project began in 2001 as Fernando Perez's side
project to make a better interactive \texttt{Python} interpreter. In the
subsequent 11 years it has grown into what's widely considered one of
the most important tools in the modern scientific \texttt{Python}
computing stack. While it does not provide any computational or data
analytical tools by itself, \texttt{IPython} is designed from the ground
up to maximize your productivity in both interactive computing and
software development. It encourages an \emph{execute-explore} workflow
instead of the typical \emph{edit-compile-run} workflow of many other
programming languages.
\end{quote}

The \texttt{IPython Notebook} uses an input-output programming paradigm
centered around the \emph{cell}. Excecuting a cell saves all of the data
into the notebook, so you can use it later. But, you can execute a cell,
write more code, and come back to the same cell to make changes, which
gives the notebook an incredible amount of versatility.

Each notebook uses an execution kernel to keep track of all of the data.
If you run into a challenge, say, an accidental infinite loop, you can
interrupt and restart the kernel from the options. Restarting the kernel
drops all of the data saved to the main memory stack, so you need to
re-execute all of your cells.

    \section{Key features of
\texttt{IPython}}\label{key-features-of-ipython}

Users of \emph{Mathematica} may feel familiar with the overall layout of
the \texttt{IPython Notebook}, but there are some important differences
(and advantages) of using the Notebook. The \texttt{IPython Notebook} is
(lovingly, but jokingly) referred to as the ``poor person's
\emph{Mathematica}'', which is unfair to both \texttt{IPython} and
\emph{Mathematica}. The Notebook serves a different purpose from
Wolfram's product, and it does so exceptionally well.

    \subsection{Tab completion}\label{tab-completion}

From \emph{Python for Data Analysis}:

One of the major improvements over the standard \texttt{Python} shell is
\emph{tab completion}, a feature common to most interactive data
analysis environments. While entering expressions in the shell, pressing
\textbf{\texttt{\textless{}Tab\textgreater{}}} will search the namespace
for any variables (objects, functions, etc.) matching the characters you
have typed so far:

\begin{Shaded}
\begin{Highlighting}[]
\NormalTok{In [}\DecValTok{1}\NormalTok{]: an_apple = }\DecValTok{27}

\NormalTok{In [}\DecValTok{2}\NormalTok{]: an_example = }\DecValTok{42}

\NormalTok{In [}\DecValTok{3}\NormalTok{]: an<Tab>}
\NormalTok{an_apple    and    an_example    }\DataTypeTok{any}
\end{Highlighting}
\end{Shaded}

In this example, not that \texttt{IPython} displayed both the two
variables defined as well as the \texttt{Python} keyword \texttt{and}
and built-in function \texttt{any}. Naturally, you can also complete
methods and attributes on any object after typing a period:

    \begin{Verbatim}[commandchars=\\\{\}]
{\color{incolor}In [{\color{incolor}3}]:} \PY{n}{b} \PY{o}{=} \PY{p}{[}\PY{l+m+mi}{1}\PY{p}{,} \PY{l+m+mi}{2}\PY{p}{,} \PY{l+m+mi}{3}\PY{p}{]}
\end{Verbatim}

    \begin{Shaded}
\begin{Highlighting}[]
\NormalTok{In [}\DecValTok{2}\NormalTok{]: b.<Tab>}
\NormalTok{b.append    b.extend    b.insert    b.remove    b.sort}
\NormalTok{b.count     b.index     b.pop       b.reverse}
\end{Highlighting}
\end{Shaded}

    Tab completion works in many contexts outside of searching the
interactive namespace and completing object or module attributes. When
typing anything that looks like a file path (even in a \texttt{Python}
string), pressing \textbf{\texttt{\textless{}Tab\textgreater{}}} will
complete anything on your computer's file system matching what you've
typed. Combined with the \texttt{\%run} command (see later section),
this functionality will undoubtedly save you many keystrokes.

Another area where tab completion saves time is in the completion of
function keyword argument (arguments that include the \texttt{=} sign).

    \subsection{Introspection}\label{introspection}

Closing a question mark (\texttt{?}) before or after a variable will
display some general information about the object:

    \begin{Verbatim}[commandchars=\\\{\}]
{\color{incolor}In [{\color{incolor}4}]:} \PY{n}{b}\PY{o}{?}
\end{Verbatim}

    \begin{verbatim}
Type:        list
String form: [1, 2, 3]
Length:      3
Docstring:
list() -> new empty list
list(iterable) -> new list initialized from iterable's items
\end{verbatim}

    This is referred to as \emph{object introspection}. If the object is a
function or instance method, the docstring, if defined, will also be
shown. Suppose we'd written the following functions:

    \begin{Verbatim}[commandchars=\\\{\}]
{\color{incolor}In [{\color{incolor}5}]:} \PY{k}{def} \PY{n+nf}{add\PYZus{}numbers}\PY{p}{(}\PY{n}{a}\PY{p}{,} \PY{n}{b}\PY{p}{)}\PY{p}{:}
            \PY{l+s+sd}{\PYZdq{}\PYZdq{}\PYZdq{}}
        \PY{l+s+sd}{    Add two numbers together}
        \PY{l+s+sd}{    }
        \PY{l+s+sd}{    Returns}
        \PY{l+s+sd}{    \PYZhy{}\PYZhy{}\PYZhy{}\PYZhy{}\PYZhy{}\PYZhy{}\PYZhy{}}
        \PY{l+s+sd}{    the\PYZus{}sum : type of arguments}
        \PY{l+s+sd}{    \PYZdq{}\PYZdq{}\PYZdq{}}
            \PY{k}{return} \PY{n}{a} \PY{o}{+} \PY{n}{b}
\end{Verbatim}

    Then using \texttt{?} shows us the docstring:

    \begin{Verbatim}[commandchars=\\\{\}]
{\color{incolor}In [{\color{incolor}6}]:} \PY{n}{add\PYZus{}numbers}\PY{o}{?}
\end{Verbatim}

    \begin{verbatim}
Type:        function
String form: <function add_numbers at 0x7facfc177488>
File:        /home/alethiometryst/mathlan/public_html/courses/python/course-material/ipynbs/<ipython-input-11-5b88597b2522>
Definition:  add_numbers(a, b)
Docstring:
Add two numbers together

Returns
-------
the_sum : type of arguments
\end{verbatim}

    Using \texttt{??} will also show the function's source code if possible:

    \begin{Verbatim}[commandchars=\\\{\}]
{\color{incolor}In [{\color{incolor}7}]:} \PY{n}{add\PYZus{}numbers}\PY{o}{??}
\end{Verbatim}

    \begin{verbatim}
Type:        function
String form: <function add_numbers at 0x7facfc177488>
File:        /home/alethiometryst/mathlan/public_html/courses/python/course-material/ipynbs/<ipython-input-11-5b88597b2522>
Definition:  add_numbers(a, b)
Source:
def add_numbers(a, b):
    """
    Add two numbers together
    
    Returns
    -------
    the_sum : type of arguments
    """
    return a + b
\end{verbatim}

    \texttt{?} has a final usage, which is for searching the
\texttt{IPython} namespace in a manner similar to the standard
\texttt{UNIX} or Windows command line. A number of characters combined
with the wildcard (*) will show all names matching the wildcard
expression. For example, we could get a list of all functions in the top
level \texttt{NumPy} namespace containing \texttt{load}:

    \begin{Verbatim}[commandchars=\\\{\}]
{\color{incolor}In [{\color{incolor}8}]:} \PY{k+kn}{import} \PY{n+nn}{numpy} \PY{k+kn}{as} \PY{n+nn}{np}
        \PY{n}{np}\PY{o}{.}\PY{o}{*}\PY{n}{load}\PY{o}{*}\PY{err}{?}
\end{Verbatim}

    \begin{verbatim}
np.load
np.loads
np.loadtxt
np.pkgload
\end{verbatim}

    \subsection{The \texttt{\%run} Command}\label{the-run-command}

Any file can be run as a \texttt{Python} program inside the environment
of your \texttt{IPython} session using the \texttt{\%run} command.

    \subsection{Keyboard Shortcuts}\label{keyboard-shortcuts}

\texttt{IPython} has many keyboard shortcuts for navigating the prompt
(which will be familiar to users of the Emacs text editor or the
\texttt{UNIX} bash shell). Here are the most commonly used elements:

\subsubsection{Command Mode (press \texttt{Esc} to
enable)}\label{command-mode-press-esc-to-enable}

\begin{longtable}[c]{@{}llll@{}}
\toprule\addlinespace
Command & Description & Command & Description
\\\addlinespace
\midrule\endhead
\texttt{Enter} & edit mode & \texttt{Ctrl-j} & move cell down
\\\addlinespace
\texttt{Shift-Enter} & run cell, select below & \texttt{a} & insert cell
above
\\\addlinespace
\texttt{Ctrl-Enter} & run cell & \texttt{b} & insert cell below
\\\addlinespace
\texttt{Alt-Enter} & run cell, insert below & \texttt{x} & cut cell
\\\addlinespace
\texttt{y} & to code & \texttt{c} & copy cell
\\\addlinespace
\texttt{m} & to markdown & \texttt{Shift-v} & paste cell above
\\\addlinespace
\texttt{r} & to raw & \texttt{v} & paste cell below
\\\addlinespace
\texttt{1} & to heading 1 & \texttt{d} & delete cell (press twice)
\\\addlinespace
\texttt{2} & to heading 2 & \texttt{Shift-m} & merge cell below
\\\addlinespace
\texttt{3} & to heading 3 & \texttt{s} & save notebook
\\\addlinespace
\texttt{4} & to heading 4 & \texttt{Ctrl-s} & save notebook
\\\addlinespace
\texttt{5} & to heading 5 & \texttt{l} & toggle line numbers
\\\addlinespace
\texttt{6} & to heading 6 & \texttt{o} & toggle output
\\\addlinespace
\texttt{Up} & select previous cell & \texttt{Shift-o} & toggle output
scrolling
\\\addlinespace
\texttt{Down} & select next cell & \texttt{q} & close pager
\\\addlinespace
\texttt{k} & select previous cell & \texttt{h} & keyboard shortcuts
\\\addlinespace
\texttt{j} & select next cell & \texttt{i} & interrupt kernel (press
twice)
\\\addlinespace
\texttt{Ctrl-k} & move cell up & \texttt{0} & restart kernel (press
twice)
\\\addlinespace
\bottomrule
\end{longtable}

\subsubsection{Edit Mode (press \texttt{Enter} to
enable)}\label{edit-mode-press-enter-to-enable}

\begin{longtable}[c]{@{}llll@{}}
\toprule\addlinespace
Command & Description & Command & Description
\\\addlinespace
\midrule\endhead
\texttt{Tab} & code completion or indent & \texttt{Ctrl-Down} & go to
cell end
\\\addlinespace
\texttt{Shift-Tab} & tooltip & \texttt{Ctrl-Left} & go one word left
\\\addlinespace
\texttt{Ctrl-{]}} & indent & \texttt{Ctrl-Right} & go one word right
\\\addlinespace
\texttt{Ctrl-{[}} & dedent & \texttt{Ctrl-Backspace} & del word before
\\\addlinespace
\texttt{Ctrl-a} & select all & \texttt{Ctrl-Delete} & del word after
\\\addlinespace
\texttt{Ctrl-z} & undo & \texttt{Esc} & command mode
\\\addlinespace
\texttt{Ctrl-Shift-z} & redo & \texttt{Ctrl-m} & command mode
\\\addlinespace
\texttt{Ctrl-y} & redo & \texttt{Shift-Enter} & run cell, select below
\\\addlinespace
\texttt{Ctrl-Home} & go to cell start & \texttt{Ctrl-Enter} & run cell,
select below
\\\addlinespace
\texttt{Ctrl-Up} & go to cell start & \texttt{Alt-Enter} & run cell,
insert below
\\\addlinespace
\texttt{Ctrl-End} & go to cell end & \texttt{Ctrl-Shift-\/-} & split
cell
\\\addlinespace
\texttt{Ctrl-s} & save notebook
\\\addlinespace
\bottomrule
\end{longtable}

    \subsection{Exception and Tracebacks}\label{exception-and-tracebacks}

If an exception is raised while executing any statement,
\texttt{IPython} will by default print a full call stack trace
(traceback) with a few lines of context around the position at each
point in the stack.

    \begin{Verbatim}[commandchars=\\\{\}]
{\color{incolor}In [{\color{incolor}9}]:} \PY{k}{def} \PY{n+nf}{func}\PY{p}{(}\PY{n}{a}\PY{p}{)}\PY{p}{:}
            \PY{k}{return} \PY{n}{a} \PY{o}{+} \PY{l+m+mi}{2}
                
        \PY{n}{func}\PY{p}{(}\PY{l+m+mi}{3}\PY{p}{,} \PY{l+m+mi}{4}\PY{p}{)}
\end{Verbatim}

    \begin{Verbatim}[commandchars=\\\{\}]

        ---------------------------------------------------------------------------

        TypeError                                 Traceback (most recent call last)

        <ipython-input-9-73c995299d78> in <module>()
          2     return a + 2
          3 
    ----> 4 func(3, 4)
    

        TypeError: func() takes exactly 1 argument (2 given)

    \end{Verbatim}

    \subsection{Magic Commands}\label{magic-commands}

\texttt{IPython} has many special commands, known as ``magic'' commands,
which are designed to facilitate common tasks and enable you to easily
control the behavior of the \texttt{IPython} system. A magic command is
any command prefixed by the percent symbol \texttt{\%}.

Magic commands can be viewed as command line programs to be run within
the \texttt{IPython} system. Many of them have additional ``command
line'' options, which can all be viewed (as you might expect) using
\texttt{?}.

Magic functions can be used by default without the percent sign, as long
as no variable is defined with the same name as the magic function. This
feature is called \emph{automagic} and can be enabled or disabled using
\texttt{\%automatic}.

Since \texttt{IPython}'s documentation is easily accessible from within
the system, I encourage you to explore all of the special commands
available by typing \texttt{\%quickref} or \texttt{\%magic}. Here are a
few more of the most critical ones for being productive in interactive
computing and \texttt{Python} development in \texttt{IPython}.

\begin{longtable}[c]{@{}ll@{}}
\toprule\addlinespace
Command & Description
\\\addlinespace
\midrule\endhead
\texttt{\%quickref} & Display the \texttt{IPython} Quick Reference Card
\\\addlinespace
\texttt{\%magic} & Display detailed documentation for all of the
available magic commands
\\\addlinespace
\texttt{\%debug} & Enter the interactive debugger at the bottom of the
last exception traceback
\\\addlinespace
\texttt{\%hist} & Print command input (and optionally output) history
\\\addlinespace
\texttt{\%pdb} & Automatically enter debugger after any exception
\\\addlinespace
\texttt{\%paste} & Execute pre-formatted \texttt{Python} code from
clipboard
\\\addlinespace
\texttt{\%cpaste} & Open a special prompt for manually pasting
\texttt{Python} code to be executed
\\\addlinespace
\texttt{\%reset} & Delete all variables / names defined in interactive
namespace
\\\addlinespace
\texttt{\%page OBJECT} & Pretty print the object and display it through
a pager
\\\addlinespace
\texttt{\%run script.py} & Run a \texttt{Python} script inside
\texttt{IPython}
\\\addlinespace
\texttt{\%prun statement} & Execute \texttt{statement} with
\texttt{cProfile} and report the profiler output
\\\addlinespace
\texttt{\%time statement} & Report the execution time of single
statement
\\\addlinespace
\texttt{\%timeit statement} & Run a (short) statement multiple times to
compute an emsemble average execution time.
\\\addlinespace
\texttt{\%who, \%who\_ls, \%whos} & Display variables defined in
interactive namespace, varying info depth
\\\addlinespace
\texttt{\%xdel variable} & Delete a variable and clear references to the
object in the \texttt{IPython} internals
\\\addlinespace
\bottomrule
\end{longtable}

    \subsubsection{Try it!}\label{try-it}

Which ``squaring'' function do you think is more efficient?

    \begin{Verbatim}[commandchars=\\\{\}]
{\color{incolor}In [{\color{incolor}10}]:} \PY{k}{def} \PY{n+nf}{square\PYZus{}one}\PY{p}{(}\PY{n}{x}\PY{p}{)}\PY{p}{:}
             \PY{k}{return} \PY{n}{x} \PY{o}{*}\PY{o}{*} \PY{l+m+mi}{2}
         
         \PY{k}{def} \PY{n+nf}{square\PYZus{}two}\PY{p}{(}\PY{n}{x}\PY{p}{)}\PY{p}{:}
             \PY{k}{return} \PY{n}{x} \PY{o}{*} \PY{n}{x}
         
         \PY{o}{\PYZpc{}}\PY{k}{timeit} square\PYZus{}one(500)
         \PY{o}{\PYZpc{}}\PY{k}{timeit} square\PYZus{}two(500)
\end{Verbatim}

    \begin{Verbatim}[commandchars=\\\{\}]
10000000 loops, best of 3: 167 ns per loop
10000000 loops, best of 3: 142 ns per loop
    \end{Verbatim}

    It's good to remember that $x\cdot x$ is always faster to compute than
$x^2$.

\texttt{\%timeit} can be used for more complicated functions. For
example, consider the Fibonacci numbers, which are computed according to
the following rule:

\begin{align*}
F_1 &= 1 \\
F_2 &= 1 \\
F_n &= F_{n-1} + F_{n-2}
\end{align*}

So, $F_3$ is the sum of $F_1$ and $F_2$, i.e. $F_3=1+1=2$. Then
$F_4 = F_2 + F_3 = 5$, and so on. We can write a \texttt{Python}
function to calculate Fibonacci numbers with two different strategies:
recursion or iteration. One might ask which implementation is more
efficient, so we can use \texttt{\%timeit} to get a good idea.

    \begin{Verbatim}[commandchars=\\\{\}]
{\color{incolor}In [{\color{incolor}11}]:} \PY{c}{\PYZsh{} Recursive implementation}
         \PY{k}{def} \PY{n+nf}{fibonacci\PYZus{}one}\PY{p}{(}\PY{n}{n}\PY{p}{)}\PY{p}{:}
             \PY{k}{if} \PY{n}{n} \PY{o}{==} \PY{l+m+mi}{1} \PY{o+ow}{or} \PY{n}{n} \PY{o}{==} \PY{l+m+mi}{2}\PY{p}{:}
                 \PY{k}{return} \PY{l+m+mi}{1}
             \PY{k}{else}\PY{p}{:}
                 \PY{k}{return} \PY{n}{fibonacci\PYZus{}one}\PY{p}{(}\PY{n}{n}\PY{o}{\PYZhy{}}\PY{l+m+mi}{1}\PY{p}{)} \PY{o}{+} \PY{n}{fibonacci\PYZus{}one}\PY{p}{(}\PY{n}{n}\PY{o}{\PYZhy{}}\PY{l+m+mi}{2}\PY{p}{)}
         
         \PY{c}{\PYZsh{} Iterative implementation}
         \PY{k}{def} \PY{n+nf}{fibonacci\PYZus{}two}\PY{p}{(}\PY{n}{n}\PY{p}{)}\PY{p}{:}
             \PY{n}{a} \PY{o}{=} \PY{l+m+mi}{1}
             \PY{n}{b} \PY{o}{=} \PY{l+m+mi}{1}
             \PY{k}{if} \PY{n}{n} \PY{o}{\PYZlt{}} \PY{l+m+mi}{3}\PY{p}{:}
                 \PY{k}{return} \PY{l+m+mi}{1}
             \PY{k}{for} \PY{n}{i} \PY{o+ow}{in} \PY{n+nb}{range} \PY{p}{(}\PY{l+m+mi}{3}\PY{p}{,} \PY{n}{n}\PY{o}{+}\PY{l+m+mi}{1}\PY{p}{)}\PY{p}{:}
                 \PY{n}{c} \PY{o}{=} \PY{n}{a}
                 \PY{n}{a} \PY{o}{+}\PY{o}{=} \PY{n}{b}
                 \PY{n}{b} \PY{o}{=} \PY{n}{c}
                 
             \PY{k}{return} \PY{n}{a}
\end{Verbatim}

    \begin{Verbatim}[commandchars=\\\{\}]
{\color{incolor}In [{\color{incolor}12}]:} \PY{o}{\PYZpc{}}\PY{k}{timeit} fibonacci\PYZus{}one(15)
         \PY{o}{\PYZpc{}}\PY{k}{timeit} fibonacci\PYZus{}two(15)
\end{Verbatim}

    \begin{Verbatim}[commandchars=\\\{\}]
1000 loops, best of 3: 186 µs per loop
1000000 loops, best of 3: 1.82 µs per loop
    \end{Verbatim}

    Remember that there are 1000 nanoseconds per microsecond. In other
words, the iterative implementation is \textbf{150 times faster} than
the recursive one.

    \subsection{Debugging}\label{debugging}

Anyone who has dealt with computers for any length of time has had to
deal with bugs in their code (I'm looking at you, 161ers). Sometimes you
make a (or many) small mistake(s) that you don't catch before running
your code. Your program won't work properly, and you need to figure out
what's causing the problem. Luckily, with \texttt{IPython}, we get to
use the interactive debugger, \texttt{\%debug}, which lets you step
through your code to help you spot errors.

Consider the following problem. You need to write a function that takes
a list of \texttt{double}s and computes the list of its multiplicative
inverses. Then, you need to plot the data (don't worry about the
\texttt{matplotlib} code; it's a demonstration). For example, if you are
given $[1, 2, 3, 4, 5]$, you should obtain

\begin{align*}
f([1,2,3,4,5]) = \left [1, \frac{1}{2}, \frac{1}{3}, \frac{1}{4}, \frac{1}{5} \right ]
\end{align*}

represented as \texttt{doubles}. Suppose your first attempt at a
solution is as follows.

    \begin{Verbatim}[commandchars=\\\{\}]
{\color{incolor}In [{\color{incolor}13}]:} \PY{k+kn}{import} \PY{n+nn}{matplotlib.pyplot} \PY{k+kn}{as} \PY{n+nn}{plt}
         \PY{o}{\PYZpc{}}\PY{k}{matplotlib} inline
\end{Verbatim}

    \begin{Verbatim}[commandchars=\\\{\}]
{\color{incolor}In [{\color{incolor}14}]:} \PY{c}{\PYZsh{} A simple, naive solution}
         \PY{k}{def} \PY{n+nf}{invert}\PY{p}{(}\PY{n}{list\PYZus{}of\PYZus{}doubles}\PY{p}{)}\PY{p}{:}
             \PY{n}{inverted\PYZus{}list} \PY{o}{=} \PY{p}{[}\PY{p}{]}
             \PY{k}{for} \PY{n}{i} \PY{o+ow}{in} \PY{n}{list\PYZus{}of\PYZus{}doubles}\PY{p}{:}
                 \PY{n}{inverted\PYZus{}list}\PY{o}{.}\PY{n}{append}\PY{p}{(}\PY{l+m+mf}{1.}\PY{o}{/}\PY{n}{i}\PY{p}{)}
             \PY{k}{return} \PY{n}{inverted\PYZus{}list}
         
         \PY{c}{\PYZsh{} A silly, useless function}
         \PY{k}{def} \PY{n+nf}{plot\PYZus{}demo}\PY{p}{(}\PY{n}{inverted\PYZus{}list}\PY{p}{)}\PY{p}{:}
             \PY{n}{plt}\PY{o}{.}\PY{n}{plot}\PY{p}{(}\PY{n}{inverted\PYZus{}list}\PY{p}{)}
             \PY{n}{plt}\PY{o}{.}\PY{n}{title}\PY{p}{(}\PY{l+s}{\PYZdq{}}\PY{l+s}{Debug Demo}\PY{l+s}{\PYZdq{}}\PY{p}{)}
\end{Verbatim}

    \begin{Verbatim}[commandchars=\\\{\}]
{\color{incolor}In [{\color{incolor}15}]:} \PY{n}{plot\PYZus{}demo}\PY{p}{(}\PY{n}{invert}\PY{p}{(}\PY{p}{[}\PY{l+m+mf}{1.}\PY{p}{,} \PY{l+m+mf}{2.}\PY{p}{,} \PY{l+m+mf}{3.}\PY{p}{,} \PY{l+m+mf}{4.}\PY{p}{,} \PY{l+m+mf}{5.}\PY{p}{,} \PY{l+m+mf}{6.}\PY{p}{,} \PY{l+m+mf}{7.}\PY{p}{]}\PY{p}{)}\PY{p}{)}
\end{Verbatim}

    \begin{center}
    \adjustimage{max size={0.9\linewidth}{0.9\paperheight}}{Using the IPython Notebook_files/Using the IPython Notebook_37_0.png}
    \end{center}
    { \hspace*{\fill} \\}
    
    Everything looks good, so what's the problem?

    \begin{Verbatim}[commandchars=\\\{\}]
{\color{incolor}In [{\color{incolor}16}]:} \PY{n}{plot\PYZus{}demo}\PY{p}{(}\PY{n}{invert}\PY{p}{(}\PY{p}{[}\PY{l+m+mf}{1.}\PY{p}{,} \PY{l+m+mf}{2.}\PY{p}{,} \PY{l+m+mf}{3.}\PY{p}{,} \PY{l+m+mf}{4.}\PY{p}{,} \PY{l+m+mf}{5.}\PY{p}{,} \PY{l+m+mf}{6.}\PY{p}{,} \PY{l+m+mf}{7.}\PY{p}{,} \PY{l+m+mf}{0.}\PY{p}{]}\PY{p}{)}\PY{p}{)}
\end{Verbatim}

    \begin{Verbatim}[commandchars=\\\{\}]

        ---------------------------------------------------------------------------

        ZeroDivisionError                         Traceback (most recent call last)

        <ipython-input-16-588f355e682e> in <module>()
    ----> 1 plot\_demo(invert([1., 2., 3., 4., 5., 6., 7., 0.]))
    

        <ipython-input-14-b7d23d10d306> in invert(list\_of\_doubles)
          3     inverted\_list = []
          4     for i in list\_of\_doubles:
    ----> 5         inverted\_list.append(1./i)
          6     return inverted\_list
          7 


        ZeroDivisionError: float division by zero

    \end{Verbatim}

    Okay, so we're getting a divide by zero\ldots{} let's run
\texttt{\%debug} if it can enlighten us on where the functions broke.

    \begin{Verbatim}[commandchars=\\\{\}]
{\color{incolor}In [{\color{incolor}17}]:} \PY{o}{\PYZpc{}}\PY{k}{debug}
\end{Verbatim}

    \begin{Verbatim}[commandchars=\\\{\}]
> \textbf{\color{lightgreen}<ipython-input-14-b7d23d10d306>}(5){\color{cyan}invert}\textbf{\color{lightblue}()}
\textbf{\color{lightgreen}      4 }\textbf{\color{yellow}    }\textbf{\color{lightgreen}for} i \textbf{\color{lightgreen}in} list\_of\_doubles\textbf{\color{yellow}:}\textbf{\color{yellow}}
\textbf{\color{lightgreen}----> 5 }\textbf{\color{yellow}        }inverted\_list\textbf{\color{yellow}.}append\textbf{\color{yellow}(}\textbf{\color{lightcyan}1.}\textbf{\color{yellow}/}i\textbf{\color{yellow})}\textbf{\color{yellow}}
\textbf{\color{lightgreen}      6 }\textbf{\color{yellow}    }\textbf{\color{lightgreen}return} inverted\_list\textbf{\color{yellow}}

ipdb> d
*** Newest frame
ipdb> u
> \textbf{\color{lightgreen}<ipython-input-16-588f355e682e>}(1){\color{cyan}<module>}\textbf{\color{lightblue}()}
\textbf{\color{lightgreen}----> 1 }\textbf{\color{yellow}}plot\_demo\textbf{\color{yellow}(}invert\textbf{\color{yellow}(}\textbf{\color{yellow}[}\textbf{\color{lightcyan}1.}\textbf{\color{yellow},} \textbf{\color{lightcyan}2.}\textbf{\color{yellow},} \textbf{\color{lightcyan}3.}\textbf{\color{yellow},} \textbf{\color{lightcyan}4.}\textbf{\color{yellow},} \textbf{\color{lightcyan}5.}\textbf{\color{yellow},} \textbf{\color{lightcyan}6.}\textbf{\color{yellow},} \textbf{\color{lightcyan}7.}\textbf{\color{yellow},} \textbf{\color{lightcyan}0.}\textbf{\color{yellow}]}\textbf{\color{yellow})}\textbf{\color{yellow})}\textbf{\color{yellow}}

ipdb> q
    \end{Verbatim}

    \subsubsection{Debugger commands}\label{debugger-commands}

This example was extremely simplified, but \texttt{\%debug} is a very
useful command to have in your toolbelt. Here are the key commands
inside the debugger to help you navigate.

\begin{longtable}[c]{@{}llll@{}}
\toprule\addlinespace
Command & Action & Command & Action
\\\addlinespace
\midrule\endhead
\texttt{h(elp)} & Display command list & \texttt{s(tep)} & Step
\emph{into} function call
\\\addlinespace
\texttt{help} \emph{command} & Show documentation for \emph{command} &
\texttt{n(ext)} & Execute current line and advance
\\\addlinespace
\texttt{c(ontinue)} & Resume program execution & \texttt{u(p) / d(own)}
& Move up/down in function call stack
\\\addlinespace
\texttt{q(uit)} & Exit debugger without executing any more code &
\texttt{a(rgs)} & Show arguments for current function
\\\addlinespace
\texttt{b(reak)} \emph{number} & Set a breakpoint at \emph{number} in
current file & \texttt{debug} \emph{statement} & Invoke \emph{statement}
in new (recursive) debugger
\\\addlinespace
\texttt{b} \emph{path/to/file.py:number} & Set breakpoint at line
\emph{number} in specified file & \texttt{l(ist)} \emph{statement} &
Show current position with context
\\\addlinespace
\texttt{w(here)} & Print full stack trace with context at current
position
\\\addlinespace
\bottomrule
\end{longtable}

    \section{Tips for Productive Code Development Using
\texttt{IPython}}\label{tips-for-productive-code-development-using-ipython}

Wes McKinney has a number of helpful tips regarding using
\texttt{IPython} for code development. Importantly, he says

\begin{quote}
Writing code in a way that makes it easy to develop, debug, and
ultimately \emph{use} interactively may be a paradigm shift for many
users. There are procedural details like code reloading that may require
some adjustment as well as coding style concerns.
\end{quote}

\begin{quote}
As such, most of this section is more of an art than a science and will
require some experimentation on your part to determine a way to write
your \texttt{Python} code that is effective and productive for you.
Ultimately you want to structure your code in a way that makes it easy
to use iteratively and to be able to explore the results of running a
program or function as effortlessly as possible.
\end{quote}

On these lines, here are some of McKinney's tips for good code design in
\texttt{Python}.

\subsubsection{Flat is better than
nested}\label{flat-is-better-than-nested}

\begin{itemize}
\itemsep1pt\parskip0pt\parsep0pt
\item
  Deeply nested code makes me think about the many layers of an onion.
\item
  When testing or debugging a function, how many layers of the onion
  must you peel back in order to reach the code of interest?
\item
  Making functions and classes as decoupled and modular aas possible
  makes them easier to test (if you are writing unit tests), debug, and
  use interactively.
\end{itemize}

\subsubsection{Overcome a fear of longer
files}\label{overcome-a-fear-of-longer-files}

\begin{itemize}
\itemsep1pt\parskip0pt\parsep0pt
\item
  If you come from a \texttt{Java} background, you may have been told to
  keep files short.
\item
  However, while developing code using \texttt{IPython}, working with 10
  small, but interconnected files (under, say, 100 lines each) is likely
  to cause you more headache in general than a single large file or two
  or three longer files.
\item
  Fewer files means fewer modules to reload and less jumping between
  files while editing, too.
\end{itemize}

    \section{Lab: Testing Efficiency}\label{lab-testing-efficiency}

One fundamental problem in computing is array (or list) sorting. There
are many different ways to sort, each providing distinct advantages. As
a general rule, the best objective way to compare sorting algorithms is
their efficiency, but this can change depending on the size of the array
(or list) under consideration.

This lab invites you to try to implement two common sorting algorithms:

\begin{itemize}
\itemsep1pt\parskip0pt\parsep0pt
\item
  Bubble sort,
\item
  Selection sort
\end{itemize}

Neither algorithm is inherently efficient, but are comparatively simple
to put to code. The bubble sort works as follows:

\begin{enumerate}
\def\labelenumi{\arabic{enumi}.}
\itemsep1pt\parskip0pt\parsep0pt
\item
  Start with a list of numbers
\item
  Check if it is sorted

  \begin{enumerate}
  \def\labelenumii{\arabic{enumii}.}
  \itemsep1pt\parskip0pt\parsep0pt
  \item
    If it is sorted, then return the list
  \item
    If not, continue
  \end{enumerate}
\item
  Compare the first and second numbers

  \begin{enumerate}
  \def\labelenumii{\arabic{enumii}.}
  \itemsep1pt\parskip0pt\parsep0pt
  \item
    If the first number is less than the second number, continue
  \item
    If the second number is less than the first number, swap, and then
    continue
  \end{enumerate}
\item
  Shift over one element of the list, and repeat.
\item
  After reaching the end of the list, go back to step (2).
\end{enumerate}

The selection sort works as follows:

\begin{enumerate}
\def\labelenumi{\arabic{enumi}.}
\itemsep1pt\parskip0pt\parsep0pt
\item
  Start with a list of numbers
\item
  Set the ``swap'' index to 0
\item
  Check if the list after the swap index is sorted

  \begin{enumerate}
  \def\labelenumii{\arabic{enumii}.}
  \itemsep1pt\parskip0pt\parsep0pt
  \item
    If it is sorted, then return the list
  \item
    If not, continue
  \end{enumerate}
\item
  Cycle through the remaining list and identify the location of the
  smallest number
\item
  Exchange the minimum with the current swap index
\item
  Increment the swap index
\item
  Go back to step(2).
\end{enumerate}

Your task: implement these sorting algorithms in \texttt{Python}, and
use \texttt{IPython} tools to determine which algorithm is more
efficient.

    \begin{Verbatim}[commandchars=\\\{\}]
{\color{incolor}In [{\color{incolor}18}]:} \PY{k+kn}{from} \PY{n+nn}{numpy} \PY{k+kn}{import} \PY{n}{random}
         
         \PY{k}{def} \PY{n+nf}{is\PYZus{}sorted}\PY{p}{(}\PY{n}{lst}\PY{p}{)}\PY{p}{:}
             \PY{k}{for} \PY{n}{i} \PY{o+ow}{in} \PY{n+nb}{range}\PY{p}{(}\PY{l+m+mi}{1}\PY{p}{,}\PY{n+nb}{len}\PY{p}{(}\PY{n}{lst}\PY{p}{)}\PY{p}{)}\PY{p}{:}
                 \PY{k}{if} \PY{n}{lst}\PY{p}{[}\PY{n}{i}\PY{p}{]} \PY{o}{\PYZlt{}} \PY{n}{lst}\PY{p}{[}\PY{n}{i}\PY{o}{\PYZhy{}}\PY{l+m+mi}{1}\PY{p}{]}\PY{p}{:}
                     \PY{k}{return} \PY{n+nb+bp}{False}
             \PY{k}{return} \PY{n+nb+bp}{True}
         
         \PY{k}{def} \PY{n+nf}{bubblesort}\PY{p}{(}\PY{n}{lst}\PY{p}{)}\PY{p}{:}
             \PY{k}{while} \PY{o+ow}{not} \PY{n}{is\PYZus{}sorted}\PY{p}{(}\PY{n}{lst}\PY{p}{)}\PY{p}{:}
                 \PY{k}{for} \PY{n}{i} \PY{o+ow}{in} \PY{n+nb}{range}\PY{p}{(}\PY{l+m+mi}{0}\PY{p}{,} \PY{n+nb}{len}\PY{p}{(}\PY{n}{lst}\PY{p}{)} \PY{o}{\PYZhy{}} \PY{l+m+mi}{1}\PY{p}{)}\PY{p}{:}
                     \PY{k}{if} \PY{n}{lst}\PY{p}{[}\PY{n}{i}\PY{o}{+}\PY{l+m+mi}{1}\PY{p}{]} \PY{o}{\PYZlt{}} \PY{n}{lst}\PY{p}{[}\PY{n}{i}\PY{p}{]}\PY{p}{:}
                         \PY{n}{temp} \PY{o}{=} \PY{n}{lst}\PY{p}{[}\PY{n}{i}\PY{p}{]}
                         \PY{n}{lst}\PY{p}{[}\PY{n}{i}\PY{p}{]} \PY{o}{=} \PY{n}{lst}\PY{p}{[}\PY{n}{i}\PY{o}{+}\PY{l+m+mi}{1}\PY{p}{]}
                         \PY{n}{lst}\PY{p}{[}\PY{n}{i}\PY{o}{+}\PY{l+m+mi}{1}\PY{p}{]} \PY{o}{=} \PY{n}{temp}
             \PY{k}{return} \PY{n}{lst}
         
         \PY{k}{def} \PY{n+nf}{selection\PYZus{}sort}\PY{p}{(}\PY{n}{lst}\PY{p}{)}\PY{p}{:}
             \PY{k}{for} \PY{n}{i}\PY{p}{,} \PY{n}{e} \PY{o+ow}{in} \PY{n+nb}{enumerate}\PY{p}{(}\PY{n}{lst}\PY{p}{)}\PY{p}{:}
                 \PY{n}{mn} \PY{o}{=} \PY{n+nb}{min}\PY{p}{(}\PY{n+nb}{range}\PY{p}{(}\PY{n}{i}\PY{p}{,}\PY{n+nb}{len}\PY{p}{(}\PY{n}{lst}\PY{p}{)}\PY{p}{)}\PY{p}{,} \PY{n}{key}\PY{o}{=}\PY{n}{lst}\PY{o}{.}\PY{n}{\PYZus{}\PYZus{}getitem\PYZus{}\PYZus{}}\PY{p}{)}
                 \PY{n}{lst}\PY{p}{[}\PY{n}{i}\PY{p}{]}\PY{p}{,} \PY{n}{lst}\PY{p}{[}\PY{n}{mn}\PY{p}{]} \PY{o}{=} \PY{n}{lst}\PY{p}{[}\PY{n}{mn}\PY{p}{]}\PY{p}{,} \PY{n}{e}
             \PY{k}{return} \PY{n}{lst}
\end{Verbatim}

    \begin{Verbatim}[commandchars=\\\{\}]
{\color{incolor}In [{\color{incolor}19}]:} \PY{o}{\PYZpc{}}\PY{k}{timeit} bubblesort(random.randn(1000))
         \PY{o}{\PYZpc{}}\PY{k}{timeit} selection\PYZus{}sort(random.randn(1000))
\end{Verbatim}

    \begin{Verbatim}[commandchars=\\\{\}]
1 loops, best of 3: 362 ms per loop
10 loops, best of 3: 82.1 ms per loop
    \end{Verbatim}


    % Add a bibliography block to the postdoc
    
    
    
    \end{document}
