
% Default to the notebook output style

    


% Inherit from the specified cell style.




    
\documentclass{article}

    
    
    \usepackage{graphicx} % Used to insert images
    \usepackage{adjustbox} % Used to constrain images to a maximum size 
    \usepackage{color} % Allow colors to be defined
    \usepackage{enumerate} % Needed for markdown enumerations to work
    \usepackage{geometry} % Used to adjust the document margins
    \usepackage{amsmath} % Equations
    \usepackage{amssymb} % Equations
    \usepackage{eurosym} % defines \euro
    \usepackage[mathletters]{ucs} % Extended unicode (utf-8) support
    \usepackage[utf8x]{inputenc} % Allow utf-8 characters in the tex document
    \usepackage{fancyvrb} % verbatim replacement that allows latex
    \usepackage{grffile} % extends the file name processing of package graphics 
                         % to support a larger range 
    % The hyperref package gives us a pdf with properly built
    % internal navigation ('pdf bookmarks' for the table of contents,
    % internal cross-reference links, web links for URLs, etc.)
    \usepackage{hyperref}
    \usepackage{longtable} % longtable support required by pandoc >1.10
    \usepackage{booktabs}  % table support for pandoc > 1.12.2
    

    
    
    \definecolor{orange}{cmyk}{0,0.4,0.8,0.2}
    \definecolor{darkorange}{rgb}{.71,0.21,0.01}
    \definecolor{darkgreen}{rgb}{.12,.54,.11}
    \definecolor{myteal}{rgb}{.26, .44, .56}
    \definecolor{gray}{gray}{0.45}
    \definecolor{lightgray}{gray}{.95}
    \definecolor{mediumgray}{gray}{.8}
    \definecolor{inputbackground}{rgb}{.95, .95, .85}
    \definecolor{outputbackground}{rgb}{.95, .95, .95}
    \definecolor{traceback}{rgb}{1, .95, .95}
    % ansi colors
    \definecolor{red}{rgb}{.6,0,0}
    \definecolor{green}{rgb}{0,.65,0}
    \definecolor{brown}{rgb}{0.6,0.6,0}
    \definecolor{blue}{rgb}{0,.145,.698}
    \definecolor{purple}{rgb}{.698,.145,.698}
    \definecolor{cyan}{rgb}{0,.698,.698}
    \definecolor{lightgray}{gray}{0.5}
    
    % bright ansi colors
    \definecolor{darkgray}{gray}{0.25}
    \definecolor{lightred}{rgb}{1.0,0.39,0.28}
    \definecolor{lightgreen}{rgb}{0.48,0.99,0.0}
    \definecolor{lightblue}{rgb}{0.53,0.81,0.92}
    \definecolor{lightpurple}{rgb}{0.87,0.63,0.87}
    \definecolor{lightcyan}{rgb}{0.5,1.0,0.83}
    
    % commands and environments needed by pandoc snippets
    % extracted from the output of `pandoc -s`
    \providecommand{\tightlist}{%
      \setlength{\itemsep}{0pt}\setlength{\parskip}{0pt}}
    \DefineVerbatimEnvironment{Highlighting}{Verbatim}{commandchars=\\\{\}}
    % Add ',fontsize=\small' for more characters per line
    \newenvironment{Shaded}{}{}
    \newcommand{\KeywordTok}[1]{\textcolor[rgb]{0.00,0.44,0.13}{\textbf{{#1}}}}
    \newcommand{\DataTypeTok}[1]{\textcolor[rgb]{0.56,0.13,0.00}{{#1}}}
    \newcommand{\DecValTok}[1]{\textcolor[rgb]{0.25,0.63,0.44}{{#1}}}
    \newcommand{\BaseNTok}[1]{\textcolor[rgb]{0.25,0.63,0.44}{{#1}}}
    \newcommand{\FloatTok}[1]{\textcolor[rgb]{0.25,0.63,0.44}{{#1}}}
    \newcommand{\CharTok}[1]{\textcolor[rgb]{0.25,0.44,0.63}{{#1}}}
    \newcommand{\StringTok}[1]{\textcolor[rgb]{0.25,0.44,0.63}{{#1}}}
    \newcommand{\CommentTok}[1]{\textcolor[rgb]{0.38,0.63,0.69}{\textit{{#1}}}}
    \newcommand{\OtherTok}[1]{\textcolor[rgb]{0.00,0.44,0.13}{{#1}}}
    \newcommand{\AlertTok}[1]{\textcolor[rgb]{1.00,0.00,0.00}{\textbf{{#1}}}}
    \newcommand{\FunctionTok}[1]{\textcolor[rgb]{0.02,0.16,0.49}{{#1}}}
    \newcommand{\RegionMarkerTok}[1]{{#1}}
    \newcommand{\ErrorTok}[1]{\textcolor[rgb]{1.00,0.00,0.00}{\textbf{{#1}}}}
    \newcommand{\NormalTok}[1]{{#1}}
    
    % Define a nice break command that doesn't care if a line doesn't already
    % exist.
    \def\br{\hspace*{\fill} \\* }
    % Math Jax compatability definitions
    \def\gt{>}
    \def\lt{<}
    % Document parameters
    \title{Time Series}
    
    
    

    % Pygments definitions
    
\makeatletter
\def\PY@reset{\let\PY@it=\relax \let\PY@bf=\relax%
    \let\PY@ul=\relax \let\PY@tc=\relax%
    \let\PY@bc=\relax \let\PY@ff=\relax}
\def\PY@tok#1{\csname PY@tok@#1\endcsname}
\def\PY@toks#1+{\ifx\relax#1\empty\else%
    \PY@tok{#1}\expandafter\PY@toks\fi}
\def\PY@do#1{\PY@bc{\PY@tc{\PY@ul{%
    \PY@it{\PY@bf{\PY@ff{#1}}}}}}}
\def\PY#1#2{\PY@reset\PY@toks#1+\relax+\PY@do{#2}}

\expandafter\def\csname PY@tok@gd\endcsname{\def\PY@tc##1{\textcolor[rgb]{0.63,0.00,0.00}{##1}}}
\expandafter\def\csname PY@tok@gu\endcsname{\let\PY@bf=\textbf\def\PY@tc##1{\textcolor[rgb]{0.50,0.00,0.50}{##1}}}
\expandafter\def\csname PY@tok@gt\endcsname{\def\PY@tc##1{\textcolor[rgb]{0.00,0.27,0.87}{##1}}}
\expandafter\def\csname PY@tok@gs\endcsname{\let\PY@bf=\textbf}
\expandafter\def\csname PY@tok@gr\endcsname{\def\PY@tc##1{\textcolor[rgb]{1.00,0.00,0.00}{##1}}}
\expandafter\def\csname PY@tok@cm\endcsname{\let\PY@it=\textit\def\PY@tc##1{\textcolor[rgb]{0.25,0.50,0.50}{##1}}}
\expandafter\def\csname PY@tok@vg\endcsname{\def\PY@tc##1{\textcolor[rgb]{0.10,0.09,0.49}{##1}}}
\expandafter\def\csname PY@tok@m\endcsname{\def\PY@tc##1{\textcolor[rgb]{0.40,0.40,0.40}{##1}}}
\expandafter\def\csname PY@tok@mh\endcsname{\def\PY@tc##1{\textcolor[rgb]{0.40,0.40,0.40}{##1}}}
\expandafter\def\csname PY@tok@go\endcsname{\def\PY@tc##1{\textcolor[rgb]{0.53,0.53,0.53}{##1}}}
\expandafter\def\csname PY@tok@ge\endcsname{\let\PY@it=\textit}
\expandafter\def\csname PY@tok@vc\endcsname{\def\PY@tc##1{\textcolor[rgb]{0.10,0.09,0.49}{##1}}}
\expandafter\def\csname PY@tok@il\endcsname{\def\PY@tc##1{\textcolor[rgb]{0.40,0.40,0.40}{##1}}}
\expandafter\def\csname PY@tok@cs\endcsname{\let\PY@it=\textit\def\PY@tc##1{\textcolor[rgb]{0.25,0.50,0.50}{##1}}}
\expandafter\def\csname PY@tok@cp\endcsname{\def\PY@tc##1{\textcolor[rgb]{0.74,0.48,0.00}{##1}}}
\expandafter\def\csname PY@tok@gi\endcsname{\def\PY@tc##1{\textcolor[rgb]{0.00,0.63,0.00}{##1}}}
\expandafter\def\csname PY@tok@gh\endcsname{\let\PY@bf=\textbf\def\PY@tc##1{\textcolor[rgb]{0.00,0.00,0.50}{##1}}}
\expandafter\def\csname PY@tok@ni\endcsname{\let\PY@bf=\textbf\def\PY@tc##1{\textcolor[rgb]{0.60,0.60,0.60}{##1}}}
\expandafter\def\csname PY@tok@nl\endcsname{\def\PY@tc##1{\textcolor[rgb]{0.63,0.63,0.00}{##1}}}
\expandafter\def\csname PY@tok@nn\endcsname{\let\PY@bf=\textbf\def\PY@tc##1{\textcolor[rgb]{0.00,0.00,1.00}{##1}}}
\expandafter\def\csname PY@tok@no\endcsname{\def\PY@tc##1{\textcolor[rgb]{0.53,0.00,0.00}{##1}}}
\expandafter\def\csname PY@tok@na\endcsname{\def\PY@tc##1{\textcolor[rgb]{0.49,0.56,0.16}{##1}}}
\expandafter\def\csname PY@tok@nb\endcsname{\def\PY@tc##1{\textcolor[rgb]{0.00,0.50,0.00}{##1}}}
\expandafter\def\csname PY@tok@nc\endcsname{\let\PY@bf=\textbf\def\PY@tc##1{\textcolor[rgb]{0.00,0.00,1.00}{##1}}}
\expandafter\def\csname PY@tok@nd\endcsname{\def\PY@tc##1{\textcolor[rgb]{0.67,0.13,1.00}{##1}}}
\expandafter\def\csname PY@tok@ne\endcsname{\let\PY@bf=\textbf\def\PY@tc##1{\textcolor[rgb]{0.82,0.25,0.23}{##1}}}
\expandafter\def\csname PY@tok@nf\endcsname{\def\PY@tc##1{\textcolor[rgb]{0.00,0.00,1.00}{##1}}}
\expandafter\def\csname PY@tok@si\endcsname{\let\PY@bf=\textbf\def\PY@tc##1{\textcolor[rgb]{0.73,0.40,0.53}{##1}}}
\expandafter\def\csname PY@tok@s2\endcsname{\def\PY@tc##1{\textcolor[rgb]{0.73,0.13,0.13}{##1}}}
\expandafter\def\csname PY@tok@vi\endcsname{\def\PY@tc##1{\textcolor[rgb]{0.10,0.09,0.49}{##1}}}
\expandafter\def\csname PY@tok@nt\endcsname{\let\PY@bf=\textbf\def\PY@tc##1{\textcolor[rgb]{0.00,0.50,0.00}{##1}}}
\expandafter\def\csname PY@tok@nv\endcsname{\def\PY@tc##1{\textcolor[rgb]{0.10,0.09,0.49}{##1}}}
\expandafter\def\csname PY@tok@s1\endcsname{\def\PY@tc##1{\textcolor[rgb]{0.73,0.13,0.13}{##1}}}
\expandafter\def\csname PY@tok@kd\endcsname{\let\PY@bf=\textbf\def\PY@tc##1{\textcolor[rgb]{0.00,0.50,0.00}{##1}}}
\expandafter\def\csname PY@tok@sh\endcsname{\def\PY@tc##1{\textcolor[rgb]{0.73,0.13,0.13}{##1}}}
\expandafter\def\csname PY@tok@sc\endcsname{\def\PY@tc##1{\textcolor[rgb]{0.73,0.13,0.13}{##1}}}
\expandafter\def\csname PY@tok@sx\endcsname{\def\PY@tc##1{\textcolor[rgb]{0.00,0.50,0.00}{##1}}}
\expandafter\def\csname PY@tok@bp\endcsname{\def\PY@tc##1{\textcolor[rgb]{0.00,0.50,0.00}{##1}}}
\expandafter\def\csname PY@tok@c1\endcsname{\let\PY@it=\textit\def\PY@tc##1{\textcolor[rgb]{0.25,0.50,0.50}{##1}}}
\expandafter\def\csname PY@tok@kc\endcsname{\let\PY@bf=\textbf\def\PY@tc##1{\textcolor[rgb]{0.00,0.50,0.00}{##1}}}
\expandafter\def\csname PY@tok@c\endcsname{\let\PY@it=\textit\def\PY@tc##1{\textcolor[rgb]{0.25,0.50,0.50}{##1}}}
\expandafter\def\csname PY@tok@mf\endcsname{\def\PY@tc##1{\textcolor[rgb]{0.40,0.40,0.40}{##1}}}
\expandafter\def\csname PY@tok@err\endcsname{\def\PY@bc##1{\setlength{\fboxsep}{0pt}\fcolorbox[rgb]{1.00,0.00,0.00}{1,1,1}{\strut ##1}}}
\expandafter\def\csname PY@tok@mb\endcsname{\def\PY@tc##1{\textcolor[rgb]{0.40,0.40,0.40}{##1}}}
\expandafter\def\csname PY@tok@ss\endcsname{\def\PY@tc##1{\textcolor[rgb]{0.10,0.09,0.49}{##1}}}
\expandafter\def\csname PY@tok@sr\endcsname{\def\PY@tc##1{\textcolor[rgb]{0.73,0.40,0.53}{##1}}}
\expandafter\def\csname PY@tok@mo\endcsname{\def\PY@tc##1{\textcolor[rgb]{0.40,0.40,0.40}{##1}}}
\expandafter\def\csname PY@tok@kn\endcsname{\let\PY@bf=\textbf\def\PY@tc##1{\textcolor[rgb]{0.00,0.50,0.00}{##1}}}
\expandafter\def\csname PY@tok@mi\endcsname{\def\PY@tc##1{\textcolor[rgb]{0.40,0.40,0.40}{##1}}}
\expandafter\def\csname PY@tok@gp\endcsname{\let\PY@bf=\textbf\def\PY@tc##1{\textcolor[rgb]{0.00,0.00,0.50}{##1}}}
\expandafter\def\csname PY@tok@o\endcsname{\def\PY@tc##1{\textcolor[rgb]{0.40,0.40,0.40}{##1}}}
\expandafter\def\csname PY@tok@kr\endcsname{\let\PY@bf=\textbf\def\PY@tc##1{\textcolor[rgb]{0.00,0.50,0.00}{##1}}}
\expandafter\def\csname PY@tok@s\endcsname{\def\PY@tc##1{\textcolor[rgb]{0.73,0.13,0.13}{##1}}}
\expandafter\def\csname PY@tok@kp\endcsname{\def\PY@tc##1{\textcolor[rgb]{0.00,0.50,0.00}{##1}}}
\expandafter\def\csname PY@tok@w\endcsname{\def\PY@tc##1{\textcolor[rgb]{0.73,0.73,0.73}{##1}}}
\expandafter\def\csname PY@tok@kt\endcsname{\def\PY@tc##1{\textcolor[rgb]{0.69,0.00,0.25}{##1}}}
\expandafter\def\csname PY@tok@ow\endcsname{\let\PY@bf=\textbf\def\PY@tc##1{\textcolor[rgb]{0.67,0.13,1.00}{##1}}}
\expandafter\def\csname PY@tok@sb\endcsname{\def\PY@tc##1{\textcolor[rgb]{0.73,0.13,0.13}{##1}}}
\expandafter\def\csname PY@tok@k\endcsname{\let\PY@bf=\textbf\def\PY@tc##1{\textcolor[rgb]{0.00,0.50,0.00}{##1}}}
\expandafter\def\csname PY@tok@se\endcsname{\let\PY@bf=\textbf\def\PY@tc##1{\textcolor[rgb]{0.73,0.40,0.13}{##1}}}
\expandafter\def\csname PY@tok@sd\endcsname{\let\PY@it=\textit\def\PY@tc##1{\textcolor[rgb]{0.73,0.13,0.13}{##1}}}

\def\PYZbs{\char`\\}
\def\PYZus{\char`\_}
\def\PYZob{\char`\{}
\def\PYZcb{\char`\}}
\def\PYZca{\char`\^}
\def\PYZam{\char`\&}
\def\PYZlt{\char`\<}
\def\PYZgt{\char`\>}
\def\PYZsh{\char`\#}
\def\PYZpc{\char`\%}
\def\PYZdl{\char`\$}
\def\PYZhy{\char`\-}
\def\PYZsq{\char`\'}
\def\PYZdq{\char`\"}
\def\PYZti{\char`\~}
% for compatibility with earlier versions
\def\PYZat{@}
\def\PYZlb{[}
\def\PYZrb{]}
\makeatother


    % Exact colors from NB
    \definecolor{incolor}{rgb}{0.0, 0.0, 0.5}
    \definecolor{outcolor}{rgb}{0.545, 0.0, 0.0}



    
    % Prevent overflowing lines due to hard-to-break entities
    \sloppy 
    % Setup hyperref package
    \hypersetup{
      breaklinks=true,  % so long urls are correctly broken across lines
      colorlinks=true,
      urlcolor=blue,
      linkcolor=darkorange,
      citecolor=darkgreen,
      }
    % Slightly bigger margins than the latex defaults
    
    \geometry{verbose,tmargin=1in,bmargin=1in,lmargin=1in,rmargin=1in}
    
    

    \begin{document}
    
    
    \maketitle
    
    

    
    \section{Time series}\label{time-series}

    From \emph{Python for Data Analysis}:

\begin{quote}
Time series data is an important form of structured data in many
different dielfds, such as finance, economics, ecology, neuroscience,
and physics. Anything that is observed or measured at many points in
time forms a time series. Many time series are \emph{fixed frequency},
which is to say that data points occur at regular intervals according to
some rule, such as every 15 seconds, every 5 minutes, or once per month.
Time series can also be \emph{irregular} without a fixed unit or time or
offset between units. How you mark and refer to time series data depends
on the application and you may have one of the following:
\end{quote}

\begin{quote}
\begin{itemize}
\itemsep1pt\parskip0pt\parsep0pt
\item
  \emph{timestamps}, specific instants in time
\item
  \emph{fixed periods}, such as the month January 2007 or the full year
  2010
\item
  \emph{intervals} of time, indicated by a start and end timestamp.
  Periods can be thought of as special cases of intervals
\item
  Experiment or elapsed time; each timestamp is a measure of time
  relative to a particular start time. For example, the diameter of a
  cookie baking each second since being placed in the oven
\end{itemize}
\end{quote}

\begin{quote}
\texttt{Pandas} provides a standard set of time series tools and data
algorithms. With this you can efficiently work with very large time
series and easily slice and dice, aggregate, and resample irregular and
fixed frequency time series. As you might guess, many of these tools are
especially useful for financial and economics applications, but you
could certainly use them to analyze server log data, too.
\end{quote}

    \begin{Verbatim}[commandchars=\\\{\}]
{\color{incolor}In [{\color{incolor} }]:} \PY{k+kn}{from} \PY{n+nn}{\PYZus{}\PYZus{}future\PYZus{}\PYZus{}} \PY{k+kn}{import} \PY{n}{division}
        \PY{k+kn}{from} \PY{n+nn}{pandas} \PY{k+kn}{import} \PY{n}{Series}\PY{p}{,} \PY{n}{DataFrame}
        \PY{k+kn}{import} \PY{n+nn}{pandas} \PY{k+kn}{as} \PY{n+nn}{pd}
        \PY{k+kn}{from} \PY{n+nn}{numpy.random} \PY{k+kn}{import} \PY{n}{randn}
        \PY{k+kn}{import} \PY{n+nn}{numpy} \PY{k+kn}{as} \PY{n+nn}{np}
        \PY{n}{pd}\PY{o}{.}\PY{n}{options}\PY{o}{.}\PY{n}{display}\PY{o}{.}\PY{n}{max\PYZus{}rows} \PY{o}{=} \PY{l+m+mi}{12}
        \PY{n}{np}\PY{o}{.}\PY{n}{set\PYZus{}printoptions}\PY{p}{(}\PY{n}{precision}\PY{o}{=}\PY{l+m+mi}{4}\PY{p}{,} \PY{n}{suppress}\PY{o}{=}\PY{n+nb+bp}{True}\PY{p}{)}
        \PY{k+kn}{import} \PY{n+nn}{matplotlib.pyplot} \PY{k+kn}{as} \PY{n+nn}{plt}
        \PY{n}{plt}\PY{o}{.}\PY{n}{rc}\PY{p}{(}\PY{l+s}{\PYZsq{}}\PY{l+s}{figure}\PY{l+s}{\PYZsq{}}\PY{p}{,} \PY{n}{figsize}\PY{o}{=}\PY{p}{(}\PY{l+m+mi}{12}\PY{p}{,} \PY{l+m+mi}{4}\PY{p}{)}\PY{p}{)}
\end{Verbatim}

    \begin{Verbatim}[commandchars=\\\{\}]
{\color{incolor}In [{\color{incolor} }]:} \PY{o}{\PYZpc{}}\PY{k}{matplotlib} inline
\end{Verbatim}

    \subsection{Date and Time Data Types and
Tools}\label{date-and-time-data-types-and-tools}

    In general, dealing with date arithmetic is \emph{hard}. Luckily,
\texttt{Python} has a robust library that implements \texttt{datetime}
objects, which handle all of the annoying bits of date manipulation in a
powerful way.

    \begin{Verbatim}[commandchars=\\\{\}]
{\color{incolor}In [{\color{incolor} }]:} \PY{k+kn}{from} \PY{n+nn}{datetime} \PY{k+kn}{import} \PY{n}{datetime}
        \PY{n}{now} \PY{o}{=} \PY{n}{datetime}\PY{o}{.}\PY{n}{now}\PY{p}{(}\PY{p}{)}
        \PY{n}{now}
\end{Verbatim}

    Every \texttt{datetime} object has a \texttt{year}, \texttt{month}, and
\texttt{day} field.

    \begin{Verbatim}[commandchars=\\\{\}]
{\color{incolor}In [{\color{incolor} }]:} \PY{n}{now}\PY{o}{.}\PY{n}{year}\PY{p}{,} \PY{n}{now}\PY{o}{.}\PY{n}{month}\PY{p}{,} \PY{n}{now}\PY{o}{.}\PY{n}{day}
\end{Verbatim}

    You can do arithmetic on \texttt{datetime} objects, which produce
\texttt{timedelta} objects.

    \begin{Verbatim}[commandchars=\\\{\}]
{\color{incolor}In [{\color{incolor} }]:} \PY{n}{delta} \PY{o}{=} \PY{n}{datetime}\PY{p}{(}\PY{l+m+mi}{2011}\PY{p}{,} \PY{l+m+mi}{1}\PY{p}{,} \PY{l+m+mi}{7}\PY{p}{)} \PY{o}{\PYZhy{}} \PY{n}{datetime}\PY{p}{(}\PY{l+m+mi}{2008}\PY{p}{,} \PY{l+m+mi}{6}\PY{p}{,} \PY{l+m+mi}{24}\PY{p}{,} \PY{l+m+mi}{8}\PY{p}{,} \PY{l+m+mi}{15}\PY{p}{)}
        \PY{n}{delta}
\end{Verbatim}

    \texttt{timedelta} objects are very similar to \texttt{datetime}
objects, with similar fields:

    \begin{Verbatim}[commandchars=\\\{\}]
{\color{incolor}In [{\color{incolor} }]:} \PY{n}{delta}\PY{o}{.}\PY{n}{days}
\end{Verbatim}

    \begin{Verbatim}[commandchars=\\\{\}]
{\color{incolor}In [{\color{incolor} }]:} \PY{n}{delta}\PY{o}{.}\PY{n}{seconds}
\end{Verbatim}

    As you expect, arithmetic between \texttt{datetime} and
\texttt{timedelta} objects produce \texttt{datetime} objects.

    \begin{Verbatim}[commandchars=\\\{\}]
{\color{incolor}In [{\color{incolor} }]:} \PY{k+kn}{from} \PY{n+nn}{datetime} \PY{k+kn}{import} \PY{n}{timedelta}
        \PY{n}{start} \PY{o}{=} \PY{n}{datetime}\PY{p}{(}\PY{l+m+mi}{2011}\PY{p}{,} \PY{l+m+mi}{1}\PY{p}{,} \PY{l+m+mi}{7}\PY{p}{)}
        \PY{n}{start} \PY{o}{+} \PY{n}{timedelta}\PY{p}{(}\PY{l+m+mi}{12}\PY{p}{)}
\end{Verbatim}

    \begin{Verbatim}[commandchars=\\\{\}]
{\color{incolor}In [{\color{incolor} }]:} \PY{n}{start} \PY{o}{\PYZhy{}} \PY{l+m+mi}{2} \PY{o}{*} \PY{n}{timedelta}\PY{p}{(}\PY{l+m+mi}{12}\PY{p}{)}
\end{Verbatim}

    \subsubsection{Converting between string and
datetime}\label{converting-between-string-and-datetime}

    In general, it is easier to format a string from a \texttt{datetime}
object than to parse a string date into a \texttt{datetime} object.

    \begin{Verbatim}[commandchars=\\\{\}]
{\color{incolor}In [{\color{incolor} }]:} \PY{n}{stamp} \PY{o}{=} \PY{n}{datetime}\PY{p}{(}\PY{l+m+mi}{2011}\PY{p}{,} \PY{l+m+mi}{1}\PY{p}{,} \PY{l+m+mi}{3}\PY{p}{)}
\end{Verbatim}

    \begin{Verbatim}[commandchars=\\\{\}]
{\color{incolor}In [{\color{incolor} }]:} \PY{n+nb}{str}\PY{p}{(}\PY{n}{stamp}\PY{p}{)}
\end{Verbatim}

    To format a string from a \texttt{datetime} object, use the
\texttt{strftime} method. You can use the standard string-formatting
delimiters that are used in computing.

    \begin{Verbatim}[commandchars=\\\{\}]
{\color{incolor}In [{\color{incolor} }]:} \PY{n}{stamp}\PY{o}{.}\PY{n}{strftime}\PY{p}{(}\PY{l+s}{\PYZsq{}}\PY{l+s}{\PYZpc{}}\PY{l+s}{Y\PYZhy{}}\PY{l+s}{\PYZpc{}}\PY{l+s}{m\PYZhy{}}\PY{l+s+si}{\PYZpc{}d}\PY{l+s}{\PYZsq{}}\PY{p}{)}
\end{Verbatim}

    To parse a string into a \texttt{datetime} object, you can use the
\texttt{strptime} method, along with the relevant format.

    \begin{Verbatim}[commandchars=\\\{\}]
{\color{incolor}In [{\color{incolor} }]:} \PY{n}{value} \PY{o}{=} \PY{l+s}{\PYZsq{}}\PY{l+s}{2011\PYZhy{}01\PYZhy{}03}\PY{l+s}{\PYZsq{}}
        \PY{n}{datetime}\PY{o}{.}\PY{n}{strptime}\PY{p}{(}\PY{n}{value}\PY{p}{,} \PY{l+s}{\PYZsq{}}\PY{l+s}{\PYZpc{}}\PY{l+s}{Y\PYZhy{}}\PY{l+s}{\PYZpc{}}\PY{l+s}{m\PYZhy{}}\PY{l+s+si}{\PYZpc{}d}\PY{l+s}{\PYZsq{}}\PY{p}{)}
\end{Verbatim}

    Of course, this being \texttt{Python}, we can easily abstract this
process to list form using comprehensions.

    \begin{Verbatim}[commandchars=\\\{\}]
{\color{incolor}In [{\color{incolor} }]:} \PY{n}{datestrs} \PY{o}{=} \PY{p}{[}\PY{l+s}{\PYZsq{}}\PY{l+s}{7/6/2011}\PY{l+s}{\PYZsq{}}\PY{p}{,} \PY{l+s}{\PYZsq{}}\PY{l+s}{8/6/2011}\PY{l+s}{\PYZsq{}}\PY{p}{]}
        \PY{p}{[}\PY{n}{datetime}\PY{o}{.}\PY{n}{strptime}\PY{p}{(}\PY{n}{x}\PY{p}{,} \PY{l+s}{\PYZsq{}}\PY{l+s}{\PYZpc{}}\PY{l+s}{m/}\PY{l+s+si}{\PYZpc{}d}\PY{l+s}{/}\PY{l+s}{\PYZpc{}}\PY{l+s}{Y}\PY{l+s}{\PYZsq{}}\PY{p}{)} \PY{k}{for} \PY{n}{x} \PY{o+ow}{in} \PY{n}{datestrs}\PY{p}{]}
\end{Verbatim}

    Without question, \texttt{datetime.strptime} is the best way to parse a
date, especially when you know the format a priori. However, it can be a
bit annoying to have to write a format spec each time, especially for
common date formats. In this case, you can use the \texttt{parser.parse}
method in the third party \texttt{dateutil} package:

    \begin{Verbatim}[commandchars=\\\{\}]
{\color{incolor}In [{\color{incolor} }]:} \PY{k+kn}{from} \PY{n+nn}{dateutil.parser} \PY{k+kn}{import} \PY{n}{parse}
        \PY{n}{parse}\PY{p}{(}\PY{l+s}{\PYZsq{}}\PY{l+s}{2011\PYZhy{}01\PYZhy{}03}\PY{l+s}{\PYZsq{}}\PY{p}{)}
\end{Verbatim}

    \texttt{dateutil} is capable of parsing almost any human-intelligible
date representation:

    \begin{Verbatim}[commandchars=\\\{\}]
{\color{incolor}In [{\color{incolor} }]:} \PY{n}{parse}\PY{p}{(}\PY{l+s}{\PYZsq{}}\PY{l+s}{Jan 31, 1997 10:45 PM}\PY{l+s}{\PYZsq{}}\PY{p}{)}
\end{Verbatim}

    In international locales, day appearing before month is very common, so
you can pass \texttt{dayfirst=True} to indicate this:

    \begin{Verbatim}[commandchars=\\\{\}]
{\color{incolor}In [{\color{incolor} }]:} \PY{n}{parse}\PY{p}{(}\PY{l+s}{\PYZsq{}}\PY{l+s}{6/12/2011}\PY{l+s}{\PYZsq{}}\PY{p}{,} \PY{n}{dayfirst}\PY{o}{=}\PY{n+nb+bp}{True}\PY{p}{)}
\end{Verbatim}

    \texttt{Pandas} is generally oriented toward working with arrays of
dates, whether used as an index or a column in a \texttt{DataFrame}. The
\texttt{to\_datetime} method parses many different kinds of date
representations. Standard date formats like ISO8601 can be parsed very
quickly.

    \begin{Verbatim}[commandchars=\\\{\}]
{\color{incolor}In [{\color{incolor} }]:} \PY{n}{datestrs}
\end{Verbatim}

    \begin{Verbatim}[commandchars=\\\{\}]
{\color{incolor}In [{\color{incolor} }]:} \PY{n}{pd}\PY{o}{.}\PY{n}{to\PYZus{}datetime}\PY{p}{(}\PY{n}{datestrs}\PY{p}{)}
\end{Verbatim}

    Notice that the \texttt{Pandas} object at work behind the scenes here is
the \texttt{DatetimeIndex}, which is a subclass of \texttt{Index}. More
on this later. \texttt{to\_datetime} also handles values that should be
considered missing (\texttt{None}, empty string, etc.):

    \begin{Verbatim}[commandchars=\\\{\}]
{\color{incolor}In [{\color{incolor} }]:} \PY{n}{idx} \PY{o}{=} \PY{n}{pd}\PY{o}{.}\PY{n}{to\PYZus{}datetime}\PY{p}{(}\PY{n}{datestrs} \PY{o}{+} \PY{p}{[}\PY{n+nb+bp}{None}\PY{p}{]}\PY{p}{)}
        \PY{n}{idx}
\end{Verbatim}

    \begin{Verbatim}[commandchars=\\\{\}]
{\color{incolor}In [{\color{incolor} }]:} \PY{n}{idx}\PY{p}{[}\PY{l+m+mi}{2}\PY{p}{]}
\end{Verbatim}

    \begin{Verbatim}[commandchars=\\\{\}]
{\color{incolor}In [{\color{incolor} }]:} \PY{n}{pd}\PY{o}{.}\PY{n}{isnull}\PY{p}{(}\PY{n}{idx}\PY{p}{)}
\end{Verbatim}

    \texttt{datetime} objects also have a number of locale-specific
formatting options for systems in other countries or languages. For
example, the abbreviated month names will be different on German or
French systems compared with English systems.

    \subsection{Time Series Basics}\label{time-series-basics}

    The most basic kind of time series object in \texttt{Pandas} is a
\texttt{Series} indexed by timestamps, which is often represented
external to \texttt{Pandas} as \texttt{Python} strings or
\texttt{datetime} objects.

    \begin{Verbatim}[commandchars=\\\{\}]
{\color{incolor}In [{\color{incolor} }]:} \PY{k+kn}{from} \PY{n+nn}{datetime} \PY{k+kn}{import} \PY{n}{datetime}
        \PY{n}{dates} \PY{o}{=} \PY{p}{[}\PY{n}{datetime}\PY{p}{(}\PY{l+m+mi}{2011}\PY{p}{,} \PY{l+m+mi}{1}\PY{p}{,} \PY{l+m+mi}{2}\PY{p}{)}\PY{p}{,} \PY{n}{datetime}\PY{p}{(}\PY{l+m+mi}{2011}\PY{p}{,} \PY{l+m+mi}{1}\PY{p}{,} \PY{l+m+mi}{5}\PY{p}{)}\PY{p}{,} \PY{n}{datetime}\PY{p}{(}\PY{l+m+mi}{2011}\PY{p}{,} \PY{l+m+mi}{1}\PY{p}{,} \PY{l+m+mi}{7}\PY{p}{)}\PY{p}{,}
                 \PY{n}{datetime}\PY{p}{(}\PY{l+m+mi}{2011}\PY{p}{,} \PY{l+m+mi}{1}\PY{p}{,} \PY{l+m+mi}{8}\PY{p}{)}\PY{p}{,} \PY{n}{datetime}\PY{p}{(}\PY{l+m+mi}{2011}\PY{p}{,} \PY{l+m+mi}{1}\PY{p}{,} \PY{l+m+mi}{10}\PY{p}{)}\PY{p}{,} \PY{n}{datetime}\PY{p}{(}\PY{l+m+mi}{2011}\PY{p}{,} \PY{l+m+mi}{1}\PY{p}{,} \PY{l+m+mi}{12}\PY{p}{)}\PY{p}{]}
        \PY{n}{ts} \PY{o}{=} \PY{n}{Series}\PY{p}{(}\PY{n}{np}\PY{o}{.}\PY{n}{random}\PY{o}{.}\PY{n}{randn}\PY{p}{(}\PY{l+m+mi}{6}\PY{p}{)}\PY{p}{,} \PY{n}{index}\PY{o}{=}\PY{n}{dates}\PY{p}{)}
        \PY{n}{ts}
\end{Verbatim}

    Under the hood, these \texttt{datetime} objects have been put in a
\texttt{DatetimeIndex}, and the variable \texttt{ts} is now of type
\texttt{TimeSeries}.

    \begin{Verbatim}[commandchars=\\\{\}]
{\color{incolor}In [{\color{incolor} }]:} \PY{n+nb}{type}\PY{p}{(}\PY{n}{ts}\PY{p}{)}
        \PY{c}{\PYZsh{} note: output changed to \PYZdq{}pandas.core.series.Series\PYZdq{}}
\end{Verbatim}

    \begin{Verbatim}[commandchars=\\\{\}]
{\color{incolor}In [{\color{incolor} }]:} \PY{n}{ts}\PY{o}{.}\PY{n}{index}
\end{Verbatim}

    Like other \texttt{Series}, arithmetic operations between
differently-indexed time series automatically align on the dates:

    \begin{Verbatim}[commandchars=\\\{\}]
{\color{incolor}In [{\color{incolor} }]:} \PY{n}{ts} \PY{o}{+} \PY{n}{ts}\PY{p}{[}\PY{p}{:}\PY{p}{:}\PY{l+m+mi}{2}\PY{p}{]}
\end{Verbatim}

    \texttt{Pandas} stores timestamps using \texttt{NumPy}'s
\texttt{datetime64} date type at the nanosecond resolution:

    \begin{Verbatim}[commandchars=\\\{\}]
{\color{incolor}In [{\color{incolor} }]:} \PY{n}{ts}\PY{o}{.}\PY{n}{index}\PY{o}{.}\PY{n}{dtype}
        \PY{c}{\PYZsh{} note: output changed from dtype(\PYZsq{}datetime64[ns]\PYZsq{}) to dtype(\PYZsq{}\PYZlt{}M8[ns]\PYZsq{})}
\end{Verbatim}

    Scalar values from a \texttt{DatetimeIndex} are \texttt{Pandas}
\texttt{Timestamp} objects

    \begin{Verbatim}[commandchars=\\\{\}]
{\color{incolor}In [{\color{incolor} }]:} \PY{n}{stamp} \PY{o}{=} \PY{n}{ts}\PY{o}{.}\PY{n}{index}\PY{p}{[}\PY{l+m+mi}{0}\PY{p}{]}
        \PY{n}{stamp}
        \PY{c}{\PYZsh{} note: output changed from \PYZlt{}Timestamp: 2011\PYZhy{}01\PYZhy{}02 00:00:00\PYZgt{} to Timestamp(\PYZsq{}2011\PYZhy{}01\PYZhy{}02 00:00:00\PYZsq{})}
\end{Verbatim}

    A \texttt{Timestamp} can be substituted anywhere you would use a
\texttt{datetime} object. Additionally, it can store frequency
information (if any) and understands how to do time zone conversions and
other kinds of manipulations. More on both of these things later.

    \subsubsection{Indexing, selection,
subsetting}\label{indexing-selection-subsetting}

    \texttt{TimeSeries} is a subclass of \texttt{Series} and thus behaves in
the same way with regard to indexing and selecting data based on label:

    \begin{Verbatim}[commandchars=\\\{\}]
{\color{incolor}In [{\color{incolor} }]:} \PY{n}{stamp} \PY{o}{=} \PY{n}{ts}\PY{o}{.}\PY{n}{index}\PY{p}{[}\PY{l+m+mi}{2}\PY{p}{]}
        \PY{n}{ts}\PY{p}{[}\PY{n}{stamp}\PY{p}{]}
\end{Verbatim}

    As a convenience, you can also pass a string that is interpretable as a
date:

    \begin{Verbatim}[commandchars=\\\{\}]
{\color{incolor}In [{\color{incolor} }]:} \PY{n}{ts}\PY{p}{[}\PY{l+s}{\PYZsq{}}\PY{l+s}{1/10/2011}\PY{l+s}{\PYZsq{}}\PY{p}{]}
\end{Verbatim}

    \begin{Verbatim}[commandchars=\\\{\}]
{\color{incolor}In [{\color{incolor} }]:} \PY{n}{ts}\PY{p}{[}\PY{l+s}{\PYZsq{}}\PY{l+s}{20110110}\PY{l+s}{\PYZsq{}}\PY{p}{]}
\end{Verbatim}

    For longer time series, a year or only a year and month can be passed to
easily select slices of data:

    \begin{Verbatim}[commandchars=\\\{\}]
{\color{incolor}In [{\color{incolor} }]:} \PY{n}{longer\PYZus{}ts} \PY{o}{=} \PY{n}{Series}\PY{p}{(}\PY{n}{np}\PY{o}{.}\PY{n}{random}\PY{o}{.}\PY{n}{randn}\PY{p}{(}\PY{l+m+mi}{1000}\PY{p}{)}\PY{p}{,}
                           \PY{n}{index}\PY{o}{=}\PY{n}{pd}\PY{o}{.}\PY{n}{date\PYZus{}range}\PY{p}{(}\PY{l+s}{\PYZsq{}}\PY{l+s}{1/1/2000}\PY{l+s}{\PYZsq{}}\PY{p}{,} \PY{n}{periods}\PY{o}{=}\PY{l+m+mi}{1000}\PY{p}{)}\PY{p}{)}
        \PY{n}{longer\PYZus{}ts}
\end{Verbatim}

    \begin{Verbatim}[commandchars=\\\{\}]
{\color{incolor}In [{\color{incolor} }]:} \PY{n}{longer\PYZus{}ts}\PY{p}{[}\PY{l+s}{\PYZsq{}}\PY{l+s}{2001}\PY{l+s}{\PYZsq{}}\PY{p}{]}
\end{Verbatim}

    \begin{Verbatim}[commandchars=\\\{\}]
{\color{incolor}In [{\color{incolor} }]:} \PY{n}{longer\PYZus{}ts}\PY{p}{[}\PY{l+s}{\PYZsq{}}\PY{l+s}{2001\PYZhy{}05}\PY{l+s}{\PYZsq{}}\PY{p}{]}
\end{Verbatim}

    Slicing with dates works just like with a regular \texttt{Series}

    \begin{Verbatim}[commandchars=\\\{\}]
{\color{incolor}In [{\color{incolor} }]:} \PY{n}{ts}\PY{p}{[}\PY{n}{datetime}\PY{p}{(}\PY{l+m+mi}{2011}\PY{p}{,} \PY{l+m+mi}{1}\PY{p}{,} \PY{l+m+mi}{7}\PY{p}{)}\PY{p}{:}\PY{p}{]}
\end{Verbatim}

    Because most time series data is ordered chronologically, you can slice
with timestamps not contained in a time series to perform a range query:

    \begin{Verbatim}[commandchars=\\\{\}]
{\color{incolor}In [{\color{incolor} }]:} \PY{n}{ts}
\end{Verbatim}

    \begin{Verbatim}[commandchars=\\\{\}]
{\color{incolor}In [{\color{incolor} }]:} \PY{n}{ts}\PY{p}{[}\PY{l+s}{\PYZsq{}}\PY{l+s}{1/6/2011}\PY{l+s}{\PYZsq{}}\PY{p}{:}\PY{l+s}{\PYZsq{}}\PY{l+s}{1/11/2011}\PY{l+s}{\PYZsq{}}\PY{p}{]}
\end{Verbatim}

    As before you can pass either a string date, \texttt{datetime}, or
\texttt{Timestamp}. Remember that slicing in this manner produces views
on the source time series just like slicing \texttt{NumPy} arrays. There
is an equivalent instance method \texttt{truncate} which slices a
\texttt{TimeSeries} between two dates:

    \begin{Verbatim}[commandchars=\\\{\}]
{\color{incolor}In [{\color{incolor} }]:} \PY{n}{ts}\PY{o}{.}\PY{n}{truncate}\PY{p}{(}\PY{n}{after}\PY{o}{=}\PY{l+s}{\PYZsq{}}\PY{l+s}{1/9/2011}\PY{l+s}{\PYZsq{}}\PY{p}{)}
\end{Verbatim}

    All of the above holds true for \texttt{DataFrame} as well, indexing on
its rows:

    \begin{Verbatim}[commandchars=\\\{\}]
{\color{incolor}In [{\color{incolor} }]:} \PY{n}{dates} \PY{o}{=} \PY{n}{pd}\PY{o}{.}\PY{n}{date\PYZus{}range}\PY{p}{(}\PY{l+s}{\PYZsq{}}\PY{l+s}{1/1/2000}\PY{l+s}{\PYZsq{}}\PY{p}{,} \PY{n}{periods}\PY{o}{=}\PY{l+m+mi}{100}\PY{p}{,} \PY{n}{freq}\PY{o}{=}\PY{l+s}{\PYZsq{}}\PY{l+s}{W\PYZhy{}WED}\PY{l+s}{\PYZsq{}}\PY{p}{)}
        \PY{n}{long\PYZus{}df} \PY{o}{=} \PY{n}{DataFrame}\PY{p}{(}\PY{n}{np}\PY{o}{.}\PY{n}{random}\PY{o}{.}\PY{n}{randn}\PY{p}{(}\PY{l+m+mi}{100}\PY{p}{,} \PY{l+m+mi}{4}\PY{p}{)}\PY{p}{,}
                            \PY{n}{index}\PY{o}{=}\PY{n}{dates}\PY{p}{,}
                            \PY{n}{columns}\PY{o}{=}\PY{p}{[}\PY{l+s}{\PYZsq{}}\PY{l+s}{Colorado}\PY{l+s}{\PYZsq{}}\PY{p}{,} \PY{l+s}{\PYZsq{}}\PY{l+s}{Texas}\PY{l+s}{\PYZsq{}}\PY{p}{,} \PY{l+s}{\PYZsq{}}\PY{l+s}{New York}\PY{l+s}{\PYZsq{}}\PY{p}{,} \PY{l+s}{\PYZsq{}}\PY{l+s}{Ohio}\PY{l+s}{\PYZsq{}}\PY{p}{]}\PY{p}{)}
        \PY{n}{long\PYZus{}df}\PY{o}{.}\PY{n}{ix}\PY{p}{[}\PY{l+s}{\PYZsq{}}\PY{l+s}{5\PYZhy{}2001}\PY{l+s}{\PYZsq{}}\PY{p}{]}
\end{Verbatim}

    \subsubsection{Time series with duplicate
indices}\label{time-series-with-duplicate-indices}

    In some applications, there may be multiple data observations falling on
a particular timestamp. Here is an example:

    \begin{Verbatim}[commandchars=\\\{\}]
{\color{incolor}In [{\color{incolor} }]:} \PY{n}{dates} \PY{o}{=} \PY{n}{pd}\PY{o}{.}\PY{n}{DatetimeIndex}\PY{p}{(}\PY{p}{[}\PY{l+s}{\PYZsq{}}\PY{l+s}{1/1/2000}\PY{l+s}{\PYZsq{}}\PY{p}{,} \PY{l+s}{\PYZsq{}}\PY{l+s}{1/2/2000}\PY{l+s}{\PYZsq{}}\PY{p}{,} \PY{l+s}{\PYZsq{}}\PY{l+s}{1/2/2000}\PY{l+s}{\PYZsq{}}\PY{p}{,} \PY{l+s}{\PYZsq{}}\PY{l+s}{1/2/2000}\PY{l+s}{\PYZsq{}}\PY{p}{,}
                                  \PY{l+s}{\PYZsq{}}\PY{l+s}{1/3/2000}\PY{l+s}{\PYZsq{}}\PY{p}{]}\PY{p}{)}
        \PY{n}{dup\PYZus{}ts} \PY{o}{=} \PY{n}{Series}\PY{p}{(}\PY{n}{np}\PY{o}{.}\PY{n}{arange}\PY{p}{(}\PY{l+m+mi}{5}\PY{p}{)}\PY{p}{,} \PY{n}{index}\PY{o}{=}\PY{n}{dates}\PY{p}{)}
        \PY{n}{dup\PYZus{}ts}
\end{Verbatim}

    We can tell that the index is not unique by checking its
\texttt{is\_unique} property:

    \begin{Verbatim}[commandchars=\\\{\}]
{\color{incolor}In [{\color{incolor} }]:} \PY{n}{dup\PYZus{}ts}\PY{o}{.}\PY{n}{index}\PY{o}{.}\PY{n}{is\PYZus{}unique}
\end{Verbatim}

    Indexing into this time series will now either produce scalar values or
slices depending on whether a timestamp is duplicated:

    \begin{Verbatim}[commandchars=\\\{\}]
{\color{incolor}In [{\color{incolor} }]:} \PY{n}{dup\PYZus{}ts}\PY{p}{[}\PY{l+s}{\PYZsq{}}\PY{l+s}{1/3/2000}\PY{l+s}{\PYZsq{}}\PY{p}{]}  \PY{c}{\PYZsh{} not duplicated}
\end{Verbatim}

    \begin{Verbatim}[commandchars=\\\{\}]
{\color{incolor}In [{\color{incolor} }]:} \PY{n}{dup\PYZus{}ts}\PY{p}{[}\PY{l+s}{\PYZsq{}}\PY{l+s}{1/2/2000}\PY{l+s}{\PYZsq{}}\PY{p}{]}  \PY{c}{\PYZsh{} duplicated}
\end{Verbatim}

    Suppose you want to aggregate the data having non-unique timestamps. One
way to do this is to use \texttt{groupby} and pass \texttt{level=0} (the
only level of indexing!):

    \begin{Verbatim}[commandchars=\\\{\}]
{\color{incolor}In [{\color{incolor} }]:} \PY{n}{grouped} \PY{o}{=} \PY{n}{dup\PYZus{}ts}\PY{o}{.}\PY{n}{groupby}\PY{p}{(}\PY{n}{level}\PY{o}{=}\PY{l+m+mi}{0}\PY{p}{)}
        \PY{n}{grouped}\PY{o}{.}\PY{n}{mean}\PY{p}{(}\PY{p}{)}
\end{Verbatim}

    \begin{Verbatim}[commandchars=\\\{\}]
{\color{incolor}In [{\color{incolor} }]:} \PY{n}{grouped}\PY{o}{.}\PY{n}{count}\PY{p}{(}\PY{p}{)}
\end{Verbatim}

    \subsection{Date ranges, Frequencies, and
Shifting}\label{date-ranges-frequencies-and-shifting}

    Generic time series in \texttt{Pandas} are assumed to be irregular; that
is, they have no fixed frequency. For many applications this is
sufficient. However, it's often desirable to work relative to a fixed
frequency, such as daily, monthly, or even 15 minutes, even if that
means introducing missing values into a time series. Fortunately
\texttt{Pandas} has a full suite of standard time series frequencies and
tools for resampling, inferring frequencies, and generating fixed
frequency date ranges. For example, in the example time series,
converting it to be fixed daily frequency can be accomplished by calling
\texttt{resample}:

    \begin{Verbatim}[commandchars=\\\{\}]
{\color{incolor}In [{\color{incolor} }]:} \PY{n}{ts}
\end{Verbatim}

    \begin{Verbatim}[commandchars=\\\{\}]
{\color{incolor}In [{\color{incolor} }]:} \PY{n}{ts}\PY{o}{.}\PY{n}{resample}\PY{p}{(}\PY{l+s}{\PYZsq{}}\PY{l+s}{D}\PY{l+s}{\PYZsq{}}\PY{p}{)}
\end{Verbatim}

    Conversion between frequencies or \emph{resampling} is a big enough
topic to have its own section later. Here, we'll see how to use the base
frequencies and multiples thereof.

    \subsubsection{Generating date ranges}\label{generating-date-ranges}

    You may have guessed that \texttt{pandas.date\_range} is responsible for
generating a \texttt{DatetimeIndex} with an indicated length according
to a particular frequency:

    \begin{Verbatim}[commandchars=\\\{\}]
{\color{incolor}In [{\color{incolor} }]:} \PY{n}{index} \PY{o}{=} \PY{n}{pd}\PY{o}{.}\PY{n}{date\PYZus{}range}\PY{p}{(}\PY{l+s}{\PYZsq{}}\PY{l+s}{4/1/2012}\PY{l+s}{\PYZsq{}}\PY{p}{,} \PY{l+s}{\PYZsq{}}\PY{l+s}{6/1/2012}\PY{l+s}{\PYZsq{}}\PY{p}{)}
        \PY{n}{index}
\end{Verbatim}

    By default, \texttt{date\_range} generates daily timestamps. If you pass
only a start or end date, you must pass a number of periods to generate:

    \begin{Verbatim}[commandchars=\\\{\}]
{\color{incolor}In [{\color{incolor} }]:} \PY{n}{pd}\PY{o}{.}\PY{n}{date\PYZus{}range}\PY{p}{(}\PY{n}{start}\PY{o}{=}\PY{l+s}{\PYZsq{}}\PY{l+s}{4/1/2012}\PY{l+s}{\PYZsq{}}\PY{p}{,} \PY{n}{periods}\PY{o}{=}\PY{l+m+mi}{20}\PY{p}{)}
\end{Verbatim}

    \begin{Verbatim}[commandchars=\\\{\}]
{\color{incolor}In [{\color{incolor} }]:} \PY{n}{pd}\PY{o}{.}\PY{n}{date\PYZus{}range}\PY{p}{(}\PY{n}{end}\PY{o}{=}\PY{l+s}{\PYZsq{}}\PY{l+s}{6/1/2012}\PY{l+s}{\PYZsq{}}\PY{p}{,} \PY{n}{periods}\PY{o}{=}\PY{l+m+mi}{20}\PY{p}{)}
\end{Verbatim}

    The start and end dates define strict boundaries for the generated date
index. For example, if you wanted a date index containing the last
business day of each month, you would pass the \texttt{'BM'} frequency
(business end of month) and only dates falling on or inside the date
interval will be included:

    \begin{Verbatim}[commandchars=\\\{\}]
{\color{incolor}In [{\color{incolor} }]:} \PY{n}{pd}\PY{o}{.}\PY{n}{date\PYZus{}range}\PY{p}{(}\PY{l+s}{\PYZsq{}}\PY{l+s}{1/1/2000}\PY{l+s}{\PYZsq{}}\PY{p}{,} \PY{l+s}{\PYZsq{}}\PY{l+s}{12/1/2000}\PY{l+s}{\PYZsq{}}\PY{p}{,} \PY{n}{freq}\PY{o}{=}\PY{l+s}{\PYZsq{}}\PY{l+s}{BM}\PY{l+s}{\PYZsq{}}\PY{p}{)}
\end{Verbatim}

    \texttt{date\_range} by default preserves the time (if any) or the start
or end timestamp:

    \begin{Verbatim}[commandchars=\\\{\}]
{\color{incolor}In [{\color{incolor} }]:} \PY{n}{pd}\PY{o}{.}\PY{n}{date\PYZus{}range}\PY{p}{(}\PY{l+s}{\PYZsq{}}\PY{l+s}{5/2/2012 12:56:31}\PY{l+s}{\PYZsq{}}\PY{p}{,} \PY{n}{periods}\PY{o}{=}\PY{l+m+mi}{5}\PY{p}{)}
\end{Verbatim}

    Sometimes you will have start or end dates with time information but
want to generate a set of timestamps \emph{normalized} to midnight as a
convention. To do this, there is a \texttt{normalize} option:

    \begin{Verbatim}[commandchars=\\\{\}]
{\color{incolor}In [{\color{incolor} }]:} \PY{n}{pd}\PY{o}{.}\PY{n}{date\PYZus{}range}\PY{p}{(}\PY{l+s}{\PYZsq{}}\PY{l+s}{5/2/2012 12:56:31}\PY{l+s}{\PYZsq{}}\PY{p}{,} \PY{n}{periods}\PY{o}{=}\PY{l+m+mi}{5}\PY{p}{,} \PY{n}{normalize}\PY{o}{=}\PY{n+nb+bp}{True}\PY{p}{)}
\end{Verbatim}

    \subsubsection{Frequencies and Date
Offsets}\label{frequencies-and-date-offsets}

    Frequencies in \texttt{Pandas} are composed of a \emph{base frequency}
and a multiplier. Base frequencies are typically referred to by a string
alias, like \texttt{'M'} for monthly or \texttt{'H'} for hourly. For
each base frequency, there is an object defined generally referred to as
a \emph{date offset}. For each example, hourly frequency can be
represented with the \texttt{Hour} class:

    \begin{Verbatim}[commandchars=\\\{\}]
{\color{incolor}In [{\color{incolor} }]:} \PY{k+kn}{from} \PY{n+nn}{pandas.tseries.offsets} \PY{k+kn}{import} \PY{n}{Hour}\PY{p}{,} \PY{n}{Minute}
        \PY{n}{hour} \PY{o}{=} \PY{n}{Hour}\PY{p}{(}\PY{p}{)}
        \PY{n}{hour}
\end{Verbatim}

    You can define a multiple of an offset by passing an integer:

    \begin{Verbatim}[commandchars=\\\{\}]
{\color{incolor}In [{\color{incolor} }]:} \PY{n}{four\PYZus{}hours} \PY{o}{=} \PY{n}{Hour}\PY{p}{(}\PY{l+m+mi}{4}\PY{p}{)}
        \PY{n}{four\PYZus{}hours}
\end{Verbatim}

    In most applications, you would never need to explicitly create one of
these objects, instead using a string alias like \texttt{'H'} or
\texttt{'4H'}. Putting an integer before the base frequency creates a
multiple:

    \begin{Verbatim}[commandchars=\\\{\}]
{\color{incolor}In [{\color{incolor} }]:} \PY{n}{pd}\PY{o}{.}\PY{n}{date\PYZus{}range}\PY{p}{(}\PY{l+s}{\PYZsq{}}\PY{l+s}{1/1/2000}\PY{l+s}{\PYZsq{}}\PY{p}{,} \PY{l+s}{\PYZsq{}}\PY{l+s}{1/3/2000 23:59}\PY{l+s}{\PYZsq{}}\PY{p}{,} \PY{n}{freq}\PY{o}{=}\PY{l+s}{\PYZsq{}}\PY{l+s}{4h}\PY{l+s}{\PYZsq{}}\PY{p}{)}
\end{Verbatim}

    Many offsets can be combined together by addition:

    \begin{Verbatim}[commandchars=\\\{\}]
{\color{incolor}In [{\color{incolor} }]:} \PY{n}{Hour}\PY{p}{(}\PY{l+m+mi}{2}\PY{p}{)} \PY{o}{+} \PY{n}{Minute}\PY{p}{(}\PY{l+m+mi}{30}\PY{p}{)}
\end{Verbatim}

    Similarly, you can pass frequency strings like \texttt{'2h30min'} which
will effectively be parsed to the same expression.

    \begin{Verbatim}[commandchars=\\\{\}]
{\color{incolor}In [{\color{incolor} }]:} \PY{n}{pd}\PY{o}{.}\PY{n}{date\PYZus{}range}\PY{p}{(}\PY{l+s}{\PYZsq{}}\PY{l+s}{1/1/2000}\PY{l+s}{\PYZsq{}}\PY{p}{,} \PY{n}{periods}\PY{o}{=}\PY{l+m+mi}{10}\PY{p}{,} \PY{n}{freq}\PY{o}{=}\PY{l+s}{\PYZsq{}}\PY{l+s}{1h30min}\PY{l+s}{\PYZsq{}}\PY{p}{)}
\end{Verbatim}

    Some frequencies describe points in time that are not evenly spaced. For
example, \texttt{'M'} (calendar month end) and \texttt{'BM'} (last
business/weekday of month) depend on the number of days in a month and,
in the latter case, whether the month ends on a weekend or not. For lack
of a better term, we will call these \emph{anchored} offsets.

    \paragraph{Week of month dates}\label{week-of-month-dates}

    One useful frequency class is ``week of month'', starting with
\texttt{WOM}. This enables you to get dates like the third Friday of
each month:

    \begin{Verbatim}[commandchars=\\\{\}]
{\color{incolor}In [{\color{incolor} }]:} \PY{n}{rng} \PY{o}{=} \PY{n}{pd}\PY{o}{.}\PY{n}{date\PYZus{}range}\PY{p}{(}\PY{l+s}{\PYZsq{}}\PY{l+s}{1/1/2012}\PY{l+s}{\PYZsq{}}\PY{p}{,} \PY{l+s}{\PYZsq{}}\PY{l+s}{9/1/2012}\PY{l+s}{\PYZsq{}}\PY{p}{,} \PY{n}{freq}\PY{o}{=}\PY{l+s}{\PYZsq{}}\PY{l+s}{WOM\PYZhy{}3FRI}\PY{l+s}{\PYZsq{}}\PY{p}{)}
        \PY{n+nb}{list}\PY{p}{(}\PY{n}{rng}\PY{p}{)}
\end{Verbatim}

    Traders of US equity options will recognize thse dates as the standard
dates of monthly expiry.

    \subsubsection{Shifting (leading and lagging)
data}\label{shifting-leading-and-lagging-data}

    ``Shifting'' refers to moving data backward and forward through time.
Both \texttt{Series} and \texttt{DataFrame} have a \texttt{shift} method
for doing naive shifts forward or backward, leaving the index
unmodified:

    \begin{Verbatim}[commandchars=\\\{\}]
{\color{incolor}In [{\color{incolor} }]:} \PY{n}{ts} \PY{o}{=} \PY{n}{Series}\PY{p}{(}\PY{n}{np}\PY{o}{.}\PY{n}{random}\PY{o}{.}\PY{n}{randn}\PY{p}{(}\PY{l+m+mi}{4}\PY{p}{)}\PY{p}{,}
                    \PY{n}{index}\PY{o}{=}\PY{n}{pd}\PY{o}{.}\PY{n}{date\PYZus{}range}\PY{p}{(}\PY{l+s}{\PYZsq{}}\PY{l+s}{1/1/2000}\PY{l+s}{\PYZsq{}}\PY{p}{,} \PY{n}{periods}\PY{o}{=}\PY{l+m+mi}{4}\PY{p}{,} \PY{n}{freq}\PY{o}{=}\PY{l+s}{\PYZsq{}}\PY{l+s}{M}\PY{l+s}{\PYZsq{}}\PY{p}{)}\PY{p}{)}
        \PY{n}{ts}
\end{Verbatim}

    \begin{Verbatim}[commandchars=\\\{\}]
{\color{incolor}In [{\color{incolor} }]:} \PY{n}{ts}\PY{o}{.}\PY{n}{shift}\PY{p}{(}\PY{l+m+mi}{2}\PY{p}{)}
\end{Verbatim}

    \begin{Verbatim}[commandchars=\\\{\}]
{\color{incolor}In [{\color{incolor} }]:} \PY{n}{ts}\PY{o}{.}\PY{n}{shift}\PY{p}{(}\PY{o}{\PYZhy{}}\PY{l+m+mi}{2}\PY{p}{)}
\end{Verbatim}

    A common use of \texttt{shift} is computing percent changes in a time
series or multiple time series as \texttt{DataFrame} columns. This is
expressed as
ts / ts.shift(1) - 1
    Because naive shifts leave the index unmodified, some data is discarded.
Thus if the frequency is known, it can be passed to \texttt{shift} to
advance the timestamps instead of simply the data

    \begin{Verbatim}[commandchars=\\\{\}]
{\color{incolor}In [{\color{incolor} }]:} \PY{n}{ts}\PY{o}{.}\PY{n}{shift}\PY{p}{(}\PY{l+m+mi}{2}\PY{p}{,} \PY{n}{freq}\PY{o}{=}\PY{l+s}{\PYZsq{}}\PY{l+s}{M}\PY{l+s}{\PYZsq{}}\PY{p}{)}
\end{Verbatim}

    Other frequencies can be passed, too, giving you a lot of flexibility in
how to lead and lag the data

    \begin{Verbatim}[commandchars=\\\{\}]
{\color{incolor}In [{\color{incolor} }]:} \PY{n}{ts}\PY{o}{.}\PY{n}{shift}\PY{p}{(}\PY{l+m+mi}{3}\PY{p}{,} \PY{n}{freq}\PY{o}{=}\PY{l+s}{\PYZsq{}}\PY{l+s}{D}\PY{l+s}{\PYZsq{}}\PY{p}{)}
\end{Verbatim}

    \begin{Verbatim}[commandchars=\\\{\}]
{\color{incolor}In [{\color{incolor} }]:} \PY{n}{ts}\PY{o}{.}\PY{n}{shift}\PY{p}{(}\PY{l+m+mi}{1}\PY{p}{,} \PY{n}{freq}\PY{o}{=}\PY{l+s}{\PYZsq{}}\PY{l+s}{3D}\PY{l+s}{\PYZsq{}}\PY{p}{)}
\end{Verbatim}

    \begin{Verbatim}[commandchars=\\\{\}]
{\color{incolor}In [{\color{incolor} }]:} \PY{n}{ts}\PY{o}{.}\PY{n}{shift}\PY{p}{(}\PY{l+m+mi}{1}\PY{p}{,} \PY{n}{freq}\PY{o}{=}\PY{l+s}{\PYZsq{}}\PY{l+s}{90T}\PY{l+s}{\PYZsq{}}\PY{p}{)}
\end{Verbatim}

    \paragraph{Shifting dates with
offsets}\label{shifting-dates-with-offsets}

    The \texttt{Pandas} date offsets can also be used with \texttt{datetime}
or \texttt{Timestamp} objects:

    \begin{Verbatim}[commandchars=\\\{\}]
{\color{incolor}In [{\color{incolor} }]:} \PY{k+kn}{from} \PY{n+nn}{pandas.tseries.offsets} \PY{k+kn}{import} \PY{n}{Day}\PY{p}{,} \PY{n}{MonthEnd}
        \PY{n}{now} \PY{o}{=} \PY{n}{datetime}\PY{p}{(}\PY{l+m+mi}{2011}\PY{p}{,} \PY{l+m+mi}{11}\PY{p}{,} \PY{l+m+mi}{17}\PY{p}{)}
        \PY{n}{now} \PY{o}{+} \PY{l+m+mi}{3} \PY{o}{*} \PY{n}{Day}\PY{p}{(}\PY{p}{)}
\end{Verbatim}

    If you add an anchored offset like \texttt{MonthEnd}, the first
increment will \texttt{roll forward} a date to the next date according
to the frequency rule:

    \begin{Verbatim}[commandchars=\\\{\}]
{\color{incolor}In [{\color{incolor} }]:} \PY{n}{now} \PY{o}{+} \PY{n}{MonthEnd}\PY{p}{(}\PY{p}{)}
\end{Verbatim}

    \begin{Verbatim}[commandchars=\\\{\}]
{\color{incolor}In [{\color{incolor} }]:} \PY{n}{now} \PY{o}{+} \PY{n}{MonthEnd}\PY{p}{(}\PY{l+m+mi}{2}\PY{p}{)}
\end{Verbatim}

    Anchored offsets can explicitly ``roll'' dates forward or backward using
their \texttt{rollforward} and \texttt{rollback} methods, respectively:

    \begin{Verbatim}[commandchars=\\\{\}]
{\color{incolor}In [{\color{incolor} }]:} \PY{n}{offset} \PY{o}{=} \PY{n}{MonthEnd}\PY{p}{(}\PY{p}{)}
        \PY{n}{offset}\PY{o}{.}\PY{n}{rollforward}\PY{p}{(}\PY{n}{now}\PY{p}{)}
\end{Verbatim}

    \begin{Verbatim}[commandchars=\\\{\}]
{\color{incolor}In [{\color{incolor} }]:} \PY{n}{offset}\PY{o}{.}\PY{n}{rollback}\PY{p}{(}\PY{n}{now}\PY{p}{)}
\end{Verbatim}

    A clever use of date offsets is to use these methods with groupby:

    \begin{Verbatim}[commandchars=\\\{\}]
{\color{incolor}In [{\color{incolor} }]:} \PY{n}{ts} \PY{o}{=} \PY{n}{Series}\PY{p}{(}\PY{n}{np}\PY{o}{.}\PY{n}{random}\PY{o}{.}\PY{n}{randn}\PY{p}{(}\PY{l+m+mi}{20}\PY{p}{)}\PY{p}{,}
                    \PY{n}{index}\PY{o}{=}\PY{n}{pd}\PY{o}{.}\PY{n}{date\PYZus{}range}\PY{p}{(}\PY{l+s}{\PYZsq{}}\PY{l+s}{1/15/2000}\PY{l+s}{\PYZsq{}}\PY{p}{,} \PY{n}{periods}\PY{o}{=}\PY{l+m+mi}{20}\PY{p}{,} \PY{n}{freq}\PY{o}{=}\PY{l+s}{\PYZsq{}}\PY{l+s}{4d}\PY{l+s}{\PYZsq{}}\PY{p}{)}\PY{p}{)}
        \PY{n}{ts}\PY{o}{.}\PY{n}{groupby}\PY{p}{(}\PY{n}{offset}\PY{o}{.}\PY{n}{rollforward}\PY{p}{)}\PY{o}{.}\PY{n}{mean}\PY{p}{(}\PY{p}{)}
\end{Verbatim}

    Of course, an easier and faster way to do this is using
\texttt{resample} (more on this to come).

    \begin{Verbatim}[commandchars=\\\{\}]
{\color{incolor}In [{\color{incolor} }]:} \PY{n}{ts}\PY{o}{.}\PY{n}{resample}\PY{p}{(}\PY{l+s}{\PYZsq{}}\PY{l+s}{M}\PY{l+s}{\PYZsq{}}\PY{p}{,} \PY{n}{how}\PY{o}{=}\PY{l+s}{\PYZsq{}}\PY{l+s}{mean}\PY{l+s}{\PYZsq{}}\PY{p}{)}
\end{Verbatim}

    \subsection{Time Zone Handling}\label{time-zone-handling}

    Working with time zones is a pain. As Americans hold on dearly to
daylight savings time, we must pay the price with difficult conversions
between time zones. Many time series users choose to work with time
series in \emph{coordinated universal time (UTC)} of which time zones
can be expressed as offsets.

In \texttt{Python} we can use the \texttt{pytz} library, based off the
Olson \emph{database} of world time zone data.

    \begin{Verbatim}[commandchars=\\\{\}]
{\color{incolor}In [{\color{incolor} }]:} \PY{k+kn}{import} \PY{n+nn}{pytz}
        \PY{n}{pytz}\PY{o}{.}\PY{n}{common\PYZus{}timezones}\PY{p}{[}\PY{o}{\PYZhy{}}\PY{l+m+mi}{5}\PY{p}{:}\PY{p}{]}
\end{Verbatim}

    To get a time zone object from \texttt{pytz}, use
\texttt{pytz.timezone}.

    \begin{Verbatim}[commandchars=\\\{\}]
{\color{incolor}In [{\color{incolor} }]:} \PY{n}{tz} \PY{o}{=} \PY{n}{pytz}\PY{o}{.}\PY{n}{timezone}\PY{p}{(}\PY{l+s}{\PYZsq{}}\PY{l+s}{US/Eastern}\PY{l+s}{\PYZsq{}}\PY{p}{)}
        \PY{n}{tz}
\end{Verbatim}

    Methods in \texttt{Pandas} will accept either time zone names or these
objects. Using the names is recommended.

    \subsubsection{Localization and
Conversion}\label{localization-and-conversion}

    By default, time series in \texttt{Pandas} are \emph{time zone naive}.
Consider the following time series:

    \begin{Verbatim}[commandchars=\\\{\}]
{\color{incolor}In [{\color{incolor} }]:} \PY{n}{rng} \PY{o}{=} \PY{n}{pd}\PY{o}{.}\PY{n}{date\PYZus{}range}\PY{p}{(}\PY{l+s}{\PYZsq{}}\PY{l+s}{3/9/2012 9:30}\PY{l+s}{\PYZsq{}}\PY{p}{,} \PY{n}{periods}\PY{o}{=}\PY{l+m+mi}{6}\PY{p}{,} \PY{n}{freq}\PY{o}{=}\PY{l+s}{\PYZsq{}}\PY{l+s}{D}\PY{l+s}{\PYZsq{}}\PY{p}{)}
        \PY{n}{ts} \PY{o}{=} \PY{n}{Series}\PY{p}{(}\PY{n}{np}\PY{o}{.}\PY{n}{random}\PY{o}{.}\PY{n}{randn}\PY{p}{(}\PY{n+nb}{len}\PY{p}{(}\PY{n}{rng}\PY{p}{)}\PY{p}{)}\PY{p}{,} \PY{n}{index}\PY{o}{=}\PY{n}{rng}\PY{p}{)}
\end{Verbatim}

    The index's \texttt{tz} field is \texttt{None}:

    \begin{Verbatim}[commandchars=\\\{\}]
{\color{incolor}In [{\color{incolor} }]:} \PY{k}{print}\PY{p}{(}\PY{n}{ts}\PY{o}{.}\PY{n}{index}\PY{o}{.}\PY{n}{tz}\PY{p}{)}
\end{Verbatim}

    Date ranges can be generated with a time zone set:

    \begin{Verbatim}[commandchars=\\\{\}]
{\color{incolor}In [{\color{incolor} }]:} \PY{n}{pd}\PY{o}{.}\PY{n}{date\PYZus{}range}\PY{p}{(}\PY{l+s}{\PYZsq{}}\PY{l+s}{3/9/2012 9:30}\PY{l+s}{\PYZsq{}}\PY{p}{,} \PY{n}{periods}\PY{o}{=}\PY{l+m+mi}{10}\PY{p}{,} \PY{n}{freq}\PY{o}{=}\PY{l+s}{\PYZsq{}}\PY{l+s}{D}\PY{l+s}{\PYZsq{}}\PY{p}{,} \PY{n}{tz}\PY{o}{=}\PY{l+s}{\PYZsq{}}\PY{l+s}{UTC}\PY{l+s}{\PYZsq{}}\PY{p}{)}
\end{Verbatim}

    Conversion from naive to \emph{localized} is handled by the
\texttt{tz\_localize} method

    \begin{Verbatim}[commandchars=\\\{\}]
{\color{incolor}In [{\color{incolor} }]:} \PY{n}{ts\PYZus{}utc} \PY{o}{=} \PY{n}{ts}\PY{o}{.}\PY{n}{tz\PYZus{}localize}\PY{p}{(}\PY{l+s}{\PYZsq{}}\PY{l+s}{UTC}\PY{l+s}{\PYZsq{}}\PY{p}{)}
        \PY{n}{ts\PYZus{}utc}
\end{Verbatim}

    \begin{Verbatim}[commandchars=\\\{\}]
{\color{incolor}In [{\color{incolor} }]:} \PY{n}{ts\PYZus{}utc}\PY{o}{.}\PY{n}{index}
\end{Verbatim}

    Once a time series has been localized to a particular time zone, it can
be converted to another time zone using \texttt{tz\_convert}.

    \begin{Verbatim}[commandchars=\\\{\}]
{\color{incolor}In [{\color{incolor} }]:} \PY{n}{ts\PYZus{}utc}\PY{o}{.}\PY{n}{tz\PYZus{}convert}\PY{p}{(}\PY{l+s}{\PYZsq{}}\PY{l+s}{US/Eastern}\PY{l+s}{\PYZsq{}}\PY{p}{)}
\end{Verbatim}

    In this case of the above time series, which straddles a DST transition
in the US/Eastern time zone, we could localize to EST and convert to,
say, UTC or Berlin time.

    \begin{Verbatim}[commandchars=\\\{\}]
{\color{incolor}In [{\color{incolor} }]:} \PY{n}{ts\PYZus{}eastern} \PY{o}{=} \PY{n}{ts}\PY{o}{.}\PY{n}{tz\PYZus{}localize}\PY{p}{(}\PY{l+s}{\PYZsq{}}\PY{l+s}{US/Eastern}\PY{l+s}{\PYZsq{}}\PY{p}{)}
        \PY{n}{ts\PYZus{}eastern}\PY{o}{.}\PY{n}{tz\PYZus{}convert}\PY{p}{(}\PY{l+s}{\PYZsq{}}\PY{l+s}{UTC}\PY{l+s}{\PYZsq{}}\PY{p}{)}
\end{Verbatim}

    \begin{Verbatim}[commandchars=\\\{\}]
{\color{incolor}In [{\color{incolor} }]:} \PY{n}{ts\PYZus{}eastern}\PY{o}{.}\PY{n}{tz\PYZus{}convert}\PY{p}{(}\PY{l+s}{\PYZsq{}}\PY{l+s}{Europe/Berlin}\PY{l+s}{\PYZsq{}}\PY{p}{)}
\end{Verbatim}

    \texttt{tz\_localize} and \texttt{tz\_convert} are also instance methods
on \texttt{DatetimeIndex}.

    \begin{Verbatim}[commandchars=\\\{\}]
{\color{incolor}In [{\color{incolor} }]:} \PY{n}{ts}\PY{o}{.}\PY{n}{index}\PY{o}{.}\PY{n}{tz\PYZus{}localize}\PY{p}{(}\PY{l+s}{\PYZsq{}}\PY{l+s}{Asia/Shanghai}\PY{l+s}{\PYZsq{}}\PY{p}{)}
\end{Verbatim}

    \subsubsection{Operations with time zone-aware Timestamp
objects}\label{operations-with-time-zone-aware-timestamp-objects}

    Similar to time series and date ranges, individual \texttt{Timestamp}
objects similarly can be localized from naive to time zone-aware and
converted from one time zone to another:

    \begin{Verbatim}[commandchars=\\\{\}]
{\color{incolor}In [{\color{incolor} }]:} \PY{n}{stamp} \PY{o}{=} \PY{n}{pd}\PY{o}{.}\PY{n}{Timestamp}\PY{p}{(}\PY{l+s}{\PYZsq{}}\PY{l+s}{2011\PYZhy{}03\PYZhy{}12 04:00}\PY{l+s}{\PYZsq{}}\PY{p}{)}
        \PY{n}{stamp\PYZus{}utc} \PY{o}{=} \PY{n}{stamp}\PY{o}{.}\PY{n}{tz\PYZus{}localize}\PY{p}{(}\PY{l+s}{\PYZsq{}}\PY{l+s}{utc}\PY{l+s}{\PYZsq{}}\PY{p}{)}
        \PY{n}{stamp\PYZus{}utc}\PY{o}{.}\PY{n}{tz\PYZus{}convert}\PY{p}{(}\PY{l+s}{\PYZsq{}}\PY{l+s}{US/Eastern}\PY{l+s}{\PYZsq{}}\PY{p}{)}
\end{Verbatim}

    You can also pass a time zone when creating the \texttt{Timestamp}.

    \begin{Verbatim}[commandchars=\\\{\}]
{\color{incolor}In [{\color{incolor} }]:} \PY{n}{stamp\PYZus{}moscow} \PY{o}{=} \PY{n}{pd}\PY{o}{.}\PY{n}{Timestamp}\PY{p}{(}\PY{l+s}{\PYZsq{}}\PY{l+s}{2011\PYZhy{}03\PYZhy{}12 04:00}\PY{l+s}{\PYZsq{}}\PY{p}{,} \PY{n}{tz}\PY{o}{=}\PY{l+s}{\PYZsq{}}\PY{l+s}{Europe/Moscow}\PY{l+s}{\PYZsq{}}\PY{p}{)}
        \PY{n}{stamp\PYZus{}moscow}
\end{Verbatim}

    Time zone-aware \texttt{Timestamp} objects internally store a UTC
timestamp value as nanoseconds since the UNIX epoch (January 1, 1970);
this UTC value is invariant between time zone conversions:

    \begin{Verbatim}[commandchars=\\\{\}]
{\color{incolor}In [{\color{incolor} }]:} \PY{n}{stamp\PYZus{}utc}\PY{o}{.}\PY{n}{value}
\end{Verbatim}

    \begin{Verbatim}[commandchars=\\\{\}]
{\color{incolor}In [{\color{incolor} }]:} \PY{n}{stamp\PYZus{}utc}\PY{o}{.}\PY{n}{tz\PYZus{}convert}\PY{p}{(}\PY{l+s}{\PYZsq{}}\PY{l+s}{US/Eastern}\PY{l+s}{\PYZsq{}}\PY{p}{)}\PY{o}{.}\PY{n}{value}
\end{Verbatim}

    When performing time arithmetic using \texttt{Pandas}'
\texttt{DateOffset} objects, daylight savings time transitions are
respected where possible

    \begin{Verbatim}[commandchars=\\\{\}]
{\color{incolor}In [{\color{incolor} }]:} \PY{c}{\PYZsh{} 30 minutes before DST transition}
        \PY{k+kn}{from} \PY{n+nn}{pandas.tseries.offsets} \PY{k+kn}{import} \PY{n}{Hour}
        \PY{n}{stamp} \PY{o}{=} \PY{n}{pd}\PY{o}{.}\PY{n}{Timestamp}\PY{p}{(}\PY{l+s}{\PYZsq{}}\PY{l+s}{2012\PYZhy{}03\PYZhy{}12 01:30}\PY{l+s}{\PYZsq{}}\PY{p}{,} \PY{n}{tz}\PY{o}{=}\PY{l+s}{\PYZsq{}}\PY{l+s}{US/Eastern}\PY{l+s}{\PYZsq{}}\PY{p}{)}
        \PY{n}{stamp}
\end{Verbatim}

    \begin{Verbatim}[commandchars=\\\{\}]
{\color{incolor}In [{\color{incolor} }]:} \PY{n}{stamp} \PY{o}{+} \PY{n}{Hour}\PY{p}{(}\PY{p}{)}
\end{Verbatim}

    \begin{Verbatim}[commandchars=\\\{\}]
{\color{incolor}In [{\color{incolor} }]:} \PY{c}{\PYZsh{} 90 minutes before DST transition}
        \PY{n}{stamp} \PY{o}{=} \PY{n}{pd}\PY{o}{.}\PY{n}{Timestamp}\PY{p}{(}\PY{l+s}{\PYZsq{}}\PY{l+s}{2012\PYZhy{}11\PYZhy{}04 00:30}\PY{l+s}{\PYZsq{}}\PY{p}{,} \PY{n}{tz}\PY{o}{=}\PY{l+s}{\PYZsq{}}\PY{l+s}{US/Eastern}\PY{l+s}{\PYZsq{}}\PY{p}{)}
        \PY{n}{stamp}
\end{Verbatim}

    \begin{Verbatim}[commandchars=\\\{\}]
{\color{incolor}In [{\color{incolor} }]:} \PY{n}{stamp} \PY{o}{+} \PY{l+m+mi}{2} \PY{o}{*} \PY{n}{Hour}\PY{p}{(}\PY{p}{)}
\end{Verbatim}

    \subsubsection{Operations between different time
zones}\label{operations-between-different-time-zones}

    If two time series with different time zones are combined, the result
will be UTC. Since the timestamps are stored under the hood in UTC, this
is a straightforward operation and requires no conversion to happen.

    \begin{Verbatim}[commandchars=\\\{\}]
{\color{incolor}In [{\color{incolor} }]:} \PY{n}{rng} \PY{o}{=} \PY{n}{pd}\PY{o}{.}\PY{n}{date\PYZus{}range}\PY{p}{(}\PY{l+s}{\PYZsq{}}\PY{l+s}{3/7/2012 9:30}\PY{l+s}{\PYZsq{}}\PY{p}{,} \PY{n}{periods}\PY{o}{=}\PY{l+m+mi}{10}\PY{p}{,} \PY{n}{freq}\PY{o}{=}\PY{l+s}{\PYZsq{}}\PY{l+s}{B}\PY{l+s}{\PYZsq{}}\PY{p}{)}
        \PY{n}{ts} \PY{o}{=} \PY{n}{Series}\PY{p}{(}\PY{n}{np}\PY{o}{.}\PY{n}{random}\PY{o}{.}\PY{n}{randn}\PY{p}{(}\PY{n+nb}{len}\PY{p}{(}\PY{n}{rng}\PY{p}{)}\PY{p}{)}\PY{p}{,} \PY{n}{index}\PY{o}{=}\PY{n}{rng}\PY{p}{)}
        \PY{n}{ts}
\end{Verbatim}

    \begin{Verbatim}[commandchars=\\\{\}]
{\color{incolor}In [{\color{incolor} }]:} \PY{n}{ts1} \PY{o}{=} \PY{n}{ts}\PY{p}{[}\PY{p}{:}\PY{l+m+mi}{7}\PY{p}{]}\PY{o}{.}\PY{n}{tz\PYZus{}localize}\PY{p}{(}\PY{l+s}{\PYZsq{}}\PY{l+s}{Europe/London}\PY{l+s}{\PYZsq{}}\PY{p}{)}
        \PY{n}{ts2} \PY{o}{=} \PY{n}{ts1}\PY{p}{[}\PY{l+m+mi}{2}\PY{p}{:}\PY{p}{]}\PY{o}{.}\PY{n}{tz\PYZus{}convert}\PY{p}{(}\PY{l+s}{\PYZsq{}}\PY{l+s}{Europe/Moscow}\PY{l+s}{\PYZsq{}}\PY{p}{)}
        \PY{n}{result} \PY{o}{=} \PY{n}{ts1} \PY{o}{+} \PY{n}{ts2}
        \PY{n}{result}\PY{o}{.}\PY{n}{index}
\end{Verbatim}


    % Add a bibliography block to the postdoc
    
    
    
    \end{document}
