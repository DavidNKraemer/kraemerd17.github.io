
% Default to the notebook output style

    


% Inherit from the specified cell style.




    
\documentclass{article}

    
    
    \usepackage{graphicx} % Used to insert images
    \usepackage{adjustbox} % Used to constrain images to a maximum size 
    \usepackage{color} % Allow colors to be defined
    \usepackage{enumerate} % Needed for markdown enumerations to work
    \usepackage{geometry} % Used to adjust the document margins
    \usepackage{amsmath} % Equations
    \usepackage{amssymb} % Equations
    \usepackage{eurosym} % defines \euro
    \usepackage[mathletters]{ucs} % Extended unicode (utf-8) support
    \usepackage[utf8x]{inputenc} % Allow utf-8 characters in the tex document
    \usepackage{fancyvrb} % verbatim replacement that allows latex
    \usepackage{grffile} % extends the file name processing of package graphics 
                         % to support a larger range 
    % The hyperref package gives us a pdf with properly built
    % internal navigation ('pdf bookmarks' for the table of contents,
    % internal cross-reference links, web links for URLs, etc.)
    \usepackage{hyperref}
    \usepackage{longtable} % longtable support required by pandoc >1.10
    \usepackage{booktabs}  % table support for pandoc > 1.12.2
    

    
    
    \definecolor{orange}{cmyk}{0,0.4,0.8,0.2}
    \definecolor{darkorange}{rgb}{.71,0.21,0.01}
    \definecolor{darkgreen}{rgb}{.12,.54,.11}
    \definecolor{myteal}{rgb}{.26, .44, .56}
    \definecolor{gray}{gray}{0.45}
    \definecolor{lightgray}{gray}{.95}
    \definecolor{mediumgray}{gray}{.8}
    \definecolor{inputbackground}{rgb}{.95, .95, .85}
    \definecolor{outputbackground}{rgb}{.95, .95, .95}
    \definecolor{traceback}{rgb}{1, .95, .95}
    % ansi colors
    \definecolor{red}{rgb}{.6,0,0}
    \definecolor{green}{rgb}{0,.65,0}
    \definecolor{brown}{rgb}{0.6,0.6,0}
    \definecolor{blue}{rgb}{0,.145,.698}
    \definecolor{purple}{rgb}{.698,.145,.698}
    \definecolor{cyan}{rgb}{0,.698,.698}
    \definecolor{lightgray}{gray}{0.5}
    
    % bright ansi colors
    \definecolor{darkgray}{gray}{0.25}
    \definecolor{lightred}{rgb}{1.0,0.39,0.28}
    \definecolor{lightgreen}{rgb}{0.48,0.99,0.0}
    \definecolor{lightblue}{rgb}{0.53,0.81,0.92}
    \definecolor{lightpurple}{rgb}{0.87,0.63,0.87}
    \definecolor{lightcyan}{rgb}{0.5,1.0,0.83}
    
    % commands and environments needed by pandoc snippets
    % extracted from the output of `pandoc -s`
    \DefineVerbatimEnvironment{Highlighting}{Verbatim}{commandchars=\\\{\}}
    % Add ',fontsize=\small' for more characters per line
    \newenvironment{Shaded}{}{}
    \newcommand{\KeywordTok}[1]{\textcolor[rgb]{0.00,0.44,0.13}{\textbf{{#1}}}}
    \newcommand{\DataTypeTok}[1]{\textcolor[rgb]{0.56,0.13,0.00}{{#1}}}
    \newcommand{\DecValTok}[1]{\textcolor[rgb]{0.25,0.63,0.44}{{#1}}}
    \newcommand{\BaseNTok}[1]{\textcolor[rgb]{0.25,0.63,0.44}{{#1}}}
    \newcommand{\FloatTok}[1]{\textcolor[rgb]{0.25,0.63,0.44}{{#1}}}
    \newcommand{\CharTok}[1]{\textcolor[rgb]{0.25,0.44,0.63}{{#1}}}
    \newcommand{\StringTok}[1]{\textcolor[rgb]{0.25,0.44,0.63}{{#1}}}
    \newcommand{\CommentTok}[1]{\textcolor[rgb]{0.38,0.63,0.69}{\textit{{#1}}}}
    \newcommand{\OtherTok}[1]{\textcolor[rgb]{0.00,0.44,0.13}{{#1}}}
    \newcommand{\AlertTok}[1]{\textcolor[rgb]{1.00,0.00,0.00}{\textbf{{#1}}}}
    \newcommand{\FunctionTok}[1]{\textcolor[rgb]{0.02,0.16,0.49}{{#1}}}
    \newcommand{\RegionMarkerTok}[1]{{#1}}
    \newcommand{\ErrorTok}[1]{\textcolor[rgb]{1.00,0.00,0.00}{\textbf{{#1}}}}
    \newcommand{\NormalTok}[1]{{#1}}
    
    % Define a nice break command that doesn't care if a line doesn't already
    % exist.
    \def\br{\hspace*{\fill} \\* }
    % Math Jax compatability definitions
    \def\gt{>}
    \def\lt{<}
    % Document parameters
    \title{Visualization with Matplotlib}
    
    
    

    % Pygments definitions
    
\makeatletter
\def\PY@reset{\let\PY@it=\relax \let\PY@bf=\relax%
    \let\PY@ul=\relax \let\PY@tc=\relax%
    \let\PY@bc=\relax \let\PY@ff=\relax}
\def\PY@tok#1{\csname PY@tok@#1\endcsname}
\def\PY@toks#1+{\ifx\relax#1\empty\else%
    \PY@tok{#1}\expandafter\PY@toks\fi}
\def\PY@do#1{\PY@bc{\PY@tc{\PY@ul{%
    \PY@it{\PY@bf{\PY@ff{#1}}}}}}}
\def\PY#1#2{\PY@reset\PY@toks#1+\relax+\PY@do{#2}}

\expandafter\def\csname PY@tok@gd\endcsname{\def\PY@tc##1{\textcolor[rgb]{0.63,0.00,0.00}{##1}}}
\expandafter\def\csname PY@tok@gu\endcsname{\let\PY@bf=\textbf\def\PY@tc##1{\textcolor[rgb]{0.50,0.00,0.50}{##1}}}
\expandafter\def\csname PY@tok@gt\endcsname{\def\PY@tc##1{\textcolor[rgb]{0.00,0.27,0.87}{##1}}}
\expandafter\def\csname PY@tok@gs\endcsname{\let\PY@bf=\textbf}
\expandafter\def\csname PY@tok@gr\endcsname{\def\PY@tc##1{\textcolor[rgb]{1.00,0.00,0.00}{##1}}}
\expandafter\def\csname PY@tok@cm\endcsname{\let\PY@it=\textit\def\PY@tc##1{\textcolor[rgb]{0.25,0.50,0.50}{##1}}}
\expandafter\def\csname PY@tok@vg\endcsname{\def\PY@tc##1{\textcolor[rgb]{0.10,0.09,0.49}{##1}}}
\expandafter\def\csname PY@tok@m\endcsname{\def\PY@tc##1{\textcolor[rgb]{0.40,0.40,0.40}{##1}}}
\expandafter\def\csname PY@tok@mh\endcsname{\def\PY@tc##1{\textcolor[rgb]{0.40,0.40,0.40}{##1}}}
\expandafter\def\csname PY@tok@go\endcsname{\def\PY@tc##1{\textcolor[rgb]{0.53,0.53,0.53}{##1}}}
\expandafter\def\csname PY@tok@ge\endcsname{\let\PY@it=\textit}
\expandafter\def\csname PY@tok@vc\endcsname{\def\PY@tc##1{\textcolor[rgb]{0.10,0.09,0.49}{##1}}}
\expandafter\def\csname PY@tok@il\endcsname{\def\PY@tc##1{\textcolor[rgb]{0.40,0.40,0.40}{##1}}}
\expandafter\def\csname PY@tok@cs\endcsname{\let\PY@it=\textit\def\PY@tc##1{\textcolor[rgb]{0.25,0.50,0.50}{##1}}}
\expandafter\def\csname PY@tok@cp\endcsname{\def\PY@tc##1{\textcolor[rgb]{0.74,0.48,0.00}{##1}}}
\expandafter\def\csname PY@tok@gi\endcsname{\def\PY@tc##1{\textcolor[rgb]{0.00,0.63,0.00}{##1}}}
\expandafter\def\csname PY@tok@gh\endcsname{\let\PY@bf=\textbf\def\PY@tc##1{\textcolor[rgb]{0.00,0.00,0.50}{##1}}}
\expandafter\def\csname PY@tok@ni\endcsname{\let\PY@bf=\textbf\def\PY@tc##1{\textcolor[rgb]{0.60,0.60,0.60}{##1}}}
\expandafter\def\csname PY@tok@nl\endcsname{\def\PY@tc##1{\textcolor[rgb]{0.63,0.63,0.00}{##1}}}
\expandafter\def\csname PY@tok@nn\endcsname{\let\PY@bf=\textbf\def\PY@tc##1{\textcolor[rgb]{0.00,0.00,1.00}{##1}}}
\expandafter\def\csname PY@tok@no\endcsname{\def\PY@tc##1{\textcolor[rgb]{0.53,0.00,0.00}{##1}}}
\expandafter\def\csname PY@tok@na\endcsname{\def\PY@tc##1{\textcolor[rgb]{0.49,0.56,0.16}{##1}}}
\expandafter\def\csname PY@tok@nb\endcsname{\def\PY@tc##1{\textcolor[rgb]{0.00,0.50,0.00}{##1}}}
\expandafter\def\csname PY@tok@nc\endcsname{\let\PY@bf=\textbf\def\PY@tc##1{\textcolor[rgb]{0.00,0.00,1.00}{##1}}}
\expandafter\def\csname PY@tok@nd\endcsname{\def\PY@tc##1{\textcolor[rgb]{0.67,0.13,1.00}{##1}}}
\expandafter\def\csname PY@tok@ne\endcsname{\let\PY@bf=\textbf\def\PY@tc##1{\textcolor[rgb]{0.82,0.25,0.23}{##1}}}
\expandafter\def\csname PY@tok@nf\endcsname{\def\PY@tc##1{\textcolor[rgb]{0.00,0.00,1.00}{##1}}}
\expandafter\def\csname PY@tok@si\endcsname{\let\PY@bf=\textbf\def\PY@tc##1{\textcolor[rgb]{0.73,0.40,0.53}{##1}}}
\expandafter\def\csname PY@tok@s2\endcsname{\def\PY@tc##1{\textcolor[rgb]{0.73,0.13,0.13}{##1}}}
\expandafter\def\csname PY@tok@vi\endcsname{\def\PY@tc##1{\textcolor[rgb]{0.10,0.09,0.49}{##1}}}
\expandafter\def\csname PY@tok@nt\endcsname{\let\PY@bf=\textbf\def\PY@tc##1{\textcolor[rgb]{0.00,0.50,0.00}{##1}}}
\expandafter\def\csname PY@tok@nv\endcsname{\def\PY@tc##1{\textcolor[rgb]{0.10,0.09,0.49}{##1}}}
\expandafter\def\csname PY@tok@s1\endcsname{\def\PY@tc##1{\textcolor[rgb]{0.73,0.13,0.13}{##1}}}
\expandafter\def\csname PY@tok@kd\endcsname{\let\PY@bf=\textbf\def\PY@tc##1{\textcolor[rgb]{0.00,0.50,0.00}{##1}}}
\expandafter\def\csname PY@tok@sh\endcsname{\def\PY@tc##1{\textcolor[rgb]{0.73,0.13,0.13}{##1}}}
\expandafter\def\csname PY@tok@sc\endcsname{\def\PY@tc##1{\textcolor[rgb]{0.73,0.13,0.13}{##1}}}
\expandafter\def\csname PY@tok@sx\endcsname{\def\PY@tc##1{\textcolor[rgb]{0.00,0.50,0.00}{##1}}}
\expandafter\def\csname PY@tok@bp\endcsname{\def\PY@tc##1{\textcolor[rgb]{0.00,0.50,0.00}{##1}}}
\expandafter\def\csname PY@tok@c1\endcsname{\let\PY@it=\textit\def\PY@tc##1{\textcolor[rgb]{0.25,0.50,0.50}{##1}}}
\expandafter\def\csname PY@tok@kc\endcsname{\let\PY@bf=\textbf\def\PY@tc##1{\textcolor[rgb]{0.00,0.50,0.00}{##1}}}
\expandafter\def\csname PY@tok@c\endcsname{\let\PY@it=\textit\def\PY@tc##1{\textcolor[rgb]{0.25,0.50,0.50}{##1}}}
\expandafter\def\csname PY@tok@mf\endcsname{\def\PY@tc##1{\textcolor[rgb]{0.40,0.40,0.40}{##1}}}
\expandafter\def\csname PY@tok@err\endcsname{\def\PY@bc##1{\setlength{\fboxsep}{0pt}\fcolorbox[rgb]{1.00,0.00,0.00}{1,1,1}{\strut ##1}}}
\expandafter\def\csname PY@tok@mb\endcsname{\def\PY@tc##1{\textcolor[rgb]{0.40,0.40,0.40}{##1}}}
\expandafter\def\csname PY@tok@ss\endcsname{\def\PY@tc##1{\textcolor[rgb]{0.10,0.09,0.49}{##1}}}
\expandafter\def\csname PY@tok@sr\endcsname{\def\PY@tc##1{\textcolor[rgb]{0.73,0.40,0.53}{##1}}}
\expandafter\def\csname PY@tok@mo\endcsname{\def\PY@tc##1{\textcolor[rgb]{0.40,0.40,0.40}{##1}}}
\expandafter\def\csname PY@tok@kn\endcsname{\let\PY@bf=\textbf\def\PY@tc##1{\textcolor[rgb]{0.00,0.50,0.00}{##1}}}
\expandafter\def\csname PY@tok@mi\endcsname{\def\PY@tc##1{\textcolor[rgb]{0.40,0.40,0.40}{##1}}}
\expandafter\def\csname PY@tok@gp\endcsname{\let\PY@bf=\textbf\def\PY@tc##1{\textcolor[rgb]{0.00,0.00,0.50}{##1}}}
\expandafter\def\csname PY@tok@o\endcsname{\def\PY@tc##1{\textcolor[rgb]{0.40,0.40,0.40}{##1}}}
\expandafter\def\csname PY@tok@kr\endcsname{\let\PY@bf=\textbf\def\PY@tc##1{\textcolor[rgb]{0.00,0.50,0.00}{##1}}}
\expandafter\def\csname PY@tok@s\endcsname{\def\PY@tc##1{\textcolor[rgb]{0.73,0.13,0.13}{##1}}}
\expandafter\def\csname PY@tok@kp\endcsname{\def\PY@tc##1{\textcolor[rgb]{0.00,0.50,0.00}{##1}}}
\expandafter\def\csname PY@tok@w\endcsname{\def\PY@tc##1{\textcolor[rgb]{0.73,0.73,0.73}{##1}}}
\expandafter\def\csname PY@tok@kt\endcsname{\def\PY@tc##1{\textcolor[rgb]{0.69,0.00,0.25}{##1}}}
\expandafter\def\csname PY@tok@ow\endcsname{\let\PY@bf=\textbf\def\PY@tc##1{\textcolor[rgb]{0.67,0.13,1.00}{##1}}}
\expandafter\def\csname PY@tok@sb\endcsname{\def\PY@tc##1{\textcolor[rgb]{0.73,0.13,0.13}{##1}}}
\expandafter\def\csname PY@tok@k\endcsname{\let\PY@bf=\textbf\def\PY@tc##1{\textcolor[rgb]{0.00,0.50,0.00}{##1}}}
\expandafter\def\csname PY@tok@se\endcsname{\let\PY@bf=\textbf\def\PY@tc##1{\textcolor[rgb]{0.73,0.40,0.13}{##1}}}
\expandafter\def\csname PY@tok@sd\endcsname{\let\PY@it=\textit\def\PY@tc##1{\textcolor[rgb]{0.73,0.13,0.13}{##1}}}

\def\PYZbs{\char`\\}
\def\PYZus{\char`\_}
\def\PYZob{\char`\{}
\def\PYZcb{\char`\}}
\def\PYZca{\char`\^}
\def\PYZam{\char`\&}
\def\PYZlt{\char`\<}
\def\PYZgt{\char`\>}
\def\PYZsh{\char`\#}
\def\PYZpc{\char`\%}
\def\PYZdl{\char`\$}
\def\PYZhy{\char`\-}
\def\PYZsq{\char`\'}
\def\PYZdq{\char`\"}
\def\PYZti{\char`\~}
% for compatibility with earlier versions
\def\PYZat{@}
\def\PYZlb{[}
\def\PYZrb{]}
\makeatother


    % Exact colors from NB
    \definecolor{incolor}{rgb}{0.0, 0.0, 0.5}
    \definecolor{outcolor}{rgb}{0.545, 0.0, 0.0}



    
    % Prevent overflowing lines due to hard-to-break entities
    \sloppy 
    % Setup hyperref package
    \hypersetup{
      breaklinks=true,  % so long urls are correctly broken across lines
      colorlinks=true,
      urlcolor=blue,
      linkcolor=darkorange,
      citecolor=darkgreen,
      }
    % Slightly bigger margins than the latex defaults
    
    \geometry{verbose,tmargin=1in,bmargin=1in,lmargin=1in,rmargin=1in}
    
    

    \begin{document}
    
    
    \maketitle
    
    

    
    \section{Overview of \texttt{matplotlib}}\label{overview-of-matplotlib}

    So far in our course, we've covered basic \texttt{Python} to more
advanced features of \texttt{Python}'s array processing and data
analysis libraries. While we have gotten into the meat of handling
numbers themselves, it would be nice to have a library of tools to
visualize these underlying data in a powerful but aesthetic way. The
solution, which has become a massive open-source project in its own
right, is \texttt{matplotlib}. From the \texttt{matplotlib}
\href{http://matplotlib.org/1.2.1/index.html}{homepage}:

\begin{quote}
\texttt{matplotlib} is a python 2D plotting library which produces
publication quality figures in a variety of hardcopy formats and
interactive environments across platforms. matplotlib can be used in
python scripts, the python and ipython shell (ala MATLAB® or
Mathematica®†), web application servers, and six graphical user
interface toolkits.
\end{quote}

\begin{quote}
\texttt{matplotlib} tries to make easy things easy and hard things
possible. You can generate plots, histograms, power spectra, bar charts,
errorcharts, scatterplots, etc, with just a few lines of code. For a
sampling, see the
\href{http://matplotlib.org/1.2.1/users/screenshots.html}{screenshots},
\href{http://matplotlib.org/1.2.1/gallery.html}{thumbnail gallery}, and
\href{http://matplotlib.org/1.2.1/examples/index.html}{examples
directory}.
\end{quote}

\begin{quote}
For the power user, you have full control of line styles, font
properties, axes properties, etc, via an object oriented interface or
via a set of functions familiar to MATLAB users.
\end{quote}

    Let's import all of the libraries we will use in this session.

    \begin{Verbatim}[commandchars=\\\{\}]
{\color{incolor}In [{\color{incolor}1}]:} \PY{k+kn}{from} \PY{n+nn}{\PYZus{}\PYZus{}future\PYZus{}\PYZus{}} \PY{k+kn}{import} \PY{n}{division}
        \PY{k+kn}{import} \PY{n+nn}{numpy} \PY{k+kn}{as} \PY{n+nn}{np}
        \PY{k+kn}{import} \PY{n+nn}{matplotlib.pyplot} \PY{k+kn}{as} \PY{n+nn}{plt}
        \PY{n}{np}\PY{o}{.}\PY{n}{random}\PY{o}{.}\PY{n}{seed}\PY{p}{(}\PY{l+m+mi}{12345}\PY{p}{)}
        \PY{n}{plt}\PY{o}{.}\PY{n}{rc}\PY{p}{(}\PY{l+s}{\PYZsq{}}\PY{l+s}{figure}\PY{l+s}{\PYZsq{}}\PY{p}{,} \PY{n}{figsize}\PY{o}{=}\PY{p}{(}\PY{l+m+mi}{5}\PY{p}{,} \PY{l+m+mi}{5}\PY{p}{)}\PY{p}{)}
        \PY{k+kn}{from} \PY{n+nn}{pandas} \PY{k+kn}{import} \PY{n}{Series}\PY{p}{,} \PY{n}{DataFrame}
        \PY{k+kn}{import} \PY{n+nn}{pandas} \PY{k+kn}{as} \PY{n+nn}{pd}
        \PY{n}{np}\PY{o}{.}\PY{n}{set\PYZus{}printoptions}\PY{p}{(}\PY{n}{precision}\PY{o}{=}\PY{l+m+mi}{4}\PY{p}{)}
\end{Verbatim}

    When you use the \texttt{IPython} notebook, you can print plots to the
output of individual cells by including the magic command:

    \begin{Verbatim}[commandchars=\\\{\}]
{\color{incolor}In [{\color{incolor}2}]:} \PY{o}{\PYZpc{}}\PY{k}{matplotlib} inline
\end{Verbatim}

    There are two ways to think about creating and displaying plots using
\texttt{matplotlib}. The first, and simpler, approach is the imperative,
``scripting'' paradigm. Modeled after the plotting functionality of
MATLAB, this gives you an easy way to generate a large quantity of
plots.

The second paradigm is the object-oriented approach, which requires a
larger amount of initial code, but with a much higher degree of
flexibility and robust functionality.

    \section{The MATLAB approach}\label{the-matlab-approach}

    The main module that we use to generate plots is the \texttt{pyplot}
submodule of \texttt{matplotlib}. According to established convention,
we import this module as follows:

    \begin{Verbatim}[commandchars=\\\{\}]
{\color{incolor}In [{\color{incolor}3}]:} \PY{k+kn}{import} \PY{n+nn}{matplotlib.pyplot} \PY{k+kn}{as} \PY{n+nn}{plt}
\end{Verbatim}

    From now on, we will use \texttt{plt} to denote methods and fields in
the \texttt{pyplot} module. Here is a simple demonstration of the MATLAB
approach to plotting.

    \begin{Verbatim}[commandchars=\\\{\}]
{\color{incolor}In [{\color{incolor}4}]:} \PY{n}{x} \PY{o}{=} \PY{n}{np}\PY{o}{.}\PY{n}{arange}\PY{p}{(}\PY{l+m+mi}{0}\PY{p}{,}\PY{l+m+mi}{10}\PY{p}{,}\PY{l+m+mf}{0.1}\PY{p}{)} \PY{c}{\PYZsh{} generates an ndarray from 0 to 9.9}
        \PY{n}{plt}\PY{o}{.}\PY{n}{plot}\PY{p}{(}\PY{n}{np}\PY{o}{.}\PY{n}{sin}\PY{p}{(}\PY{n}{x}\PY{p}{)}\PY{p}{)}
\end{Verbatim}

            \begin{Verbatim}[commandchars=\\\{\}]
{\color{outcolor}Out[{\color{outcolor}4}]:} [<matplotlib.lines.Line2D at 0x7fe665a9a4d0>]
\end{Verbatim}
        
    \begin{center}
    \adjustimage{max size={0.9\linewidth}{0.9\paperheight}}{Visualization with Matplotlib_files/Visualization with Matplotlib_11_1.png}
    \end{center}
    { \hspace*{\fill} \\}
    
    The \texttt{plot} function takes an array-like type and produces a line
plot of the array. If you give \texttt{plot} a single array, it will
implicitly assume that you mean to plot coordinate pairs $(i,arr[i])$,
where $i$ is an integer.

Instead, you can pass two arrays $x$ and $y$ (of the same size), which
produces a plot of coordinates $(x[i], y[i])$.

    \begin{Verbatim}[commandchars=\\\{\}]
{\color{incolor}In [{\color{incolor}5}]:} \PY{n}{x} \PY{o}{=} \PY{n}{np}\PY{o}{.}\PY{n}{arange}\PY{p}{(}\PY{l+m+mi}{0}\PY{p}{,}\PY{l+m+mf}{2.0}\PY{o}{*}\PY{n}{np}\PY{o}{.}\PY{n}{pi}\PY{p}{,} \PY{l+m+mf}{0.01}\PY{p}{)}
        \PY{n}{plt}\PY{o}{.}\PY{n}{plot}\PY{p}{(}\PY{n}{np}\PY{o}{.}\PY{n}{cos}\PY{p}{(}\PY{n}{x}\PY{p}{)}\PY{p}{,}\PY{n}{np}\PY{o}{.}\PY{n}{sin}\PY{p}{(}\PY{n}{x}\PY{p}{)}\PY{p}{)}
        \PY{n}{plt}\PY{o}{.}\PY{n}{xlim}\PY{p}{(}\PY{p}{[}\PY{o}{\PYZhy{}}\PY{l+m+mf}{1.1}\PY{p}{,}\PY{l+m+mf}{1.1}\PY{p}{]}\PY{p}{)}
        \PY{n}{plt}\PY{o}{.}\PY{n}{ylim}\PY{p}{(}\PY{p}{[}\PY{o}{\PYZhy{}}\PY{l+m+mf}{1.1}\PY{p}{,}\PY{l+m+mf}{1.1}\PY{p}{]}\PY{p}{)}
\end{Verbatim}

            \begin{Verbatim}[commandchars=\\\{\}]
{\color{outcolor}Out[{\color{outcolor}5}]:} (-1.1, 1.1)
\end{Verbatim}
        
    \begin{center}
    \adjustimage{max size={0.9\linewidth}{0.9\paperheight}}{Visualization with Matplotlib_files/Visualization with Matplotlib_13_1.png}
    \end{center}
    { \hspace*{\fill} \\}
    
    It is very simple to customize the style of the plots.

    \begin{Verbatim}[commandchars=\\\{\}]
{\color{incolor}In [{\color{incolor}6}]:} \PY{n}{plt}\PY{o}{.}\PY{n}{plot}\PY{o}{?}
\end{Verbatim}

    \subparagraph{Line color and marker
arguments}\label{line-color-and-marker-arguments}

By passing in an optional character argument, you can specify the color
of the line being plotted.

\begin{longtable}[c]{@{}ll@{}}
\toprule\addlinespace
Character & Color
\\\addlinespace
\midrule\endhead
\texttt{'b'} & blue
\\\addlinespace
\texttt{'g'} & green
\\\addlinespace
\texttt{'r'} & red
\\\addlinespace
\texttt{'c'} & cyan
\\\addlinespace
\texttt{'m'} & magenta
\\\addlinespace
\texttt{'y'} & yellow
\\\addlinespace
\texttt{'k'} & black
\\\addlinespace
\texttt{'w'} & white
\\\addlinespace
\bottomrule
\end{longtable}

Alternatively, you can specify a custom (e.g.~hexadecimal) color by
passing a \texttt{color=\#123456} argument.

For customizing the line marker shapes, you can specify from a number of
built-in arguments.

\begin{longtable}[c]{@{}llll@{}}
\toprule\addlinespace
Character & Description & Character & Description
\\\addlinespace
\midrule\endhead
\texttt{'-'} & solid line style & \texttt{'3'} & tri\_left marker
\\\addlinespace
\texttt{'-\/-'} & dashed line style & \texttt{'4'} & tri\_right marker
\\\addlinespace
\texttt{'-.'} & dash-dot line style & \texttt{'s'} & square marker
\\\addlinespace
\texttt{':'} & dotted line style & \texttt{'p'} & pentagon marker
\\\addlinespace
\texttt{'.'} & point marker & \texttt{'*'} & star marker
\\\addlinespace
\texttt{','} & pixel marker & \texttt{'h'} & hexagon1 marker
\\\addlinespace
\texttt{'o'} & circle marker & \texttt{'H'} & hexagon2 marker
\\\addlinespace
\texttt{'v'} & triangle\_down marker & \texttt{'+'} & plus marker
\\\addlinespace
\texttt{'\^{}'} & triangle\_up marker & \texttt{'x'} & x marker
\\\addlinespace
\texttt{'\textless{}'} & triangle\_left marker & \texttt{'D'} & diamond
marker
\\\addlinespace
\texttt{'\textgreater{}'} & triangle\_right marker & \texttt{'d'} &
thin\_diamond marker
\\\addlinespace
\texttt{'1'} & tri\_down marker & \texttt{'\_'} & hline marker
\\\addlinespace
\texttt{'2'} & tri\_up marker
\\\addlinespace
\bottomrule
\end{longtable}

    There are even more keyword arguments, but we won't go into the details
here. Here is a simple example:

    \begin{Verbatim}[commandchars=\\\{\}]
{\color{incolor}In [{\color{incolor}7}]:} \PY{n}{x} \PY{o}{=} \PY{n}{np}\PY{o}{.}\PY{n}{arange}\PY{p}{(}\PY{l+m+mi}{0}\PY{p}{,}\PY{l+m+mi}{10}\PY{p}{,}\PY{l+m+mf}{0.2}\PY{p}{)}
        \PY{n}{plt}\PY{o}{.}\PY{n}{plot}\PY{p}{(}\PY{n}{np}\PY{o}{.}\PY{n}{sin}\PY{p}{(}\PY{n}{x}\PY{p}{)}\PY{p}{,} \PY{l+s}{\PYZsq{}}\PY{l+s}{1}\PY{l+s}{\PYZsq{}}\PY{p}{)}
        \PY{n}{plt}\PY{o}{.}\PY{n}{plot}\PY{p}{(}\PY{n}{np}\PY{o}{.}\PY{n}{cos}\PY{p}{(}\PY{n}{x}\PY{p}{)}\PY{p}{,} \PY{l+s}{\PYZsq{}}\PY{l+s}{:}\PY{l+s}{\PYZsq{}}\PY{p}{)}
        \PY{n}{plt}\PY{o}{.}\PY{n}{plot}\PY{p}{(}\PY{n}{np}\PY{o}{.}\PY{n}{sqrt}\PY{p}{(}\PY{n}{x}\PY{p}{)}\PY{p}{,} \PY{l+s}{\PYZsq{}}\PY{l+s}{m}\PY{l+s}{\PYZsq{}}\PY{p}{,} \PY{n}{drawstyle}\PY{o}{=}\PY{l+s}{\PYZsq{}}\PY{l+s}{steps\PYZhy{}post}\PY{l+s}{\PYZsq{}}\PY{p}{)}
\end{Verbatim}

            \begin{Verbatim}[commandchars=\\\{\}]
{\color{outcolor}Out[{\color{outcolor}7}]:} [<matplotlib.lines.Line2D at 0x7fe665920e10>]
\end{Verbatim}
        
    \begin{center}
    \adjustimage{max size={0.9\linewidth}{0.9\paperheight}}{Visualization with Matplotlib_files/Visualization with Matplotlib_18_1.png}
    \end{center}
    { \hspace*{\fill} \\}
    
    Notice that calling \texttt{plot} several times in one cell allows you
to plot several graphs on one figure.

Now, let's customize the title, labels, legend, and ticks of the plot.
In the MATLAB paradigm, we can use simple figure methods like
\texttt{title}, \texttt{xlabel}, and \texttt{ylabel}, as well as call
the \texttt{legend} method. To specify a legend string, include the
optional argument \texttt{label} in the plot method.

    \begin{Verbatim}[commandchars=\\\{\}]
{\color{incolor}In [{\color{incolor}8}]:} \PY{n}{x} \PY{o}{=} \PY{n}{np}\PY{o}{.}\PY{n}{random}\PY{o}{.}\PY{n}{randn}\PY{p}{(}\PY{l+m+mi}{1000}\PY{p}{)}
        \PY{n}{y} \PY{o}{=} \PY{n}{np}\PY{o}{.}\PY{n}{random}\PY{o}{.}\PY{n}{randn}\PY{p}{(}\PY{l+m+mi}{1000}\PY{p}{)}
        
        \PY{n}{plt}\PY{o}{.}\PY{n}{plot}\PY{p}{(}\PY{n}{x}\PY{o}{.}\PY{n}{cumsum}\PY{p}{(}\PY{p}{)}\PY{p}{,} \PY{l+s}{\PYZsq{}}\PY{l+s}{k}\PY{l+s}{\PYZsq{}}\PY{p}{,} \PY{n}{label}\PY{o}{=}\PY{l+s}{\PYZsq{}}\PY{l+s}{A random walk (\PYZdl{}X(n)\PYZdl{})}\PY{l+s}{\PYZsq{}}\PY{p}{)}
        \PY{n}{plt}\PY{o}{.}\PY{n}{plot}\PY{p}{(}\PY{n}{y}\PY{o}{.}\PY{n}{cumsum}\PY{p}{(}\PY{p}{)}\PY{p}{,} \PY{l+s}{\PYZsq{}}\PY{l+s}{r\PYZhy{}\PYZhy{}}\PY{l+s}{\PYZsq{}}\PY{p}{,} \PY{n}{label}\PY{o}{=}\PY{l+s}{\PYZsq{}}\PY{l+s}{Another walk (\PYZdl{}Y(n)\PYZdl{})}\PY{l+s}{\PYZsq{}}\PY{p}{)}
        
        \PY{c}{\PYZsh{} Title and labels}
        \PY{n}{plt}\PY{o}{.}\PY{n}{title}\PY{p}{(}\PY{l+s}{\PYZsq{}}\PY{l+s}{This is the figure title}\PY{l+s}{\PYZsq{}}\PY{p}{)}
        \PY{n}{plt}\PY{o}{.}\PY{n}{xlabel}\PY{p}{(}\PY{l+s}{\PYZsq{}}\PY{l+s}{The horizontal axis (\PYZdl{}n\PYZdl{})}\PY{l+s}{\PYZsq{}}\PY{p}{)}
        \PY{n}{plt}\PY{o}{.}\PY{n}{ylabel}\PY{p}{(}\PY{l+s}{\PYZsq{}}\PY{l+s}{The vertical axis}\PY{l+s}{\PYZsq{}}\PY{p}{)}
        
        \PY{c}{\PYZsh{} Tick values}
        \PY{n}{plt}\PY{o}{.}\PY{n}{xticks}\PY{p}{(}\PY{n+nb}{range}\PY{p}{(}\PY{l+m+mi}{0}\PY{p}{,}\PY{l+m+mi}{1001}\PY{p}{,}\PY{l+m+mi}{250}\PY{p}{)}\PY{p}{,} \PY{n}{rotation}\PY{o}{=}\PY{l+m+mi}{30}\PY{p}{)}
        \PY{n}{plt}\PY{o}{.}\PY{n}{yticks}\PY{p}{(}\PY{n+nb}{range}\PY{p}{(}\PY{o}{\PYZhy{}}\PY{l+m+mi}{50}\PY{p}{,}\PY{l+m+mi}{51}\PY{p}{,} \PY{l+m+mi}{20}\PY{p}{)}\PY{p}{)}
        
        \PY{n}{plt}\PY{o}{.}\PY{n}{legend}\PY{p}{(}\PY{n}{loc}\PY{o}{=}\PY{l+s}{\PYZsq{}}\PY{l+s}{best}\PY{l+s}{\PYZsq{}}\PY{p}{)} \PY{c}{\PYZsh{} especially useful for random data}
\end{Verbatim}

            \begin{Verbatim}[commandchars=\\\{\}]
{\color{outcolor}Out[{\color{outcolor}8}]:} <matplotlib.legend.Legend at 0x7fe6658deb90>
\end{Verbatim}
        
    \begin{center}
    \adjustimage{max size={0.9\linewidth}{0.9\paperheight}}{Visualization with Matplotlib_files/Visualization with Matplotlib_20_1.png}
    \end{center}
    { \hspace*{\fill} \\}
    
    If you notice carefully, \texttt{matplotlib} can render \LaTeX in
title, axis, and legend strings. Simply include the \LaTeX dollar sign
and \texttt{matplotlib} will do the rest for you. Now, there are tricky
grey-areas with this functionality. For example, if you want to typeset
the Greek letter $\tau$ on your plots, \texttt{matplotlib} will not
properly interpret the string. (Why is this? See if you can figure out
why.) To force \texttt{matplotlib} to interpret strings literally, you
can instead write \texttt{r'\$\textbackslash{}tau\$'}, which will tell
\texttt{matplotlib} to ignore the formatting ambiguity.

There are tons of ways to customize your plots further, but we'll leave
this to your exploration of the \texttt{matplotlib} documentation.

    \section{The object-oriented
approach}\label{the-object-oriented-approach}

    Whereas in the MATLAB approach, all plotting activity was centered
around the \texttt{matplotlib} \emph{figure}, the object-oriented
approach shifts this attention to the \emph{axis}.

When you begin plotting, you first initialize a figure and then add axes
to the figure. Each axis now functions as its own plotting environment,
which allows you to specify all of the previous functions nearly
identically as before.

Why go to all of this work to specify the axis objects? The immediate
advantage is that you can now easily construct \textbf{several} axes on
one figure, which is an ability I have personally found incredibly
useful.

Here is a simple way to get started:

    \begin{Verbatim}[commandchars=\\\{\}]
{\color{incolor}In [{\color{incolor}9}]:} \PY{n}{fig} \PY{o}{=} \PY{n}{plt}\PY{o}{.}\PY{n}{figure}\PY{p}{(}\PY{n}{figsize}\PY{o}{=}\PY{p}{(}\PY{l+m+mi}{9}\PY{p}{,}\PY{l+m+mi}{3}\PY{p}{)}\PY{p}{)} \PY{c}{\PYZsh{} instantiate a new figure object}
        
        \PY{c}{\PYZsh{} Add three axes aligned horizontally}
        \PY{n}{ax1} \PY{o}{=} \PY{n}{fig}\PY{o}{.}\PY{n}{add\PYZus{}subplot}\PY{p}{(}\PY{l+m+mi}{1}\PY{p}{,}\PY{l+m+mi}{3}\PY{p}{,}\PY{l+m+mi}{1}\PY{p}{)}
        \PY{n}{ax2} \PY{o}{=} \PY{n}{fig}\PY{o}{.}\PY{n}{add\PYZus{}subplot}\PY{p}{(}\PY{l+m+mi}{1}\PY{p}{,}\PY{l+m+mi}{3}\PY{p}{,}\PY{l+m+mi}{2}\PY{p}{)}
        \PY{n}{ax3} \PY{o}{=} \PY{n}{fig}\PY{o}{.}\PY{n}{add\PYZus{}subplot}\PY{p}{(}\PY{l+m+mi}{1}\PY{p}{,}\PY{l+m+mi}{3}\PY{p}{,}\PY{l+m+mi}{3}\PY{p}{)}
        
        \PY{n}{x} \PY{o}{=} \PY{n}{np}\PY{o}{.}\PY{n}{arange}\PY{p}{(}\PY{l+m+mf}{0.0}\PY{p}{,} \PY{l+m+mf}{10.0}\PY{p}{,} \PY{l+m+mf}{0.1}\PY{p}{)}
        
        \PY{c}{\PYZsh{} Simple plot}
        \PY{n}{ax1}\PY{o}{.}\PY{n}{plot}\PY{p}{(}\PY{n}{x}\PY{p}{,} \PY{n}{np}\PY{o}{.}\PY{n}{tan}\PY{p}{(}\PY{n}{x}\PY{p}{)}\PY{p}{)}
        
        \PY{c}{\PYZsh{} Histogram plot}
        \PY{n}{ax2}\PY{o}{.}\PY{n}{hist}\PY{p}{(}\PY{n}{np}\PY{o}{.}\PY{n}{random}\PY{o}{.}\PY{n}{randn}\PY{p}{(}\PY{l+m+mi}{1000}\PY{p}{)}\PY{p}{,} \PY{n}{bins}\PY{o}{=}\PY{l+m+mi}{100}\PY{p}{)}
        
        \PY{c}{\PYZsh{} Scatter plot, parametric}
        \PY{n}{ax3}\PY{o}{.}\PY{n}{scatter}\PY{p}{(}\PY{n}{np}\PY{o}{.}\PY{n}{sin}\PY{p}{(}\PY{n}{x}\PY{p}{)}\PY{p}{,} \PY{n}{x}\PY{o}{*}\PY{n}{x}\PY{p}{)}
        
        \PY{n}{ax1}\PY{o}{.}\PY{n}{set\PYZus{}xlabel}\PY{p}{(}\PY{l+s}{\PYZdq{}}\PY{l+s}{Axis 1}\PY{l+s}{\PYZdq{}}\PY{p}{)}
        \PY{n}{ax2}\PY{o}{.}\PY{n}{set\PYZus{}xlabel}\PY{p}{(}\PY{l+s}{\PYZdq{}}\PY{l+s}{Axis 2}\PY{l+s}{\PYZdq{}}\PY{p}{)}
        \PY{n}{ax3}\PY{o}{.}\PY{n}{set\PYZus{}xlabel}\PY{p}{(}\PY{l+s}{\PYZdq{}}\PY{l+s}{Axis 3}\PY{l+s}{\PYZdq{}}\PY{p}{)}
        
        \PY{n}{ax1}\PY{o}{.}\PY{n}{set\PYZus{}title}\PY{p}{(}\PY{l+s}{r\PYZdq{}}\PY{l+s}{\PYZdl{}}\PY{l+s}{\PYZbs{}}\PY{l+s}{tan(x)\PYZdl{}}\PY{l+s}{\PYZdq{}}\PY{p}{,} \PY{n}{fontsize}\PY{o}{=}\PY{l+m+mi}{16}\PY{p}{)}
        \PY{n}{ax2}\PY{o}{.}\PY{n}{set\PYZus{}title}\PY{p}{(}\PY{l+s}{\PYZdq{}}\PY{l+s}{Random sampling}\PY{l+s}{\PYZdq{}}\PY{p}{)}
        \PY{n}{ax3}\PY{o}{.}\PY{n}{set\PYZus{}title}\PY{p}{(}\PY{l+s}{r\PYZdq{}}\PY{l+s}{\PYZdl{}(}\PY{l+s}{\PYZbs{}}\PY{l+s}{sin(x),x\PYZca{}2)\PYZdl{}}\PY{l+s}{\PYZdq{}}\PY{p}{,} \PY{n}{fontsize}\PY{o}{=}\PY{l+m+mi}{16}\PY{p}{)}
\end{Verbatim}

            \begin{Verbatim}[commandchars=\\\{\}]
{\color{outcolor}Out[{\color{outcolor}9}]:} <matplotlib.text.Text at 0x7fe665722d50>
\end{Verbatim}
        
    \begin{center}
    \adjustimage{max size={0.9\linewidth}{0.9\paperheight}}{Visualization with Matplotlib_files/Visualization with Matplotlib_24_1.png}
    \end{center}
    { \hspace*{\fill} \\}
    
    Here, we construct subplots by using the \texttt{add\_subplot} method.
The first two arguments of \texttt{add\_subplot} indicate the number of
rows and columns, respectively. Notice that for axes objects, we use
\texttt{set\_xlabel} and \texttt{set\_title} instead of \texttt{xlabel}
and \texttt{title}, but otherwise the functions work as one might expect
compared to the MATLAB approach. This is generally the case for axes
methods.

Beyond \texttt{plot}, \texttt{matplotlib} provides a host of other
plotting methods, depending on exactly what your visualization needs. On
display here is the \texttt{hist} and \texttt{scatter} methods.
\texttt{hist} takes an array and plots the distribution of values of the
array in a bar chart. \texttt{scatter} is similar to \texttt{plot}, but
it requires exactly two arrays to generate coordinate pairs.

One way to deal with a large number of axes is to think about them as
iterable objects. This can \emph{dramatically} reduce the amount of code
requisite to do sophisticated plots. For example:

    \begin{Verbatim}[commandchars=\\\{\}]
{\color{incolor}In [{\color{incolor}10}]:} \PY{n}{fig}\PY{p}{,} \PY{n}{axes} \PY{o}{=} \PY{n}{plt}\PY{o}{.}\PY{n}{subplots}\PY{p}{(}\PY{l+m+mi}{3}\PY{p}{,}\PY{l+m+mi}{3}\PY{p}{,} \PY{n}{figsize}\PY{o}{=}\PY{p}{(}\PY{l+m+mi}{10}\PY{p}{,}\PY{l+m+mi}{10}\PY{p}{)}\PY{p}{,} \PY{n}{sharex}\PY{o}{=}\PY{n+nb+bp}{True}\PY{p}{,} \PY{n}{sharey}\PY{o}{=}\PY{n+nb+bp}{True}\PY{p}{)}
         
         \PY{k}{for} \PY{n}{i} \PY{o+ow}{in} \PY{n+nb}{range}\PY{p}{(}\PY{l+m+mi}{3}\PY{p}{)}\PY{p}{:}
             \PY{k}{for} \PY{n}{j} \PY{o+ow}{in} \PY{n+nb}{range}\PY{p}{(}\PY{l+m+mi}{3}\PY{p}{)}\PY{p}{:}
                 \PY{n}{x} \PY{o}{=} \PY{n}{np}\PY{o}{.}\PY{n}{random}\PY{o}{.}\PY{n}{randn}\PY{p}{(}\PY{l+m+mi}{100}\PY{p}{)}
                 \PY{n}{axes}\PY{p}{[}\PY{n}{i}\PY{p}{,} \PY{n}{j}\PY{p}{]}\PY{o}{.}\PY{n}{hist}\PY{p}{(}\PY{n}{x}\PY{p}{,} \PY{n}{color}\PY{o}{=}\PY{l+s}{\PYZsq{}}\PY{l+s}{g}\PY{l+s}{\PYZsq{}}\PY{p}{,} \PY{n}{alpha}\PY{o}{=}\PY{l+m+mf}{0.5}\PY{p}{)}
                 \PY{n}{axes}\PY{p}{[}\PY{n}{i}\PY{p}{,} \PY{n}{j}\PY{p}{]}\PY{o}{.}\PY{n}{set\PYZus{}title}\PY{p}{(}\PY{l+s}{\PYZdq{}}\PY{l+s}{Realization }\PY{l+s+si}{\PYZpc{}i}\PY{l+s}{,}\PY{l+s+si}{\PYZpc{}i}\PY{l+s}{\PYZdq{}} \PY{o}{\PYZpc{}} \PY{p}{(}\PY{n}{i}\PY{o}{+}\PY{l+m+mi}{1}\PY{p}{,}\PY{n}{j}\PY{o}{+}\PY{l+m+mi}{1}\PY{p}{)}\PY{p}{)}
         
         \PY{n}{plt}\PY{o}{.}\PY{n}{subplots\PYZus{}adjust}\PY{p}{(}\PY{n}{wspace}\PY{o}{=}\PY{l+m+mf}{0.2}\PY{p}{,}\PY{n}{hspace}\PY{o}{=}\PY{l+m+mf}{0.2}\PY{p}{)}
\end{Verbatim}

    \begin{center}
    \adjustimage{max size={0.9\linewidth}{0.9\paperheight}}{Visualization with Matplotlib_files/Visualization with Matplotlib_26_0.png}
    \end{center}
    { \hspace*{\fill} \\}
    
    Of note, you can specify whether two (or in the above case,
\emph{every}) subplots share an x or y-axis. This can be a nice
technique to reduce the clutter around a plot. Another function that is
useful for figure formatting is \texttt{subplots\_adjust}, which allows
you to specify the spacing between plots and the margins from the
borders of the aggregate figure.

    \section{Plotting functions}\label{plotting-functions}

    Here we will use the MATLAB approach just for brevity of code. We have
already seen \texttt{plot} fairly extensively, so now we will explore
other \texttt{matplotlib} plotting functions that you might want to
explore.

    \subsection{\texttt{bar} and \texttt{barh}}\label{bar-and-barh}

    The \texttt{bar} and \texttt{barh} methods allow you to generate bar
plots, with the distinction that \texttt{bar} orients the rectangles of
the plot along the vertical axis while \texttt{barh} orients along the
horizontal axis. Outside from orientation, both work identically (we
will from now on assume \texttt{bar}).

\texttt{bar} takes an array denoting the x-coordinates of the left sides
of the bars and an array denoting the heights of the bars. Optionally,
you can add a scalar value (or array, for each bar) to denote the width
of every bar. As with every plotting function, you can then specify
color, transparency (alpha), and the legend label of the bars. For bar
plots, you can add additional options, \texttt{xerr} and \texttt{yerr},
to specify the error bars in the x and y directions for the plot.

As a first example, we simply create a bar chart denoting increasing
values:

    \begin{Verbatim}[commandchars=\\\{\}]
{\color{incolor}In [{\color{incolor}11}]:} \PY{n}{vals} \PY{o}{=} \PY{n}{np}\PY{o}{.}\PY{n}{arange}\PY{p}{(}\PY{l+m+mi}{0}\PY{p}{,}\PY{l+m+mi}{10}\PY{p}{,}\PY{l+m+mi}{1}\PY{p}{)}
         
         \PY{n}{plt}\PY{o}{.}\PY{n}{bar}\PY{p}{(}\PY{n}{vals}\PY{p}{,} \PY{n}{vals} \PY{o}{+} \PY{l+m+mi}{1}\PY{p}{,} \PY{l+m+mi}{1}\PY{p}{)}
\end{Verbatim}

            \begin{Verbatim}[commandchars=\\\{\}]
{\color{outcolor}Out[{\color{outcolor}11}]:} <Container object of 10 artists>
\end{Verbatim}
        
    \begin{center}
    \adjustimage{max size={0.9\linewidth}{0.9\paperheight}}{Visualization with Matplotlib_files/Visualization with Matplotlib_32_1.png}
    \end{center}
    { \hspace*{\fill} \\}
    
    Of course, we can make this much more sophisticated. Here's a fun
example that demonstrates some of the main features of \texttt{bar}.

    \begin{Verbatim}[commandchars=\\\{\}]
{\color{incolor}In [{\color{incolor}12}]:} \PY{n}{width} \PY{o}{=} \PY{l+m+mf}{0.2}
         \PY{n}{rows} \PY{o}{=} \PY{n}{np}\PY{o}{.}\PY{n}{arange}\PY{p}{(}\PY{l+m+mi}{0}\PY{p}{,}\PY{l+m+mi}{10}\PY{p}{,} \PY{n}{width}\PY{p}{)}
         
         \PY{n}{plt}\PY{o}{.}\PY{n}{figure}\PY{p}{(}\PY{n}{figsize}\PY{o}{=}\PY{p}{(}\PY{l+m+mi}{10}\PY{p}{,}\PY{l+m+mi}{5}\PY{p}{)}\PY{p}{)}
         
         \PY{n}{data1} \PY{o}{=} \PY{l+m+mf}{1.0} \PY{o}{\PYZhy{}} \PY{l+m+mf}{2.0}\PY{o}{/}\PY{p}{(}\PY{n}{rows} \PY{o}{+}\PY{l+m+mf}{1.0}\PY{p}{)} \PY{o}{*} \PY{n}{np}\PY{o}{.}\PY{n}{sin}\PY{p}{(}\PY{n}{rows}\PY{p}{)}
         \PY{n}{data2} \PY{o}{=} \PY{l+m+mf}{2.}\PY{o}{/}\PY{p}{(}\PY{n}{rows} \PY{o}{+} \PY{l+m+mf}{1.}\PY{p}{)}\PY{o}{*}\PY{o}{*}\PY{l+m+mi}{2} \PY{o}{*} \PY{n}{np}\PY{o}{.}\PY{n}{abs}\PY{p}{(}\PY{n}{np}\PY{o}{.}\PY{n}{cos}\PY{p}{(}\PY{n}{rows}\PY{p}{)}\PY{p}{)}
         
         \PY{n}{plt}\PY{o}{.}\PY{n}{bar}\PY{p}{(}\PY{n}{rows}\PY{p}{,} \PY{n}{data1}\PY{p}{,} \PY{n}{width}\PY{p}{,} \PY{n}{color}\PY{o}{=}\PY{l+s}{\PYZdq{}}\PY{l+s}{y}\PY{l+s}{\PYZdq{}}\PY{p}{,} \PY{n}{alpha}\PY{o}{=}\PY{l+m+mf}{0.7}\PY{p}{,} 
                 \PY{n}{label}\PY{o}{=}\PY{l+s}{\PYZdq{}}\PY{l+s}{Perceived Knowledge of C}\PY{l+s}{\PYZdq{}}\PY{p}{)}
         \PY{n}{plt}\PY{o}{.}\PY{n}{bar}\PY{p}{(}\PY{n}{rows}\PY{p}{,} \PY{n}{data2}\PY{p}{,} \PY{n}{width}\PY{p}{,} \PY{n}{color}\PY{o}{=}\PY{l+s}{\PYZdq{}}\PY{l+s}{b}\PY{l+s}{\PYZdq{}}\PY{p}{,} \PY{n}{alpha}\PY{o}{=}\PY{l+m+mf}{0.7}\PY{p}{,} \PY{n}{label}\PY{o}{=}\PY{l+s}{\PYZdq{}}\PY{l+s}{Happiness}\PY{l+s}{\PYZdq{}}\PY{p}{)}
         \PY{n}{plt}\PY{o}{.}\PY{n}{legend}\PY{p}{(}\PY{n}{loc}\PY{o}{=}\PY{l+s}{\PYZdq{}}\PY{l+s}{best}\PY{l+s}{\PYZdq{}}\PY{p}{)}
         
         \PY{n}{plt}\PY{o}{.}\PY{n}{xticks}\PY{p}{(}\PY{p}{[}\PY{l+m+mf}{0.2}\PY{p}{,}\PY{l+m+mf}{2.0}\PY{p}{,}\PY{l+m+mf}{4.0}\PY{p}{,}\PY{l+m+mf}{7.0}\PY{p}{,}\PY{l+m+mf}{9.0}\PY{p}{]}\PY{p}{,}
                    \PY{p}{(}\PY{l+s}{\PYZdq{}}\PY{l+s}{None}\PY{l+s}{\PYZdq{}}\PY{p}{,} \PY{l+s}{\PYZdq{}}\PY{l+s}{Pointer syntax}\PY{l+s}{\PYZdq{}}\PY{p}{,} \PY{l+s}{\PYZdq{}}\PY{l+s}{Pointer arithmetic}\PY{l+s}{\PYZdq{}}\PY{p}{,} 
                     \PY{l+s}{\PYZdq{}}\PY{l+s}{Passing pointers}\PY{l+s}{\PYZdq{}}\PY{p}{,} 
                     \PY{l+s}{\PYZdq{}}\PY{l+s}{Function pointers}\PY{l+s}{\PYZdq{}}\PY{p}{)}\PY{p}{,} \PY{n}{rotation}\PY{o}{=}\PY{l+m+mi}{0}\PY{p}{)}
         
         \PY{n}{plt}\PY{o}{.}\PY{n}{xlabel}\PY{p}{(}\PY{l+s}{\PYZdq{}}\PY{l+s}{Material covered in CSC 161}\PY{l+s}{\PYZdq{}}\PY{p}{,} \PY{n}{fontsize}\PY{o}{=}\PY{l+m+mi}{16}\PY{p}{)}
         \PY{n}{plt}\PY{o}{.}\PY{n}{yticks}\PY{p}{(}\PY{p}{[}\PY{p}{]}\PY{p}{)}
         \PY{n}{plt}\PY{o}{.}\PY{n}{title}\PY{p}{(}\PY{l+s}{\PYZdq{}}\PY{l+s}{C is a frustrating language}\PY{l+s}{\PYZdq{}}\PY{p}{,} \PY{n}{fontsize}\PY{o}{=}\PY{l+m+mi}{20}\PY{p}{)}
\end{Verbatim}

            \begin{Verbatim}[commandchars=\\\{\}]
{\color{outcolor}Out[{\color{outcolor}12}]:} <matplotlib.text.Text at 0x7fe664d51390>
\end{Verbatim}
        
    \begin{center}
    \adjustimage{max size={0.9\linewidth}{0.9\paperheight}}{Visualization with Matplotlib_files/Visualization with Matplotlib_34_1.png}
    \end{center}
    { \hspace*{\fill} \\}
    
    \begin{Verbatim}[commandchars=\\\{\}]
{\color{incolor}In [{\color{incolor}13}]:} \PY{n}{x} \PY{o}{=} \PY{n}{np}\PY{o}{.}\PY{n}{arange}\PY{p}{(}\PY{l+m+mi}{0}\PY{p}{,}\PY{l+m+mi}{10}\PY{p}{,}\PY{l+m+mi}{1}\PY{p}{)}
         
         \PY{n}{plt}\PY{o}{.}\PY{n}{barh}\PY{p}{(}\PY{n}{x}\PY{p}{,} \PY{n}{np}\PY{o}{.}\PY{n}{cos}\PY{p}{(}\PY{n}{x}\PY{p}{)}\PY{p}{,} \PY{l+m+mf}{0.85}\PY{p}{,} \PY{n}{align}\PY{o}{=}\PY{l+s}{\PYZsq{}}\PY{l+s}{center}\PY{l+s}{\PYZsq{}}\PY{p}{,} 
                  \PY{n}{xerr}\PY{o}{=} \PY{l+m+mf}{0.05} \PY{o}{*} \PY{n}{np}\PY{o}{.}\PY{n}{random}\PY{o}{.}\PY{n}{rand}\PY{p}{(}\PY{n}{np}\PY{o}{.}\PY{n}{size}\PY{p}{(}\PY{n}{x}\PY{p}{)}\PY{p}{)}\PY{p}{,} \PY{n}{alpha}\PY{o}{=}\PY{l+m+mf}{0.4}\PY{p}{)} 
         
         \PY{n}{plt}\PY{o}{.}\PY{n}{yticks}\PY{p}{(}\PY{n}{x}\PY{p}{[}\PY{p}{:}\PY{p}{]}\PY{p}{,} \PY{p}{(}\PY{l+s}{\PYZdq{}}\PY{l+s}{This}\PY{l+s}{\PYZdq{}}\PY{p}{,} \PY{l+s}{\PYZdq{}}\PY{l+s}{Is}\PY{l+s}{\PYZdq{}}\PY{p}{,} \PY{l+s}{\PYZdq{}}\PY{l+s}{A}\PY{l+s}{\PYZdq{}}\PY{p}{,} \PY{l+s}{\PYZdq{}}\PY{l+s}{Bar}\PY{l+s}{\PYZdq{}}\PY{p}{,} \PY{l+s}{\PYZdq{}}\PY{l+s}{Plot}\PY{l+s}{\PYZdq{}}\PY{p}{,} \PY{l+s}{\PYZdq{}}\PY{l+s}{With}\PY{l+s}{\PYZdq{}}\PY{p}{,} \PY{l+s}{\PYZdq{}}\PY{l+s}{Custom}\PY{l+s}{\PYZdq{}}\PY{p}{,} 
                        \PY{l+s}{\PYZdq{}}\PY{l+s}{Tags}\PY{l+s}{\PYZdq{}}\PY{p}{,} \PY{l+s}{\PYZdq{}}\PY{l+s}{For}\PY{l+s}{\PYZdq{}}\PY{p}{,} \PY{l+s}{\PYZdq{}}\PY{l+s}{You}\PY{l+s}{\PYZdq{}}\PY{p}{,} \PY{l+s}{\PYZdq{}}\PY{l+s}{To}\PY{l+s}{\PYZdq{}}\PY{p}{,} \PY{l+s}{\PYZdq{}}\PY{l+s}{See}\PY{l+s}{\PYZdq{}}\PY{p}{)}\PY{p}{)}
\end{Verbatim}

            \begin{Verbatim}[commandchars=\\\{\}]
{\color{outcolor}Out[{\color{outcolor}13}]:} ([<matplotlib.axis.YTick at 0x7fe66482cb50>,
           <matplotlib.axis.YTick at 0x7fe664774550>,
           <matplotlib.axis.YTick at 0x7fe6646fca50>,
           <matplotlib.axis.YTick at 0x7fe6646467d0>,
           <matplotlib.axis.YTick at 0x7fe664646f10>,
           <matplotlib.axis.YTick at 0x7fe664650690>,
           <matplotlib.axis.YTick at 0x7fe664650dd0>,
           <matplotlib.axis.YTick at 0x7fe664659550>,
           <matplotlib.axis.YTick at 0x7fe664659c90>,
           <matplotlib.axis.YTick at 0x7fe664665410>],
          <a list of 10 Text yticklabel objects>)
\end{Verbatim}
        
    \begin{center}
    \adjustimage{max size={0.9\linewidth}{0.9\paperheight}}{Visualization with Matplotlib_files/Visualization with Matplotlib_35_1.png}
    \end{center}
    { \hspace*{\fill} \\}
    
    \subsection{Plotting functions in
pandas}\label{plotting-functions-in-pandas}

    It's important to know how to build plots manually from data stored in
\texttt{NumPy}. However, we can also use \texttt{Pandas} to produce
high-quality \texttt{matplotlib} plots from existing \texttt{Series} and
\texttt{DataFrame} objects with considerable ease and flexibility.

This section will walk through a variety of options you have at your
disposal when building visualizations with \texttt{Pandas}, but is by no
means exhaustive. For more information, check out the
\href{http://pandas.pydata.org/pandas-docs/stable/}{Pandas
Documentation}.

    \subsubsection{Line plots}\label{line-plots}

    Given a \texttt{Series} object, one natural approach to plotting is with
line plots. This is the default behavior of the method
\texttt{Series.plot}, displayed below.

    \begin{Verbatim}[commandchars=\\\{\}]
{\color{incolor}In [{\color{incolor}14}]:} \PY{n}{s} \PY{o}{=} \PY{n}{Series}\PY{p}{(}\PY{n}{np}\PY{o}{.}\PY{n}{random}\PY{o}{.}\PY{n}{randn}\PY{p}{(}\PY{l+m+mi}{10}\PY{p}{)}\PY{o}{.}\PY{n}{cumsum}\PY{p}{(}\PY{p}{)}\PY{p}{,} \PY{n}{index}\PY{o}{=}\PY{n}{np}\PY{o}{.}\PY{n}{arange}\PY{p}{(}\PY{l+m+mi}{0}\PY{p}{,} \PY{l+m+mi}{100}\PY{p}{,} \PY{l+m+mi}{10}\PY{p}{)}\PY{p}{)}
         \PY{n}{s}\PY{o}{.}\PY{n}{plot}\PY{p}{(}\PY{p}{)}
\end{Verbatim}

            \begin{Verbatim}[commandchars=\\\{\}]
{\color{outcolor}Out[{\color{outcolor}14}]:} <matplotlib.axes.\_subplots.AxesSubplot at 0x7fe6659664d0>
\end{Verbatim}
        
    \begin{center}
    \adjustimage{max size={0.9\linewidth}{0.9\paperheight}}{Visualization with Matplotlib_files/Visualization with Matplotlib_40_1.png}
    \end{center}
    { \hspace*{\fill} \\}
    
    This extends naturally to the \texttt{DataFrame} object. Since
\texttt{DataFrame} objects already label the columns of their internal
data, it is also easy to produce legends.

    \begin{Verbatim}[commandchars=\\\{\}]
{\color{incolor}In [{\color{incolor}15}]:} \PY{n}{df} \PY{o}{=} \PY{n}{DataFrame}\PY{p}{(}\PY{n}{np}\PY{o}{.}\PY{n}{random}\PY{o}{.}\PY{n}{randn}\PY{p}{(}\PY{l+m+mi}{10}\PY{p}{,} \PY{l+m+mi}{4}\PY{p}{)}\PY{o}{.}\PY{n}{cumsum}\PY{p}{(}\PY{l+m+mi}{0}\PY{p}{)}\PY{p}{,}
                        \PY{n}{columns}\PY{o}{=}\PY{p}{[}\PY{l+s}{\PYZsq{}}\PY{l+s}{A}\PY{l+s}{\PYZsq{}}\PY{p}{,} \PY{l+s}{\PYZsq{}}\PY{l+s}{B}\PY{l+s}{\PYZsq{}}\PY{p}{,} \PY{l+s}{\PYZsq{}}\PY{l+s}{C}\PY{l+s}{\PYZsq{}}\PY{p}{,} \PY{l+s}{\PYZsq{}}\PY{l+s}{D}\PY{l+s}{\PYZsq{}}\PY{p}{]}\PY{p}{,}
                        \PY{n}{index}\PY{o}{=}\PY{n}{np}\PY{o}{.}\PY{n}{arange}\PY{p}{(}\PY{l+m+mi}{0}\PY{p}{,} \PY{l+m+mi}{100}\PY{p}{,} \PY{l+m+mi}{10}\PY{p}{)}\PY{p}{)}
         \PY{n}{df}\PY{o}{.}\PY{n}{plot}\PY{p}{(}\PY{p}{)}
\end{Verbatim}

            \begin{Verbatim}[commandchars=\\\{\}]
{\color{outcolor}Out[{\color{outcolor}15}]:} <matplotlib.axes.\_subplots.AxesSubplot at 0x7fe664594910>
\end{Verbatim}
        
    \begin{center}
    \adjustimage{max size={0.9\linewidth}{0.9\paperheight}}{Visualization with Matplotlib_files/Visualization with Matplotlib_42_1.png}
    \end{center}
    { \hspace*{\fill} \\}
    
    \subsubsection{Bar plots}\label{bar-plots}

    Of course, there are many more types of visualization than line plots.
In general, one can specify the type of plot a \texttt{Series} or
\texttt{DataFrame} generates by changing the optional parameter
\texttt{kind}.

Here is an example showcasing the \texttt{bar} and \texttt{barh} plots
we saw earlier. Additionally, we can specify the specific axis we want
as the base of the plot. \texttt{Pandas} takes care of the formatting as
well.

Plotting with \texttt{Pandas} plays nicely with both the MATLAB-style of
generation, as with the prior examples, or with the Object-oriented
paradigm, as below.

    \begin{Verbatim}[commandchars=\\\{\}]
{\color{incolor}In [{\color{incolor}16}]:} \PY{n}{fig}\PY{p}{,} \PY{n}{axes} \PY{o}{=} \PY{n}{plt}\PY{o}{.}\PY{n}{subplots}\PY{p}{(}\PY{l+m+mi}{2}\PY{p}{,} \PY{l+m+mi}{1}\PY{p}{)}
         \PY{n}{data} \PY{o}{=} \PY{n}{Series}\PY{p}{(}\PY{n}{np}\PY{o}{.}\PY{n}{random}\PY{o}{.}\PY{n}{rand}\PY{p}{(}\PY{l+m+mi}{16}\PY{p}{)}\PY{p}{,} \PY{n}{index}\PY{o}{=}\PY{n+nb}{list}\PY{p}{(}\PY{l+s}{\PYZsq{}}\PY{l+s}{abcdefghijklmnop}\PY{l+s}{\PYZsq{}}\PY{p}{)}\PY{p}{)}
         \PY{n}{data}\PY{o}{.}\PY{n}{plot}\PY{p}{(}\PY{n}{kind}\PY{o}{=}\PY{l+s}{\PYZsq{}}\PY{l+s}{bar}\PY{l+s}{\PYZsq{}}\PY{p}{,} \PY{n}{ax}\PY{o}{=}\PY{n}{axes}\PY{p}{[}\PY{l+m+mi}{0}\PY{p}{]}\PY{p}{,} \PY{n}{color}\PY{o}{=}\PY{l+s}{\PYZsq{}}\PY{l+s}{k}\PY{l+s}{\PYZsq{}}\PY{p}{,} \PY{n}{alpha}\PY{o}{=}\PY{l+m+mf}{0.7}\PY{p}{)}
         \PY{n}{data}\PY{o}{.}\PY{n}{plot}\PY{p}{(}\PY{n}{kind}\PY{o}{=}\PY{l+s}{\PYZsq{}}\PY{l+s}{barh}\PY{l+s}{\PYZsq{}}\PY{p}{,} \PY{n}{ax}\PY{o}{=}\PY{n}{axes}\PY{p}{[}\PY{l+m+mi}{1}\PY{p}{]}\PY{p}{,} \PY{n}{color}\PY{o}{=}\PY{l+s}{\PYZsq{}}\PY{l+s}{k}\PY{l+s}{\PYZsq{}}\PY{p}{,} \PY{n}{alpha}\PY{o}{=}\PY{l+m+mf}{0.7}\PY{p}{)}
\end{Verbatim}

            \begin{Verbatim}[commandchars=\\\{\}]
{\color{outcolor}Out[{\color{outcolor}16}]:} <matplotlib.axes.\_subplots.AxesSubplot at 0x7fe6643cb0d0>
\end{Verbatim}
        
    \begin{center}
    \adjustimage{max size={0.9\linewidth}{0.9\paperheight}}{Visualization with Matplotlib_files/Visualization with Matplotlib_45_1.png}
    \end{center}
    { \hspace*{\fill} \\}
    
    Again, \texttt{DataFrame} objects have similar functionality. Consider
the following \texttt{DataFrame}.

    \begin{Verbatim}[commandchars=\\\{\}]
{\color{incolor}In [{\color{incolor}17}]:} \PY{n}{df} \PY{o}{=} \PY{n}{DataFrame}\PY{p}{(}\PY{n}{np}\PY{o}{.}\PY{n}{random}\PY{o}{.}\PY{n}{rand}\PY{p}{(}\PY{l+m+mi}{6}\PY{p}{,} \PY{l+m+mi}{4}\PY{p}{)}\PY{p}{,}
                        \PY{n}{index}\PY{o}{=}\PY{p}{[}\PY{l+s}{\PYZsq{}}\PY{l+s}{one}\PY{l+s}{\PYZsq{}}\PY{p}{,} \PY{l+s}{\PYZsq{}}\PY{l+s}{two}\PY{l+s}{\PYZsq{}}\PY{p}{,} \PY{l+s}{\PYZsq{}}\PY{l+s}{three}\PY{l+s}{\PYZsq{}}\PY{p}{,} \PY{l+s}{\PYZsq{}}\PY{l+s}{four}\PY{l+s}{\PYZsq{}}\PY{p}{,} \PY{l+s}{\PYZsq{}}\PY{l+s}{five}\PY{l+s}{\PYZsq{}}\PY{p}{,} \PY{l+s}{\PYZsq{}}\PY{l+s}{six}\PY{l+s}{\PYZsq{}}\PY{p}{]}\PY{p}{,}
                        \PY{n}{columns}\PY{o}{=}\PY{n}{pd}\PY{o}{.}\PY{n}{Index}\PY{p}{(}\PY{p}{[}\PY{l+s}{\PYZsq{}}\PY{l+s}{A}\PY{l+s}{\PYZsq{}}\PY{p}{,} \PY{l+s}{\PYZsq{}}\PY{l+s}{B}\PY{l+s}{\PYZsq{}}\PY{p}{,} \PY{l+s}{\PYZsq{}}\PY{l+s}{C}\PY{l+s}{\PYZsq{}}\PY{p}{,} \PY{l+s}{\PYZsq{}}\PY{l+s}{D}\PY{l+s}{\PYZsq{}}\PY{p}{]}\PY{p}{,} \PY{n}{name}\PY{o}{=}\PY{l+s}{\PYZsq{}}\PY{l+s}{Genus}\PY{l+s}{\PYZsq{}}\PY{p}{)}\PY{p}{)}
         \PY{n}{df}
\end{Verbatim}

            \begin{Verbatim}[commandchars=\\\{\}]
{\color{outcolor}Out[{\color{outcolor}17}]:} Genus         A         B         C         D
         one    0.836035  0.383135  0.413375  0.823983
         two    0.748730  0.808821  0.582242  0.856516
         three  0.611993  0.401634  0.489481  0.027948
         four   0.823606  0.948357  0.974807  0.023504
         five   0.324714  0.376265  0.946262  0.934007
         six    0.569295  0.366941  0.979173  0.634197
\end{Verbatim}
        
    We can plot the data in a bar graph just as we would for a
\texttt{Series} object. Notice that the legend by default is not fixed
to any particular location on the plot. This is the \texttt{"best"}
parameter choice for legend location. You can hide the legend by
specifying \texttt{legend=False}.

    \begin{Verbatim}[commandchars=\\\{\}]
{\color{incolor}In [{\color{incolor}18}]:} \PY{n}{df}\PY{o}{.}\PY{n}{plot}\PY{p}{(}\PY{n}{kind}\PY{o}{=}\PY{l+s}{\PYZsq{}}\PY{l+s}{bar}\PY{l+s}{\PYZsq{}}\PY{p}{)}
\end{Verbatim}

            \begin{Verbatim}[commandchars=\\\{\}]
{\color{outcolor}Out[{\color{outcolor}18}]:} <matplotlib.axes.\_subplots.AxesSubplot at 0x7fe664491310>
\end{Verbatim}
        
    \begin{center}
    \adjustimage{max size={0.9\linewidth}{0.9\paperheight}}{Visualization with Matplotlib_files/Visualization with Matplotlib_49_1.png}
    \end{center}
    { \hspace*{\fill} \\}
    
    Bar plots can also be stacked by using the \texttt{stacked} parameter.
Notice also that when using \texttt{bar} or \texttt{barh},
\texttt{Pandas} takes care of aligning the data properly to its index
label, using the column label for the legend.

    \begin{Verbatim}[commandchars=\\\{\}]
{\color{incolor}In [{\color{incolor}19}]:} \PY{n}{df}\PY{o}{.}\PY{n}{plot}\PY{p}{(}\PY{n}{kind}\PY{o}{=}\PY{l+s}{\PYZsq{}}\PY{l+s}{barh}\PY{l+s}{\PYZsq{}}\PY{p}{,} \PY{n}{stacked}\PY{o}{=}\PY{n+nb+bp}{True}\PY{p}{,} \PY{n}{alpha}\PY{o}{=}\PY{l+m+mf}{0.5}\PY{p}{)}
\end{Verbatim}

            \begin{Verbatim}[commandchars=\\\{\}]
{\color{outcolor}Out[{\color{outcolor}19}]:} <matplotlib.axes.\_subplots.AxesSubplot at 0x7fe664094890>
\end{Verbatim}
        
    \begin{center}
    \adjustimage{max size={0.9\linewidth}{0.9\paperheight}}{Visualization with Matplotlib_files/Visualization with Matplotlib_51_1.png}
    \end{center}
    { \hspace*{\fill} \\}
    
    Now, Wes Mckinney has a collection of restaurant tip data in
\texttt{csv} format that you can download \href{}{here}. We can load up
the file and put it directly into a \texttt{DataFrame} object using
\texttt{read\_csv} (more on this next week!).

    \begin{Verbatim}[commandchars=\\\{\}]
{\color{incolor}In [{\color{incolor}20}]:} \PY{n}{tips} \PY{o}{=} \PY{n}{pd}\PY{o}{.}\PY{n}{read\PYZus{}csv}\PY{p}{(}\PY{l+s}{\PYZsq{}}\PY{l+s}{mckinney\PYZhy{}files/tips.csv}\PY{l+s}{\PYZsq{}}\PY{p}{)}
         
         \PY{n}{tips}\PY{o}{.}\PY{n}{head}\PY{p}{(}\PY{p}{)} \PY{c}{\PYZsh{} head specifies to display a reasonable amount of output.}
\end{Verbatim}

            \begin{Verbatim}[commandchars=\\\{\}]
{\color{outcolor}Out[{\color{outcolor}20}]:}    total\_bill   tip     sex smoker  day    time  size
         0       16.99  1.01  Female     No  Sun  Dinner     2
         1       10.34  1.66    Male     No  Sun  Dinner     3
         2       21.01  3.50    Male     No  Sun  Dinner     3
         3       23.68  3.31    Male     No  Sun  Dinner     2
         4       24.59  3.61  Female     No  Sun  Dinner     4
\end{Verbatim}
        
    We want to cross-tabulate between the day of the week and the size of
the party. In other words, we want to count how many parties of one were
seated on Friday; how many parties of two; etc, for each day of the week
we have data (Friday, Saturday, Sunday, and Thursday). To do this we
will use \texttt{crosstab}.

First, let's look at the columns formed by \texttt{tips}.

    \begin{Verbatim}[commandchars=\\\{\}]
{\color{incolor}In [{\color{incolor}21}]:} \PY{n}{tips}\PY{o}{.}\PY{n}{columns}
\end{Verbatim}

            \begin{Verbatim}[commandchars=\\\{\}]
{\color{outcolor}Out[{\color{outcolor}21}]:} Index([u'total\_bill', u'tip', u'sex', u'smoker', u'day', u'time', u'size'], dtype='object')
\end{Verbatim}
        
    We can cross-tabulate between the day of the week (Thursday, Friday,
Saturday, or Sunday) and the number of guests per party (1-6), using
\texttt{crosstab}.

    \begin{Verbatim}[commandchars=\\\{\}]
{\color{incolor}In [{\color{incolor}22}]:} \PY{n}{party\PYZus{}counts} \PY{o}{=} \PY{n}{pd}\PY{o}{.}\PY{n}{crosstab}\PY{p}{(}\PY{n}{tips}\PY{p}{[}\PY{l+s}{\PYZdq{}}\PY{l+s}{day}\PY{l+s}{\PYZdq{}}\PY{p}{]}\PY{p}{,} \PY{n}{tips}\PY{p}{[}\PY{l+s}{\PYZdq{}}\PY{l+s}{size}\PY{l+s}{\PYZdq{}}\PY{p}{]}\PY{p}{)}
\end{Verbatim}

    \begin{Verbatim}[commandchars=\\\{\}]
{\color{incolor}In [{\color{incolor}23}]:} \PY{n}{party\PYZus{}counts}
\end{Verbatim}

            \begin{Verbatim}[commandchars=\\\{\}]
{\color{outcolor}Out[{\color{outcolor}23}]:} size  1   2   3   4  5  6
         day                      
         Fri   1  16   1   1  0  0
         Sat   2  53  18  13  1  0
         Sun   0  39  15  18  3  1
         Thur  1  48   4   5  1  3
\end{Verbatim}
        
    Now, we can proceed to the analysis. One type of plot would simply show
the breakdown of guests given a day of the week.

    \begin{Verbatim}[commandchars=\\\{\}]
{\color{incolor}In [{\color{incolor}24}]:} \PY{n}{party\PYZus{}counts}\PY{o}{.}\PY{n}{plot}\PY{p}{(}\PY{n}{kind}\PY{o}{=}\PY{l+s}{\PYZsq{}}\PY{l+s}{barh}\PY{l+s}{\PYZsq{}}\PY{p}{)}
\end{Verbatim}

            \begin{Verbatim}[commandchars=\\\{\}]
{\color{outcolor}Out[{\color{outcolor}24}]:} <matplotlib.axes.\_subplots.AxesSubplot at 0x7fe6651ade90>
\end{Verbatim}
        
    \begin{center}
    \adjustimage{max size={0.9\linewidth}{0.9\paperheight}}{Visualization with Matplotlib_files/Visualization with Matplotlib_60_1.png}
    \end{center}
    { \hspace*{\fill} \\}
    
    Not so enlightening, because there is wide variation in the data. Let's
restrict our analysis to parties with a size between 2 and 5
(inclusive).

    \begin{Verbatim}[commandchars=\\\{\}]
{\color{incolor}In [{\color{incolor}25}]:} \PY{c}{\PYZsh{} Not many 1\PYZhy{} and 6\PYZhy{}person parties}
         \PY{n}{party\PYZus{}counts} \PY{o}{=} \PY{n}{party\PYZus{}counts}\PY{o}{.}\PY{n}{ix}\PY{p}{[}\PY{p}{:}\PY{p}{,} \PY{l+m+mi}{2}\PY{p}{:}\PY{l+m+mi}{5}\PY{p}{]}
\end{Verbatim}

    We can ``normalize'' the daily data by dividing the values of a
particular entry by the sum of the values along that row. Notice that we
use \texttt{astype(float)} to make sure that no integer division
problems are encountered, and we specify \texttt{axis=0} to say that we
are normalizing along the day, not the party.

    \begin{Verbatim}[commandchars=\\\{\}]
{\color{incolor}In [{\color{incolor}26}]:} \PY{c}{\PYZsh{} Normalize to sum to 1}
         \PY{n}{party\PYZus{}pcts} \PY{o}{=} \PY{n}{party\PYZus{}counts}\PY{o}{.}\PY{n}{div}\PY{p}{(}\PY{n}{party\PYZus{}counts}\PY{o}{.}\PY{n}{sum}\PY{p}{(}\PY{l+m+mi}{1}\PY{p}{)}\PY{o}{.}\PY{n}{astype}\PY{p}{(}\PY{n+nb}{float}\PY{p}{)}\PY{p}{,} \PY{n}{axis}\PY{o}{=}\PY{l+m+mi}{0}\PY{p}{)}
         \PY{n}{party\PYZus{}pcts}
\end{Verbatim}

            \begin{Verbatim}[commandchars=\\\{\}]
{\color{outcolor}Out[{\color{outcolor}26}]:} size         2         3         4         5
         day                                         
         Fri   0.888889  0.055556  0.055556  0.000000
         Sat   0.623529  0.211765  0.152941  0.011765
         Sun   0.520000  0.200000  0.240000  0.040000
         Thur  0.827586  0.068966  0.086207  0.017241
\end{Verbatim}
        
    Given this new percentage data, it might make more sense to stack the
bars so we can see how the distribution changes from day to day.

    \begin{Verbatim}[commandchars=\\\{\}]
{\color{incolor}In [{\color{incolor}27}]:} \PY{n}{party\PYZus{}pcts}\PY{o}{.}\PY{n}{plot}\PY{p}{(}\PY{n}{kind}\PY{o}{=}\PY{l+s}{\PYZsq{}}\PY{l+s}{bar}\PY{l+s}{\PYZsq{}}\PY{p}{,} \PY{n}{stacked}\PY{o}{=}\PY{n+nb+bp}{True}\PY{p}{)}
         
         \PY{n}{plt}\PY{o}{.}\PY{n}{legend}\PY{p}{(}\PY{n}{loc}\PY{o}{=}\PY{l+s}{\PYZsq{}}\PY{l+s}{center left}\PY{l+s}{\PYZsq{}}\PY{p}{,} \PY{n}{bbox\PYZus{}to\PYZus{}anchor}\PY{o}{=}\PY{p}{(}\PY{l+m+mf}{1.0}\PY{p}{,} \PY{l+m+mf}{0.5}\PY{p}{)}\PY{p}{)}
\end{Verbatim}

            \begin{Verbatim}[commandchars=\\\{\}]
{\color{outcolor}Out[{\color{outcolor}27}]:} <matplotlib.legend.Legend at 0x7fe664ef7990>
\end{Verbatim}
        
    \begin{center}
    \adjustimage{max size={0.9\linewidth}{0.9\paperheight}}{Visualization with Matplotlib_files/Visualization with Matplotlib_66_1.png}
    \end{center}
    { \hspace*{\fill} \\}
    
    Parties seem to get much larger on weekends, while couples dominate
during weekdays. Not bad for a short analysis!

    \subsubsection{Histograms and density
plots}\label{histograms-and-density-plots}

    Of course, one might also ask what the distribution of tip percentages a
server can expect to see at a given night. \texttt{Pandas} helps us
answer this with the easy integration of histograms and density plots.

    First, let's calculate tip percentages. Luckily, the tip data and the
total bill data are already given, so adding a new column is simple.

    \begin{Verbatim}[commandchars=\\\{\}]
{\color{incolor}In [{\color{incolor}28}]:} \PY{n}{plt}\PY{o}{.}\PY{n}{figure}\PY{p}{(}\PY{p}{)}
         
         \PY{n}{tips}\PY{p}{[}\PY{l+s}{\PYZsq{}}\PY{l+s}{tip\PYZus{}pct}\PY{l+s}{\PYZsq{}}\PY{p}{]} \PY{o}{=} \PY{n}{tips}\PY{p}{[}\PY{l+s}{\PYZsq{}}\PY{l+s}{tip}\PY{l+s}{\PYZsq{}}\PY{p}{]} \PY{o}{/} \PY{n}{tips}\PY{p}{[}\PY{l+s}{\PYZsq{}}\PY{l+s}{total\PYZus{}bill}\PY{l+s}{\PYZsq{}}\PY{p}{]}
         \PY{n}{tips}\PY{p}{[}\PY{l+s}{\PYZsq{}}\PY{l+s}{tip\PYZus{}pct}\PY{l+s}{\PYZsq{}}\PY{p}{]}\PY{o}{.}\PY{n}{hist}\PY{p}{(}\PY{n}{bins}\PY{o}{=}\PY{l+m+mi}{50}\PY{p}{)}
         
         \PY{n}{plt}\PY{o}{.}\PY{n}{title}\PY{p}{(}\PY{l+s}{\PYZdq{}}\PY{l+s}{Histogram of tip percentages}\PY{l+s}{\PYZdq{}}\PY{p}{)}
\end{Verbatim}

            \begin{Verbatim}[commandchars=\\\{\}]
{\color{outcolor}Out[{\color{outcolor}28}]:} <matplotlib.text.Text at 0x7fe664f8ccd0>
\end{Verbatim}
        
    \begin{center}
    \adjustimage{max size={0.9\linewidth}{0.9\paperheight}}{Visualization with Matplotlib_files/Visualization with Matplotlib_71_1.png}
    \end{center}
    { \hspace*{\fill} \\}
    
    Looks like the average is right about 15\%. Not too shocking. Perhaps
instead of a histogram of bins, you want to show a smooth distribution
of density. To do so, we can simply choose the \texttt{kde} style of
plot.

    \begin{Verbatim}[commandchars=\\\{\}]
{\color{incolor}In [{\color{incolor}29}]:} \PY{n}{tips}\PY{p}{[}\PY{l+s}{\PYZsq{}}\PY{l+s}{tip\PYZus{}pct}\PY{l+s}{\PYZsq{}}\PY{p}{]}\PY{o}{.}\PY{n}{plot}\PY{p}{(}\PY{n}{kind}\PY{o}{=}\PY{l+s}{\PYZsq{}}\PY{l+s}{kde}\PY{l+s}{\PYZsq{}}\PY{p}{)}
\end{Verbatim}

            \begin{Verbatim}[commandchars=\\\{\}]
{\color{outcolor}Out[{\color{outcolor}29}]:} <matplotlib.axes.\_subplots.AxesSubplot at 0x7fe664ae2f90>
\end{Verbatim}
        
    \begin{center}
    \adjustimage{max size={0.9\linewidth}{0.9\paperheight}}{Visualization with Matplotlib_files/Visualization with Matplotlib_73_1.png}
    \end{center}
    { \hspace*{\fill} \\}
    
    \paragraph{Quiz:}\label{quiz}

Are the above two plots \texttt{Series} plots or \texttt{DataFrame}
plots?

    You can actually plot histograms and density plots together. Consider
the following random data samples.

    \begin{Verbatim}[commandchars=\\\{\}]
{\color{incolor}In [{\color{incolor}30}]:} \PY{n}{comp1} \PY{o}{=} \PY{n}{np}\PY{o}{.}\PY{n}{random}\PY{o}{.}\PY{n}{normal}\PY{p}{(}\PY{l+m+mi}{0}\PY{p}{,} \PY{l+m+mi}{1}\PY{p}{,} \PY{n}{size}\PY{o}{=}\PY{l+m+mi}{200}\PY{p}{)}  \PY{c}{\PYZsh{} N(0, 1)}
         \PY{n}{comp2} \PY{o}{=} \PY{n}{np}\PY{o}{.}\PY{n}{random}\PY{o}{.}\PY{n}{normal}\PY{p}{(}\PY{l+m+mi}{10}\PY{p}{,} \PY{l+m+mi}{2}\PY{p}{,} \PY{n}{size}\PY{o}{=}\PY{l+m+mi}{200}\PY{p}{)}  \PY{c}{\PYZsh{} N(10, 4)}
\end{Verbatim}

    By having one cell plot both a histogram and a kernel density estimate
plot, we can overlay the two of them together to form a solid
understanding of the distribution of data in the set.

    \begin{Verbatim}[commandchars=\\\{\}]
{\color{incolor}In [{\color{incolor}31}]:} \PY{n}{values} \PY{o}{=} \PY{n}{Series}\PY{p}{(}\PY{n}{np}\PY{o}{.}\PY{n}{concatenate}\PY{p}{(}\PY{p}{[}\PY{n}{comp1}\PY{p}{,} \PY{n}{comp2}\PY{p}{]}\PY{p}{)}\PY{p}{)}
         \PY{n}{values}\PY{o}{.}\PY{n}{hist}\PY{p}{(}\PY{n}{bins}\PY{o}{=}\PY{l+m+mi}{100}\PY{p}{,} \PY{n}{alpha}\PY{o}{=}\PY{l+m+mf}{0.3}\PY{p}{,} \PY{n}{color}\PY{o}{=}\PY{l+s}{\PYZsq{}}\PY{l+s}{g}\PY{l+s}{\PYZsq{}}\PY{p}{,} \PY{n}{normed}\PY{o}{=}\PY{n+nb+bp}{True}\PY{p}{)}
         \PY{n}{values}\PY{o}{.}\PY{n}{plot}\PY{p}{(}\PY{n}{kind}\PY{o}{=}\PY{l+s}{\PYZsq{}}\PY{l+s}{kde}\PY{l+s}{\PYZsq{}}\PY{p}{,} \PY{n}{style}\PY{o}{=}\PY{l+s}{\PYZsq{}}\PY{l+s}{k\PYZhy{}\PYZhy{}}\PY{l+s}{\PYZsq{}}\PY{p}{)}
\end{Verbatim}

            \begin{Verbatim}[commandchars=\\\{\}]
{\color{outcolor}Out[{\color{outcolor}31}]:} <matplotlib.axes.\_subplots.AxesSubplot at 0x7fe65c3960d0>
\end{Verbatim}
        
    \begin{center}
    \adjustimage{max size={0.9\linewidth}{0.9\paperheight}}{Visualization with Matplotlib_files/Visualization with Matplotlib_78_1.png}
    \end{center}
    { \hspace*{\fill} \\}
    
    This allows us to provide a considerable amount of information compactly
into one figure. We can also do it without losing the essence of the
visualization.

    \subsubsection{Scatter plots}\label{scatter-plots}

    When infering the relationship between two series of data, the scatter
plot can provide significant assistance to visualize correlation.
Consider the following economic data, which you can download as a
\texttt{csv} file from \href{}{here}.

    \begin{Verbatim}[commandchars=\\\{\}]
{\color{incolor}In [{\color{incolor}32}]:} \PY{n}{macro} \PY{o}{=} \PY{n}{pd}\PY{o}{.}\PY{n}{read\PYZus{}csv}\PY{p}{(}\PY{l+s}{\PYZsq{}}\PY{l+s}{mckinney\PYZhy{}files/macrodata.csv}\PY{l+s}{\PYZsq{}}\PY{p}{)}
         \PY{n}{macro}\PY{o}{.}\PY{n}{set\PYZus{}index}\PY{p}{(}\PY{p}{[}\PY{l+s}{\PYZsq{}}\PY{l+s}{year}\PY{l+s}{\PYZsq{}}\PY{p}{,}\PY{l+s}{\PYZsq{}}\PY{l+s}{quarter}\PY{l+s}{\PYZsq{}}\PY{p}{]}\PY{p}{)}\PY{o}{.}\PY{n}{tail}\PY{p}{(}\PY{p}{)}
\end{Verbatim}

            \begin{Verbatim}[commandchars=\\\{\}]
{\color{outcolor}Out[{\color{outcolor}32}]:}                 realgdp  realcons   realinv  realgovt  realdpi      cpi  \textbackslash{}
         year quarter                                                              
         2008 3        13324.600    9267.7  1990.693   991.551   9838.3  216.889   
              4        13141.920    9195.3  1857.661  1007.273   9920.4  212.174   
         2009 1        12925.410    9209.2  1558.494   996.287   9926.4  212.671   
              2        12901.504    9189.0  1456.678  1023.528  10077.5  214.469   
              3        12990.341    9256.0  1486.398  1044.088  10040.6  216.385   
         
                           m1  tbilrate  unemp      pop  infl  realint  
         year quarter                                                   
         2008 3        1474.7      1.17    6.0  305.270 -3.16     4.33  
              4        1576.5      0.12    6.9  305.952 -8.79     8.91  
         2009 1        1592.8      0.22    8.1  306.547  0.94    -0.71  
              2        1653.6      0.18    9.2  307.226  3.37    -3.19  
              3        1673.9      0.12    9.6  308.013  3.56    -3.44  
\end{Verbatim}
        
    This is a macroeconomic dataset containing the following metrics: * real
gross domestic product * real aggregate consumption * real investment *
real government investment * real disposable income * consumer prices *
M1 money stock * Treasury bill 1-month yields * unemployment rate *
population * inflation * real interest rates

It's a fair bet that this is more information than we want to process at
the moment, so we can define a new \texttt{DataFrame} considering only
the essence of the data we need in this example.

Wes McKinney then takes the data and applies transformations to make the
visualization easier.

    \begin{Verbatim}[commandchars=\\\{\}]
{\color{incolor}In [{\color{incolor}33}]:} \PY{n}{data} \PY{o}{=} \PY{n}{macro}\PY{p}{[}\PY{p}{[}\PY{l+s}{\PYZsq{}}\PY{l+s}{cpi}\PY{l+s}{\PYZsq{}}\PY{p}{,} \PY{l+s}{\PYZsq{}}\PY{l+s}{m1}\PY{l+s}{\PYZsq{}}\PY{p}{,} \PY{l+s}{\PYZsq{}}\PY{l+s}{tbilrate}\PY{l+s}{\PYZsq{}}\PY{p}{,} \PY{l+s}{\PYZsq{}}\PY{l+s}{unemp}\PY{l+s}{\PYZsq{}}\PY{p}{]}\PY{p}{]}
         \PY{n}{trans\PYZus{}data} \PY{o}{=} \PY{n}{np}\PY{o}{.}\PY{n}{log}\PY{p}{(}\PY{n}{data}\PY{p}{)}\PY{o}{.}\PY{n}{diff}\PY{p}{(}\PY{p}{)}\PY{o}{.}\PY{n}{dropna}\PY{p}{(}\PY{p}{)}
         \PY{n}{trans\PYZus{}data}\PY{p}{[}\PY{o}{\PYZhy{}}\PY{l+m+mi}{5}\PY{p}{:}\PY{p}{]}
\end{Verbatim}

            \begin{Verbatim}[commandchars=\\\{\}]
{\color{outcolor}Out[{\color{outcolor}33}]:}           cpi        m1  tbilrate     unemp
         198 -0.007904  0.045361 -0.396881  0.105361
         199 -0.021979  0.066753 -2.277267  0.139762
         200  0.002340  0.010286  0.606136  0.160343
         201  0.008419  0.037461 -0.200671  0.127339
         202  0.008894  0.012202 -0.405465  0.042560
\end{Verbatim}
        
    How does the change in the size of M1 correspond to changes in the
unemployment rate? Let's find out!

    \begin{Verbatim}[commandchars=\\\{\}]
{\color{incolor}In [{\color{incolor}34}]:} \PY{n}{plt}\PY{o}{.}\PY{n}{scatter}\PY{p}{(}\PY{n}{trans\PYZus{}data}\PY{p}{[}\PY{l+s}{\PYZsq{}}\PY{l+s}{m1}\PY{l+s}{\PYZsq{}}\PY{p}{]}\PY{p}{,} \PY{n}{trans\PYZus{}data}\PY{p}{[}\PY{l+s}{\PYZsq{}}\PY{l+s}{unemp}\PY{l+s}{\PYZsq{}}\PY{p}{]}\PY{p}{,} \PY{n}{alpha}\PY{o}{=}\PY{l+m+mf}{0.5}\PY{p}{)}
         \PY{n}{plt}\PY{o}{.}\PY{n}{xlabel}\PY{p}{(}\PY{l+s}{\PYZdq{}}\PY{l+s}{Change in \PYZdl{}}\PY{l+s}{\PYZbs{}}\PY{l+s}{log M\PYZus{}1\PYZdl{}}\PY{l+s}{\PYZdq{}}\PY{p}{)}
         \PY{n}{plt}\PY{o}{.}\PY{n}{ylabel}\PY{p}{(}\PY{l+s}{\PYZdq{}}\PY{l+s}{Change in unemployment}\PY{l+s}{\PYZdq{}}\PY{p}{)}
         \PY{n}{plt}\PY{o}{.}\PY{n}{title}\PY{p}{(}\PY{l+s}{\PYZsq{}}\PY{l+s}{Changes in log }\PY{l+s+si}{\PYZpc{}s}\PY{l+s}{ vs. log }\PY{l+s+si}{\PYZpc{}s}\PY{l+s}{\PYZsq{}} \PY{o}{\PYZpc{}} \PY{p}{(}\PY{l+s}{\PYZsq{}}\PY{l+s}{m1}\PY{l+s}{\PYZsq{}}\PY{p}{,} \PY{l+s}{\PYZsq{}}\PY{l+s}{unemp}\PY{l+s}{\PYZsq{}}\PY{p}{)}\PY{p}{)}
\end{Verbatim}

            \begin{Verbatim}[commandchars=\\\{\}]
{\color{outcolor}Out[{\color{outcolor}34}]:} <matplotlib.text.Text at 0x7fe65bfb2750>
\end{Verbatim}
        
    \begin{center}
    \adjustimage{max size={0.9\linewidth}{0.9\paperheight}}{Visualization with Matplotlib_files/Visualization with Matplotlib_86_1.png}
    \end{center}
    { \hspace*{\fill} \\}
    
    It looks like increases in the money supply may have a positive effect
on the unemployment rate. Although, it is difficult to say exactly how
(we might need a model to infer anything more). Certainly, unemployment
seems to be decreasing when the money supply shrinks, according to the
data.

    Suppose you have a new dataset and you have no idea how the various
series are related. One quick approach to get a feel for the
relationships, which you can later expand upon in a more thorough
analysis, is the scatter matrix. Given $n$ series of data,
\texttt{scatter\_matrix} produces an $n\times n$ matrix of scatter plots
corresponding to pairs of data.

    The question is what to do on the main diagonal; a scatter plot of a
data series with itself is quite uninteresting. Instead, the default
behavior is to produce a histogram of the data series, but you can
specify this to be a \texttt{kde} plot using the \texttt{diag} optional
parameter.

    \begin{Verbatim}[commandchars=\\\{\}]
{\color{incolor}In [{\color{incolor}35}]:} \PY{n}{pd}\PY{o}{.}\PY{n}{scatter\PYZus{}matrix}\PY{p}{(}\PY{n}{trans\PYZus{}data}\PY{p}{,} \PY{n}{diagonal}\PY{o}{=}\PY{l+s}{\PYZsq{}}\PY{l+s}{kde}\PY{l+s}{\PYZsq{}}\PY{p}{,} \PY{n}{color}\PY{o}{=}\PY{l+s}{\PYZsq{}}\PY{l+s}{k}\PY{l+s}{\PYZsq{}}\PY{p}{)}
\end{Verbatim}

            \begin{Verbatim}[commandchars=\\\{\}]
{\color{outcolor}Out[{\color{outcolor}35}]:} array([[<matplotlib.axes.\_subplots.AxesSubplot object at 0x7fe65c01ad50>,
                 <matplotlib.axes.\_subplots.AxesSubplot object at 0x7fe65bf280d0>,
                 <matplotlib.axes.\_subplots.AxesSubplot object at 0x7fe65be3cbd0>,
                 <matplotlib.axes.\_subplots.AxesSubplot object at 0x7fe65bdac350>],
                [<matplotlib.axes.\_subplots.AxesSubplot object at 0x7fe65bd2f150>,
                 <matplotlib.axes.\_subplots.AxesSubplot object at 0x7fe65bd5a310>,
                 <matplotlib.axes.\_subplots.AxesSubplot object at 0x7fe65bc94d90>,
                 <matplotlib.axes.\_subplots.AxesSubplot object at 0x7fe65bb99fd0>],
                [<matplotlib.axes.\_subplots.AxesSubplot object at 0x7fe65bb7bf90>,
                 <matplotlib.axes.\_subplots.AxesSubplot object at 0x7fe65bb0c210>,
                 <matplotlib.axes.\_subplots.AxesSubplot object at 0x7fe665a00f50>,
                 <matplotlib.axes.\_subplots.AxesSubplot object at 0x7fe65ba2ced0>],
                [<matplotlib.axes.\_subplots.AxesSubplot object at 0x7fe65b9b1b10>,
                 <matplotlib.axes.\_subplots.AxesSubplot object at 0x7fe65b921190>,
                 <matplotlib.axes.\_subplots.AxesSubplot object at 0x7fe65b898e90>,
                 <matplotlib.axes.\_subplots.AxesSubplot object at 0x7fe65b873a10>]], dtype=object)
\end{Verbatim}
        
    \begin{center}
    \adjustimage{max size={0.9\linewidth}{0.9\paperheight}}{Visualization with Matplotlib_files/Visualization with Matplotlib_90_1.png}
    \end{center}
    { \hspace*{\fill} \\}
    
    \section{Image Processing}\label{image-processing}

    In the module we will start an application of \texttt{scipy} and
\texttt{numpy} in order to manipulate images. For further resources we
are using ideas and from
\texttt{http://scipy-lectures.github.io/advanced/image\_processing/}.

    \subsection{Displaying Files}\label{displaying-files}

    First we need to import the \texttt{scipy} and \texttt{numpy} into our
file. After doing this we want to write an array into a file.

    \begin{Verbatim}[commandchars=\\\{\}]
{\color{incolor}In [{\color{incolor}36}]:} \PY{o}{\PYZpc{}}\PY{k}{matplotlib} inline
         \PY{k+kn}{from} \PY{n+nn}{scipy} \PY{k+kn}{import} \PY{n}{misc}
\end{Verbatim}

    \begin{Verbatim}[commandchars=\\\{\}]
{\color{incolor}In [{\color{incolor}37}]:} \PY{n}{l} \PY{o}{=} \PY{n}{misc}\PY{o}{.}\PY{n}{lena}\PY{p}{(}\PY{p}{)}
         \PY{n}{misc}\PY{o}{.}\PY{n}{imsave}\PY{p}{(}\PY{l+s}{\PYZsq{}}\PY{l+s}{lena.png}\PY{l+s}{\PYZsq{}}\PY{p}{,} \PY{n}{l}\PY{p}{)} \PY{c}{\PYZsh{} uses the Image module (PIL)}
\end{Verbatim}

    \begin{Verbatim}[commandchars=\\\{\}]
{\color{incolor}In [{\color{incolor}38}]:} \PY{k+kn}{import} \PY{n+nn}{matplotlib.pyplot} \PY{k+kn}{as} \PY{n+nn}{plt}
         \PY{n}{plt}\PY{o}{.}\PY{n}{imshow}\PY{p}{(}\PY{n}{l}\PY{p}{)}
\end{Verbatim}

            \begin{Verbatim}[commandchars=\\\{\}]
{\color{outcolor}Out[{\color{outcolor}38}]:} <matplotlib.image.AxesImage at 0x7fe65b2bc890>
\end{Verbatim}
        
    \begin{center}
    \adjustimage{max size={0.9\linewidth}{0.9\paperheight}}{Visualization with Matplotlib_files/Visualization with Matplotlib_97_1.png}
    \end{center}
    { \hspace*{\fill} \\}
    
    We can also change the color of our image to reflect the original
greyscale.

    \begin{Verbatim}[commandchars=\\\{\}]
{\color{incolor}In [{\color{incolor}39}]:} \PY{n}{plt}\PY{o}{.}\PY{n}{imshow}\PY{p}{(}\PY{n}{l}\PY{p}{,} \PY{n}{cmap}\PY{o}{=}\PY{n}{plt}\PY{o}{.}\PY{n}{cm}\PY{o}{.}\PY{n}{gray}\PY{p}{)}
\end{Verbatim}

            \begin{Verbatim}[commandchars=\\\{\}]
{\color{outcolor}Out[{\color{outcolor}39}]:} <matplotlib.image.AxesImage at 0x7fe65b1aef90>
\end{Verbatim}
        
    \begin{center}
    \adjustimage{max size={0.9\linewidth}{0.9\paperheight}}{Visualization with Matplotlib_files/Visualization with Matplotlib_99_1.png}
    \end{center}
    { \hspace*{\fill} \\}
    
    We can increase the contrast by changing the mininimum and maximum
values.

    \begin{Verbatim}[commandchars=\\\{\}]
{\color{incolor}In [{\color{incolor}40}]:} \PY{n}{plt}\PY{o}{.}\PY{n}{imshow}\PY{p}{(}\PY{n}{l}\PY{p}{,} \PY{n}{cmap}\PY{o}{=}\PY{n}{plt}\PY{o}{.}\PY{n}{cm}\PY{o}{.}\PY{n}{gray}\PY{p}{,} \PY{n}{vmin}\PY{o}{=}\PY{l+m+mi}{100}\PY{p}{,} \PY{n}{vmax}\PY{o}{=}\PY{l+m+mi}{200}\PY{p}{)}
         \PY{n}{plt}\PY{o}{.}\PY{n}{axis}\PY{p}{(}\PY{l+s}{\PYZsq{}}\PY{l+s}{off}\PY{l+s}{\PYZsq{}}\PY{p}{)} \PY{c}{\PYZsh{} Remove axes and ticks}
\end{Verbatim}

            \begin{Verbatim}[commandchars=\\\{\}]
{\color{outcolor}Out[{\color{outcolor}40}]:} (-0.5, 511.5, 511.5, -0.5)
\end{Verbatim}
        
    \begin{center}
    \adjustimage{max size={0.9\linewidth}{0.9\paperheight}}{Visualization with Matplotlib_files/Visualization with Matplotlib_101_1.png}
    \end{center}
    { \hspace*{\fill} \\}
    
    An interesting image processing technique is drawing contour lines. We
can do this using \texttt{plt.contour}.

    \begin{Verbatim}[commandchars=\\\{\}]
{\color{incolor}In [{\color{incolor}41}]:} \PY{n}{plt}\PY{o}{.}\PY{n}{imshow}\PY{p}{(}\PY{n}{l}\PY{p}{,} \PY{n}{cmap}\PY{o}{=}\PY{n}{plt}\PY{o}{.}\PY{n}{cm}\PY{o}{.}\PY{n}{gray}\PY{p}{,}\PY{n}{vmin}\PY{o}{=}\PY{l+m+mi}{100}\PY{p}{,} \PY{n}{vmax}\PY{o}{=}\PY{l+m+mi}{200}\PY{p}{)}
         \PY{n}{plt}\PY{o}{.}\PY{n}{contour}\PY{p}{(}\PY{n}{l}\PY{p}{,} \PY{p}{[}\PY{l+m+mi}{60}\PY{p}{,} \PY{l+m+mi}{150}\PY{p}{]}\PY{p}{)}
         \PY{n}{plt}\PY{o}{.}\PY{n}{axis}\PY{p}{(}\PY{l+s}{\PYZsq{}}\PY{l+s}{off}\PY{l+s}{\PYZsq{}}\PY{p}{)}
\end{Verbatim}

            \begin{Verbatim}[commandchars=\\\{\}]
{\color{outcolor}Out[{\color{outcolor}41}]:} (-0.5, 511.5, 511.5, -0.5)
\end{Verbatim}
        
    \begin{center}
    \adjustimage{max size={0.9\linewidth}{0.9\paperheight}}{Visualization with Matplotlib_files/Visualization with Matplotlib_103_1.png}
    \end{center}
    { \hspace*{\fill} \\}
    
    We can inspect individual elements for intensity variation using
\texttt{interpolation='nearest'}.

    \begin{Verbatim}[commandchars=\\\{\}]
{\color{incolor}In [{\color{incolor}42}]:} \PY{n}{plt}\PY{o}{.}\PY{n}{imshow}\PY{p}{(}\PY{n}{l}\PY{p}{[}\PY{l+m+mi}{200}\PY{p}{:}\PY{l+m+mi}{220}\PY{p}{,} \PY{l+m+mi}{200}\PY{p}{:}\PY{l+m+mi}{220}\PY{p}{]}\PY{p}{,} \PY{n}{cmap}\PY{o}{=}\PY{n}{plt}\PY{o}{.}\PY{n}{cm}\PY{o}{.}\PY{n}{gray}\PY{p}{)}
         \PY{n}{plt}\PY{o}{.}\PY{n}{imshow}\PY{p}{(}\PY{n}{l}\PY{p}{[}\PY{l+m+mi}{200}\PY{p}{:}\PY{l+m+mi}{220}\PY{p}{,} \PY{l+m+mi}{200}\PY{p}{:}\PY{l+m+mi}{220}\PY{p}{]}\PY{p}{,} \PY{n}{cmap}\PY{o}{=}\PY{n}{plt}\PY{o}{.}\PY{n}{cm}\PY{o}{.}\PY{n}{gray}\PY{p}{,} 
                    \PY{n}{interpolation}\PY{o}{=}\PY{l+s}{\PYZsq{}}\PY{l+s}{nearest}\PY{l+s}{\PYZsq{}}\PY{p}{)}
\end{Verbatim}

            \begin{Verbatim}[commandchars=\\\{\}]
{\color{outcolor}Out[{\color{outcolor}42}]:} <matplotlib.image.AxesImage at 0x7fe6599603d0>
\end{Verbatim}
        
    \begin{center}
    \adjustimage{max size={0.9\linewidth}{0.9\paperheight}}{Visualization with Matplotlib_files/Visualization with Matplotlib_105_1.png}
    \end{center}
    { \hspace*{\fill} \\}
    
    \subsection{Basic Image Manipulations}\label{basic-image-manipulations}

    Images are arrays. Consequently, we can use array manipulations that we
used from \texttt{numpy}.

    \begin{Verbatim}[commandchars=\\\{\}]
{\color{incolor}In [{\color{incolor}43}]:} \PY{k+kn}{import} \PY{n+nn}{scipy}
         \PY{k+kn}{import} \PY{n+nn}{numpy} \PY{k+kn}{as} \PY{n+nn}{np}
\end{Verbatim}

    \begin{Verbatim}[commandchars=\\\{\}]
{\color{incolor}In [{\color{incolor}44}]:} \PY{n}{lena} \PY{o}{=} \PY{n}{scipy}\PY{o}{.}\PY{n}{misc}\PY{o}{.}\PY{n}{lena}\PY{p}{(}\PY{p}{)}
         \PY{n}{lena}\PY{p}{[}\PY{l+m+mi}{10}\PY{p}{:}\PY{l+m+mi}{13}\PY{p}{,} \PY{l+m+mi}{20}\PY{p}{:}\PY{l+m+mi}{23}\PY{p}{]}
         \PY{n}{lena}\PY{p}{[}\PY{l+m+mi}{100}\PY{p}{:}\PY{l+m+mi}{120}\PY{p}{]} \PY{o}{=} \PY{l+m+mi}{255}
         
         \PY{n}{lx}\PY{p}{,} \PY{n}{ly} \PY{o}{=} \PY{n}{lena}\PY{o}{.}\PY{n}{shape}
         \PY{n}{X}\PY{p}{,} \PY{n}{Y} \PY{o}{=} \PY{n}{np}\PY{o}{.}\PY{n}{ogrid}\PY{p}{[}\PY{l+m+mi}{0}\PY{p}{:}\PY{n}{lx}\PY{p}{,} \PY{l+m+mi}{0}\PY{p}{:}\PY{n}{ly}\PY{p}{]}
         \PY{n}{mask} \PY{o}{=} \PY{p}{(}\PY{n}{X} \PY{o}{\PYZhy{}} \PY{n}{lx}\PY{o}{/}\PY{l+m+mi}{2}\PY{p}{)}\PY{o}{*}\PY{o}{*}\PY{l+m+mi}{2} \PY{o}{+} \PY{p}{(}\PY{n}{Y} \PY{o}{\PYZhy{}} \PY{n}{ly}\PY{o}{/}\PY{l+m+mi}{2}\PY{p}{)}\PY{o}{*}\PY{o}{*}\PY{l+m+mi}{2} \PY{o}{\PYZgt{}} \PY{n}{lx}\PY{o}{*}\PY{n}{ly}\PY{o}{/}\PY{l+m+mi}{4}
         \PY{n}{lena}\PY{p}{[}\PY{n}{mask}\PY{p}{]} \PY{o}{=} \PY{l+m+mi}{0}
         \PY{n}{lena}\PY{p}{[}\PY{n+nb}{range}\PY{p}{(}\PY{l+m+mi}{400}\PY{p}{)}\PY{p}{,} \PY{n+nb}{range}\PY{p}{(}\PY{l+m+mi}{400}\PY{p}{)}\PY{p}{]} \PY{o}{=} \PY{l+m+mi}{255}
         
         \PY{n}{plt}\PY{o}{.}\PY{n}{figure}\PY{p}{(}\PY{n}{figsize}\PY{o}{=}\PY{p}{(}\PY{l+m+mi}{3}\PY{p}{,} \PY{l+m+mi}{3}\PY{p}{)}\PY{p}{)}
         \PY{n}{plt}\PY{o}{.}\PY{n}{axes}\PY{p}{(}\PY{p}{[}\PY{l+m+mi}{0}\PY{p}{,} \PY{l+m+mi}{0}\PY{p}{,} \PY{l+m+mi}{1}\PY{p}{,} \PY{l+m+mi}{1}\PY{p}{]}\PY{p}{)}
         \PY{n}{plt}\PY{o}{.}\PY{n}{imshow}\PY{p}{(}\PY{n}{lena}\PY{p}{,} \PY{n}{cmap}\PY{o}{=}\PY{n}{plt}\PY{o}{.}\PY{n}{cm}\PY{o}{.}\PY{n}{gray}\PY{p}{)}
         \PY{n}{plt}\PY{o}{.}\PY{n}{axis}\PY{p}{(}\PY{l+s}{\PYZsq{}}\PY{l+s}{off}\PY{l+s}{\PYZsq{}}\PY{p}{)}
\end{Verbatim}

            \begin{Verbatim}[commandchars=\\\{\}]
{\color{outcolor}Out[{\color{outcolor}44}]:} (-0.5, 511.5, 511.5, -0.5)
\end{Verbatim}
        
    \begin{center}
    \adjustimage{max size={0.9\linewidth}{0.9\paperheight}}{Visualization with Matplotlib_files/Visualization with Matplotlib_109_1.png}
    \end{center}
    { \hspace*{\fill} \\}
    
    \subsection{Geometric Transformations}\label{geometric-transformations}

    We can easily rotate and flip the image using the \texttt{numpy}
library.

    \begin{Verbatim}[commandchars=\\\{\}]
{\color{incolor}In [{\color{incolor}45}]:} \PY{k+kn}{from} \PY{n+nn}{scipy} \PY{k+kn}{import} \PY{n}{ndimage}
         
         \PY{n}{lena} \PY{o}{=} \PY{n}{scipy}\PY{o}{.}\PY{n}{misc}\PY{o}{.}\PY{n}{lena}\PY{p}{(}\PY{p}{)}
         \PY{n}{lx}\PY{p}{,} \PY{n}{ly} \PY{o}{=} \PY{n}{lena}\PY{o}{.}\PY{n}{shape}
         
         \PY{c}{\PYZsh{} Cropping}
         
         \PY{n}{crop\PYZus{}lena} \PY{o}{=} \PY{n}{lena}\PY{p}{[}\PY{n}{lx}\PY{o}{/}\PY{l+m+mi}{4}\PY{p}{:}\PY{o}{\PYZhy{}}\PY{n}{lx}\PY{o}{/}\PY{l+m+mi}{4}\PY{p}{,} \PY{n}{ly}\PY{o}{/}\PY{l+m+mi}{4}\PY{p}{:}\PY{o}{\PYZhy{}}\PY{n}{ly}\PY{o}{/}\PY{l+m+mi}{4}\PY{p}{]}
         \PY{c}{\PYZsh{} up \PYZlt{}\PYZhy{}\PYZgt{} down flip}
         \PY{n}{flip\PYZus{}ud\PYZus{}lena} \PY{o}{=} \PY{n}{np}\PY{o}{.}\PY{n}{flipud}\PY{p}{(}\PY{n}{lena}\PY{p}{)}
         \PY{c}{\PYZsh{} rotation}
         \PY{n}{rotate\PYZus{}lena} \PY{o}{=} \PY{n}{ndimage}\PY{o}{.}\PY{n}{rotate}\PY{p}{(}\PY{n}{lena}\PY{p}{,} \PY{l+m+mi}{45}\PY{p}{)}
         \PY{n}{rotate\PYZus{}lena\PYZus{}noreshape} \PY{o}{=} \PY{n}{ndimage}\PY{o}{.}\PY{n}{rotate}\PY{p}{(}\PY{n}{lena}\PY{p}{,} \PY{l+m+mi}{45}\PY{p}{,} \PY{n}{reshape}\PY{o}{=}\PY{n+nb+bp}{False}\PY{p}{)}
         
         \PY{n}{plt}\PY{o}{.}\PY{n}{figure}\PY{p}{(}\PY{n}{figsize}\PY{o}{=}\PY{p}{(}\PY{l+m+mf}{12.5}\PY{p}{,} \PY{l+m+mf}{2.5}\PY{p}{)}\PY{p}{)}
         
         
         \PY{n}{plt}\PY{o}{.}\PY{n}{subplot}\PY{p}{(}\PY{l+m+mi}{151}\PY{p}{)}
         \PY{n}{plt}\PY{o}{.}\PY{n}{imshow}\PY{p}{(}\PY{n}{lena}\PY{p}{,} \PY{n}{cmap}\PY{o}{=}\PY{n}{plt}\PY{o}{.}\PY{n}{cm}\PY{o}{.}\PY{n}{gray}\PY{p}{)}
         \PY{n}{plt}\PY{o}{.}\PY{n}{axis}\PY{p}{(}\PY{l+s}{\PYZsq{}}\PY{l+s}{off}\PY{l+s}{\PYZsq{}}\PY{p}{)}
         \PY{n}{plt}\PY{o}{.}\PY{n}{subplot}\PY{p}{(}\PY{l+m+mi}{152}\PY{p}{)}
         \PY{n}{plt}\PY{o}{.}\PY{n}{imshow}\PY{p}{(}\PY{n}{crop\PYZus{}lena}\PY{p}{,} \PY{n}{cmap}\PY{o}{=}\PY{n}{plt}\PY{o}{.}\PY{n}{cm}\PY{o}{.}\PY{n}{gray}\PY{p}{)}
         \PY{n}{plt}\PY{o}{.}\PY{n}{axis}\PY{p}{(}\PY{l+s}{\PYZsq{}}\PY{l+s}{off}\PY{l+s}{\PYZsq{}}\PY{p}{)}
         \PY{n}{plt}\PY{o}{.}\PY{n}{subplot}\PY{p}{(}\PY{l+m+mi}{153}\PY{p}{)}
         \PY{n}{plt}\PY{o}{.}\PY{n}{imshow}\PY{p}{(}\PY{n}{flip\PYZus{}ud\PYZus{}lena}\PY{p}{,} \PY{n}{cmap}\PY{o}{=}\PY{n}{plt}\PY{o}{.}\PY{n}{cm}\PY{o}{.}\PY{n}{gray}\PY{p}{)}
         \PY{n}{plt}\PY{o}{.}\PY{n}{axis}\PY{p}{(}\PY{l+s}{\PYZsq{}}\PY{l+s}{off}\PY{l+s}{\PYZsq{}}\PY{p}{)}
         \PY{n}{plt}\PY{o}{.}\PY{n}{subplot}\PY{p}{(}\PY{l+m+mi}{154}\PY{p}{)}
         \PY{n}{plt}\PY{o}{.}\PY{n}{imshow}\PY{p}{(}\PY{n}{rotate\PYZus{}lena}\PY{p}{,} \PY{n}{cmap}\PY{o}{=}\PY{n}{plt}\PY{o}{.}\PY{n}{cm}\PY{o}{.}\PY{n}{gray}\PY{p}{)}
         \PY{n}{plt}\PY{o}{.}\PY{n}{axis}\PY{p}{(}\PY{l+s}{\PYZsq{}}\PY{l+s}{off}\PY{l+s}{\PYZsq{}}\PY{p}{)}
         \PY{n}{plt}\PY{o}{.}\PY{n}{subplot}\PY{p}{(}\PY{l+m+mi}{155}\PY{p}{)}
         \PY{n}{plt}\PY{o}{.}\PY{n}{imshow}\PY{p}{(}\PY{n}{rotate\PYZus{}lena\PYZus{}noreshape}\PY{p}{,} \PY{n}{cmap}\PY{o}{=}\PY{n}{plt}\PY{o}{.}\PY{n}{cm}\PY{o}{.}\PY{n}{gray}\PY{p}{)}
         \PY{n}{plt}\PY{o}{.}\PY{n}{axis}\PY{p}{(}\PY{l+s}{\PYZsq{}}\PY{l+s}{off}\PY{l+s}{\PYZsq{}}\PY{p}{)}
         
         \PY{n}{plt}\PY{o}{.}\PY{n}{subplots\PYZus{}adjust}\PY{p}{(}\PY{n}{wspace}\PY{o}{=}\PY{l+m+mf}{0.02}\PY{p}{,} \PY{n}{hspace}\PY{o}{=}\PY{l+m+mf}{0.3}\PY{p}{,} \PY{n}{top}\PY{o}{=}\PY{l+m+mi}{1}\PY{p}{,} \PY{n}{bottom}\PY{o}{=}\PY{l+m+mf}{0.1}\PY{p}{,} \PY{n}{left}\PY{o}{=}\PY{l+m+mi}{0}\PY{p}{,}
                             \PY{n}{right}\PY{o}{=}\PY{l+m+mi}{1}\PY{p}{)}
\end{Verbatim}

    \begin{center}
    \adjustimage{max size={0.9\linewidth}{0.9\paperheight}}{Visualization with Matplotlib_files/Visualization with Matplotlib_112_0.png}
    \end{center}
    { \hspace*{\fill} \\}
    
    \subsection{Image Filtering}\label{image-filtering}

    We can filter images by replacing the value of the pixels by a function
of adjacent pixels. In the example below we use two different filters.
The Gaussian filter sets the value of a pixel to the weighted average of
the value of neighboring pixels, where nearby pixels have greater
weights. The uniform filter is simply the average value of the pixels a
set distance away.

    \begin{Verbatim}[commandchars=\\\{\}]
{\color{incolor}In [{\color{incolor}46}]:} \PY{n}{lena} \PY{o}{=} \PY{n}{scipy}\PY{o}{.}\PY{n}{misc}\PY{o}{.}\PY{n}{lena}\PY{p}{(}\PY{p}{)}
         \PY{n}{blurred\PYZus{}lena} \PY{o}{=} \PY{n}{ndimage}\PY{o}{.}\PY{n}{gaussian\PYZus{}filter}\PY{p}{(}\PY{n}{lena}\PY{p}{,} \PY{n}{sigma}\PY{o}{=}\PY{l+m+mi}{3}\PY{p}{)}
         \PY{n}{very\PYZus{}blurred} \PY{o}{=} \PY{n}{ndimage}\PY{o}{.}\PY{n}{gaussian\PYZus{}filter}\PY{p}{(}\PY{n}{lena}\PY{p}{,} \PY{n}{sigma}\PY{o}{=}\PY{l+m+mi}{5}\PY{p}{)}
         \PY{n}{local\PYZus{}mean} \PY{o}{=} \PY{n}{ndimage}\PY{o}{.}\PY{n}{uniform\PYZus{}filter}\PY{p}{(}\PY{n}{lena}\PY{p}{,} \PY{n}{size}\PY{o}{=}\PY{l+m+mi}{11}\PY{p}{)}
         
         
         \PY{n}{plt}\PY{o}{.}\PY{n}{figure}\PY{p}{(}\PY{n}{figsize}\PY{o}{=}\PY{p}{(}\PY{l+m+mi}{9}\PY{p}{,} \PY{l+m+mi}{3}\PY{p}{)}\PY{p}{)}
         \PY{n}{plt}\PY{o}{.}\PY{n}{subplot}\PY{p}{(}\PY{l+m+mi}{131}\PY{p}{)}
         \PY{n}{plt}\PY{o}{.}\PY{n}{imshow}\PY{p}{(}\PY{n}{blurred\PYZus{}lena}\PY{p}{,} \PY{n}{cmap}\PY{o}{=}\PY{n}{plt}\PY{o}{.}\PY{n}{cm}\PY{o}{.}\PY{n}{gray}\PY{p}{)}
         \PY{n}{plt}\PY{o}{.}\PY{n}{axis}\PY{p}{(}\PY{l+s}{\PYZsq{}}\PY{l+s}{off}\PY{l+s}{\PYZsq{}}\PY{p}{)}
         \PY{n}{plt}\PY{o}{.}\PY{n}{subplot}\PY{p}{(}\PY{l+m+mi}{132}\PY{p}{)}
         \PY{n}{plt}\PY{o}{.}\PY{n}{imshow}\PY{p}{(}\PY{n}{very\PYZus{}blurred}\PY{p}{,} \PY{n}{cmap}\PY{o}{=}\PY{n}{plt}\PY{o}{.}\PY{n}{cm}\PY{o}{.}\PY{n}{gray}\PY{p}{)}
         \PY{n}{plt}\PY{o}{.}\PY{n}{axis}\PY{p}{(}\PY{l+s}{\PYZsq{}}\PY{l+s}{off}\PY{l+s}{\PYZsq{}}\PY{p}{)}
         \PY{n}{plt}\PY{o}{.}\PY{n}{subplot}\PY{p}{(}\PY{l+m+mi}{133}\PY{p}{)}
         \PY{n}{plt}\PY{o}{.}\PY{n}{imshow}\PY{p}{(}\PY{n}{local\PYZus{}mean}\PY{p}{,} \PY{n}{cmap}\PY{o}{=}\PY{n}{plt}\PY{o}{.}\PY{n}{cm}\PY{o}{.}\PY{n}{gray}\PY{p}{)}
         \PY{n}{plt}\PY{o}{.}\PY{n}{axis}\PY{p}{(}\PY{l+s}{\PYZsq{}}\PY{l+s}{off}\PY{l+s}{\PYZsq{}}\PY{p}{)}
         
         \PY{n}{plt}\PY{o}{.}\PY{n}{subplots\PYZus{}adjust}\PY{p}{(}\PY{n}{wspace}\PY{o}{=}\PY{l+m+mi}{0}\PY{p}{,} \PY{n}{hspace}\PY{o}{=}\PY{l+m+mf}{0.}\PY{p}{,} \PY{n}{top}\PY{o}{=}\PY{l+m+mf}{0.99}\PY{p}{,} \PY{n}{bottom}\PY{o}{=}\PY{l+m+mf}{0.01}\PY{p}{,}
                             \PY{n}{left}\PY{o}{=}\PY{l+m+mf}{0.01}\PY{p}{,} \PY{n}{right}\PY{o}{=}\PY{l+m+mf}{0.99}\PY{p}{)}
\end{Verbatim}

    \begin{center}
    \adjustimage{max size={0.9\linewidth}{0.9\paperheight}}{Visualization with Matplotlib_files/Visualization with Matplotlib_115_0.png}
    \end{center}
    { \hspace*{\fill} \\}
    
    \subsection{Image Sharpening}\label{image-sharpening}

    We can also sharpen a blurred image. The following shows the original
image followed by a blurred image and a resharpened image.

    \begin{Verbatim}[commandchars=\\\{\}]
{\color{incolor}In [{\color{incolor}47}]:} \PY{n}{l} \PY{o}{=} \PY{n}{scipy}\PY{o}{.}\PY{n}{misc}\PY{o}{.}\PY{n}{lena}\PY{p}{(}\PY{p}{)}
         \PY{n}{blurred\PYZus{}l} \PY{o}{=} \PY{n}{ndimage}\PY{o}{.}\PY{n}{gaussian\PYZus{}filter}\PY{p}{(}\PY{n}{l}\PY{p}{,} \PY{l+m+mi}{3}\PY{p}{)}
         
         \PY{n}{filter\PYZus{}blurred\PYZus{}l} \PY{o}{=} \PY{n}{ndimage}\PY{o}{.}\PY{n}{gaussian\PYZus{}filter}\PY{p}{(}\PY{n}{blurred\PYZus{}l}\PY{p}{,} \PY{l+m+mi}{1}\PY{p}{)}
         
         \PY{n}{alpha} \PY{o}{=} \PY{l+m+mi}{30}
         \PY{n}{sharpened} \PY{o}{=} \PY{n}{blurred\PYZus{}l} \PY{o}{+} \PY{n}{alpha} \PY{o}{*} \PY{p}{(}\PY{n}{blurred\PYZus{}l} \PY{o}{\PYZhy{}} \PY{n}{filter\PYZus{}blurred\PYZus{}l}\PY{p}{)}
         
         \PY{n}{plt}\PY{o}{.}\PY{n}{figure}\PY{p}{(}\PY{n}{figsize}\PY{o}{=}\PY{p}{(}\PY{l+m+mi}{12}\PY{p}{,} \PY{l+m+mi}{4}\PY{p}{)}\PY{p}{)}
         
         \PY{n}{plt}\PY{o}{.}\PY{n}{subplot}\PY{p}{(}\PY{l+m+mi}{131}\PY{p}{)}
         \PY{n}{plt}\PY{o}{.}\PY{n}{imshow}\PY{p}{(}\PY{n}{l}\PY{p}{,} \PY{n}{cmap}\PY{o}{=}\PY{n}{plt}\PY{o}{.}\PY{n}{cm}\PY{o}{.}\PY{n}{gray}\PY{p}{)}
         \PY{n}{plt}\PY{o}{.}\PY{n}{axis}\PY{p}{(}\PY{l+s}{\PYZsq{}}\PY{l+s}{off}\PY{l+s}{\PYZsq{}}\PY{p}{)}
         \PY{n}{plt}\PY{o}{.}\PY{n}{subplot}\PY{p}{(}\PY{l+m+mi}{132}\PY{p}{)}
         \PY{n}{plt}\PY{o}{.}\PY{n}{imshow}\PY{p}{(}\PY{n}{blurred\PYZus{}l}\PY{p}{,} \PY{n}{cmap}\PY{o}{=}\PY{n}{plt}\PY{o}{.}\PY{n}{cm}\PY{o}{.}\PY{n}{gray}\PY{p}{)}
         \PY{n}{plt}\PY{o}{.}\PY{n}{axis}\PY{p}{(}\PY{l+s}{\PYZsq{}}\PY{l+s}{off}\PY{l+s}{\PYZsq{}}\PY{p}{)}
         \PY{n}{plt}\PY{o}{.}\PY{n}{subplot}\PY{p}{(}\PY{l+m+mi}{133}\PY{p}{)}
         \PY{n}{plt}\PY{o}{.}\PY{n}{imshow}\PY{p}{(}\PY{n}{sharpened}\PY{p}{,} \PY{n}{cmap}\PY{o}{=}\PY{n}{plt}\PY{o}{.}\PY{n}{cm}\PY{o}{.}\PY{n}{gray}\PY{p}{)}
         \PY{n}{plt}\PY{o}{.}\PY{n}{axis}\PY{p}{(}\PY{l+s}{\PYZsq{}}\PY{l+s}{off}\PY{l+s}{\PYZsq{}}\PY{p}{)}
\end{Verbatim}

            \begin{Verbatim}[commandchars=\\\{\}]
{\color{outcolor}Out[{\color{outcolor}47}]:} (-0.5, 511.5, 511.5, -0.5)
\end{Verbatim}
        
    \begin{center}
    \adjustimage{max size={0.9\linewidth}{0.9\paperheight}}{Visualization with Matplotlib_files/Visualization with Matplotlib_118_1.png}
    \end{center}
    { \hspace*{\fill} \\}
    
    \subsection{Denoising}\label{denoising}

    Applying the filters we learned to help us blur and sharpen images allow
us to denoise an image. However, these filters are not without problems.
The Gaussian filter smoothes out the noise, but it also smoothes out the
edges of the picture. A median picture smoothes the noise, but it
preserves the edges better than the Gaussian filter.

    \begin{Verbatim}[commandchars=\\\{\}]
{\color{incolor}In [{\color{incolor}48}]:} \PY{n}{l} \PY{o}{=} \PY{n}{scipy}\PY{o}{.}\PY{n}{misc}\PY{o}{.}\PY{n}{lena}\PY{p}{(}\PY{p}{)}
         \PY{n}{l} \PY{o}{=} \PY{n}{l}\PY{p}{[}\PY{l+m+mi}{230}\PY{p}{:}\PY{l+m+mi}{290}\PY{p}{,} \PY{l+m+mi}{220}\PY{p}{:}\PY{l+m+mi}{320}\PY{p}{]}
         
         \PY{n}{noisy} \PY{o}{=} \PY{n}{l} \PY{o}{+} \PY{l+m+mf}{0.4}\PY{o}{*}\PY{n}{l}\PY{o}{.}\PY{n}{std}\PY{p}{(}\PY{p}{)}\PY{o}{*}\PY{n}{np}\PY{o}{.}\PY{n}{random}\PY{o}{.}\PY{n}{random}\PY{p}{(}\PY{n}{l}\PY{o}{.}\PY{n}{shape}\PY{p}{)}
         
         \PY{n}{gauss\PYZus{}denoised} \PY{o}{=} \PY{n}{ndimage}\PY{o}{.}\PY{n}{gaussian\PYZus{}filter}\PY{p}{(}\PY{n}{noisy}\PY{p}{,} \PY{l+m+mi}{2}\PY{p}{)}
         \PY{n}{med\PYZus{}denoised} \PY{o}{=} \PY{n}{ndimage}\PY{o}{.}\PY{n}{median\PYZus{}filter}\PY{p}{(}\PY{n}{noisy}\PY{p}{,} \PY{l+m+mi}{3}\PY{p}{)}
         
         
         \PY{n}{plt}\PY{o}{.}\PY{n}{figure}\PY{p}{(}\PY{n}{figsize}\PY{o}{=}\PY{p}{(}\PY{l+m+mi}{12}\PY{p}{,}\PY{l+m+mf}{2.8}\PY{p}{)}\PY{p}{)}
         
         \PY{n}{plt}\PY{o}{.}\PY{n}{subplot}\PY{p}{(}\PY{l+m+mi}{131}\PY{p}{)}
         \PY{n}{plt}\PY{o}{.}\PY{n}{imshow}\PY{p}{(}\PY{n}{noisy}\PY{p}{,} \PY{n}{cmap}\PY{o}{=}\PY{n}{plt}\PY{o}{.}\PY{n}{cm}\PY{o}{.}\PY{n}{gray}\PY{p}{,} \PY{n}{vmin}\PY{o}{=}\PY{l+m+mi}{40}\PY{p}{,} \PY{n}{vmax}\PY{o}{=}\PY{l+m+mi}{220}\PY{p}{)}
         \PY{n}{plt}\PY{o}{.}\PY{n}{axis}\PY{p}{(}\PY{l+s}{\PYZsq{}}\PY{l+s}{off}\PY{l+s}{\PYZsq{}}\PY{p}{)}
         \PY{n}{plt}\PY{o}{.}\PY{n}{title}\PY{p}{(}\PY{l+s}{\PYZsq{}}\PY{l+s}{noisy}\PY{l+s}{\PYZsq{}}\PY{p}{,} \PY{n}{fontsize}\PY{o}{=}\PY{l+m+mi}{20}\PY{p}{)}
         \PY{n}{plt}\PY{o}{.}\PY{n}{subplot}\PY{p}{(}\PY{l+m+mi}{132}\PY{p}{)}
         \PY{n}{plt}\PY{o}{.}\PY{n}{imshow}\PY{p}{(}\PY{n}{gauss\PYZus{}denoised}\PY{p}{,} \PY{n}{cmap}\PY{o}{=}\PY{n}{plt}\PY{o}{.}\PY{n}{cm}\PY{o}{.}\PY{n}{gray}\PY{p}{,} \PY{n}{vmin}\PY{o}{=}\PY{l+m+mi}{40}\PY{p}{,} \PY{n}{vmax}\PY{o}{=}\PY{l+m+mi}{220}\PY{p}{)}
         \PY{n}{plt}\PY{o}{.}\PY{n}{axis}\PY{p}{(}\PY{l+s}{\PYZsq{}}\PY{l+s}{off}\PY{l+s}{\PYZsq{}}\PY{p}{)}
         \PY{n}{plt}\PY{o}{.}\PY{n}{title}\PY{p}{(}\PY{l+s}{\PYZsq{}}\PY{l+s}{Gaussian filter}\PY{l+s}{\PYZsq{}}\PY{p}{,} \PY{n}{fontsize}\PY{o}{=}\PY{l+m+mi}{20}\PY{p}{)}
         \PY{n}{plt}\PY{o}{.}\PY{n}{subplot}\PY{p}{(}\PY{l+m+mi}{133}\PY{p}{)}
         \PY{n}{plt}\PY{o}{.}\PY{n}{imshow}\PY{p}{(}\PY{n}{med\PYZus{}denoised}\PY{p}{,} \PY{n}{cmap}\PY{o}{=}\PY{n}{plt}\PY{o}{.}\PY{n}{cm}\PY{o}{.}\PY{n}{gray}\PY{p}{,} \PY{n}{vmin}\PY{o}{=}\PY{l+m+mi}{40}\PY{p}{,} \PY{n}{vmax}\PY{o}{=}\PY{l+m+mi}{220}\PY{p}{)}
         \PY{n}{plt}\PY{o}{.}\PY{n}{axis}\PY{p}{(}\PY{l+s}{\PYZsq{}}\PY{l+s}{off}\PY{l+s}{\PYZsq{}}\PY{p}{)}
         \PY{n}{plt}\PY{o}{.}\PY{n}{title}\PY{p}{(}\PY{l+s}{\PYZsq{}}\PY{l+s}{Median filter}\PY{l+s}{\PYZsq{}}\PY{p}{,} \PY{n}{fontsize}\PY{o}{=}\PY{l+m+mi}{20}\PY{p}{)}
         
         \PY{n}{plt}\PY{o}{.}\PY{n}{subplots\PYZus{}adjust}\PY{p}{(}\PY{n}{wspace}\PY{o}{=}\PY{l+m+mf}{0.02}\PY{p}{,} \PY{n}{hspace}\PY{o}{=}\PY{l+m+mf}{0.02}\PY{p}{,} \PY{n}{top}\PY{o}{=}\PY{l+m+mf}{0.9}\PY{p}{,} \PY{n}{bottom}\PY{o}{=}\PY{l+m+mi}{0}\PY{p}{,} \PY{n}{left}\PY{o}{=}\PY{l+m+mi}{0}\PY{p}{,}
                             \PY{n}{right}\PY{o}{=}\PY{l+m+mi}{1}\PY{p}{)}
\end{Verbatim}

    \begin{center}
    \adjustimage{max size={0.9\linewidth}{0.9\paperheight}}{Visualization with Matplotlib_files/Visualization with Matplotlib_121_0.png}
    \end{center}
    { \hspace*{\fill} \\}
    
    The median filter is better when working with straight edges
(low-curviture images).

    \begin{Verbatim}[commandchars=\\\{\}]
{\color{incolor}In [{\color{incolor}49}]:} \PY{n}{im} \PY{o}{=} \PY{n}{np}\PY{o}{.}\PY{n}{zeros}\PY{p}{(}\PY{p}{(}\PY{l+m+mi}{20}\PY{p}{,} \PY{l+m+mi}{20}\PY{p}{)}\PY{p}{)}
         \PY{n}{im}\PY{p}{[}\PY{l+m+mi}{5}\PY{p}{:}\PY{o}{\PYZhy{}}\PY{l+m+mi}{5}\PY{p}{,} \PY{l+m+mi}{5}\PY{p}{:}\PY{o}{\PYZhy{}}\PY{l+m+mi}{5}\PY{p}{]} \PY{o}{=} \PY{l+m+mi}{1}
         \PY{n}{im} \PY{o}{=} \PY{n}{ndimage}\PY{o}{.}\PY{n}{distance\PYZus{}transform\PYZus{}bf}\PY{p}{(}\PY{n}{im}\PY{p}{)}
         \PY{n}{im\PYZus{}noise} \PY{o}{=} \PY{n}{im} \PY{o}{+} \PY{l+m+mf}{0.2}\PY{o}{*}\PY{n}{np}\PY{o}{.}\PY{n}{random}\PY{o}{.}\PY{n}{randn}\PY{p}{(}\PY{o}{*}\PY{n}{im}\PY{o}{.}\PY{n}{shape}\PY{p}{)}
         
         \PY{n}{im\PYZus{}med} \PY{o}{=} \PY{n}{ndimage}\PY{o}{.}\PY{n}{median\PYZus{}filter}\PY{p}{(}\PY{n}{im\PYZus{}noise}\PY{p}{,} \PY{l+m+mi}{3}\PY{p}{)}
         
         \PY{n}{plt}\PY{o}{.}\PY{n}{figure}\PY{p}{(}\PY{n}{figsize}\PY{o}{=}\PY{p}{(}\PY{l+m+mi}{12}\PY{p}{,} \PY{l+m+mi}{5}\PY{p}{)}\PY{p}{)}
         
         \PY{n}{plt}\PY{o}{.}\PY{n}{subplot}\PY{p}{(}\PY{l+m+mi}{141}\PY{p}{)}
         \PY{n}{plt}\PY{o}{.}\PY{n}{imshow}\PY{p}{(}\PY{n}{im}\PY{p}{,} \PY{n}{interpolation}\PY{o}{=}\PY{l+s}{\PYZsq{}}\PY{l+s}{nearest}\PY{l+s}{\PYZsq{}}\PY{p}{)}
         \PY{n}{plt}\PY{o}{.}\PY{n}{axis}\PY{p}{(}\PY{l+s}{\PYZsq{}}\PY{l+s}{off}\PY{l+s}{\PYZsq{}}\PY{p}{)}
         \PY{n}{plt}\PY{o}{.}\PY{n}{title}\PY{p}{(}\PY{l+s}{\PYZsq{}}\PY{l+s}{Original image}\PY{l+s}{\PYZsq{}}\PY{p}{,} \PY{n}{fontsize}\PY{o}{=}\PY{l+m+mi}{20}\PY{p}{)}
         \PY{n}{plt}\PY{o}{.}\PY{n}{subplot}\PY{p}{(}\PY{l+m+mi}{142}\PY{p}{)}
         \PY{n}{plt}\PY{o}{.}\PY{n}{imshow}\PY{p}{(}\PY{n}{im\PYZus{}noise}\PY{p}{,} \PY{n}{interpolation}\PY{o}{=}\PY{l+s}{\PYZsq{}}\PY{l+s}{nearest}\PY{l+s}{\PYZsq{}}\PY{p}{,} \PY{n}{vmin}\PY{o}{=}\PY{l+m+mi}{0}\PY{p}{,} \PY{n}{vmax}\PY{o}{=}\PY{l+m+mi}{5}\PY{p}{)}
         \PY{n}{plt}\PY{o}{.}\PY{n}{axis}\PY{p}{(}\PY{l+s}{\PYZsq{}}\PY{l+s}{off}\PY{l+s}{\PYZsq{}}\PY{p}{)}
         \PY{n}{plt}\PY{o}{.}\PY{n}{title}\PY{p}{(}\PY{l+s}{\PYZsq{}}\PY{l+s}{Noisy image}\PY{l+s}{\PYZsq{}}\PY{p}{,} \PY{n}{fontsize}\PY{o}{=}\PY{l+m+mi}{20}\PY{p}{)}
         \PY{n}{plt}\PY{o}{.}\PY{n}{subplot}\PY{p}{(}\PY{l+m+mi}{143}\PY{p}{)}
         \PY{n}{plt}\PY{o}{.}\PY{n}{imshow}\PY{p}{(}\PY{n}{im\PYZus{}med}\PY{p}{,} \PY{n}{interpolation}\PY{o}{=}\PY{l+s}{\PYZsq{}}\PY{l+s}{nearest}\PY{l+s}{\PYZsq{}}\PY{p}{,} \PY{n}{vmin}\PY{o}{=}\PY{l+m+mi}{0}\PY{p}{,} \PY{n}{vmax}\PY{o}{=}\PY{l+m+mi}{5}\PY{p}{)}
         \PY{n}{plt}\PY{o}{.}\PY{n}{axis}\PY{p}{(}\PY{l+s}{\PYZsq{}}\PY{l+s}{off}\PY{l+s}{\PYZsq{}}\PY{p}{)}
         \PY{n}{plt}\PY{o}{.}\PY{n}{title}\PY{p}{(}\PY{l+s}{\PYZsq{}}\PY{l+s}{Median filter}\PY{l+s}{\PYZsq{}}\PY{p}{,} \PY{n}{fontsize}\PY{o}{=}\PY{l+m+mi}{20}\PY{p}{)}
         \PY{n}{plt}\PY{o}{.}\PY{n}{subplot}\PY{p}{(}\PY{l+m+mi}{144}\PY{p}{)}
         \PY{n}{plt}\PY{o}{.}\PY{n}{imshow}\PY{p}{(}\PY{n}{np}\PY{o}{.}\PY{n}{abs}\PY{p}{(}\PY{n}{im} \PY{o}{\PYZhy{}} \PY{n}{im\PYZus{}med}\PY{p}{)}\PY{p}{,} \PY{n}{cmap}\PY{o}{=}\PY{n}{plt}\PY{o}{.}\PY{n}{cm}\PY{o}{.}\PY{n}{hot}\PY{p}{,} \PY{n}{vmin}\PY{o}{=}\PY{l+m+mi}{0}\PY{p}{,} \PY{n}{vmax}\PY{o}{=}\PY{l+m+mi}{5}\PY{p}{,} \PY{n}{interpolation}\PY{o}{=}\PY{l+s}{\PYZsq{}}\PY{l+s}{nearest}\PY{l+s}{\PYZsq{}}\PY{p}{)}
         \PY{n}{plt}\PY{o}{.}\PY{n}{axis}\PY{p}{(}\PY{l+s}{\PYZsq{}}\PY{l+s}{off}\PY{l+s}{\PYZsq{}}\PY{p}{)}
         \PY{n}{plt}\PY{o}{.}\PY{n}{title}\PY{p}{(}\PY{l+s}{\PYZsq{}}\PY{l+s}{Error}\PY{l+s}{\PYZsq{}}\PY{p}{,} \PY{n}{fontsize}\PY{o}{=}\PY{l+m+mi}{20}\PY{p}{)}
         
         
         \PY{n}{plt}\PY{o}{.}\PY{n}{subplots\PYZus{}adjust}\PY{p}{(}\PY{n}{wspace}\PY{o}{=}\PY{l+m+mf}{0.02}\PY{p}{,} \PY{n}{hspace}\PY{o}{=}\PY{l+m+mf}{0.02}\PY{p}{,} \PY{n}{top}\PY{o}{=}\PY{l+m+mf}{0.9}\PY{p}{,} \PY{n}{bottom}\PY{o}{=}\PY{l+m+mi}{0}\PY{p}{,} \PY{n}{left}\PY{o}{=}\PY{l+m+mi}{0}\PY{p}{,} \PY{n}{right}\PY{o}{=}\PY{l+m+mi}{1}\PY{p}{)}
\end{Verbatim}

    \begin{center}
    \adjustimage{max size={0.9\linewidth}{0.9\paperheight}}{Visualization with Matplotlib_files/Visualization with Matplotlib_123_0.png}
    \end{center}
    { \hspace*{\fill} \\}
    
    \subsubsection{Try It!}\label{try-it}

\begin{enumerate}
\def\labelenumi{\arabic{enumi}.}
\itemsep1pt\parskip0pt\parsep0pt
\item
  Try adding noise to the image \texttt{Lena}. Once you have a noisy
  image try using a median and Gaussian filter to smooth the image.
\item
  Create an error chart to measure the error between the two techniques.
\item
  Try using a new filter like \texttt{ndimage.maximum\_filter}, and
  \texttt{ndimage.percentile\_filter} on the image of concentric
  squares.
\item
  Try using a non-rank filter like \texttt{scipy.signal.wiener}.
\end{enumerate}


    % Add a bibliography block to the postdoc
    
    
    
    \end{document}
